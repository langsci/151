\documentclass[output=paper,
modfonts
]{LSP/langsci}


%% add all extra packages you need to load to this file 
% \usepackage{todo} %% removed,cna use todonotes instead. % Jason reactivated
% \usepackage{graphicx} % not needed because forest loads tikz, which loads graphicx
\usepackage{tabularx}
\usepackage{amsmath} 
\usepackage{multicol}
\usepackage{lipsum}
\usepackage{longtable}
\usepackage{booktabs}
\usepackage[normalem]{ulem}
%\usepackage{tikz} % not needed because forest loads tikz
\usepackage{phonrule} % for SPE-style phonological rules
\usepackage{pst-all} % loads the main pstricks tools; for arrow diagrams in Hale.tex
%\usepackage{leipzig} % for gloss abbreviations
\usepackage[% for automatic cross-referencing
compress,%
capitalize,% labels are always capitalized in LSP style
noabbrev]% labels are always spelled out in LSP style
{cleveref}

% based on http://tex.stackexchange.com/a/318983/42880 for using gb4e examples with cleveref
\crefname{xnumi}{}{}
\creflabelformat{xnumi}{(#2#1#3)}
\crefrangeformat{xnumi}{(#3#1#4)--(#5#2#6)}
\crefname{xnumii}{}{}
\creflabelformat{xnumii}{(#2#1#3)}
\crefrangeformat{xnumii}{(#3#1#4)--(#5#2#6)}

%\usepackage[notcite,notref]{showkeys} %%removed, not helping CB.
%\usepackage{showidx} %%remove for final compiling - shows index keys at top of page.
 
\usepackage{langsci/styles/langsci-gb4e}  
 \usepackage{pifont}
% % OT tableaux                                                
% \usepackage{pstricks,colortab}  
\usepackage{multirow} % used in OT tableaux
\usepackage{rotating} %needed for angled text%
\usepackage{colortbl} % for cell shading
 
 \usepackage{avm}  
\usepackage[linguistics]{forest} 
\usetikzlibrary{matrix,fit} % for matrix of nodes in Kaisse and Bat-El


\usepackage{hhline}
\newcommand{\cgr}{\cellcolor[gray]{0.8}}
\newcommand{\cn}{\centering}



\newcommand{\reff}[1]{(\ref{#1})}
%\usepackage{newtxtext,newtxmath}


%\usepackage[normalem] {ulem}
\usepackage{qtree}
%\usepackage{natbib}
%\usepackage{tikz}
%\usepackage{gb4e}
\usepackage{phonrule}  
%\bibliographystyle{humannat}



\usepackage{minibox}

%\include{psheader-metr}

\def\bl#1{$_{\textrm{{\footnotesize #1}}}$}

%%add all your local new commands to this file

\newcommand{\form}[1]{\mbox{\emph{#1}}}
\newcommand{\uf}[1]{\mbox{/#1/}}

% borrowed from expex and converted from plan tex to latex
\newcommand{\judge}[1]{{\upshape #1\hspace{0.1em}}}
\newcommand{\ljudge}[1]{\makebox[0pt][r]{\judge{#1}}}

\newcommand\tikzmark[1]{\tikz[remember picture, baseline=(#1.base)] \node[anchor=base,inner sep=0pt, outer sep=0pt] (#1) {#1};} % for adding decorations, arrows, lines, etc. to text
\newcommand\tikzmarknamed[2]{\tikz[remember picture, baseline=(#1.base)] \node[anchor=base,inner sep=0pt, outer sep=0pt] (#1) {#2};} % for adding decorations, arrows, lines, etc. to text
\newcommand\tikzmarkfullnamed[2]{\tikz[remember picture, baseline=(#1.base)] \node[anchor=base,inner sep=0pt, outer sep=0pt] (#1) {\vphantom{X}#2};} % for adding decorations, arrows, lines, etc. to text; this one works best for decorations above a line of text because it adds in the heigh of a capital letter and takes two arguments - one for the node name and one for the printed text

\newcommand{\sub}[1]{$_{\text{#1}}$} % for non-math subscripts
\newcommand{\subit}[1]{\sub{\textit{#1}}} % for italics non-math subscripts
\newcommand{\1}{\rlap{$'$}\xspace} % for the prime in X' (the \rlap command allows the prime to be ignored for horizontal spacing in trees, and the \xspace command allows you to use this in normal text without adding \ afterwards). This isn't crucial, but it helps the formatting to look a little better.

% Aissen:
\newcommand\tikzmarkfull[1]{\tikz[remember picture, baseline=(#1.base)] \node[anchor=base,inner sep=0pt, outer sep=0pt] (#1) {\vphantom{X}#1};} % for adding decorations, arrows, lines, etc. to text; this one works best for decorations above a line of text because it adds in the heigh of a capital letter and takes one argument that serves as the name and the printed text
\newcommand{\bridgeover}[2]{% for a line that creates a bridge over text, connecting two nodes
	\begin{tikzpicture}[remember picture,overlay]
	\draw[thick,shorten >=3pt,shorten <=3pt] (#1.north) |- +(0ex,2.5ex) -| (#2.north);
	\end{tikzpicture}
}
\newcommand{\bridgeoverht}[3]{% for a line that creates a bridge over text, connecting two nodes and specifing the height of the bridge
	\begin{tikzpicture}[remember picture,overlay]
	\draw[thick,shorten >=3pt,shorten <=3pt] (#2.north) |- +(0ex,#1) -| (#3.north);
	\end{tikzpicture}
}
\newcommand{\bridgeoverex}{\vspace*{3ex}} % place before an example that has a \bridgeover so that there is enough vertical space for it

% Chung:
\newcommand{\lefttabular}[1]{\begin{tabular}{p{0.5in}}#1\end{tabular}}

% Kaisse:
\newcommand{\mgmorph}[1]{|(#1)| {#1}}
\newcommand{\mgone}[2][$\times$]{\node at (#2.base) [above=2ex] (1#2) {\vphantom{X}#1};}
\newcommand{\mgtwo}[2][$\times$]{\mgone{#2} \node at (#2.base) [above=4.5ex] (2#2) {\vphantom{X}#1};}
\newcommand{\mgthree}[2][$\times$]{\mgtwo{#2} \node at (#2.base) [above=7ex] (3#2) {\vphantom{X}#1};}
\newcommand{\mgftl}[1]{\node at (1#1) [left] {(};}
\newcommand{\mgftr}[1]{\node at (1#1) [right] {)};}
\newcommand{\mgfoot}[2]{\mgftl{#1}\mgftr{#2}}
\newcommand{\mgldelim}[2]{\node at (#2.west) [left,inner sep = 0pt, outer sep = 0pt] {#1};}
\newcommand{\mgrdelim}[2]{\node at (#2.east) [right,inner sep = 0pt, outer sep = 0pt] {#1};}

\newcommand{\squish}{\hspace*{-3pt}}

% Kavitskaya:
\newcommand{\assoc}[2]{\draw (#1.south) -- (#2.north);}
\newcolumntype{L}{>{\raggedright\arraybackslash}X}

% Lepic & Padden:
\newcommand{\fitpic}[1]{\resizebox{\hsize}{!}{\includegraphics{#1}}} % from http://tex.stackexchange.com/a/148965/42880
\newcommand{\signpic}[1]{\includegraphics[width=1.36in]{#1}}
\newcolumntype{C}{>{\centering\arraybackslash}X}

% Spencer:

\newcommand{\textex}[1]{\textit{#1}\xspace}
\newcommand{\lxm}[1]{\textsc{#1}\xspace}

% Thrainsson:

\renewcommand{\textasciitilde}{\char`~} % for use with TTF/OTF fonts (see comments on http://tex.stackexchange.com/a/317/42880)
\newcommand{\tikzarrow}[2]{% for an arrow connecting two nodes
\begin{tikzpicture}[remember picture,overlay]
\draw[thick,shorten >=3pt,shorten <=3pt,->,>=stealth] (#1) -- (#2);
\end{tikzpicture}
}

\newlength{\padding}
\setlength{\padding}{0.5em}
\newcommand{\lesspadding}{\hspace*{-\padding}}
\newcommand{\feat}[1]{\lesspadding#1\lesspadding}

% Hammond

\usepackage[]{graphicx}\usepackage[]{xcolor}
%% maxwidth is the original width if it is less than linewidth
%% otherwise use linewidth (to make sure the graphics do not exceed the margin)
\makeatletter
\def\maxwidth{ %
  \ifdim\Gin@nat@width>\linewidth
    \linewidth
  \else
    \Gin@nat@width
  \fi
}
\makeatother

\definecolor{fgcolor}{rgb}{0.345, 0.345, 0.345}
\newcommand{\hlnum}[1]{\textcolor[rgb]{0.686,0.059,0.569}{#1}}%
\newcommand{\hlstr}[1]{\textcolor[rgb]{0.192,0.494,0.8}{#1}}%
\newcommand{\hlcom}[1]{\textcolor[rgb]{0.678,0.584,0.686}{\textit{#1}}}%
\newcommand{\hlopt}[1]{\textcolor[rgb]{0,0,0}{#1}}%
\newcommand{\hlstd}[1]{\textcolor[rgb]{0.345,0.345,0.345}{#1}}%
\newcommand{\hlkwa}[1]{\textcolor[rgb]{0.161,0.373,0.58}{\textbf{#1}}}%
\newcommand{\hlkwb}[1]{\textcolor[rgb]{0.69,0.353,0.396}{#1}}%
\newcommand{\hlkwc}[1]{\textcolor[rgb]{0.333,0.667,0.333}{#1}}%
\newcommand{\hlkwd}[1]{\textcolor[rgb]{0.737,0.353,0.396}{\textbf{#1}}}%
\let\hlipl\hlkwb

\usepackage{framed}
\makeatletter
\newenvironment{kframe}{%
 \def\at@end@of@kframe{}%
 \ifinner\ifhmode%
  \def\at@end@of@kframe{\end{minipage}}%
  \begin{minipage}{\columnwidth}%
 \fi\fi%
 \def\FrameCommand##1{\hskip\@totalleftmargin \hskip-\fboxsep
 \colorbox{shadecolor}{##1}\hskip-\fboxsep
     % There is no \\@totalrightmargin, so:
     \hskip-\linewidth \hskip-\@totalleftmargin \hskip\columnwidth}%
 \MakeFramed {\advance\hsize-\width
   \@totalleftmargin\z@ \linewidth\hsize
   \@setminipage}}%
 {\par\unskip\endMakeFramed%
 \at@end@of@kframe}
\makeatother

\definecolor{shadecolor}{rgb}{.97, .97, .97}
\definecolor{messagecolor}{rgb}{0, 0, 0}
\definecolor{warningcolor}{rgb}{1, 0, 1}
\definecolor{errorcolor}{rgb}{1, 0, 0}
\newenvironment{knitrout}{}{} % an empty environment to be redefined in TeX

\usepackage{alltt}

%revised version started: 12/17/16

%NEEDS: allbib.bib - already added to the master bibliography file.
%cited references only: bibexport -o mhTMP.bib main1-blx.aux
%PLUS sramh-img*, sramh.tex

%added stuff
\newcommand{\add}[1]{\textcolor{blue}{#1}}
%deleted stuff
\newcommand{\del}[1]{\textcolor{red}{(removed: #1)}}
%uncomment these to turn off colors
\renewcommand{\add}[1]{#1}
\renewcommand{\del}[1]{}

%shortcuts
\newcommand{\w}{\ili{Welsh}}
\newcommand{\e}{\ili{English}}
\newcommand{\io}{Input Optimization}




 \newcommand{\hand}{\ding{43}}
% \newcommand{\rot}[1]{\begin{rotate}{90}#1\end{rotate}} %shortcut for angled text%  
% \newcommand{\rotcon}[1]{\raisebox{-5ex}{\hspace*{1ex}\rot{\hspace*{1ex}#1}}}

%% add all extra packages you need to load to this file 
% \usepackage{todo} %% removed,cna use todonotes instead. % Jason reactivated
% \usepackage{graphicx} % not needed because forest loads tikz, which loads graphicx
\usepackage{tabularx}
\usepackage{amsmath} 
\usepackage{multicol}
\usepackage{lipsum}
\usepackage{longtable}
\usepackage{booktabs}
\usepackage[normalem]{ulem}
%\usepackage{tikz} % not needed because forest loads tikz
\usepackage{phonrule} % for SPE-style phonological rules
\usepackage{pst-all} % loads the main pstricks tools; for arrow diagrams in Hale.tex
%\usepackage{leipzig} % for gloss abbreviations
\usepackage[% for automatic cross-referencing
compress,%
capitalize,% labels are always capitalized in LSP style
noabbrev]% labels are always spelled out in LSP style
{cleveref}

% based on http://tex.stackexchange.com/a/318983/42880 for using gb4e examples with cleveref
\crefname{xnumi}{}{}
\creflabelformat{xnumi}{(#2#1#3)}
\crefrangeformat{xnumi}{(#3#1#4)--(#5#2#6)}
\crefname{xnumii}{}{}
\creflabelformat{xnumii}{(#2#1#3)}
\crefrangeformat{xnumii}{(#3#1#4)--(#5#2#6)}

%\usepackage[notcite,notref]{showkeys} %%removed, not helping CB.
%\usepackage{showidx} %%remove for final compiling - shows index keys at top of page.
 
\usepackage{langsci/styles/langsci-gb4e}  
 \usepackage{pifont}
% % OT tableaux                                                
% \usepackage{pstricks,colortab}  
\usepackage{multirow} % used in OT tableaux
\usepackage{rotating} %needed for angled text%
\usepackage{colortbl} % for cell shading
 
 \usepackage{avm}  
\usepackage[linguistics]{forest} 
\usetikzlibrary{matrix,fit} % for matrix of nodes in Kaisse and Bat-El


\usepackage{hhline}
\newcommand{\cgr}{\cellcolor[gray]{0.8}}
\newcommand{\cn}{\centering}



\newcommand{\reff}[1]{(\ref{#1})}
%\usepackage{newtxtext,newtxmath}


%\usepackage[normalem] {ulem}
\usepackage{qtree}
%\usepackage{natbib}
%\usepackage{tikz}
%\usepackage{gb4e}
\usepackage{phonrule}  
%\bibliographystyle{humannat}



\usepackage{minibox}

%\include{psheader-metr}

\def\bl#1{$_{\textrm{{\footnotesize #1}}}$}
\usepackage{arydshln}
\usepackage{rotating}

%%add all your local new commands to this file

\newcommand{\form}[1]{\mbox{\emph{#1}}}
\newcommand{\uf}[1]{\mbox{/#1/}}

% borrowed from expex and converted from plan tex to latex
\newcommand{\judge}[1]{{\upshape #1\hspace{0.1em}}}
\newcommand{\ljudge}[1]{\makebox[0pt][r]{\judge{#1}}}

\newcommand\tikzmark[1]{\tikz[remember picture, baseline=(#1.base)] \node[anchor=base,inner sep=0pt, outer sep=0pt] (#1) {#1};} % for adding decorations, arrows, lines, etc. to text
\newcommand\tikzmarknamed[2]{\tikz[remember picture, baseline=(#1.base)] \node[anchor=base,inner sep=0pt, outer sep=0pt] (#1) {#2};} % for adding decorations, arrows, lines, etc. to text
\newcommand\tikzmarkfullnamed[2]{\tikz[remember picture, baseline=(#1.base)] \node[anchor=base,inner sep=0pt, outer sep=0pt] (#1) {\vphantom{X}#2};} % for adding decorations, arrows, lines, etc. to text; this one works best for decorations above a line of text because it adds in the heigh of a capital letter and takes two arguments - one for the node name and one for the printed text

\newcommand{\sub}[1]{$_{\text{#1}}$} % for non-math subscripts
\newcommand{\subit}[1]{\sub{\textit{#1}}} % for italics non-math subscripts
\newcommand{\1}{\rlap{$'$}\xspace} % for the prime in X' (the \rlap command allows the prime to be ignored for horizontal spacing in trees, and the \xspace command allows you to use this in normal text without adding \ afterwards). This isn't crucial, but it helps the formatting to look a little better.

% Aissen:
\newcommand\tikzmarkfull[1]{\tikz[remember picture, baseline=(#1.base)] \node[anchor=base,inner sep=0pt, outer sep=0pt] (#1) {\vphantom{X}#1};} % for adding decorations, arrows, lines, etc. to text; this one works best for decorations above a line of text because it adds in the heigh of a capital letter and takes one argument that serves as the name and the printed text
\newcommand{\bridgeover}[2]{% for a line that creates a bridge over text, connecting two nodes
	\begin{tikzpicture}[remember picture,overlay]
	\draw[thick,shorten >=3pt,shorten <=3pt] (#1.north) |- +(0ex,2.5ex) -| (#2.north);
	\end{tikzpicture}
}
\newcommand{\bridgeoverht}[3]{% for a line that creates a bridge over text, connecting two nodes and specifing the height of the bridge
	\begin{tikzpicture}[remember picture,overlay]
	\draw[thick,shorten >=3pt,shorten <=3pt] (#2.north) |- +(0ex,#1) -| (#3.north);
	\end{tikzpicture}
}
\newcommand{\bridgeoverex}{\vspace*{3ex}} % place before an example that has a \bridgeover so that there is enough vertical space for it

% Chung:
\newcommand{\lefttabular}[1]{\begin{tabular}{p{0.5in}}#1\end{tabular}}

% Kaisse:
\newcommand{\mgmorph}[1]{|(#1)| {#1}}
\newcommand{\mgone}[2][$\times$]{\node at (#2.base) [above=2ex] (1#2) {\vphantom{X}#1};}
\newcommand{\mgtwo}[2][$\times$]{\mgone{#2} \node at (#2.base) [above=4.5ex] (2#2) {\vphantom{X}#1};}
\newcommand{\mgthree}[2][$\times$]{\mgtwo{#2} \node at (#2.base) [above=7ex] (3#2) {\vphantom{X}#1};}
\newcommand{\mgftl}[1]{\node at (1#1) [left] {(};}
\newcommand{\mgftr}[1]{\node at (1#1) [right] {)};}
\newcommand{\mgfoot}[2]{\mgftl{#1}\mgftr{#2}}
\newcommand{\mgldelim}[2]{\node at (#2.west) [left,inner sep = 0pt, outer sep = 0pt] {#1};}
\newcommand{\mgrdelim}[2]{\node at (#2.east) [right,inner sep = 0pt, outer sep = 0pt] {#1};}

\newcommand{\squish}{\hspace*{-3pt}}

% Kavitskaya:
\newcommand{\assoc}[2]{\draw (#1.south) -- (#2.north);}
\newcolumntype{L}{>{\raggedright\arraybackslash}X}

% Lepic & Padden:
\newcommand{\fitpic}[1]{\resizebox{\hsize}{!}{\includegraphics{#1}}} % from http://tex.stackexchange.com/a/148965/42880
\newcommand{\signpic}[1]{\includegraphics[width=1.36in]{#1}}
\newcolumntype{C}{>{\centering\arraybackslash}X}

% Spencer:

\newcommand{\textex}[1]{\textit{#1}\xspace}
\newcommand{\lxm}[1]{\textsc{#1}\xspace}

% Thrainsson:

\renewcommand{\textasciitilde}{\char`~} % for use with TTF/OTF fonts (see comments on http://tex.stackexchange.com/a/317/42880)
\newcommand{\tikzarrow}[2]{% for an arrow connecting two nodes
\begin{tikzpicture}[remember picture,overlay]
\draw[thick,shorten >=3pt,shorten <=3pt,->,>=stealth] (#1) -- (#2);
\end{tikzpicture}
}

\newlength{\padding}
\setlength{\padding}{0.5em}
\newcommand{\lesspadding}{\hspace*{-\padding}}
\newcommand{\feat}[1]{\lesspadding#1\lesspadding}

% Hammond

\usepackage[]{graphicx}\usepackage[]{xcolor}
%% maxwidth is the original width if it is less than linewidth
%% otherwise use linewidth (to make sure the graphics do not exceed the margin)
\makeatletter
\def\maxwidth{ %
  \ifdim\Gin@nat@width>\linewidth
    \linewidth
  \else
    \Gin@nat@width
  \fi
}
\makeatother

\definecolor{fgcolor}{rgb}{0.345, 0.345, 0.345}
\newcommand{\hlnum}[1]{\textcolor[rgb]{0.686,0.059,0.569}{#1}}%
\newcommand{\hlstr}[1]{\textcolor[rgb]{0.192,0.494,0.8}{#1}}%
\newcommand{\hlcom}[1]{\textcolor[rgb]{0.678,0.584,0.686}{\textit{#1}}}%
\newcommand{\hlopt}[1]{\textcolor[rgb]{0,0,0}{#1}}%
\newcommand{\hlstd}[1]{\textcolor[rgb]{0.345,0.345,0.345}{#1}}%
\newcommand{\hlkwa}[1]{\textcolor[rgb]{0.161,0.373,0.58}{\textbf{#1}}}%
\newcommand{\hlkwb}[1]{\textcolor[rgb]{0.69,0.353,0.396}{#1}}%
\newcommand{\hlkwc}[1]{\textcolor[rgb]{0.333,0.667,0.333}{#1}}%
\newcommand{\hlkwd}[1]{\textcolor[rgb]{0.737,0.353,0.396}{\textbf{#1}}}%
\let\hlipl\hlkwb

\usepackage{framed}
\makeatletter
\newenvironment{kframe}{%
 \def\at@end@of@kframe{}%
 \ifinner\ifhmode%
  \def\at@end@of@kframe{\end{minipage}}%
  \begin{minipage}{\columnwidth}%
 \fi\fi%
 \def\FrameCommand##1{\hskip\@totalleftmargin \hskip-\fboxsep
 \colorbox{shadecolor}{##1}\hskip-\fboxsep
     % There is no \\@totalrightmargin, so:
     \hskip-\linewidth \hskip-\@totalleftmargin \hskip\columnwidth}%
 \MakeFramed {\advance\hsize-\width
   \@totalleftmargin\z@ \linewidth\hsize
   \@setminipage}}%
 {\par\unskip\endMakeFramed%
 \at@end@of@kframe}
\makeatother

\definecolor{shadecolor}{rgb}{.97, .97, .97}
\definecolor{messagecolor}{rgb}{0, 0, 0}
\definecolor{warningcolor}{rgb}{1, 0, 1}
\definecolor{errorcolor}{rgb}{1, 0, 0}
\newenvironment{knitrout}{}{} % an empty environment to be redefined in TeX

\usepackage{alltt}

%revised version started: 12/17/16

%NEEDS: allbib.bib - already added to the master bibliography file.
%cited references only: bibexport -o mhTMP.bib main1-blx.aux
%PLUS sramh-img*, sramh.tex

%added stuff
\newcommand{\add}[1]{\textcolor{blue}{#1}}
%deleted stuff
\newcommand{\del}[1]{\textcolor{red}{(removed: #1)}}
%uncomment these to turn off colors
\renewcommand{\add}[1]{#1}
\renewcommand{\del}[1]{}

%shortcuts
\newcommand{\w}{\ili{Welsh}}
\newcommand{\e}{\ili{English}}
\newcommand{\io}{Input Optimization}




 \newcommand{\hand}{\ding{43}}
% \newcommand{\rot}[1]{\begin{rotate}{90}#1\end{rotate}} %shortcut for angled text%  
% \newcommand{\rotcon}[1]{\raisebox{-5ex}{\hspace*{1ex}\rot{\hspace*{1ex}#1}}}

%% add all extra packages you need to load to this file 
% \usepackage{todo} %% removed,cna use todonotes instead. % Jason reactivated
% \usepackage{graphicx} % not needed because forest loads tikz, which loads graphicx
\usepackage{tabularx}
\usepackage{amsmath} 
\usepackage{multicol}
\usepackage{lipsum}
\usepackage{longtable}
\usepackage{booktabs}
\usepackage[normalem]{ulem}
%\usepackage{tikz} % not needed because forest loads tikz
\usepackage{phonrule} % for SPE-style phonological rules
\usepackage{pst-all} % loads the main pstricks tools; for arrow diagrams in Hale.tex
%\usepackage{leipzig} % for gloss abbreviations
\usepackage[% for automatic cross-referencing
compress,%
capitalize,% labels are always capitalized in LSP style
noabbrev]% labels are always spelled out in LSP style
{cleveref}

% based on http://tex.stackexchange.com/a/318983/42880 for using gb4e examples with cleveref
\crefname{xnumi}{}{}
\creflabelformat{xnumi}{(#2#1#3)}
\crefrangeformat{xnumi}{(#3#1#4)--(#5#2#6)}
\crefname{xnumii}{}{}
\creflabelformat{xnumii}{(#2#1#3)}
\crefrangeformat{xnumii}{(#3#1#4)--(#5#2#6)}

%\usepackage[notcite,notref]{showkeys} %%removed, not helping CB.
%\usepackage{showidx} %%remove for final compiling - shows index keys at top of page.
 
\usepackage{langsci/styles/langsci-gb4e}  
 \usepackage{pifont}
% % OT tableaux                                                
% \usepackage{pstricks,colortab}  
\usepackage{multirow} % used in OT tableaux
\usepackage{rotating} %needed for angled text%
\usepackage{colortbl} % for cell shading
 
 \usepackage{avm}  
\usepackage[linguistics]{forest} 
\usetikzlibrary{matrix,fit} % for matrix of nodes in Kaisse and Bat-El


\usepackage{hhline}
\newcommand{\cgr}{\cellcolor[gray]{0.8}}
\newcommand{\cn}{\centering}



\newcommand{\reff}[1]{(\ref{#1})}
%\usepackage{newtxtext,newtxmath}


%\usepackage[normalem] {ulem}
\usepackage{qtree}
%\usepackage{natbib}
%\usepackage{tikz}
%\usepackage{gb4e}
\usepackage{phonrule}  
%\bibliographystyle{humannat}



\usepackage{minibox}

%\include{psheader-metr}

\def\bl#1{$_{\textrm{{\footnotesize #1}}}$}
\usepackage{arydshln}
\usepackage{rotating}

%%add all your local new commands to this file

\newcommand{\form}[1]{\mbox{\emph{#1}}}
\newcommand{\uf}[1]{\mbox{/#1/}}

% borrowed from expex and converted from plan tex to latex
\newcommand{\judge}[1]{{\upshape #1\hspace{0.1em}}}
\newcommand{\ljudge}[1]{\makebox[0pt][r]{\judge{#1}}}

\newcommand\tikzmark[1]{\tikz[remember picture, baseline=(#1.base)] \node[anchor=base,inner sep=0pt, outer sep=0pt] (#1) {#1};} % for adding decorations, arrows, lines, etc. to text
\newcommand\tikzmarknamed[2]{\tikz[remember picture, baseline=(#1.base)] \node[anchor=base,inner sep=0pt, outer sep=0pt] (#1) {#2};} % for adding decorations, arrows, lines, etc. to text
\newcommand\tikzmarkfullnamed[2]{\tikz[remember picture, baseline=(#1.base)] \node[anchor=base,inner sep=0pt, outer sep=0pt] (#1) {\vphantom{X}#2};} % for adding decorations, arrows, lines, etc. to text; this one works best for decorations above a line of text because it adds in the heigh of a capital letter and takes two arguments - one for the node name and one for the printed text

\newcommand{\sub}[1]{$_{\text{#1}}$} % for non-math subscripts
\newcommand{\subit}[1]{\sub{\textit{#1}}} % for italics non-math subscripts
\newcommand{\1}{\rlap{$'$}\xspace} % for the prime in X' (the \rlap command allows the prime to be ignored for horizontal spacing in trees, and the \xspace command allows you to use this in normal text without adding \ afterwards). This isn't crucial, but it helps the formatting to look a little better.

% Aissen:
\newcommand\tikzmarkfull[1]{\tikz[remember picture, baseline=(#1.base)] \node[anchor=base,inner sep=0pt, outer sep=0pt] (#1) {\vphantom{X}#1};} % for adding decorations, arrows, lines, etc. to text; this one works best for decorations above a line of text because it adds in the heigh of a capital letter and takes one argument that serves as the name and the printed text
\newcommand{\bridgeover}[2]{% for a line that creates a bridge over text, connecting two nodes
	\begin{tikzpicture}[remember picture,overlay]
	\draw[thick,shorten >=3pt,shorten <=3pt] (#1.north) |- +(0ex,2.5ex) -| (#2.north);
	\end{tikzpicture}
}
\newcommand{\bridgeoverht}[3]{% for a line that creates a bridge over text, connecting two nodes and specifing the height of the bridge
	\begin{tikzpicture}[remember picture,overlay]
	\draw[thick,shorten >=3pt,shorten <=3pt] (#2.north) |- +(0ex,#1) -| (#3.north);
	\end{tikzpicture}
}
\newcommand{\bridgeoverex}{\vspace*{3ex}} % place before an example that has a \bridgeover so that there is enough vertical space for it

% Chung:
\newcommand{\lefttabular}[1]{\begin{tabular}{p{0.5in}}#1\end{tabular}}

% Kaisse:
\newcommand{\mgmorph}[1]{|(#1)| {#1}}
\newcommand{\mgone}[2][$\times$]{\node at (#2.base) [above=2ex] (1#2) {\vphantom{X}#1};}
\newcommand{\mgtwo}[2][$\times$]{\mgone{#2} \node at (#2.base) [above=4.5ex] (2#2) {\vphantom{X}#1};}
\newcommand{\mgthree}[2][$\times$]{\mgtwo{#2} \node at (#2.base) [above=7ex] (3#2) {\vphantom{X}#1};}
\newcommand{\mgftl}[1]{\node at (1#1) [left] {(};}
\newcommand{\mgftr}[1]{\node at (1#1) [right] {)};}
\newcommand{\mgfoot}[2]{\mgftl{#1}\mgftr{#2}}
\newcommand{\mgldelim}[2]{\node at (#2.west) [left,inner sep = 0pt, outer sep = 0pt] {#1};}
\newcommand{\mgrdelim}[2]{\node at (#2.east) [right,inner sep = 0pt, outer sep = 0pt] {#1};}

\newcommand{\squish}{\hspace*{-3pt}}

% Kavitskaya:
\newcommand{\assoc}[2]{\draw (#1.south) -- (#2.north);}
\newcolumntype{L}{>{\raggedright\arraybackslash}X}

% Lepic & Padden:
\newcommand{\fitpic}[1]{\resizebox{\hsize}{!}{\includegraphics{#1}}} % from http://tex.stackexchange.com/a/148965/42880
\newcommand{\signpic}[1]{\includegraphics[width=1.36in]{#1}}
\newcolumntype{C}{>{\centering\arraybackslash}X}

% Spencer:

\newcommand{\textex}[1]{\textit{#1}\xspace}
\newcommand{\lxm}[1]{\textsc{#1}\xspace}

% Thrainsson:

\renewcommand{\textasciitilde}{\char`~} % for use with TTF/OTF fonts (see comments on http://tex.stackexchange.com/a/317/42880)
\newcommand{\tikzarrow}[2]{% for an arrow connecting two nodes
\begin{tikzpicture}[remember picture,overlay]
\draw[thick,shorten >=3pt,shorten <=3pt,->,>=stealth] (#1) -- (#2);
\end{tikzpicture}
}

\newlength{\padding}
\setlength{\padding}{0.5em}
\newcommand{\lesspadding}{\hspace*{-\padding}}
\newcommand{\feat}[1]{\lesspadding#1\lesspadding}

% Hammond

\usepackage[]{graphicx}\usepackage[]{xcolor}
%% maxwidth is the original width if it is less than linewidth
%% otherwise use linewidth (to make sure the graphics do not exceed the margin)
\makeatletter
\def\maxwidth{ %
  \ifdim\Gin@nat@width>\linewidth
    \linewidth
  \else
    \Gin@nat@width
  \fi
}
\makeatother

\definecolor{fgcolor}{rgb}{0.345, 0.345, 0.345}
\newcommand{\hlnum}[1]{\textcolor[rgb]{0.686,0.059,0.569}{#1}}%
\newcommand{\hlstr}[1]{\textcolor[rgb]{0.192,0.494,0.8}{#1}}%
\newcommand{\hlcom}[1]{\textcolor[rgb]{0.678,0.584,0.686}{\textit{#1}}}%
\newcommand{\hlopt}[1]{\textcolor[rgb]{0,0,0}{#1}}%
\newcommand{\hlstd}[1]{\textcolor[rgb]{0.345,0.345,0.345}{#1}}%
\newcommand{\hlkwa}[1]{\textcolor[rgb]{0.161,0.373,0.58}{\textbf{#1}}}%
\newcommand{\hlkwb}[1]{\textcolor[rgb]{0.69,0.353,0.396}{#1}}%
\newcommand{\hlkwc}[1]{\textcolor[rgb]{0.333,0.667,0.333}{#1}}%
\newcommand{\hlkwd}[1]{\textcolor[rgb]{0.737,0.353,0.396}{\textbf{#1}}}%
\let\hlipl\hlkwb

\usepackage{framed}
\makeatletter
\newenvironment{kframe}{%
 \def\at@end@of@kframe{}%
 \ifinner\ifhmode%
  \def\at@end@of@kframe{\end{minipage}}%
  \begin{minipage}{\columnwidth}%
 \fi\fi%
 \def\FrameCommand##1{\hskip\@totalleftmargin \hskip-\fboxsep
 \colorbox{shadecolor}{##1}\hskip-\fboxsep
     % There is no \\@totalrightmargin, so:
     \hskip-\linewidth \hskip-\@totalleftmargin \hskip\columnwidth}%
 \MakeFramed {\advance\hsize-\width
   \@totalleftmargin\z@ \linewidth\hsize
   \@setminipage}}%
 {\par\unskip\endMakeFramed%
 \at@end@of@kframe}
\makeatother

\definecolor{shadecolor}{rgb}{.97, .97, .97}
\definecolor{messagecolor}{rgb}{0, 0, 0}
\definecolor{warningcolor}{rgb}{1, 0, 1}
\definecolor{errorcolor}{rgb}{1, 0, 0}
\newenvironment{knitrout}{}{} % an empty environment to be redefined in TeX

\usepackage{alltt}

%revised version started: 12/17/16

%NEEDS: allbib.bib - already added to the master bibliography file.
%cited references only: bibexport -o mhTMP.bib main1-blx.aux
%PLUS sramh-img*, sramh.tex

%added stuff
\newcommand{\add}[1]{\textcolor{blue}{#1}}
%deleted stuff
\newcommand{\del}[1]{\textcolor{red}{(removed: #1)}}
%uncomment these to turn off colors
\renewcommand{\add}[1]{#1}
\renewcommand{\del}[1]{}

%shortcuts
\newcommand{\w}{\ili{Welsh}}
\newcommand{\e}{\ili{English}}
\newcommand{\io}{Input Optimization}




 \newcommand{\hand}{\ding{43}}
% \newcommand{\rot}[1]{\begin{rotate}{90}#1\end{rotate}} %shortcut for angled text%  
% \newcommand{\rotcon}[1]{\raisebox{-5ex}{\hspace*{1ex}\rot{\hspace*{1ex}#1}}}

%\input{localpackages.tex}
\usepackage{arydshln}
\usepackage{rotating}

%\input{localcommands.tex}
\newcommand{\tworow}[1]{\multirow{2}{*}{#1}}


\newcommand{\tworow}[1]{\multirow{2}{*}{#1}}


\newcommand{\tworow}[1]{\multirow{2}{*}{#1}}



\title{Compensatory lengthening and structure preservation revisited yet again}

\author{%
Darya Kavitskaya\affiliation{University of California, Berkeley}
}

% \chapterDOI{} %will be filled in at production
% \epigram{}

\abstract{
In their seminal paper, \citet{deChene1979} make a strong claim that pre-existing vowel length contrast is a necessary condition for the phonologization of vowel length through compensatory lengthening. Compensatory lengthening is thus predicted to be always a structure-preserving change. Since that time, the claim has been challenged in numerous works (\citealt{gess1998,hock1986k,morin1992}, among others). A closer examination of the cited counterexamples to the de Chene and Anderson’s claim reveals certain generalizations. Some apparent counterexamples, such as Samothraki Greek \citep{kiparsky2011k}, involve the full vocalization stage of the consonant with the subsequent coalescence of that consonant with the preceding vowel. In other cases, such as Old French \citep{gess1998} and Komi Ižma \citep{hausenberg1998}, heterosyllabic or heteromorphemic identical vowel sequences are attested elsewhere in the language. The former cases involve the reanalysis of vowel length before weakened consonants that is indeed strengthened by the independent existence of the vowel length contrast in the languages in question, in support of de Chene and Anderson’s claim. The latter cases are not truly compensatory, and phonemic vowel length is introduced into the language through coalescence.
}

\begin{document}
\maketitle



\section{Introduction}

In their seminal paper on compensatory lengthening, \citet{deChene1979} make a strong claim that the independent existence of a
vowel length contrast is a necessary condition for the phonologization
of vowel length through compensatory lengthening. Compensatory
lengthening is thus predicted to be always a structure-preserving change
that cannot introduce contrastive vowel length into a language. Since
that time, the generalization in its stronger version (certain sound
changes are always structure preserving) or in its weaker version
(structure preservation is a tendency in sound change) has been accepted
and developed by linguists otherwise advocating very diverse and
sometimes incompatible approaches to sound change, in particular, in
research programs by Paul Kiparsky \citep{kiparsky1995,kiparsky2003} and Juliette
Blevins \citep{blevins2004a,blevins2009k}. However, the generalization has also been
challenged in several works. For instance, \citet{gess1998} takes issue with
de Chene and Anderson's claim, suggesting that in general, ``structure
preservation is irrelevant as a theoretical construct'' and proceeds to
argue that de Chene and Anderson's interpretation of the Old French
data, which is their main example, is incorrect, and that in Old French
compensatory lengthening happened before the introduction of the other
sources of length distinction into the language, contrary to de Chene
and Anderson's analysis. Compensatory lengthening through onset loss,
such as in Samothraki Greek \citep{topintzi2006k,kiparsky2011k,katsika2015}, is also a potential counterexample to the claim that
CL is a structure-preserving change, along with the case of Occitan
\citep{morin1992}. In other languages without pre-existing vowel length
contrast, such as Andalusian Spanish \citep{hock1986k}, Ilokano \citep{hayes1989k}
and the Ngajan dialect of Dyirbal \citep{dixon1990}, vowel length that is the
result of CL remains allophonic and predictable. In yet another type of
cases, such as Komi Ižma \citep{harms1967,harms1968,hausenberg1998}, vowel
length from CL appears to be quasi-phonemic and on its way to
phonologization.

Compensatory lengthening is a common sound change that occurred
independently in many languages of the world, and only a few potential
counterexamples to de Chene and Anderson's claim have been reported. In
principle, we could have been done simply restating this observation
that supports a weaker but less controversial claim that there is a
tendency for compensatory lengthening to occur in languages with
pre-existing vowel length, in the spirit of proposals about
structure-preserving sound change by either \citet{kiparsky2003} or \citet{blevins2009k}, but we will proceed to examining the most widely discussed
counterexamples to de Chene and Anderson's claim. A closer examination
of these counterexamples reveals certain generalizations. The working
analyses of some cases proposed in the literature involve the full
vocalization stage of the consonant with the subsequent coalescence with
the preceding vowel, such as in Samothraki Greek \citep{sumner1999,kiparsky2011k}. In other cases, such as Old French \citep{gess1998} and Komi Ižma
\citep{hausenberg1998}, heterosyllabic or heteromorphemic long vowels (or
rather vowel sequences) are attested elsewhere in the language. We shall
argue that the cases of compensatory lengthening that do not involve
full vocalization \citep{hayes1989k,kavitskaya2002} are in a sense truly
compensatory, as opposed to instances of consonant vocalization and
subsequent vowel coalescence. The former cases involve the reanalysis of
vowel length before weakened consonants that is indeed strengthened by
the independent existence of the vowel length contrast in the languages
in question. In the latter cases, phonemic vowel length is introduced
into the language through coalescence.

\section{The problem}

Compensatory lengthening (CL) through consonant loss is defined as a
process whereby a vowel lengthens in compensation for the loss of a
tautosyllabic consonant. CL through coda loss is the most typologically
wide spread process, as either a diachronic change or a synchronic
alternation.\footnote{We will not address CL through vowel loss in this
  paper.} An example of this kind of CL in the Ižma dialect of Komi (a
Uralic language of the Permian subgroup) is shown in
\cref{tab:codalosski} \citep{harms1967,harms1968,deChene1979}:

\begin{table}
\begin{tabular}{lllll}
	\lsptoprule
	&  Stem & Past \textsc{1sg} & Infinitive & \\
	\midrule
	a. & lɨj- & lɨj-i & lɨj-nɨ & `shoot' \\
	& mun- & mun-i & mun-nɨ & `go' \\
	b. & kɨl- & kɨl-i & kɨː-nɨ & `hear' \\
	& sulal- & sulal-i & suloː-nɨ & `stand'\\
	\lspbottomrule
\end{tabular}
\caption{CL through coda loss in Komi Ižma \citep[after][104]{harms1968}.}
\label{tab:codalosski}
\end{table}

In Komi Ižma, the lateral /l/ deletes in the coda position with the
lengthening of the preceding vowel, as illustrated in (b) of \cref{tab:codalosski}.\footnote{The
  lengthened /a/ surfaces as {[}oː{]}.} DeChene \& Anderson \citeyearpar{deChene1979}
propose that CL through consonant loss should be analyzed as an instance
of the conversion of coda consonants, /l/ in the case of Komi Ižma, to
glides (either semivocalic or laryngeal), /w/ in the case of Komi Ižma,
with the subsequent monophthongization of the resulting vowel-glide
sequence in the syllable nucleus, as in, for example, *kɨl.nɨ
\textgreater{} *kɨw.nɨ \textgreater{} kɨː.nɨ `to hear', with the
intermediate stage unattested in Komi Ižma but present in other dialects
of Komi, such as Vychegda Komi (Lytkin 1966, Lytkin and Teplyashina
1976).

De Chene and Anderson \citeyearpar[508]{deChene1979} emphasize that their account is
phonetic in nature and accounts for CL as a historical sound change, and
not as a synchronic alternation:

\begin{quote}
We will argue that these processes can be understood as the transition
of the consonant, through loss or reduction of its occlusion, to an
eventual glide G. It is the monophthongization of the resulting sequence
(X)VG(Y) which gives rise to a syllable nucleus that is interpreted as
distinctively long. In consequence, cases of apparent compensatory
lengthening can be analysed (as far as their phonetic bases are
concerned) as a combination of consonantal weakening in certain
positions followed by monophthongization; and compensatory lengthening
per se can be eliminated as an independent member of any inventory of
phonetic process-types.
\end{quote}

This insight into the phonetics of CL serves as the basis for the
analysis developed in \citet{kavitskaya2002}, who maintains that CL is the
result of the reanalysis of the longer phonetic duration of vowels as
phonological length with the loss of tautosyllabic consonants.
\citet{kavitskaya2002} maintains that vowels are more likely to be reanalyzed
as phonologically long in the environment of more sonorous consonants
after the loss of the said consonants, which makes the differences in
vowel length unpredictable. De Chene and Anderson's \citeyearpar{deChene1979} analysis of
CL as a process whereby consonants weaken to glides supports
Kavitskaya's phonetic analysis, which is shown in \cref{tab:codalossdia}:%\todo{We cross-checked this diagram against Kavitskaya (2001) and noticed that the onset of the vowel bar was not at the beginning of the first C (as shown in this MS) but more toward the midpoint of C1. I discussed this with Ryan Bennett, and he agreed that maybe it should start and end at the C midpoints in CVX and start at the C1 midpoint and end before C2 in CVY. Dasha, please check that this looks okay.}

\begin{table}
\newlength{\shiftamt}
\setlength{\shiftamt}{0.6em}
\newcommand{\shiftleft}{\hspace*{-\shiftamt}}
\begin{tabular}[t]{llll}
	\lsptoprule
	& Stage 1 & Stage 2 & Phonologization \\
	& (before consonant loss) & (consonant loss) & \\
	\midrule
	CVX & \shiftleft\begin{tikzpicture}[baseline=(C1.base)]
	\matrix[matrix of nodes,ampersand replacement=\&,column sep=0.5em]
	{|(C1)| C \& |(V)| V \& |(C2)| C \\
	};
	\draw ([yshift=-1ex]C1.south) rectangle ([yshift=-2.5ex]C2.south);
	\end{tikzpicture} & \shiftleft\begin{tikzpicture}[baseline=(C1.base)]
	\matrix[matrix of nodes,ampersand replacement=\&,column sep=0.5em]
	{|(C1)| C \& |(V)| V \& |(C2)| \phantom{C} \\
	};
	\draw ([yshift=-1ex]C1.south) rectangle ([yshift=-2.5ex]C2.south);
	\end{tikzpicture} & \shiftleft\begin{tikzpicture}[baseline=(C1.base)]
	\matrix[matrix of nodes,ampersand replacement=\&,column sep=0.5em]
	{|(C1)| C \& |(V)| Vː \\
	};
	\end{tikzpicture}\\
	CVY & \shiftleft\begin{tikzpicture}[baseline=(C1.base)]
	\matrix[matrix of nodes,ampersand replacement=\&,column sep=0.5em]
	{|(C1)| C \& |(V)| V \& |(C2)| C \\
	};
	\draw ([yshift=-1ex]C1.south) rectangle ([yshift=-2.5ex]C2.south west);
	\end{tikzpicture} & \shiftleft\begin{tikzpicture}[baseline=(C1.base)]
	\matrix[matrix of nodes,ampersand replacement=\&,column sep=0.5em]
	{|(C1)| C \& |(V)| V \& |(C2)| \phantom{C} \\
	};
	\draw ([yshift=-1ex]C1.south) rectangle ([yshift=-2.5ex]C2.south west);
	\end{tikzpicture}  \vspace*{2ex} & \shiftleft\begin{tikzpicture}[baseline=(C1.base)]
	\matrix[matrix of nodes,ampersand replacement=\&,column sep=0.5em]
	{|(C1)| C \& |(V)| V \\
	};
	\end{tikzpicture}\\
	\lspbottomrule
\end{tabular}
\caption{CL through coda loss \citep[9]{kavitskaya2002}.}
\label{tab:codalossdia}
\end{table}

The schematic representation in \cref{tab:codalossdia} considers two possible situations
where the consonants X and Y are lost. Prior to the deletion of the
consonants, both vowels are correctly analyzed as phonologically short.
In the case when the listener mishears the more sonorous consonant X as
absent, the longer transitions are reinterpreted as a part of the vowel,
which is subsequently reanalyzed as long. The vocalic transitions to the
less sonorant consonant Y are shorter, and with the loss of this
consonant, there is no reinterpretation of vowel length based on its
duration. The divide between X and Y is arbitrary, and the more sonorous
the deleting consonant is, the more likely its deletion to be
compensated by the lengthening of the vowel.

Several later accounts of CL are mostly phonological. The most
well-known of those is an account by \citet{hayes1989k}, who analyzes CL
through consonant loss as the deletion of a weight-bearing coda while
preserving its weight and reassigning it to the preceding vowel, as
illustrated in \cref{ex:codalosskitrees} for Komi Ižma. The account holds that when the
underlying coda /l/ is deleted, its mora is left behind (in an
intermediate stage) and spreads to the preceding vowel, making it
bimoraic and thus long:

\ea\label{ex:codalosskitrees}CL through coda loss in Komi Ižma (after \citealt{hayes1989k})\\
\begin{tikzpicture}[baseline=(m1.base)]
\matrix[matrix of nodes,row sep=1.25em,text depth=0.2ex,text height=1.4ex]
{      & |(s1)| σ &       &       & |(s2)| σ \\
	& |(m1)| μ & |(m2)| μ &       & |(m3)| μ \\
	|(k)| k & |(i1)| ɨ & |(l)| l & |(n)| n & |(i2)| ɨ \\
};
\assoc{s1}{k}
\assoc{s1}{m1}\assoc{m1}{i1}
\assoc{s1}{m2}\assoc{m2}{l}
\assoc{s2}{n}
\assoc{s2}{m3}\assoc{m3}{i2}
\end{tikzpicture} →
\begin{tikzpicture}[baseline=(m1.base)]
\matrix[matrix of nodes,row sep=1.25em,text depth=0.2ex,text height=1.4ex]
{      & |(s1)| σ &       &       & |(s2)| σ \\
	& |(m1)| μ & |(m2)| μ &       & |(m3)| μ \\
	|(k)| k & |(i1)| ɨ & |(l)|  & |(n)| n & |(i2)| ɨ \\
};
\assoc{s1}{k}
\assoc{s1}{m1}\assoc{m1}{i1}
\assoc{s1}{m2}
\assoc{s2}{n}
\assoc{s2}{m3}\assoc{m3}{i2}
\end{tikzpicture} →
\begin{tikzpicture}[baseline=(m1.base)]
\matrix[matrix of nodes,row sep=1.25em,text depth=0.2ex,text height=1.4ex]
{      & |(s1)| σ &       &       & |(s2)| σ \\
	& |(m1)| μ & |(m2)| μ &       & |(m3)| μ \\
	|(k)| k & |(i1)| ɨː & |(l)|  & |(n)| n & |(i2)| ɨ \\
};
\assoc{s1}{k}
\assoc{s1}{m1}\assoc{m1}{i1}
\assoc{s1}{m2}
\assoc{s2}{n}
\assoc{s2}{m3}\assoc{m3}{i2}
\draw[dashed] (m2.south) -- (i1.north);
\end{tikzpicture}
\z 

The reason for the necessity of the phonetic explanation in de Chene and
Anderson's analysis and its conspicuous absence from Hayes analysis lies
in the difference between the general approaches to CL taken by these
two accounts. De Chene and Anderson's account carefully distinguishes
between a sequence of phonetic processes that consist of the weakening
of occlusion followed by a monophthongization of the resulting
vowel-glide sequences and the phonological reinterpretation of some of
the outputs of this monophthongization as long vowels. While Hayes uses
historical examples to illustrate his points (one of the examples being
Attic Greek, where CL is arguably only a historical process with no
synchronic alternations), the account he proposes is synchronic in
nature and does not consider either phonetic or phonological stages of
the sound change analyzed by de Chene and Anderson.

One of the important predictions of de Chene and Anderson's account
concerns the systemic constraints on the phonologization of vowel
length. They propose that the phonologization of vowel lengthening as a
result of CL can happen if and only if the language in question has
a pre-existing vowel length contrast. This prediction does not follow
directly from de Chene and Anderson's analysis, nor it is necessary for
the accounts in the spirit of Hayes. It a sense, it is not a prediction
per se, but rather a generalization about the nature of CL as a sound
change. In the following sections, we will consider several
counterexamples to this claim, discuss the similarities among these
examples, and offer some speculation on why de Chene and Anderson's
generalization is at least a tendency in the languages of the world.

\section{CL with no pre-existing vowel length: Apparent counterexamples}

As can be inferred from an (admittedly small) survey of languages with
CL in \citet{kavitskaya2002}, CL is more often a structure-preserving sound
change. In the majority of the cases of CL, this tendency indeed holds:
out of 80 languages with historical CL sound changes listed in
Kavitskaya's \citeyearpar{kavitskaya2002} survey, 72 or 90\% occur in languages with
pre-existing long/short vowel contrasts, while only 8 or 10\% are found
in languages without a pre-existing vowel length contrast.\footnote{This
  information is not explicitly present in \citet{kavitskaya2002} and was
  compiled by \citet{blevins2009k}.} These 8 languages constitute
counterexamples to the stronger version of the claim, which holds that
CL as a sound change is always structure-preserving. However, first, the
presence of counterexamples does not make the tendency false (it is just
not a universal). Second, there seems to be an important difference
between the cases that are structure-preserving and the cases in which
vowel length (mostly allophonic) is potentially introduced into a
language through CL.

\subsection{Old French}

Old French is the central example used by de Chene and Anderson to
illustrate that CL as a sound change does not happen unless contrastive
vowel length is independently present in the language. According to \citet[527--528]{deChene1979}, stated after \citet[79,191]{pope1934}, the
diphthong {[}aw{]}, inherited both from Indo-European and from Vulgar
Latin, monophthongized to a short {[}o{]} in French by the middle of the
9th century, as in \cref{ex:monoawo}. The loss of other consonants,
such as velars before \emph{l} and \emph{n} and \emph{p/b} and
\emph{t/d} before \emph{p/b}, \emph{t/d}, and \emph{s}, that took place
at approximately the same time, was not accompanied by CL either, as
exemplified in \cref{ex:ofcloss}:

\ea\label{ex:oldfrench1}
	\ea\label{ex:monoawo}Monophthongization of {[}aw{]} to {[}o{]} in Old French (circa 850 AD)\\
	\form{or} \textless{} \form{aurum} `gold'\\
	\form{oser} \textless{} \form{ausare} `to dare'\\
	\form{forge} `forge' \textless{} \form{*faurga} \textless{} \form{fabrica} `workshop'\\
	\form{parole} \textless{} \form{paraula} \textless{} \form{parab(o)la} `word'
	
	\ex\label{ex:ofcloss}Loss of consonants g, k, p, and d in Old French (before 850 AD)\\
	\form{agneau} {[}aɲo{]} \textless{} \form{agnellum} `lamb'\\
	\form{maille} {[}may{]} \textless{} \form{mac(u)lam} `stain'\\
	\form{route} `road' \textless{} \form{(via)} \form{rupta} `broken road'\\
	\form{après} `after' \textless{} \form{adpressum} `near'
	\z
\z

Another wave of  monophthongization, presumably through the weakening
of the coda {[}l{]} to a labiovelar glide, happened in Old French by the
16th century, this time resulting in a long {[}oː{]},
as illustrated in \cref{ex:monoalawo}. The loss of other pre-consonantal consonants,
such as nasals (complete by the middle of the 16th
century) and fricatives \emph{s} and \emph{z} (earlier), was accompanied
by CL. Examples of the loss of fricatives are in \cref{ex:offricloss}, and CL through
the loss of nasals is illustrated in \cref{ex:ofnasloss}.

The examples in \cref{ex:offricloss} show the orthographic \emph{s} that was preserved
in such words until 1740 \citep[520]{deChene1979}. However,
\citet[151]{pope1934} mentions that 12th century poetry suggests that the
fricative had begun to drop before voiced consonants by this period.
French loanwords in English, such as \emph{blame}, \emph{male}, and
\emph{isle}, do not have a pronounced {[}s{]}, which adds more support
to this conclusion:

\ea\label{ex:oldfrench2}
	\ea\label{ex:monoalawo}Monophthongization of {[}aw{]} \textless{} {[}al{]} to {[}oː{]} in Old French (16th century)\\
	\form{autre} {[}o:tr{]} \textless{} \form{alterum} `other'\\
	\form{aube} {[}o:b{]} `dawn' \textless{} \form{alba} `white' (fem.sg.)
	
	\ex\label{ex:offricloss}Loss of fricatives with CL in Old French (12th century)\\
	\form{mȇler} (ModFrench) \textless{} \form{mesler} (Old French) \textless{} \form{misculāre} `to mix'\\
	\form{île} (ModFrench) \textless{} \form{isle} (Old French) \textless{} \form{insula} `island'
	
	\ex\label{ex:ofnasloss}Loss of nasals with CL in Old French (16th century)\\
	\form{fendre} {[}fãːdr{]} \textless{} \form{findere} `to split'\\
	\form{rompre} {[}rɔ̃ːpr{]} \textless{} \form{rumpere} `to break'
	\z
\z

De Chene and Anderson claim that the difference between the outcomes of
the two Old French monophthongizations, as well as between the loss of
consonants without and with vowel lengthening, lies in the fact that in
the 9th century, the Old French vowel system did not
exhibit contrastive vowel length, and so vowels did not lengthen as a
response to the loss of consonants, while by some time in the
12th to 16th century a vowel length
distinction was introduced into the system independently, and this
pre-existing vowel length contrast made it possible for the reanalysis
of the vowels that preceded the lost consonants as long.

It is argued in \citet{deChene1985} that languages typically acquire vowel
length through vowel coalescence (dubbed as vowel hiatus or geminate
vowel clusters \citep[520]{deChene1979}). De Chene and
Anderson \citeyearpar{deChene1979} state that French obeys this rule and acquires long
vowels through the deletion of intervocalic consonants and subsequent
vowel coalescence in the period between the changes exemplified in \cref{ex:oldfrench1}
and \cref{ex:oldfrench2}. The examples of consonant loss and vowel coalescence are
presented in \cref{tab:VCVloss}.

\begin{table}
\begin{tabular}{llll}
\lsptoprule
\multicolumn{1}{c}{Modern French} & \multicolumn{1}{c}{Old French} & \multicolumn{1}{c}{Latin} & \\
\midrule
bâiller	& baailler 	& bataculare 	& `to yawn'\\
		& graal 	& gradalem 	& `dish' \\
		& aates 	& adaptas 	& `suitable' (fem.acc.pl)\\
sceau 	& seel 	& sigillum 		& `seal'\\
\lspbottomrule
\end{tabular}
\caption{Intervocalic consonant loss between identical vowels (\citealt{deChene1979} after \citealt{pope1934}).}
\label{tab:VCVloss}
\end{table}

\citet{gess1998} takes issue with de Chene and Anderson's claim that CL is
only possible in languages with a preexisting vowel length contrast. He
argues with the claim on the basis of the evidence from Old French. The
objection is that the putative long vowels are treated as disyllabic in
12th and 13th century poetry in Old
French, as shown in \cref{ex:octosyll} for one of the examples in \cref{tab:VCVloss}. From the
scansion of the octosyllabic line in \cref{ex:octosyll}, it is evident that
\emph{graal} `dish' consists of two syllables for the purposes of poetic
syllabification:

\ea\label{ex:octosyll}Le Roman de Perceval, late 12th century (\citealt[3, 11, 76--77]{roach1959}; \citealt[357]{gess1998})\\
%\tikzmark{Ce} \tikzmark{est} \tikzmark{le} \tikzmark{con}\tikzmark{tes} \tikzmark{de} \tikzmarknamed{graal}{\textsc{graal}},\\\vspace*{\baselineskip}
\begin{tikzpicture}
\matrix[matrix of nodes,column sep=0pt,row sep=1ex,inner sep=0pt,outer sep=0pt]
{Ce & { } & est & { } & le & { } & con & tes & { } & de & { } & \textsc{graal} & , \\
	1 & & 2 & & 3 & & 4 & 5 & & 6 & & 7 \ 8 & \\
};
\end{tikzpicture}\\
`This is the story of the Grail,'\\
Dont li quens li bailla le livre.\\
`About which the count gave him the book.'
\begin{tikzpicture}[remember picture,overlay]
%\node at (Ce) [below] {1};
%\node at (est) [below] {2};
%\node at (le) [below] {3};
%\node at (con) [below] {4};
%\node at (tes) [below] {5};
%\node at (de) [below] {6};
%\node[xshift=-1ex] at (graal) [below] {7};
%\node[xshift=1ex] at (graal) [below] {8};
\end{tikzpicture}
\z

The evidence from the scansion provided by \citet{gess1998} is questionable
since syllabification in poetry is often conservative and reflects an
earlier stage of the language. The scansion is also consistent with the
possibility that vowel coalescence has already happened and long vowels
scan as two syllables, with a poetic line becoming mora-counting rather
than syllable-counting (see discussion in \citet[76, 84ff]{deChene1985}
about such developments in Japanese and Tongan).\footnote{I am grateful
  to a reviewer for the discussion of this point.} If this is the case,
then Old French does not constitute a counterexample to de Chene and
Anderson's generalization.

The metrical evidence presented by \citet{gess1998} is thus inconclusive.
However, even if Gess' interpretation is correct and indeed his examples
illustrate that at the time of CL in Old French there were
heterosyllabic sequences of identical vowels, it could have been
sufficient to strengthen the possibility of CL, as we shall further
discuss for another example in the next section.

\subsection{Komi Ižma}

It would be informative now to return to Komi Ižma, which does not have
contrastive long vowels in the inventory, or any other allophonically
long vowels, except for those that are derived by CL \citep{lytkin1966,lytkin1976,hausenberg1998}. Thus, in principle, Komi
Ižma constitutes a counterexample to de Chene and Anderson's claim
interpreted broadly, as noticed by \citet{gess1998}. The forms in \cref{tab:clcodaloski}
illustrate CL alternations in Komi Ižma:

\begin{table}
\begin{tabular}{llll}
\lsptoprule
Stem	&	Past 1\textsc{sg}	&	Infinitive		&	\\
\midrule
kɨl- 		&	kɨli 			&	kɨːnɨ 			&	`to hear' \\
sulal- 		&	sulali 			&	suloːnɨ 			&	`to stand'\\
\midrule
Indefinite & 	Definite		&	Dative		&	\\
\midrule
pi 			&	pijɨs 			&	pilɨ 			&	`son'\\
piː 			&	pilɨs 			&	piːlɨ 			&	`cloud'\\
vɘː 			&	vɘlɨs 			&	vɘːlɨ 			&	`horse'\\
\lspbottomrule
\end{tabular}
\caption{CL through coda loss in Komi Ižma (after \citealt[104--105]{harms1968}).}
\label{tab:clcodaloski}
\end{table}

The deletion of \emph{l} in Komi Ižma went through the stage of the loss
of the occlusion of the liquid to the labiovelar glide \emph{w},
followed by the monophthongization of the \emph{Vw} sequence.\footnote{Syllable-final
  /l/ frequently undergoes vocalization; cf., for instance,
  \emph{l}-vocalization in BCS (South Slavic): beo /bel/
  `white-\textsc{masc'} (vs.\ bela `white-\textsc{fem'}), video /videl/
  `see-\textsc{past.masc}' (vs.\ videla `see-\textsc{past-fem}').} The
diphthongal stage is synchronically attested in related dialects of
Komi, spoken in Vychegda and Syktyvkar, and there is also a dialect
group in Komi that preserves the lateral (cf. vɘː /vɘl/ `horse' in Komi
Ižma vs.\ vɘv /vɘl/ `horse' in Vychegda Komi vs.\ vɘl /vɘl/ `horse' in
Komi Yazva) (\citealt[44--49]{lytkin1966}, \citealt[106--115]{lytkin1976}).\footnote{Yet another dialect of Komi, Komi Inva,
  vocalizes /l/ into {[}w{]} in all positions \citep[44--49]{lytkin1966}.}

De Chene and Anderson \citeyearpar{deChene1979} maintain that Komi Ižma data do not
counterexemplify their generalization since the language has
heteromorphemic long vowels (or vowel sequences), so sufficient contrast
in vowel duration is present for CL to go through. \citet[309]{hausenberg1998} states that in dialects long vowels may develop through
assimilation in forms like \emph{una-an} \emph{\textless{} una-ën}
`many', \emph{baba-as} \emph{\textless{} baba-ïs} `his wife'.

In a narrower sense, Komi Ižma is not a counterexample since CL does not
introduce vowel length contrast into the language: vowel length is
allophonic and predictable, and even though `son' and `cloud' look like
a minimal pair, they are underlyingly /pi/ `son' vs.\ /pil/ `cloud'.
\citet[13]{abondolo1998} calls vowel quantity in Komi Ižma ``nascent'', thus
interpreting vowel length distinction in the language as quasi-phonemic
and possibly on its way to phonemicization. However, another view on the
facts of Komi Ižma CL is possible that provides additional evidence in
support of de Chene and Anderson's claim.\footnote{I am much indebted to
  a reviewer for the following discussion of vowel length and
  morphology.} These long vowels can arise through either inflection, as
in \cref{tab:clcodaloski}, or derivation, as in ‎\cref{ex:clkicross}:

\ea\label{ex:clkicross}CL in Komi Ižma (\citealt[309]{collinder1957} via \citealt[525]{deChene1979})\\
		\gll perna-al-as \\
		cross-\textsc{verb-3sg.pres}\\
		\glt `he hangs (\textsc{trans}), as a cross on one's breast'
\z

In an important paper that defines the place of morphology in grammar,
\citet{anderson1982} proposes that the traditional category of inflection is
the subset of morphology that is relevant to the syntax. As a
consequence, inflection depends on the results of syntactic operations
and is post-syntactic, while derivation happens before syntax. Thus,
according to Anderson's model, the units of lexical storage are stems
that ``include all internal structure of a derivational sort'' \citep[592]{anderson1982}. Endorsing this approach amounts to saying that, since long
vowels resulting from the addition of derivational material are robustly
attested in Komi Ižma, the language has lexical long vowels even if none
of them are morpheme-internal.

\subsection{Samothraki Greek}

One of the languages in which CL introduces phonemic vowel length into a
system without pre-existing vowel length contrast is a dialect of Greek
spoken on the island of Samothraki \citep{newton1972a,newton1972b,hayes1989k,katsanis1996,sumner1999,kavitskaya2002,topintzi2006k,kiparsky2011k,katsika2015}. Samothraki Greek is not a usual case
of CL in yet another respect since it is the loss of the onset, not the
coda, that triggers tautosyllabic vowel lengthening, as illustrated in
\cref{tab:clolosssamgk}.

In Samothraki Greek, the prevocalic \emph{r} deletes with the
lengthening of the following vowel in a) the absolute word-initial
position onset of either stressed or unstressed syllable, as in \cref{tab:clolosssamgk}a,
and b) after a consonant in a complex onset, both in biconsonantal
clusters, as in \cref{tab:clolosssamgk}b, and triconsonantal clusters, as in \cref{tab:clolosssamgk}c, both
in stressed and unstressed word-initial and word-medial/final syllables:

\begin{table}
\begin{tabular}{llll}
\lsptoprule
	&	Standard Greek	&	Samothraki Greek	&		\\
\midrule
a.	&	 ˈɾi.zɐ 			&	ˈiː.zɐ 			&	`root' \\
	&	ɾɛ.ˈvi.θçɐ 		&	iː.ˈvi.θçɐ 		&	`chickpeas'\\
	&	ɾɔ.ˈðɐ.ci.nɐ 	&	uː.ˈðɐ.ci.nɐ 		&	`peaches'\\[0.75ex]
b.	&	ˈvɾi.si 			&	ˈviːs 			&	`faucet'\\
	&	ˈθɾi.mi			&	 ˈθiːm 			&	`shard'\\[0.75ex]
c.	&	 ˈɐ.sprɔs 		&	ˈɐ.spuːs 			&	`white' \\
\lspbottomrule
\end{tabular}
\caption{CL through onset loss in Samothraki Greek (after \citealt{katsika2015}).\protect\footnotemark}
\label{tab:clolosssamgk}
\end{table}
\footnotetext{In Samothraki Greek, unstressed high vowels /i/ and /u/ delete unless the deletion creates phonotactically unacceptable structures. Unstressed mid vowels /ɛ/ and /ɔ/ raise to {[}i{]} and {[}u{]} \citep[79]{newton1972b}.}

The examples in \cref{tab:samgkalt} show the synchronic status of \emph{r}-deletion in
Samothraki Greek: the rhotic surfaces in the coda and as a first
consonant in a complex onset, but deletes intervocalically. On the basis
of such alternations, \citet{kiparsky2011k} argues that the presence of r-zero
alternations constitutes evidence for the synchronic status of CL in the
language:

\begin{table}
\caption{Alternations in Samothraki Greek (from \citealt[7]{katsika2015}).}
\label{tab:samgkalt}
\begin{tabular}{llll}
	\lsptoprule
  ˈçɛɾ 	&	`hand'	&	 pɔ.ˈðɐɾ 		& `foot'\\
  ˈçɛ.ɾjɐ 	&	`hands'	&	 pɔ.ˈðɐ.ɾʝɐ 	& `feet'\\
  çi.ˈu.ðʝɐ 	&	`little hands' &	pɔ.ðɐ.ˈu.ðʝɐ 	& `little feet'\\
\lspbottomrule
 \end{tabular}
\end{table}
 
However, as \citet{katsika2015} point out, there are no
synchronic alternations where the deletion of /r/ is accompanied by
vowel lengthening. In other words, there are no attested examples in
which one member of a semantically related pair has a surface {[}ɾ{]},
while the other exhibits a long vowel as a consequence of the
\emph{r}-deletion. On the basis of this, \citet{katsika2015}
conclude that it would be more accurate to analyze r-zero alternation as
a synchronic process, and CL through the loss of \emph{r} as a sound
change in Samothraki Greek.

CL through onset loss presents a problem for the theories that treat CL
as weight conservation (such as \citealt{hayes1989k}), which predict that only the
deletion of coda consonants can result in vowel lengthening. It is
generally assumed that, unlike codas, onsets cannot bear weight and do
not count as moraic.\footnote{\citet{ryan2014} presents statistical evidence
  from stress and meter showing that onsets are factors in syllable
  weight, though they are subordinate to the rhyme with respect to
  weight. For the discussion of the possibility of moraic onsets, see
  \citet{curtis2003}, \citet{davis1999k}, among others.} Several such problematic
cases, including CL through onset loss in Samothraki Greek, are
reanalyzed in \citet{hayes1989k}. Hayes extends Newton's \citeyearpar{newton1972b} idea that
\emph{rC} clusters underwent vowel epenthesis of the form VrC → VriC →
ViC and proposes that identical vowel epenthesis happened in \emph{Cr}
clusters as well, yielding CrVi → CVirVi → CViː. The deletion of the
intervocalic \emph{r} could then be followed by vowel coalescence, just
like in other VrV → Vː cases in Samothraki Greek.

However, as shown by \citet{topintzi2006k}, the Samothraki Greek CL resists
such a reanalysis since the deletion of the word-initial \emph{r} cannot
be accounted for by metathesis. In addition, \citet{kiparsky2011k} claims that
Hayes' analysis is problematic because it incorrectly predicts the
merger of the outputs of the \emph{r}-deletion from CrV and VrV. While
after the loss of \emph{r}, the original *rV sequence where the vowel is
accented becomes a long vowel accented on the first mora, as in
\emph{θrími → θíim} `shard', the original *VrV sequence where the second
vowel is accented becomes a long vowel accented on the second mora, as
in \emph{xará → xaá} `joy'. However, \citet[91]{heisenberg1934} notes that if
\emph{r}-deletion results in a sequence of identical vowels with the
stress on the second vowel, the stress shifts from the second vowel to
the first one, as in /karávi/ → {[}káav{]} `ship'. \citet[79]{newton1972b}
interprets the stress shift as evidence for vowel contraction
(coalescence), while Heisenberg (1934: 90) and \citet{margariti-Rogka2011} ascertain that the vowels remain separate and belong to
different syllables in such cases.

While the moraic weight approach does not seem to account for the
Samothraki Greek CL, \citet{kavitskaya2002} proposes a phonetic/historical
account. According to \citet[99]{kavitskaya2002}, \emph{r} is vocalic enough
to be reinterpreted as additional vowel length. \citet{kiparsky2011k} argues
that neither purely phonetic models nor purely phonological (weight
conservation) models are sufficient to account for CL in Samothraki
Greek. He develops an account that relies on the observation that
\emph{r} is excluded from the onset position cross-linguistically (Zec
2007). Typologically, high sonority segments are dispreferred in the
onset, which is evident from the fact that many languages, such as
Korean, various Turkic languages, Basque, Piro, Telefol, etc., do not
allow rhotics in word-initial or syllable-initial positions \citep{deLacy2001,smith2003k} even though they have some type of \emph{r} in their
consonant inventories. Languages employ different strategies to avoid
onset rhotics, such as prothesis, deletion, fortition, anti-gemination,
and incorporation into the nucleus \citep[26]{kiparsky2011k}. Specifically, in
Samothraki Greek the prohibition on the rhotic in the onset is resolved
through the latter strategy: the rhotic is syllabified as a part of the
nucleus so that the \emph{r} and the following vowel form a rising
diphthong, and then deletes with CL. \citet{katsika2015}
develop an articulatory phonetic account of Samothraki Greek CL that
builds both on \citet{kavitskaya2002} and \citet{kiparsky2011k}. To resolve the
dispreference for the onset rhotic, the tongue tip constriction of the
\emph{r} is deleted, but the tongue body constriction is kept,
preserving some of the segmental and temporal information of the
\emph{r}. The resulting segment is highly vocalic and is subsequently
incorporated into the nucleus. Thus, Katsika and Kavitskaya's \citet{katsika2015}
account provides articulatory motivation to Kiparsky's idea that in
Samothraki, the onset \emph{r} goes through a vocalic stage followed by
the coalescence with the following vowel.

We can thus conclude that the best analysis of Samothraki Greek CL
treats it as a two-stage process, under which the vocalization of the
onset \emph{r} happens first, followed by the coalescence of the two
vocalic elements.\footnote{A reviewer points out that there are cases
  when the deletion of a coda consonant happens simultaneously with
  intervocalic deletion of the same consonant, as, for example, in
  Turkish (\citealt{deChene1979}, \citealt[23]{kavitskaya2002}). The
  reviewer suggests that this renders such examples consistent with de
  Chene and Anderson's generalization. If the same was the case in
  Samothraki Greek, and the coalescence was phonetically complete, at
  the time of \emph{r}-deletion, this, by itself, would be enough to
  exclude Samothraki Greek from potential counterexamples to de Chene
  and Anderson's generalization. I believe, however, that CL through
  onset loss in Samothraki Greek is best re-analyzed as vowel
  coalescence, in the spirit of \citet{kiparsky2011k}.}

\subsection{Towards the explanation of CL as a sound change}

From the point of view of contrast maintenance and loss, CL can be
described as the loss of contrast in a certain position. In the case of
CL through the loss of consonants, it is usually the coda consonant that
deletes with the lengthening of the tautosyllabic vowel. In the system,
where there are no phonologically long vowels, the result of this
process could in principle be the introduction of a new vowel length
contrast (the phonologization of vowel length in a narrow sense).
However, in the case of the pre-existing vowel length distinction, the
result is the introduction of the merger of the new long vowels with
existing long vowels (the phonologization of vowel length in a certain
position, in a broader sense of phonologization).

On the basis of the examples discussed above as well as other instances
of CL, we can conclude that CL as a sound change should indeed be
defined as the lengthening of the vowel after the loss of the
tautosyllabic consonant as a result of the phonological reanalysis of
the additional vowel length, either in the spirit of \citet{deChene1979} or of \citet{kavitskaya2002}. De Chene \& Anderson \citeyearpar{deChene1979}
dub this process ``monophthongization'', that is, roughly, a vowel shift
under which a monosyllabic vowel of two vowel qualities becomes a
monosyllabic vowel of one vowel quality (as exemplified by Old French
and Komi Ižma, among many others). From our admittedly incomplete survey
of CL, some kind of a pre-existing length contrast is a necessary
condition for CL in the cases where such reanalysis is involved, and no
clear cases that counterexemplify this prediction have yet been found if
this contrast is interpreted as including heterosyllabic and
heteromorphemic sequences of identical vowels. Thus, CL is best
described as phonologization in a broader sense, that is, a merger of
existing long vowels with new long vowels that are the result of CL. It
is possible that Samothraki Greek could also be reanalyzed along these
lines (as discussed in footnote 10), but, according to Kiparsky's \citeyearpar{kiparsky2011k}
account and the phonetic evidence amounted in \citet{katsika2015}, it stands out since the sound change goes through an
intermediate stage, whereby the consonant becomes a full vocalic entity.
CL in this case is a misnomer, and Samothraki Greek is really an
instance of vowel coalescence, which is a well-known and uncontroversial
source of vowel length in the languages of the world.

We can thus conclude that Samothraki Greek is not a counterexample to
the generalization because the lost consonant vocalizes completely and
then vowel coalescence happens, with the result that is reminiscent of
CL as it has the initial stage of consonant plus vowel and a final stage
of a long vowel, but is not, in fact, CL, but rather an instance of VV
\textgreater{} Vː coalescence. In turn, Old French is not a
counterexample because, even if Gess's analysis of the Old French on the
basis of the metrical scansion is correct, the presence of a sequence of
identical heterosyllabic vowels, which are likely to be phonetically
identical to long vowels, provides sufficient contrast. Finally, Komi
Ižma is not a counterexample to the most restricted version of the
generalization because the presence of a sequence of identical
heteromorphemic vowels is sufficient contrast.

\section{Sound change, mergers, splits, and contrast}

A broad question that remains to be discussed is the reason for why the
CL sound change that proceeds by gliding followed by monophthongization
\citep{deChene1979} tends to be structure-preserving, that is,
is more likely to acquire vowel length in certain positions with the
loss of the consonant if vowel length is already contrastive elsewhere
in the system?

Two distinct proposals in the literature address the question of the
relevance of structure preservation to sound change. One view on the
structure-dependence of sound change is expressed in \citet{kiparsky1995,kiparsky2003} and is to various extent present in other work by Paul Kiparsky.
Another view on structure-preserving sound change is presented in
\citet{blevins2004a} and developed in \citet{blevins2009k}. As \citet{anderson2016k}
notes, Blevins and Kiparsky advocate quite different views on the
explanation of sound change. While \citet{blevins2004a,blevins2006b} puts the main
burden of explanation of the sound change on the phonetic factors,
\citet{kiparsky2006} in a critique of Blevins' program views individual
grammars as a result of both ``what change can produce and of what the
theory of grammar allows'' \citep[17]{anderson2016k}. Interestingly, both
Blevins and Kiparsky see a place for structure preservation in the
theory of sound change, either as belonging to the grammar \citep{kiparsky1995,kiparsky2003} or emerging through acquisition \citep{blevins2004a,blevins2009k}.

\citet[328]{kiparsky2003} comments on ``the textbook story'' of
phonologization, where redundant features become phonemic with the loss
of conditioning environment (e.g., in the CL sound change, vowel length
phonologizes with the loss of the tautosyllabic consonant). However, as
\citet{kiparsky2003} points out, in many similar cases the redundant features
fail to phonologize and disappear with the loss of the conditioning
environment. Kiparsky goes on to posit a priming effect, which is a
diachronic manifestation of structure preservation, formulated as in
\cref{ex:priming}:

\ea\label{ex:priming}Priming effect in phonologization \citep[328]{kiparsky2003}\\
Redundant features are likely to be phonologized if the language's phonological representations have a class node to host them.
\z

\citet{kiparsky2003} distinguishes between the two types of sound change,
perception-based and articulation-based, and claims that while
perception-based changes, such as CL, metathesis, tonogenesis, and
assimilation, are more likely to be structure-preserving
(phonologization in a broad sense, as defined in section 3.4),
articulation-based changes, such as lenition, umlaut, etc., are usually
structure-changing (phonologization in a narrow sense, as defined in
section 3.4). Among the structure-changing processes that create long
vowels are vowel coalescence and also vowel lengthening in specific
prosodic conditions (for instance, under stress). \citet[329]{kiparsky2003}
notes that \citet[333--335]{korhonen1969} suggested that only certain
allophones have a functional load that allows for the phonemicization
with the loss of conditioning environment. \citet{korhonen1969} dubs these
allophones \emph{quasi-phonemes}. Having claimed that the classical
phoneme is ``something of a straightjacket'' when it comes to
understanding of the introduction and loss of phonological contrast,
\citet{kiparsky2013k} proposes a system, where he distinguishes between
contrastiveness, as a structural notion, and distinctiveness, as a
perceptual notion, as shown in \cref{tab:contrast}:

\begin{table}
\caption{Contrastiveness vs.\ distinctiveness \citep{kiparsky2013k}.}
\label{tab:contrast}
\begin{tabular}{lll}
\lsptoprule
& Contrastive & Non-contrastive\tabularnewline
\midrule
Distinctive & Phonemes & Quasi-phonemes\tabularnewline
Non-distinctive & Near contrasts & Allophones\tabularnewline
\lspbottomrule
\end{tabular}
\end{table}

By the system in \cref{tab:contrast}, quasi-phonemes are not contrastive, but
distinctive, and thus they represent a necessary stage to the secondary
split. Since distinctiveness is a perceptually-defined notion, only
those sound changes that are perceptually-based are predicted to follow
this pattern. As was discussed in section 3.3, vowel length in Komi Ižma
is quasi-phonemic and thus is a likely candidate for the phonologization
of vowel length in the language.

\citet{blevins2004a,blevins2009k} pursues a research agenda that is very different
from Kiparsky's theory of sound change. However, she also notes that
certain sound changes tend to be structure-preserving, and that these
changes tend to be perceptually-based. \citet{blevins2004a} posits a
principle of structural analogy, stated in \cref{ex:analogy}:

\ea\label{ex:analogy}Structural Analogy \citep[154]{blevins2004a}\\
In the course of language acquisition, the existence of a
(non-ambiguous) phonological contrast between A and B will result in
more instances of sound change involving shifts of \emph{ambiguous elements} to A or B than if no contrast between A and B existed.
\z

The consequence of such principle for sound change is a tendency towards
structure preservation. \citet{blevins2009k} presents an overview of two known
cases of sound changes that have this tendency, such as CL \citep{deChene1979,kavitskaya2002} and metathesis \citep{blevins1998,blevins2004a,blevins2004d,hume2004} and then proceeds to a case study
of the Principle of Structural Analogy, unstressed vowel syncope in
Austronesian.

According to \citet{blevins2009k}, unstressed vowel syncope in Austronesian is
a perceptually-based sound change that is the result of the ambiguous
vocalic status of hypoarticulated short unstressed vowel. The loss of
the second vowel in a CV.CV.CV sequence creates a structure CVC.CV where
the first syllable is closed. The Principle of Structural Analogy
predicts that languages that contrast open and closed syllables will
have a stronger tendency towards this kind of syncope. Indeed, as
\citet{blevins2009k} shows, the prediction is borne out.

Blevins' \citeyearpar{blevins2009k} example is different from the case of CL in an
interesting and fundamental way. While CL as a sound change amounts to
the introduction of a new allophone and potentially a new phoneme with a
positional loss of a segment, unstressed vowel syncope is the
introduction of a new prosodic structure with a positional loss of a
segment.\footnote{As a reviewer notes, length is prosodic structure as
  well, and in this sense, there is little difference between the case
  discussed by Blevins and the cases of CL. However, while CL
  (potentially) introduces a new element to the inventory of phonemes,
  Blevins discusses an example that introduces a new structure to the
  inventory of syllables.} This example provides additional support to
the generalization that the presence of contrast in the system affects
sound change that potentially creates similar structures.

\section{Conclusions}

De Chene and Anderson \citeyearpar{deChene1979} had an important insight about the
structure-preserving nature of CL that holds in the majority of the
languages with this sound change and thus cannot be ignored. We have
presented examples in which systemic considerations play an important
role in the phonologization of newly introduced phonetic detail in
perception-based sound changes, such as vowel duration in CL. We have
shown a way to address potential counterexamples to the generalization,
reanalyzing CL in Samothraki Greek as vowel coalescence and arguing that
in the cases of Old French and Komi Ižma the presence of identical
tautosyllabic vowels elsewhere in the system might have constituted a
sufficient contrast for the phonologization of vowel length through
CL.


%\section*{Abbreviations}
%\section*{Acknowledgements}

\printbibliography[heading=subbibliography,notkeyword=this]

% \todos

\end{document}

