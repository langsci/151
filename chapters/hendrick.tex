\documentclass[output=paper,
modfonts
]{LSP/langsci}


%% add all extra packages you need to load to this file 
% \usepackage{todo} %% removed,cna use todonotes instead. % Jason reactivated
% \usepackage{graphicx} % not needed because forest loads tikz, which loads graphicx
\usepackage{tabularx}
\usepackage{amsmath} 
\usepackage{multicol}
\usepackage{lipsum}
\usepackage{longtable}
\usepackage{booktabs}
\usepackage[normalem]{ulem}
%\usepackage{tikz} % not needed because forest loads tikz
\usepackage{phonrule} % for SPE-style phonological rules
\usepackage{pst-all} % loads the main pstricks tools; for arrow diagrams in Hale.tex
%\usepackage{leipzig} % for gloss abbreviations
\usepackage[% for automatic cross-referencing
compress,%
capitalize,% labels are always capitalized in LSP style
noabbrev]% labels are always spelled out in LSP style
{cleveref}

% based on http://tex.stackexchange.com/a/318983/42880 for using gb4e examples with cleveref
\crefname{xnumi}{}{}
\creflabelformat{xnumi}{(#2#1#3)}
\crefrangeformat{xnumi}{(#3#1#4)--(#5#2#6)}
\crefname{xnumii}{}{}
\creflabelformat{xnumii}{(#2#1#3)}
\crefrangeformat{xnumii}{(#3#1#4)--(#5#2#6)}

%\usepackage[notcite,notref]{showkeys} %%removed, not helping CB.
%\usepackage{showidx} %%remove for final compiling - shows index keys at top of page.
 
\usepackage{langsci/styles/langsci-gb4e}  
 \usepackage{pifont}
% % OT tableaux                                                
% \usepackage{pstricks,colortab}  
\usepackage{multirow} % used in OT tableaux
\usepackage{rotating} %needed for angled text%
\usepackage{colortbl} % for cell shading
 
 \usepackage{avm}  
\usepackage[linguistics]{forest} 
\usetikzlibrary{matrix,fit} % for matrix of nodes in Kaisse and Bat-El


\usepackage{hhline}
\newcommand{\cgr}{\cellcolor[gray]{0.8}}
\newcommand{\cn}{\centering}



\newcommand{\reff}[1]{(\ref{#1})}
%\usepackage{newtxtext,newtxmath}


%\usepackage[normalem] {ulem}
\usepackage{qtree}
%\usepackage{natbib}
%\usepackage{tikz}
%\usepackage{gb4e}
\usepackage{phonrule}  
%\bibliographystyle{humannat}



\usepackage{minibox}

%\include{psheader-metr}

\def\bl#1{$_{\textrm{{\footnotesize #1}}}$}

%%%add all your local new commands to this file

\newcommand{\form}[1]{\mbox{\emph{#1}}}
\newcommand{\uf}[1]{\mbox{/#1/}}

% borrowed from expex and converted from plan tex to latex
\newcommand{\judge}[1]{{\upshape #1\hspace{0.1em}}}
\newcommand{\ljudge}[1]{\makebox[0pt][r]{\judge{#1}}}

\newcommand\tikzmark[1]{\tikz[remember picture, baseline=(#1.base)] \node[anchor=base,inner sep=0pt, outer sep=0pt] (#1) {#1};} % for adding decorations, arrows, lines, etc. to text
\newcommand\tikzmarknamed[2]{\tikz[remember picture, baseline=(#1.base)] \node[anchor=base,inner sep=0pt, outer sep=0pt] (#1) {#2};} % for adding decorations, arrows, lines, etc. to text
\newcommand\tikzmarkfullnamed[2]{\tikz[remember picture, baseline=(#1.base)] \node[anchor=base,inner sep=0pt, outer sep=0pt] (#1) {\vphantom{X}#2};} % for adding decorations, arrows, lines, etc. to text; this one works best for decorations above a line of text because it adds in the heigh of a capital letter and takes two arguments - one for the node name and one for the printed text

\newcommand{\sub}[1]{$_{\text{#1}}$} % for non-math subscripts
\newcommand{\subit}[1]{\sub{\textit{#1}}} % for italics non-math subscripts
\newcommand{\1}{\rlap{$'$}\xspace} % for the prime in X' (the \rlap command allows the prime to be ignored for horizontal spacing in trees, and the \xspace command allows you to use this in normal text without adding \ afterwards). This isn't crucial, but it helps the formatting to look a little better.

% Aissen:
\newcommand\tikzmarkfull[1]{\tikz[remember picture, baseline=(#1.base)] \node[anchor=base,inner sep=0pt, outer sep=0pt] (#1) {\vphantom{X}#1};} % for adding decorations, arrows, lines, etc. to text; this one works best for decorations above a line of text because it adds in the heigh of a capital letter and takes one argument that serves as the name and the printed text
\newcommand{\bridgeover}[2]{% for a line that creates a bridge over text, connecting two nodes
	\begin{tikzpicture}[remember picture,overlay]
	\draw[thick,shorten >=3pt,shorten <=3pt] (#1.north) |- +(0ex,2.5ex) -| (#2.north);
	\end{tikzpicture}
}
\newcommand{\bridgeoverht}[3]{% for a line that creates a bridge over text, connecting two nodes and specifing the height of the bridge
	\begin{tikzpicture}[remember picture,overlay]
	\draw[thick,shorten >=3pt,shorten <=3pt] (#2.north) |- +(0ex,#1) -| (#3.north);
	\end{tikzpicture}
}
\newcommand{\bridgeoverex}{\vspace*{3ex}} % place before an example that has a \bridgeover so that there is enough vertical space for it

% Chung:
\newcommand{\lefttabular}[1]{\begin{tabular}{p{0.5in}}#1\end{tabular}}

% Kaisse:
\newcommand{\mgmorph}[1]{|(#1)| {#1}}
\newcommand{\mgone}[2][$\times$]{\node at (#2.base) [above=2ex] (1#2) {\vphantom{X}#1};}
\newcommand{\mgtwo}[2][$\times$]{\mgone{#2} \node at (#2.base) [above=4.5ex] (2#2) {\vphantom{X}#1};}
\newcommand{\mgthree}[2][$\times$]{\mgtwo{#2} \node at (#2.base) [above=7ex] (3#2) {\vphantom{X}#1};}
\newcommand{\mgftl}[1]{\node at (1#1) [left] {(};}
\newcommand{\mgftr}[1]{\node at (1#1) [right] {)};}
\newcommand{\mgfoot}[2]{\mgftl{#1}\mgftr{#2}}
\newcommand{\mgldelim}[2]{\node at (#2.west) [left,inner sep = 0pt, outer sep = 0pt] {#1};}
\newcommand{\mgrdelim}[2]{\node at (#2.east) [right,inner sep = 0pt, outer sep = 0pt] {#1};}

\newcommand{\squish}{\hspace*{-3pt}}

% Kavitskaya:
\newcommand{\assoc}[2]{\draw (#1.south) -- (#2.north);}
\newcolumntype{L}{>{\raggedright\arraybackslash}X}

% Lepic & Padden:
\newcommand{\fitpic}[1]{\resizebox{\hsize}{!}{\includegraphics{#1}}} % from http://tex.stackexchange.com/a/148965/42880
\newcommand{\signpic}[1]{\includegraphics[width=1.36in]{#1}}
\newcolumntype{C}{>{\centering\arraybackslash}X}

% Spencer:

\newcommand{\textex}[1]{\textit{#1}\xspace}
\newcommand{\lxm}[1]{\textsc{#1}\xspace}

% Thrainsson:

\renewcommand{\textasciitilde}{\char`~} % for use with TTF/OTF fonts (see comments on http://tex.stackexchange.com/a/317/42880)
\newcommand{\tikzarrow}[2]{% for an arrow connecting two nodes
\begin{tikzpicture}[remember picture,overlay]
\draw[thick,shorten >=3pt,shorten <=3pt,->,>=stealth] (#1) -- (#2);
\end{tikzpicture}
}

\newlength{\padding}
\setlength{\padding}{0.5em}
\newcommand{\lesspadding}{\hspace*{-\padding}}
\newcommand{\feat}[1]{\lesspadding#1\lesspadding}

% Hammond

\usepackage[]{graphicx}\usepackage[]{xcolor}
%% maxwidth is the original width if it is less than linewidth
%% otherwise use linewidth (to make sure the graphics do not exceed the margin)
\makeatletter
\def\maxwidth{ %
  \ifdim\Gin@nat@width>\linewidth
    \linewidth
  \else
    \Gin@nat@width
  \fi
}
\makeatother

\definecolor{fgcolor}{rgb}{0.345, 0.345, 0.345}
\newcommand{\hlnum}[1]{\textcolor[rgb]{0.686,0.059,0.569}{#1}}%
\newcommand{\hlstr}[1]{\textcolor[rgb]{0.192,0.494,0.8}{#1}}%
\newcommand{\hlcom}[1]{\textcolor[rgb]{0.678,0.584,0.686}{\textit{#1}}}%
\newcommand{\hlopt}[1]{\textcolor[rgb]{0,0,0}{#1}}%
\newcommand{\hlstd}[1]{\textcolor[rgb]{0.345,0.345,0.345}{#1}}%
\newcommand{\hlkwa}[1]{\textcolor[rgb]{0.161,0.373,0.58}{\textbf{#1}}}%
\newcommand{\hlkwb}[1]{\textcolor[rgb]{0.69,0.353,0.396}{#1}}%
\newcommand{\hlkwc}[1]{\textcolor[rgb]{0.333,0.667,0.333}{#1}}%
\newcommand{\hlkwd}[1]{\textcolor[rgb]{0.737,0.353,0.396}{\textbf{#1}}}%
\let\hlipl\hlkwb

\usepackage{framed}
\makeatletter
\newenvironment{kframe}{%
 \def\at@end@of@kframe{}%
 \ifinner\ifhmode%
  \def\at@end@of@kframe{\end{minipage}}%
  \begin{minipage}{\columnwidth}%
 \fi\fi%
 \def\FrameCommand##1{\hskip\@totalleftmargin \hskip-\fboxsep
 \colorbox{shadecolor}{##1}\hskip-\fboxsep
     % There is no \\@totalrightmargin, so:
     \hskip-\linewidth \hskip-\@totalleftmargin \hskip\columnwidth}%
 \MakeFramed {\advance\hsize-\width
   \@totalleftmargin\z@ \linewidth\hsize
   \@setminipage}}%
 {\par\unskip\endMakeFramed%
 \at@end@of@kframe}
\makeatother

\definecolor{shadecolor}{rgb}{.97, .97, .97}
\definecolor{messagecolor}{rgb}{0, 0, 0}
\definecolor{warningcolor}{rgb}{1, 0, 1}
\definecolor{errorcolor}{rgb}{1, 0, 0}
\newenvironment{knitrout}{}{} % an empty environment to be redefined in TeX

\usepackage{alltt}

%revised version started: 12/17/16

%NEEDS: allbib.bib - already added to the master bibliography file.
%cited references only: bibexport -o mhTMP.bib main1-blx.aux
%PLUS sramh-img*, sramh.tex

%added stuff
\newcommand{\add}[1]{\textcolor{blue}{#1}}
%deleted stuff
\newcommand{\del}[1]{\textcolor{red}{(removed: #1)}}
%uncomment these to turn off colors
\renewcommand{\add}[1]{#1}
\renewcommand{\del}[1]{}

%shortcuts
\newcommand{\w}{\ili{Welsh}}
\newcommand{\e}{\ili{English}}
\newcommand{\io}{Input Optimization}




 \newcommand{\hand}{\ding{43}}
% \newcommand{\rot}[1]{\begin{rotate}{90}#1\end{rotate}} %shortcut for angled text%  
% \newcommand{\rotcon}[1]{\raisebox{-5ex}{\hspace*{1ex}\rot{\hspace*{1ex}#1}}}

%% add all extra packages you need to load to this file 
% \usepackage{todo} %% removed,cna use todonotes instead. % Jason reactivated
% \usepackage{graphicx} % not needed because forest loads tikz, which loads graphicx
\usepackage{tabularx}
\usepackage{amsmath} 
\usepackage{multicol}
\usepackage{lipsum}
\usepackage{longtable}
\usepackage{booktabs}
\usepackage[normalem]{ulem}
%\usepackage{tikz} % not needed because forest loads tikz
\usepackage{phonrule} % for SPE-style phonological rules
\usepackage{pst-all} % loads the main pstricks tools; for arrow diagrams in Hale.tex
%\usepackage{leipzig} % for gloss abbreviations
\usepackage[% for automatic cross-referencing
compress,%
capitalize,% labels are always capitalized in LSP style
noabbrev]% labels are always spelled out in LSP style
{cleveref}

% based on http://tex.stackexchange.com/a/318983/42880 for using gb4e examples with cleveref
\crefname{xnumi}{}{}
\creflabelformat{xnumi}{(#2#1#3)}
\crefrangeformat{xnumi}{(#3#1#4)--(#5#2#6)}
\crefname{xnumii}{}{}
\creflabelformat{xnumii}{(#2#1#3)}
\crefrangeformat{xnumii}{(#3#1#4)--(#5#2#6)}

%\usepackage[notcite,notref]{showkeys} %%removed, not helping CB.
%\usepackage{showidx} %%remove for final compiling - shows index keys at top of page.
 
\usepackage{langsci/styles/langsci-gb4e}  
 \usepackage{pifont}
% % OT tableaux                                                
% \usepackage{pstricks,colortab}  
\usepackage{multirow} % used in OT tableaux
\usepackage{rotating} %needed for angled text%
\usepackage{colortbl} % for cell shading
 
 \usepackage{avm}  
\usepackage[linguistics]{forest} 
\usetikzlibrary{matrix,fit} % for matrix of nodes in Kaisse and Bat-El


\usepackage{hhline}
\newcommand{\cgr}{\cellcolor[gray]{0.8}}
\newcommand{\cn}{\centering}



\newcommand{\reff}[1]{(\ref{#1})}
%\usepackage{newtxtext,newtxmath}


%\usepackage[normalem] {ulem}
\usepackage{qtree}
%\usepackage{natbib}
%\usepackage{tikz}
%\usepackage{gb4e}
\usepackage{phonrule}  
%\bibliographystyle{humannat}



\usepackage{minibox}

%\include{psheader-metr}

\def\bl#1{$_{\textrm{{\footnotesize #1}}}$}
\usepackage{arydshln}
\usepackage{rotating}

%%add all your local new commands to this file

\newcommand{\form}[1]{\mbox{\emph{#1}}}
\newcommand{\uf}[1]{\mbox{/#1/}}

% borrowed from expex and converted from plan tex to latex
\newcommand{\judge}[1]{{\upshape #1\hspace{0.1em}}}
\newcommand{\ljudge}[1]{\makebox[0pt][r]{\judge{#1}}}

\newcommand\tikzmark[1]{\tikz[remember picture, baseline=(#1.base)] \node[anchor=base,inner sep=0pt, outer sep=0pt] (#1) {#1};} % for adding decorations, arrows, lines, etc. to text
\newcommand\tikzmarknamed[2]{\tikz[remember picture, baseline=(#1.base)] \node[anchor=base,inner sep=0pt, outer sep=0pt] (#1) {#2};} % for adding decorations, arrows, lines, etc. to text
\newcommand\tikzmarkfullnamed[2]{\tikz[remember picture, baseline=(#1.base)] \node[anchor=base,inner sep=0pt, outer sep=0pt] (#1) {\vphantom{X}#2};} % for adding decorations, arrows, lines, etc. to text; this one works best for decorations above a line of text because it adds in the heigh of a capital letter and takes two arguments - one for the node name and one for the printed text

\newcommand{\sub}[1]{$_{\text{#1}}$} % for non-math subscripts
\newcommand{\subit}[1]{\sub{\textit{#1}}} % for italics non-math subscripts
\newcommand{\1}{\rlap{$'$}\xspace} % for the prime in X' (the \rlap command allows the prime to be ignored for horizontal spacing in trees, and the \xspace command allows you to use this in normal text without adding \ afterwards). This isn't crucial, but it helps the formatting to look a little better.

% Aissen:
\newcommand\tikzmarkfull[1]{\tikz[remember picture, baseline=(#1.base)] \node[anchor=base,inner sep=0pt, outer sep=0pt] (#1) {\vphantom{X}#1};} % for adding decorations, arrows, lines, etc. to text; this one works best for decorations above a line of text because it adds in the heigh of a capital letter and takes one argument that serves as the name and the printed text
\newcommand{\bridgeover}[2]{% for a line that creates a bridge over text, connecting two nodes
	\begin{tikzpicture}[remember picture,overlay]
	\draw[thick,shorten >=3pt,shorten <=3pt] (#1.north) |- +(0ex,2.5ex) -| (#2.north);
	\end{tikzpicture}
}
\newcommand{\bridgeoverht}[3]{% for a line that creates a bridge over text, connecting two nodes and specifing the height of the bridge
	\begin{tikzpicture}[remember picture,overlay]
	\draw[thick,shorten >=3pt,shorten <=3pt] (#2.north) |- +(0ex,#1) -| (#3.north);
	\end{tikzpicture}
}
\newcommand{\bridgeoverex}{\vspace*{3ex}} % place before an example that has a \bridgeover so that there is enough vertical space for it

% Chung:
\newcommand{\lefttabular}[1]{\begin{tabular}{p{0.5in}}#1\end{tabular}}

% Kaisse:
\newcommand{\mgmorph}[1]{|(#1)| {#1}}
\newcommand{\mgone}[2][$\times$]{\node at (#2.base) [above=2ex] (1#2) {\vphantom{X}#1};}
\newcommand{\mgtwo}[2][$\times$]{\mgone{#2} \node at (#2.base) [above=4.5ex] (2#2) {\vphantom{X}#1};}
\newcommand{\mgthree}[2][$\times$]{\mgtwo{#2} \node at (#2.base) [above=7ex] (3#2) {\vphantom{X}#1};}
\newcommand{\mgftl}[1]{\node at (1#1) [left] {(};}
\newcommand{\mgftr}[1]{\node at (1#1) [right] {)};}
\newcommand{\mgfoot}[2]{\mgftl{#1}\mgftr{#2}}
\newcommand{\mgldelim}[2]{\node at (#2.west) [left,inner sep = 0pt, outer sep = 0pt] {#1};}
\newcommand{\mgrdelim}[2]{\node at (#2.east) [right,inner sep = 0pt, outer sep = 0pt] {#1};}

\newcommand{\squish}{\hspace*{-3pt}}

% Kavitskaya:
\newcommand{\assoc}[2]{\draw (#1.south) -- (#2.north);}
\newcolumntype{L}{>{\raggedright\arraybackslash}X}

% Lepic & Padden:
\newcommand{\fitpic}[1]{\resizebox{\hsize}{!}{\includegraphics{#1}}} % from http://tex.stackexchange.com/a/148965/42880
\newcommand{\signpic}[1]{\includegraphics[width=1.36in]{#1}}
\newcolumntype{C}{>{\centering\arraybackslash}X}

% Spencer:

\newcommand{\textex}[1]{\textit{#1}\xspace}
\newcommand{\lxm}[1]{\textsc{#1}\xspace}

% Thrainsson:

\renewcommand{\textasciitilde}{\char`~} % for use with TTF/OTF fonts (see comments on http://tex.stackexchange.com/a/317/42880)
\newcommand{\tikzarrow}[2]{% for an arrow connecting two nodes
\begin{tikzpicture}[remember picture,overlay]
\draw[thick,shorten >=3pt,shorten <=3pt,->,>=stealth] (#1) -- (#2);
\end{tikzpicture}
}

\newlength{\padding}
\setlength{\padding}{0.5em}
\newcommand{\lesspadding}{\hspace*{-\padding}}
\newcommand{\feat}[1]{\lesspadding#1\lesspadding}

% Hammond

\usepackage[]{graphicx}\usepackage[]{xcolor}
%% maxwidth is the original width if it is less than linewidth
%% otherwise use linewidth (to make sure the graphics do not exceed the margin)
\makeatletter
\def\maxwidth{ %
  \ifdim\Gin@nat@width>\linewidth
    \linewidth
  \else
    \Gin@nat@width
  \fi
}
\makeatother

\definecolor{fgcolor}{rgb}{0.345, 0.345, 0.345}
\newcommand{\hlnum}[1]{\textcolor[rgb]{0.686,0.059,0.569}{#1}}%
\newcommand{\hlstr}[1]{\textcolor[rgb]{0.192,0.494,0.8}{#1}}%
\newcommand{\hlcom}[1]{\textcolor[rgb]{0.678,0.584,0.686}{\textit{#1}}}%
\newcommand{\hlopt}[1]{\textcolor[rgb]{0,0,0}{#1}}%
\newcommand{\hlstd}[1]{\textcolor[rgb]{0.345,0.345,0.345}{#1}}%
\newcommand{\hlkwa}[1]{\textcolor[rgb]{0.161,0.373,0.58}{\textbf{#1}}}%
\newcommand{\hlkwb}[1]{\textcolor[rgb]{0.69,0.353,0.396}{#1}}%
\newcommand{\hlkwc}[1]{\textcolor[rgb]{0.333,0.667,0.333}{#1}}%
\newcommand{\hlkwd}[1]{\textcolor[rgb]{0.737,0.353,0.396}{\textbf{#1}}}%
\let\hlipl\hlkwb

\usepackage{framed}
\makeatletter
\newenvironment{kframe}{%
 \def\at@end@of@kframe{}%
 \ifinner\ifhmode%
  \def\at@end@of@kframe{\end{minipage}}%
  \begin{minipage}{\columnwidth}%
 \fi\fi%
 \def\FrameCommand##1{\hskip\@totalleftmargin \hskip-\fboxsep
 \colorbox{shadecolor}{##1}\hskip-\fboxsep
     % There is no \\@totalrightmargin, so:
     \hskip-\linewidth \hskip-\@totalleftmargin \hskip\columnwidth}%
 \MakeFramed {\advance\hsize-\width
   \@totalleftmargin\z@ \linewidth\hsize
   \@setminipage}}%
 {\par\unskip\endMakeFramed%
 \at@end@of@kframe}
\makeatother

\definecolor{shadecolor}{rgb}{.97, .97, .97}
\definecolor{messagecolor}{rgb}{0, 0, 0}
\definecolor{warningcolor}{rgb}{1, 0, 1}
\definecolor{errorcolor}{rgb}{1, 0, 0}
\newenvironment{knitrout}{}{} % an empty environment to be redefined in TeX

\usepackage{alltt}

%revised version started: 12/17/16

%NEEDS: allbib.bib - already added to the master bibliography file.
%cited references only: bibexport -o mhTMP.bib main1-blx.aux
%PLUS sramh-img*, sramh.tex

%added stuff
\newcommand{\add}[1]{\textcolor{blue}{#1}}
%deleted stuff
\newcommand{\del}[1]{\textcolor{red}{(removed: #1)}}
%uncomment these to turn off colors
\renewcommand{\add}[1]{#1}
\renewcommand{\del}[1]{}

%shortcuts
\newcommand{\w}{\ili{Welsh}}
\newcommand{\e}{\ili{English}}
\newcommand{\io}{Input Optimization}




 \newcommand{\hand}{\ding{43}}
% \newcommand{\rot}[1]{\begin{rotate}{90}#1\end{rotate}} %shortcut for angled text%  
% \newcommand{\rotcon}[1]{\raisebox{-5ex}{\hspace*{1ex}\rot{\hspace*{1ex}#1}}}

%% add all extra packages you need to load to this file 
% \usepackage{todo} %% removed,cna use todonotes instead. % Jason reactivated
% \usepackage{graphicx} % not needed because forest loads tikz, which loads graphicx
\usepackage{tabularx}
\usepackage{amsmath} 
\usepackage{multicol}
\usepackage{lipsum}
\usepackage{longtable}
\usepackage{booktabs}
\usepackage[normalem]{ulem}
%\usepackage{tikz} % not needed because forest loads tikz
\usepackage{phonrule} % for SPE-style phonological rules
\usepackage{pst-all} % loads the main pstricks tools; for arrow diagrams in Hale.tex
%\usepackage{leipzig} % for gloss abbreviations
\usepackage[% for automatic cross-referencing
compress,%
capitalize,% labels are always capitalized in LSP style
noabbrev]% labels are always spelled out in LSP style
{cleveref}

% based on http://tex.stackexchange.com/a/318983/42880 for using gb4e examples with cleveref
\crefname{xnumi}{}{}
\creflabelformat{xnumi}{(#2#1#3)}
\crefrangeformat{xnumi}{(#3#1#4)--(#5#2#6)}
\crefname{xnumii}{}{}
\creflabelformat{xnumii}{(#2#1#3)}
\crefrangeformat{xnumii}{(#3#1#4)--(#5#2#6)}

%\usepackage[notcite,notref]{showkeys} %%removed, not helping CB.
%\usepackage{showidx} %%remove for final compiling - shows index keys at top of page.
 
\usepackage{langsci/styles/langsci-gb4e}  
 \usepackage{pifont}
% % OT tableaux                                                
% \usepackage{pstricks,colortab}  
\usepackage{multirow} % used in OT tableaux
\usepackage{rotating} %needed for angled text%
\usepackage{colortbl} % for cell shading
 
 \usepackage{avm}  
\usepackage[linguistics]{forest} 
\usetikzlibrary{matrix,fit} % for matrix of nodes in Kaisse and Bat-El


\usepackage{hhline}
\newcommand{\cgr}{\cellcolor[gray]{0.8}}
\newcommand{\cn}{\centering}



\newcommand{\reff}[1]{(\ref{#1})}
%\usepackage{newtxtext,newtxmath}


%\usepackage[normalem] {ulem}
\usepackage{qtree}
%\usepackage{natbib}
%\usepackage{tikz}
%\usepackage{gb4e}
\usepackage{phonrule}  
%\bibliographystyle{humannat}



\usepackage{minibox}

%\include{psheader-metr}

\def\bl#1{$_{\textrm{{\footnotesize #1}}}$}
\usepackage{arydshln}
\usepackage{rotating}

%%add all your local new commands to this file

\newcommand{\form}[1]{\mbox{\emph{#1}}}
\newcommand{\uf}[1]{\mbox{/#1/}}

% borrowed from expex and converted from plan tex to latex
\newcommand{\judge}[1]{{\upshape #1\hspace{0.1em}}}
\newcommand{\ljudge}[1]{\makebox[0pt][r]{\judge{#1}}}

\newcommand\tikzmark[1]{\tikz[remember picture, baseline=(#1.base)] \node[anchor=base,inner sep=0pt, outer sep=0pt] (#1) {#1};} % for adding decorations, arrows, lines, etc. to text
\newcommand\tikzmarknamed[2]{\tikz[remember picture, baseline=(#1.base)] \node[anchor=base,inner sep=0pt, outer sep=0pt] (#1) {#2};} % for adding decorations, arrows, lines, etc. to text
\newcommand\tikzmarkfullnamed[2]{\tikz[remember picture, baseline=(#1.base)] \node[anchor=base,inner sep=0pt, outer sep=0pt] (#1) {\vphantom{X}#2};} % for adding decorations, arrows, lines, etc. to text; this one works best for decorations above a line of text because it adds in the heigh of a capital letter and takes two arguments - one for the node name and one for the printed text

\newcommand{\sub}[1]{$_{\text{#1}}$} % for non-math subscripts
\newcommand{\subit}[1]{\sub{\textit{#1}}} % for italics non-math subscripts
\newcommand{\1}{\rlap{$'$}\xspace} % for the prime in X' (the \rlap command allows the prime to be ignored for horizontal spacing in trees, and the \xspace command allows you to use this in normal text without adding \ afterwards). This isn't crucial, but it helps the formatting to look a little better.

% Aissen:
\newcommand\tikzmarkfull[1]{\tikz[remember picture, baseline=(#1.base)] \node[anchor=base,inner sep=0pt, outer sep=0pt] (#1) {\vphantom{X}#1};} % for adding decorations, arrows, lines, etc. to text; this one works best for decorations above a line of text because it adds in the heigh of a capital letter and takes one argument that serves as the name and the printed text
\newcommand{\bridgeover}[2]{% for a line that creates a bridge over text, connecting two nodes
	\begin{tikzpicture}[remember picture,overlay]
	\draw[thick,shorten >=3pt,shorten <=3pt] (#1.north) |- +(0ex,2.5ex) -| (#2.north);
	\end{tikzpicture}
}
\newcommand{\bridgeoverht}[3]{% for a line that creates a bridge over text, connecting two nodes and specifing the height of the bridge
	\begin{tikzpicture}[remember picture,overlay]
	\draw[thick,shorten >=3pt,shorten <=3pt] (#2.north) |- +(0ex,#1) -| (#3.north);
	\end{tikzpicture}
}
\newcommand{\bridgeoverex}{\vspace*{3ex}} % place before an example that has a \bridgeover so that there is enough vertical space for it

% Chung:
\newcommand{\lefttabular}[1]{\begin{tabular}{p{0.5in}}#1\end{tabular}}

% Kaisse:
\newcommand{\mgmorph}[1]{|(#1)| {#1}}
\newcommand{\mgone}[2][$\times$]{\node at (#2.base) [above=2ex] (1#2) {\vphantom{X}#1};}
\newcommand{\mgtwo}[2][$\times$]{\mgone{#2} \node at (#2.base) [above=4.5ex] (2#2) {\vphantom{X}#1};}
\newcommand{\mgthree}[2][$\times$]{\mgtwo{#2} \node at (#2.base) [above=7ex] (3#2) {\vphantom{X}#1};}
\newcommand{\mgftl}[1]{\node at (1#1) [left] {(};}
\newcommand{\mgftr}[1]{\node at (1#1) [right] {)};}
\newcommand{\mgfoot}[2]{\mgftl{#1}\mgftr{#2}}
\newcommand{\mgldelim}[2]{\node at (#2.west) [left,inner sep = 0pt, outer sep = 0pt] {#1};}
\newcommand{\mgrdelim}[2]{\node at (#2.east) [right,inner sep = 0pt, outer sep = 0pt] {#1};}

\newcommand{\squish}{\hspace*{-3pt}}

% Kavitskaya:
\newcommand{\assoc}[2]{\draw (#1.south) -- (#2.north);}
\newcolumntype{L}{>{\raggedright\arraybackslash}X}

% Lepic & Padden:
\newcommand{\fitpic}[1]{\resizebox{\hsize}{!}{\includegraphics{#1}}} % from http://tex.stackexchange.com/a/148965/42880
\newcommand{\signpic}[1]{\includegraphics[width=1.36in]{#1}}
\newcolumntype{C}{>{\centering\arraybackslash}X}

% Spencer:

\newcommand{\textex}[1]{\textit{#1}\xspace}
\newcommand{\lxm}[1]{\textsc{#1}\xspace}

% Thrainsson:

\renewcommand{\textasciitilde}{\char`~} % for use with TTF/OTF fonts (see comments on http://tex.stackexchange.com/a/317/42880)
\newcommand{\tikzarrow}[2]{% for an arrow connecting two nodes
\begin{tikzpicture}[remember picture,overlay]
\draw[thick,shorten >=3pt,shorten <=3pt,->,>=stealth] (#1) -- (#2);
\end{tikzpicture}
}

\newlength{\padding}
\setlength{\padding}{0.5em}
\newcommand{\lesspadding}{\hspace*{-\padding}}
\newcommand{\feat}[1]{\lesspadding#1\lesspadding}

% Hammond

\usepackage[]{graphicx}\usepackage[]{xcolor}
%% maxwidth is the original width if it is less than linewidth
%% otherwise use linewidth (to make sure the graphics do not exceed the margin)
\makeatletter
\def\maxwidth{ %
  \ifdim\Gin@nat@width>\linewidth
    \linewidth
  \else
    \Gin@nat@width
  \fi
}
\makeatother

\definecolor{fgcolor}{rgb}{0.345, 0.345, 0.345}
\newcommand{\hlnum}[1]{\textcolor[rgb]{0.686,0.059,0.569}{#1}}%
\newcommand{\hlstr}[1]{\textcolor[rgb]{0.192,0.494,0.8}{#1}}%
\newcommand{\hlcom}[1]{\textcolor[rgb]{0.678,0.584,0.686}{\textit{#1}}}%
\newcommand{\hlopt}[1]{\textcolor[rgb]{0,0,0}{#1}}%
\newcommand{\hlstd}[1]{\textcolor[rgb]{0.345,0.345,0.345}{#1}}%
\newcommand{\hlkwa}[1]{\textcolor[rgb]{0.161,0.373,0.58}{\textbf{#1}}}%
\newcommand{\hlkwb}[1]{\textcolor[rgb]{0.69,0.353,0.396}{#1}}%
\newcommand{\hlkwc}[1]{\textcolor[rgb]{0.333,0.667,0.333}{#1}}%
\newcommand{\hlkwd}[1]{\textcolor[rgb]{0.737,0.353,0.396}{\textbf{#1}}}%
\let\hlipl\hlkwb

\usepackage{framed}
\makeatletter
\newenvironment{kframe}{%
 \def\at@end@of@kframe{}%
 \ifinner\ifhmode%
  \def\at@end@of@kframe{\end{minipage}}%
  \begin{minipage}{\columnwidth}%
 \fi\fi%
 \def\FrameCommand##1{\hskip\@totalleftmargin \hskip-\fboxsep
 \colorbox{shadecolor}{##1}\hskip-\fboxsep
     % There is no \\@totalrightmargin, so:
     \hskip-\linewidth \hskip-\@totalleftmargin \hskip\columnwidth}%
 \MakeFramed {\advance\hsize-\width
   \@totalleftmargin\z@ \linewidth\hsize
   \@setminipage}}%
 {\par\unskip\endMakeFramed%
 \at@end@of@kframe}
\makeatother

\definecolor{shadecolor}{rgb}{.97, .97, .97}
\definecolor{messagecolor}{rgb}{0, 0, 0}
\definecolor{warningcolor}{rgb}{1, 0, 1}
\definecolor{errorcolor}{rgb}{1, 0, 0}
\newenvironment{knitrout}{}{} % an empty environment to be redefined in TeX

\usepackage{alltt}

%revised version started: 12/17/16

%NEEDS: allbib.bib - already added to the master bibliography file.
%cited references only: bibexport -o mhTMP.bib main1-blx.aux
%PLUS sramh-img*, sramh.tex

%added stuff
\newcommand{\add}[1]{\textcolor{blue}{#1}}
%deleted stuff
\newcommand{\del}[1]{\textcolor{red}{(removed: #1)}}
%uncomment these to turn off colors
\renewcommand{\add}[1]{#1}
\renewcommand{\del}[1]{}

%shortcuts
\newcommand{\w}{\ili{Welsh}}
\newcommand{\e}{\ili{English}}
\newcommand{\io}{Input Optimization}




 \newcommand{\hand}{\ding{43}}
% \newcommand{\rot}[1]{\begin{rotate}{90}#1\end{rotate}} %shortcut for angled text%  
% \newcommand{\rotcon}[1]{\raisebox{-5ex}{\hspace*{1ex}\rot{\hspace*{1ex}#1}}}

%\input{localpackages.tex}
\usepackage{arydshln}
\usepackage{rotating}

%\input{localcommands.tex}
\newcommand{\tworow}[1]{\multirow{2}{*}{#1}}


\newcommand{\tworow}[1]{\multirow{2}{*}{#1}}


\newcommand{\tworow}[1]{\multirow{2}{*}{#1}}



\title{{\textit{OF}} as a phrasal affix in the English determiner system}
\author{Randall Hendrick\affiliation{University of North Carolina at Chapel Hill}}


\abstract{
This paper examines the distribution of {\textit{of}} in the complex determiner system of English nominal expressions.  The hypothesis is advanced that {\textit{of}} in degree phrase inversion constructions (expressions like {\textit{more influential of a book}}) is a phrasal affix in the sense of Anderson's classic work asserting the autonomy of morphological theory.  The phrasal affix analysis can be distinguished from syntactic accounts that treat {\textit{of}} as a syntactic head of phrase.  It is argued that the phrasal affix account can capture {\textit{of's}} sensitivity to morphological sub-category features and to second position more naturally than the syntactic head of phrase analysis. In contrast to the syntactic head analysis, the phrasal affix analysis also has the advantage of explaining why {\textit{of}} fails to exhibit typical syntactic properties such as selection, dislocation, conjunction, and adjunction of focus particles.  The analysis extends neatly to the use of {\textit{of}} in complex determiners involving fractions. 
}


\begin{document}
\maketitle


\section{Introduction}

{\citet{Anderson92}} has argued that the classical distinction between words and affixes can be profitably extended to the phrasal domain, a line of argumentation that is extended in {\citet{Anderson05}}.  On this view, some properties of constructions might be better understood if analyzed as involving affixes attached to phrases, parallel to the attachment of affixes to words.   Anderson identifies the possessive {\textit{'s}} as the prototype of such a phrasal affix, but  (special) clitics more generally are identified as phrasal affixes in this sense.  In the same spirit of enlarging the explanatory work of morphological theory into the domain of phrasal syntax, {\citet{Anderson05}} argues carefully that some verb second phenomena should be given a morphological explanation parallel to second position clitics.  Both claims have proven provocative because they seem to compete with other rather popular analyses that make use of syntactic movement.{\footnote{The true difference between these types of analyses may be less than meets the eye if one assumes a syntactic engine like that advocated in {\citet{Chomsky95}}.  There is a real substantive difference in the embrace of optimality theoretic explanations in {\citet{Anderson05}}, but I will ignore that issue here in the belief that the existence of phrasal affixes is conceptually independent of any commitment to optimality theoretic explanations.}}  In this paper I will argue that {\textit{of}} as it occurs in the pre-determiner system of English provides compelling corroboration for the attempt to extend the range of morphological analysis into at least some areas of phrasal syntax because alternative syntactic explanations for these instances of {\textit{of}} are relatively weak.

The center of our attention is on the formative {\textit{of}} as it occurs in sentences like ({\ref{ex2a}})-({\ref{ex2b}}).  These sentences appear to have some string  before the indefinite D {\textit{a(n)}} as part of the bracketed nominal constituent.  They have been the focus of attention in a number of works, including {\citet{Bolinger72}}, {\citet{Bresnan73}}, {\citet{Hendrick90}}, and {\citet{Kennedy00}}, among others.  {\citet{Kim11}} present a wide variety of naturally occurring examples of the construction.  A range of proposals have been offered to provide syntactic accounts of such sentences.  Such sentences seem closely related to ({\ref{ex3}}), in which the adjective abuts {\textit{a}} directly without the cushioning of {\textit{of}}, and ({\ref{ex1}}), in which the adjectival constituent appears to the right of {\textit{a}} rather than to its left.  {\citet{Kennedy00}} label ({\ref{ex3}}) and ({\ref{ex2a}}) as {\textit{inverted DegP} (i.e., inverted degree phrases), suggesting that they originate as ({\ref{ex1}})  and subsequently move to a higher position.

\begin{exe}
\ex \label{ex2a}  It is difficult to find [ more influential of a book ]
\ex \label{ex2b}  [ How long of a vacation ] did she take?
\ex \label{ex1}  It is difficult to find [ a more influential book ]
\ex \label{ex3}  It is difficult to find [ more influential a book ]
\end{exe}

The degree phrases that are implicated in this inversion are semantically fairly heterogenous:  a great many, like ({\ref{ex2a}}), appear to be evaluative in the sense of {\citet{Keenan85}}, but others, like ({\ref{ex2b}}), are extensional and amenable to a model theoretic interpretation.

There is another family of constructions that exhibit {\textit{of}} preceding indefinite $ [_{D}$ $a(n) ]$.  These are the complex determiners that have figured prominently in the study of the semantics of determiners and general quantifiers.  {\citet{Keenan86}} noted a range of complex determiners as part of their attempt to establish the semantic conservativity of a large set of determiners.  Many of those determiners include {\textit{of}} to the left of {\textit{a}}, and denote proportions or frequencies, as noted in {\citet{Peters06}}.  The examples in ({\ref{ex4}}) list some of the complex determiners involving {\textit{of}}.{\footnote{ {\citet{Peters06}} observe sentences like {\textit{two of ten interviewed persons have not answered the question}}.  The analysis in the text does not generalize to such an example because it does not exhibit $[_{D}$ $a(n) ]$.  Unlike the examples examined in the text, a plural NP is involved here as well.   The {\textit{of}} in this example would have to be accounted for separately.  The fact that such examples can occur with {\textit{out}} before {\textit{of}} in contrast to ({\ref{ex2a}}) could be evidence supporting this inference.} }

\begin{exe}
\ex\label{ex4}
\begin{xlist}
\ex [ ] {They sang for half of an hour.}
\ex [ ] {She ate two thirds of an orange.}
\ex [ ] {The tornado damaged one quarter of an entire neighborhood.}
\end{xlist}
\end{exe} 

 My principal claim is that {\textit{of}} is inserted in ({\ref{ex4a}}) as an affix to \=D, formally marking the relation between the head D and its specifier when $\alpha$ is phonologically non-null.

\begin{exe}
 \ex \label{ex4a}
\Tree [.DP [.DegP $\alpha$ ] [.{\=D} [.D a ]  [.NP $\beta$ ]]]
\end{exe}
 
This affixed {\textit{of}} is evinced in all of ({\ref{ex2a}}), {\ref{ex2b}} and ({\ref{ex4}}).  As a marker of a formal relation, it itself is not semantically evaluated, and in this sense, it is not meaningful.  As an adjunct, It is also syntactically inert for movement processes that target specifiers, heads or complements.{\footnote{See {\citet{Kaufman10}} for a broadly similar analysis of Tagalog and other Austronesian clitic systems.  In lexical theories of the sort that Anderson has generally favored, the syntactic inertness of the phrasal clitic will follow from the post-syntactic affixation of the phrasal clitic.}}

I have framed my claim in terms of the DP hypothesis in ({\ref{ex4a}}), where D is the head of the entire nominal phrase.  If one were instead to assume the NP hypothesis, where N is the head of the nominal phrase, we would have structures like ({\ref{ex4b}}).  In this case we would say that {\textit{of}} is adjoined as an affix of \=D, formally marking the relation between \=D and the phrase $\alpha$ to its left when $\alpha$ is non-null.
\begin{exe}
\ex\label{ex4b}
\Tree [.NP [.DP [.DegP  $\alpha$ ] [.$\bar{D}$ [.D a ] ]] [.$\bar{N}$ [.N $\beta$ ]]]

\end{exe}
 
\section{OF as a Phrasal Affix}
There is a prima facie case to be made that {\textit{of}} in examples like ({\ref{ex2a}}) and ({\ref{ex4}}) are phrasal affixes.  The insertion of {\textit{of}} is sensitive to morphological properties and shows second position effects common to other putative examples of phrasal clitics that Anderson identifies.
\subsection{OF as Sensitive to Morphological Properties}
The inversion of degree phrases in examples like ({\ref{ex2a}}) does not apply generally.  It is unnatural if the degree phrases are an instance of unmodified synthetic comparatives, as illustrated in ({\ref{ex4c}}).
\begin{exe}
\ex \label{ex4c}
\begin{xlist}
\ex [ ] {We are shopping for a better/prettier/finer vase.}
\ex [*] {We are shopping for better/prettier/finer a vase.}
\ex [*] {We are shopping for better/prettier/finer of a vase.}
\end{xlist}
\end{exe}

When synthetic comparatives are modified, inversion of degree phrases, as in ({\ref{ex5}}), is acceptable for some speakers, although others reject the inversion of degree phrases even when modified.
\begin{exe}
\ex \label{ex5}
\begin{xlist}
\ex [*] {We have never seen a any better/prettier/finer vase.}
\ex [ ] {We have never seen any better/prettier/finer a vase.}
\ex [ ]  {We have never seen any better/prettier/finer of a vase.}
\end{xlist}
\end{exe}

Speakers who reject all synthetic comparatives in ({\ref{ex4c}})--({\ref{ex5}}) treat inversion of degree phrases as sensitive to that morphological class.  Speakers who accept ({\ref{ex5}}) in contrast to ({\ref{ex4b}}) appear to treat the inversion of degree phrases as sensitive to the prosodic contour of the phrase.  Degree phrases composed of a single prosodic word, as in ({\ref{ex4c}}), are excluded.  When more prosodic material is added, the same synthetic comparatives appear acceptable once again, as in ({\ref{ex5}}).  


The inversion is also sensitive to the morphological properties of the D to its right.  The D cannot be definite, as shown by ({\ref{ex6}}b) and ({\ref{ex7}}b).  Of the indefinite D's, it is only able to employ {\textit{a(n)}, and resists co-occuring with {\textit{some}} or {\textit{no}}.

\begin{exe}
\ex \label{ex6}
\begin{xlist}
\ex [ ] {How long a novel did she write?}
\ex [*] {How long the novel did she write?}
\ex [*] {How long some novels did she write?}
\ex [*] {How long no novel did she write?}
\end{xlist}


\ex \label{ex7}
\begin{xlist}
\ex [ ] {She wrote more fascinating a novel this time.}
\ex [*] {She wrote more fascinating the novel this time.}
\ex [*] {She wrote more fascinating some novels.}
\ex [*]{She wrote more fascinating no novel.}
\end{xlist}

\end{exe}

\begin{table}[ht]

\centering
\begin{tabular}{| c | c | c |}
\hline
Indefinite D & Singular N  &  Plural N \\
\hline
a & {a novel} & {*a novels}  \\
some & {*some novel}  & {some novels} \\
no & {no novel} & {no novels} \\ 
\hline

\end{tabular}
\caption{Co-occurrence of indefinite D's with Count Nouns}
\end{table}

If {\textit{a(n)}}  has the features [+INDEF, -PLURAL]  while {\textit{some}} is [+INDEF, +PLURAL] and {\textit{no}} is [+INDEF] but lacks a feature specification for plurality, we can stipulate for the explicitness of description that degree inversion is restricted to D's that carry the feature specification +INDEF, -PLURAL.\footnote{It is worth noting that there is another formative {\textit{some}} that is stressed and is definite.  In addition, the table is offered only to make explicit the morphological use of {\textit{indefinite}} in the text.  It is not a theoretical claim.  If we were to decide that we needed to specify {\textit{no}} as PLURAL we could appeal to a feature CARD for cardinality and stipulate that degree phrase inversion could only occur with D's specified as -CARD.  By the same token, we might have reason for adopting a morphological theory that avoided features altogether.  In any scenario we need to restrict Degree Inversion to the D {\textit{a(n)}} and not other members of the D class, much as {\textit{$/n/$}} is added to {\textit{$/a/$}} and not other indefinites.}

\subsection{OF as Sensitive to Second Position}
The {\textit{of}} described in the preceding subsection is not capable of bearing stress.  Nor is it able to appear initially in a nominal domain.  When it surfaces, it always occupies second position in the nominal expression.  To see this point, consider sentences like ({\ref{ex8}}) that contain two pre-nominal adjectives.  If {\textit{long}} is part of a wh-phrase, it preposes (as in ({\ref{ex8}}b)), and if {\textit{romantic}} is part of a degree phrase, it may prepose, as in ({\ref{ex8}}c).  However, if both are before {\textit{a}}, the result is the unacceptable ({\ref{ex8}}d).

\begin{exe}
\ex\label{ex8}
\begin{xlist}
\ex [ ] {{\textit{Jane Eyre}} is a long romantic novel.}
\ex  [ ] {How long of a romantic novel is {\textit{Jane Eyre}}?}
\ex [ ]{ {\textit{Jane Eyre}} is more romantic of a long novel (than {\textit{Middlemarch}}).}
\ex  [*]{How long more romantic of a novel is {\textit{Jane Eyre}}?}
\end{xlist}
\end{exe}
We begin to capture these facts if we assume that the preposed wh-phrase in ({\ref{ex8}}b) and degree phrase in ({\ref{ex8}}c) target a (unique) specifier position for movement (i.e. specifier of DP).  We might accomplish this by way of ({\ref{ex50}}a) that gives {\textit{a(n)}} an EPP feature, parallel to what is often assumed for T.  We could then posit a spell out rule that prefixes {\textit{of}}.
\begin{exe}  
\ex\label{ex50}
\begin{xlist}
\ex  There is an optional feature on $[_{D}$ $a ]$ that requires a filled specifier.  
\ex  This feature is optionally spelled out as the phrasal prefix {\textit{of}}.
\end{xlist}
\end{exe}

\section{Could {\textit{OF}} be a Syntactic Head of Phrase?}
Some researchers have suggested that the {\textit{of}} that occurs in the Degree Inversion structures is a syntactic head of phrase.{\footnote{This structure is advocated in {\citet{Kennedy00}} and is assumed in other work, for example, {\citet{Borroff06}}.  The labels of constituents differ in other analyses, in particular {\citet{Matushansky02}} and {\citet{Kim11}}, but the constituency is broadly the same.  The crucial point is to contrast analyses that treat {\textit{of}} as a head of phrase from the analysis defended in the text where {\textit{of}} is a phrasal affix.}} On this view we should assign {\textit{how long of a vacation}} the structure in ({\ref{ex60}}), whereas the proposal I have suggested in ({\ref{ex50}}) is ({\ref{ex60a}}), if we assign {\textit{of}} to the category F and adopt an item and arrangement approach to affixation in order to facilitate the comparison of the two theoretical viewpoints.{\footnote{In a word and paradigm approach {\textit{of}} might not be given a structural position at all but simply spelled out at the left bracket \=D.  This point of view is probably closer to Anderson's.  I believe that the structure in ({\ref{ex60a}}) gives some insight into why {\textit{of}} occurs between the D and its specifier, as I explain at the end of this paper.}}
\begin{exe}
\ex\label{ex60}
\Tree [.FP [\qroof{how long}.DegP   ] [.{\=F}  [.F  of  ] [.DP [.{\=D}  [.D a ]  [.NP [\qroof{\sout{how long}}.DegP [.$\bar{N}$ [.N vacation ]]]]]]]]

\ex\label{ex60a}

\Tree [.DP [\qroof{how long}.DegP   [.{\=D}  [.F  of  ]  [.{\=D}  [.D a ] [.NP  [\qroof{\sout{how long}}.DegP  [.$\bar{N}$ [.N vacation ]]]]]]]]

\end{exe}
Some authors, following the lead of {\citet{Bennis98}}, have tried to treat {\textit{of}} as parallel to a copular construction in which predicates have been claimed to invert in ({\ref{ex60b}).
\begin{exe}
\ex \label{ex60b} 
\begin{xlist}
\ex Jill was a much better friend.
\ex  A much better friend was Jill.
\end{xlist}
\end{exe}

Den Dikken {\citeyearpar{denDikken06}} offers an extended analysis in this vein over a range of languages.  However, I find compelling the argumentation in {\citet{Heycock99}} that rejects the inversion analysis in ({\ref{ex60b}}) in favor of viewing these as identity statements.  On this view there is no basis for the parallel between ({\ref{ex60b}}) and ({\ref{ex60}}), as Heycock and Kroch note.

Four reasons lead us to avoid a structure like ({\ref{ex60}}).{\footnote{Some work has suggested that the fronted {\textit{how long}} must originate as a predicate adjective.  See, for example, {\citet{Wood11}} and \citet{Troseth09}.  This seems unlikely, at least for English.  Expressions like {\textit{a beautiful dancer}} are ambiguous between a predicative reading in which the dancer is beautiful and a non-predicative reading in which the dancer dances beautifully.  The ambiguity is preserved in {\textit{how beautiful (of) a dancer}}.  This fact is surprising if the preposed degree phrase could only arise from a predicate adjective.  These facts are general and can be reproduced with {\textit{a skillful manager}}.}}  First, heads of phrases are able to stand in a selection relation with other heads of phrases.  Yet, there is no lexical head that ever selects the FP of ({\ref{ex60}}).  This fact is accidental if {\textit{of}} is a head of phrase, but it is a necessary consequence if it is not actually a head, as in ({\ref{ex60a}}).{\footnote{This challenge can be given a sharper formulation if we accept the stipulation that the complement of V is reserved for selected arguments of V.  This stipulation requires that V select {\textit{of}}, if {\textit{of}} is a head of phrase.  To the extent that every V that selects DP also selects this {\textit{of}} we miss a regularity in the system of selection, a cost that is not incurred if {\textit{of}} is provided as a phrasal affix.  One can imagine mechanisms that allow {\textit{of}}, if it were a head of phrase, to inherit argument properties of its complement DP, but such mechanisms would obscure the generalization that {\textit{of}} is syntactically inert generally.}} Second,  the {\textit{of}} phrase in ({\ref{ex60}}) is immune to syntactic dislocation.  Other {\textit{of}} phrases are able to prepose or postpose, as shown in ({\ref{ex61}}) and ({\ref{ex61a}}).  It is also possible to strand {\textit{of}} in examples parallel to ({\ref{ex61b}}).  In this sense, they are syntactically active.

\begin{exe}
\ex\label{ex61}
\begin{xlist}
\ex [ ]{Three of these books have been reviewed.}
\ex  [ ]{Of these books three have been reviewed.}
\end{xlist}

\ex\label{ex61a}
\begin{xlist}
\ex  [ ]{The announcement of the Nobel Prize in Chemistry was delayed.}
\ex  [ ]{The announcement was delayed of the Nobel Prize in Chemistry.}
\end{xlist}

\ex\label{ex61b}  What book did the newspaper publish a review of?

\end{exe}

Yet ({\ref{ex60}}) stands in contrast to examples like ({\ref{ex61}}) in being unable to be preposed, postposed, or stranded.  They appear to be syntactically inert.{\footnote{A reviewer observes that this second challenge can be met by stipulating that only maximal phrasal projections are available for movement.  While this defense of the head of phrase analysis will side step the second challenge, it does not generalize to meet the other challenges.  For this reason the phrasal affix account seems to me to offer a more straight forward explanation, all other things being equal.}}
\begin{exe}
\ex \label{ex62}
\begin{xlist}
\ex  [ ]{More extensive of an experiment was in the planning stages.}
\ex [*] {Of an experiment more extensive was in the planning stages.}
\end{xlist}
\ex \label{ex62a}
\begin{xlist}
\ex [ ]{ More extensive of an experiment was delayed}
\ex [*] {More extensive was delayed of an experiment}
\end{xlist}
\ex \label{ex62b} *What (kind of) experiment did the journal report more extensive of?
\end{exe}
However, if \textit{of} is not a head of phrase with its own phrasal projection as in ({\ref{ex60a}}), the lack of syntactic dislocation is expected.

A third challenge to the analysis in ({\ref{ex60}}) involves conjunction.  Locative prepositions like {\textit{on}} can be repeated members of a conjunction with {\textit{only}}, presumably for pragmatic reasons related to defeasing conversational implicatures that would otherwise obtain.  The same seems true of \textit{of} that introduces the complement of derived nominals like {\textit{destruction}}.
\begin{exe}
\ex\label{ex63}
\begin{xlist}
\ex [ ]{ You may place the notebook on, and only on, the desk.}
\ex  [ ]{The tornado caused the destruction of, and only of, one residence.}
\ex  [ ]{You may place the notebook only on the desk.}
\ex  [ ]{The tornado caused the destruction only of one residence.}
\end{xlist}
\end{exe}
In contrast, the {\textit{of}} that appears with degree phrase inversion is unable to appear in similar conjunctions.  If {\textit{of}} is syntactically and semantically inert, as the phrasal affix analysis contends, it could not conjoin and co-occur with {\textit{only}} to defease any conversational implicatures.
\begin{exe}
\ex \label{ex64}
\begin{xlist}
\ex [ ]{You can't find more valuable of a proposal.}
\ex [*]{You can't find more valuable only of a proposal.}
\ex [*]{You can't find more valuable of, and only of, a proposal.}
\ex [ ]{ How long of a vacation did they take?}
\ex [*] {How long of, and only of, a vacation did they take?}
\end{xlist}
\end{exe}

A fourth reason to be skeptical of the analysis in ({\ref{ex60}}) is that it is unable to generalize to explain the appearance of {\textit{of}} in fractions.  Fractions have been treated as complex determiners since the classic work of {\citet{Keenan86}} on the conservativity of determiners.  Fractions exhibit this semantic property, a fact that is easy to explain if they are (complex) determiners, but that looks accidental if they are assimilated to constructions that take {\textit{of}} to be a head of phrase, like the partitive constructions discussed in the next section.  The use of {\textit{of}} in the fraction in ({\ref{ex75}}) parallels its use in degree phrase inversion.  In both cases {\textit{of}} is adjacent to the indefinite D {\textit{a(n)}}, in both cases {\textit{of}} is optional, and in both cases {\textit{of}} lacks the ability to be lexically selected, or syntactically preposed or stranded as illustrated in ({\ref{ex75}}).  The phrasal affix analysis in ({\ref{ex60a}}) can provide a unified analysis, on the assumption that fractions occur in specifier of DP.{\footnote{It is tempting to extend this analysis of fractions to proportions like {\textit{six of ten adults don't vote}}.  Proportions like this do not make use of the D {\textit{a(n)}} and are plural rather than singular.  Without a more detailed analysis of the proper structural analysis of cardinals like {\textit{ten}}, and an understanding of why proportions admit some form of dislocation (e.g. {\textit{(out) of (every) ten adults six don't vote}}) the extension of the analysis will have to remain incomplete.}}
\begin{exe}
\ex\label{ex75}
\begin{xlist}
\ex [ ] {The recipe called for one half of a cup.}
\ex [*]{Of a cup the recipe called for one half.}
\ex [*]{A cup the recipe called for one half of.}
\ex [*]{The recipe called for one half of, and only of, a cup.}
\end{xlist}
\end{exe}

\section{Partitive {\textit{OF}} Is a Syntactic Head of Phrase }
There is a second construction involving {\textit{of}} which should be distinguished from {\textit{of}} as a phrasal affix in Degree Inversion structures.  This second type of construction is exemplified in ({\ref{ex9e}}) and has traditionally been labeled a partitive.
\begin{exe}
\ex\label{ex9e}
\begin{xlist}
\ex She invited ten of the students to the party.
\ex  Jack offered to buy any three of her paintings.
\ex  Sandy knows the answer to the last (one) of those questions.
\end{xlist}
\ex\label{ex9f} \Tree [.DP D [.NP [.N $\alpha$ ] [.PP [.P of ] [.DP $\beta$ ]]]]

\end{exe}
The {\textit{of}} in these partitives is a syntactic head of phrase, appearing in structures like ({\ref{ex9f}}), where {\textit{of}} heads a PP.  With the exception of {\textit{half (of) the}}, partitive {\textit{of}} appears obligatorily, like other heads of phrases and in contrast to the pattern of {\textit{of}} with Degree Phrase Inversion.  In addition, the {\textit{of}} phrase can be topicalized as a syntactic constituent.  In this respect it differs from {\textit{of}} in Degree Inversion constructions like ({\ref{ex62}}).
\begin{exe}
\ex\label{ex9g}
\begin{xlist}
\ex Of the students, she invited ten to the party.
\ex Of her new paintings, Jack offered to buy any three.
\ex Of those questions, Sandy knows the answer to the last (one).
\end{xlist}
\end{exe}
The {\textit{of}} phrase can also be co-ordinated, a standard test for constituency.  Examples are somewhat limited, presumably because of semantic restrictions on partitives.  However, naturally occurring examples are not difficult to find.{\footnote{({\ref{ex9h}}a) occurs at https://www.sicknotweak.com/2016/08/its-a-piece-of-you-make-it-yours/ as accessed September 24, 2016.  ({\ref{ex9h}}b) occurs at http://www.ballet-dance.com/200505/articles/GloriaGovrin20041200.htm and was accessed September 24, 2016.}}

\begin{exe}
\ex\label{ex9h}
\begin{xlist}
\ex If you accept it to be a part of you and of your life, you gain control of the illness.
\ex I try to be very aware of all of the students and of their strengths and weaknesses.
 \end{xlist}
\end{exe}
In contrast, Degree Inversion structures do not show the same support for co-ordination.
\begin{exe}
\ex\label{ex9i}  
\begin{xlist}
\ex [*]{More expensive of a house or of a car would be hard to find.}
\ex  [*]{How good of a novel or of a play did she write?}
\end{xlist}
\end{exe}
It is also possible to use expressions like {\textit{only}} with partitive {\textit{of}}, in contrast to the {\textit{of}} in Degree Inversion structures.
\begin{exe}
\ex\label{ex9j}  
\begin{xlist}
\ex [ ]{She invited ten of the linguistics students, and only of the linguistics students, to the party.}
\ex  [*]{How good of a novel, and only of a novel, did she write?}
\end{xlist}
\end{exe}

My claim that {\textit{of}} in Degree Inversion structures is structurally distinct from {\textit{of}} in partitives can be better appreciated if it is contrasted with the classic analysis presented in {\citet{Bresnan73}}.{\footnote{ {\citet{Stockwell73}} also distinguished partitives with {\textit{of}} from other constructions where {\textit{of}} surfaces.  {\citet{Selkirk77}} distinguishes true partitives from pseudo-partitives like {\textit{two pounds of turkey}} in part by pointing to differences in the distribution of {\textit{of}} in the two types of constructions.}}
  Bresnan suggests that {\textit{of}} should be inserted in examples like ({\ref{ex9b}}) by an elegant rule like ({\ref{ex9a}}).{\footnote{Q in ({\ref{ex9a}}) is intended to denote quantified expressions like {\textit{more}} and {\textit{enough}}. }

\begin{exe}
\ex\label{ex9a} \phonb{$\emptyset$}{\textit{of}}{Q}{D N} 
\ex\label{ex9b}  She has enough of a problem as it is.
\ex\label{ex9c}  more of an egg
\ex\label{ex9d}  more of the egg
\end{exe}
({\ref{ex9a}}) is similar to the phrasal clitic analysis I suggested earlier in that {\textit{of}} is inserted rather than being treated as a syntactic head of phrase.  It differs from the phrasal clitic analysis in two ways:  (i)  it does not require Q to be complex, and (ii) it does not require D to be {\textit{a(n)}}.  

The first of these differences is more apparent than real.  I say this because Bresnan argues that {\textit{more}} originates as a syntactically complex constituent {\textit{-er much}}, and that this complex constituent is replaced in the course of the derivation with {\textit{more}}.  Bresnan analyzes {\textit{enough}} in a parallel complex structure, with the difference that {\textit{enough}} necessarily co-occurs with a null specifier.  If this complex structure is correct, there is a point in the derivation where ({\ref{ex9c}}) is not a counter-example to our phrasal affix account of {\textit{of}}; we only need to insure that the phrasal clitic {\textit{of}} is inserted prior to {\textit{-er much}} suppletion.

Consider now the second way in which rule ({\ref{ex9a}}) differs from the phrasal clitic analysis that I suggested in section 2.  By inserting {\textit{of}} before any D, the rule in ({\ref{ex9a}}) loses the important empirical generalizations about the way in which {\textit{of}} in ({\ref{ex9d}}) patterns like a syntactic head of a phrase in terms of conjunction, the distribution of {\textit{only}}, and the possibility of movement.  It does not code naturally the way in which the {\textit{of}} in Degree Inversion constructions patterns differently along these dimensions.  In the absence of other evidence, considerations of simplicity would make ({\ref{ex9a}}) quite attractive, but in this case, that very simplicity obscures important ways in which {\textit{of}} patterns in distinct, rather than haphazard, ways.  Moreover these distinct patterns have a semantic reflex.  {\citet{Peters06}} argue that semantic definiteness is a necessary characteristic of partitives. From this perspective the distinction between ({\ref{ex9c}}) and ({\ref{ex9d}}) that the phrasal affix analysis encodes is a natural one.

 
                 
                
\section{{\textit{OF}} in Complements of Nominalization and Syntactic Case Theory}
Since {\citet{Chomsky81}} it has been popular to say that the presence of {\textit{of}} after nominalizations like {\textit{picture}} in ({\ref{ex90}}) is required to provide the nominalization's complement ({\textit{Mary}} in this example) with Case.  The details of the required derivation have rarely been made precise.{\footnote{{\citet{Harley98}} is a notable exception to this trend.}}} This imprecision is remedied in an interesting way in {\citet{Kayne02}}.  There, Kayne gives a careful description of the insertion of {\textit{of}} in ({\ref{ex90}}) and links it to an explanation for why the {\textit{of}} in derived nominals like ({\ref{ex90}}) can show stranding effects like ({\ref{ex91}}), superficially in violation of subjacency requirements.  

\begin{exe}
\ex\label{ex90}  John was admiring a picture of Mary.
\ex\label{ex91}  Who was John admiring a picture of?
\ex\label{ex92}  John was  $[_{VP} $  admiring $ [_{DP} $ a  $[_{NP} $  picture  of Mary ]]]]]
\ex\label{ex93}  John was [ {\textit{of}} $[_{KP}$ $K-{\textit{of}}$   $[_{VP}$ admiring $[_{DP}$ a $[_{NP}$ picture Mary ]]]]]
\end{exe}

If one thought that the structure of ({\ref{ex90}}) was like ({\ref{ex92}}) where {\textit{of}} formed a constituent with the logical complement of picture, {\textit{Mary}}, and was internal to the VP, one has trouble explaining in a non-stipulative fashion why ({\ref{ex91}}) is not a subjacency violation (hence the debate between {\citet{Bach76}} and {\citet{Chomsky77}}).  Kayne's suggestion is that there is an abstract functional head K-{\textit{of}} that is merged into the syntactic structure outside of the VP.  On his assumption (defended in {\citet{Emonds2000}}) that every lexical noun requires Case valuation from an appropriate head, {\textit{Mary}} will be unable to be so valued by the verb {\textit{admiring}}.  Instead it will be forced to move to specifier of K-{\textit{of}}.  The formative {\textit{of}}  is merged above KP and provides another specifier position for the remnant VP to move to.  This derivation will provide the linear order observed in ({\ref{ex90}}) without treating {\textit{a picture of Mary}} as a constituent.  In this way the challenge of ({\ref{ex91}}) to subjacency is side-stepped.

Kayne sees a fundamental unity between the presence of {\textit{of}} in nominalizations like ({\ref{ex90}}) and its appearance in constructions like ({\ref{ex94}}) and ({\ref{ex95}}).{\footnote{Examples like ({\ref{ex94}}) are the {\textit{pseudo-partitives}} of {\citet{Selkirk77}}.}}

\begin{exe}
\ex\label{ex94}  He has lots of money.
\ex\label{ex95}  They bought too big of a house.
\end{exe}
Kayne suggests that {\textit{lots}} in ({\ref{ex94}}) and {\textit{big}} in ({\ref{ex95}}) have a nominal property and, for that reason, have a Case feature that needs valuation.   Because they intervene between the verbs {\textit{has}} and {\textit{bought}} and {\textit{money}} and {\textit{house}} respectively, they prevent the verbs from valuing the Case feature on those nouns.  This intervention forces the need for {\textit{of}}.  The derivation of ({\ref{ex94}}) and ({\ref{ex95}}) parallels ({\ref{ex93}}).  A functional head {\textit{of}} (and K-{\textit{of}} ) is merged above VP.  The nominals {\textit{money}} and {\textit{house}} raise to specifier of K-{\textit{of}} where their Case features can be valued.  The  remnant VP subsequently moves to a specifier position on the left in order to get the appropriate left to right linear order.

Kayne's analysis of ({\ref{ex90}}) makes {\textit{of}} deeply syntactic in the sense that it takes {\textit{of}} to be a head of phrase that enters into selection relations and that also provides a landing site for other putative syntactic operations.  It is also syntactic in the sense that, whether it is ultimately judged to be true or not, it ultimately aims to resolve a syntactic problem (regarding the ability of ({\ref{ex91}}) to avoid subjacency like island effects).   The extension of the analysis to the Degree Inversion construction in ({\ref{ex95}}) is not deeply syntactic in the same sense:  the {\textit{of}} in ({\ref{ex95}}) does not enter into other independently required syntactic operations.  For example, it is unable to feed topicalization or clefting to strand the preposition:

\begin{exe}
\ex\label{ex98} 
\begin{xlist}
\ex [ ]  {(It is) Mary John was admiring a picture of.}
\ex [ ] {(It is ) Money he has lots of.}
\ex [*] {(It is) a house John bought too big of.}
\end{xlist}
\end{exe}
Further, the {\textit{of}} in ({\ref{ex95}}) is optional, while it is obligatory in ({\ref{ex94}}) and ({\ref{ex90}}).  This fact leads Kayne to posit that in some situations we should stipulate that multiple Case evaluation is available, allowing both {\textit{big}} and {\textit{house}} to be Case valued by the verb, thereby circumventing the intervention effect.  We should also note that unlike ({\ref{ex90}}), there is no syntactic problem that the Case-intervention account of ({\ref{ex95}}) provides a resolution to.  

My claim is that there is no deeply syntactic foundation for the putative structure ({\ref{ex60}}), and in this sense there is a fundamental asymmetry between the Case licensing {\textit{of}} in ({\ref{ex90}}) and the {\textit{of}} that appears in Degree Inversion structures like ({\ref{ex95}}).  This lack of a deeply syntactic account of {\textit{of}} in Degree Inversion constructions is what opens the door for the phrasal affix account.{\footnote{It is possible to posit a syntactic feature, say +EPP, on some instances of {\textit{a(n)}} and stipulate that the feature triggers {\textit{of}} as its spell out phonologically.  While I think this style of analysis could be descriptively successful, it would not be either deeply syntactic or deeply morphological as I have been using these terms.  It would stipulate that {\textit{of}} appears to the left of {\textit{a(n)}}, for example, rather than linking its appearance to a second position effect.}} Parity of reasoning prompts the question of whether the analysis of {\textit{of}} in degree inversion structures like ({\ref{ex95}}) is deeply morphological in any comparable sense.  I would like to suggest that, rather than simply being a competing analysis, the phrasal affix analysis of {\textit{of}} is deeply morphological, offering a response to a fundamental morphological puzzle.   From this vantage, the contention that there are phrasal affixes poses the puzzle of how these affixes interact with the syntax of phrases and whether that interaction is principled in any morphological sense. Of course, Anderson's perspective on this puzzle is to say that phrasal affixes are syntactically inert because such morphological operations are ordered as a block after syntactic operations by virtue of the architecture of grammars.  This paper has offered a different view that does not necessarily assume a late, or even a block, ordering of phrasal affixation.  The analysis of {\textit{of}} as a phrasal affix advanced here treats phrasal affixes as structural adjuncts, and by virtue of that structural relation, as inert to canonically syntactic operations such as movement.  

\section{Envoi}
This paper has provided an analysis of {\textit{of}} as it appears in Degree Inversion constructions and in fractions.  It has been suggested that {\textit{of}} in these constructions marks a formal relation between the indefinite D {\textit{a(n)}} and its specifier, and that structurally it is a \=D adjunct.  This occurrence of {\textit{of}} is distinct from its presence after derived nominals, which has been attributed by a number of researchers to Case theoretic requirements, and the {\textit{of}} that appears in partitive constructions.  This paper has defended the spirit of Anderson's claim that phrases can be provided with inflectional affixes by showing that the claim can offer an explanation for phenomena that are, from the syntactic point of view, unprincipled.  In the process it draws attention to the kind of phenomena a morphological analysis is well suited for:  formatives that are syntactically inert but sensitive to morphological class features, prosody, and to linear position (second position).  I have also suggested that phrasal affixes should have the syntax of structural adjuncts.  Whether this last point is valuable depends on a substantive theory of adjuncts and a comparison of other putative instances of phrasal affixes.  For example, it has been suggested in {\citet{Chomsky86}}, and argued for in some empirical detail in {\citet{McCloskey96}}, that adjunction to lexically selected phrasal projections is prohibited.  If this adjunction prohibition is correct, and if phrasal affixes must enter a derivation as adjuncts, we expect them to avoid adjunction to lexically selected phrasal projections.  In effect, they must enter the derivation adjoined to a lower phrasal projection.  I draw attention to this implication because it can provide some foundation for the second position effect that the phrasal affix {\textit{of}} exhibits.  The intuition here is that if {\textit{of}} were adjoined to the DP, it would violate the prohibition on adjunction to a lexically selected phrase.  It can only appear on the lower phrase, \=D in ({\ref{ex4a}}), in effect producing a type of second positioning.  Alternatively, one could stipulate that the phrasal affix {\textit{of}} carries a constraint against appearing on the left edge of a phrase.  A similar tension between syntactic and morphophonemic edge effects is present in explanations of verb second phenomena in various languages, as {\citet{Anderson05}} has observed, especially in regard to the description of Swiss Rumantsch.  The choice between these explanatory crossroads will have to remain undetermined for the moment since it will inevitably depend on assumptions about the relationship between hierarchical order (i.e. c-command) and linear order.  In forthcoming work on linking elements in Austronesian compounding, I hope to produce some reason for preferring the explanation in terms of adjunction that I sketched here.

%\section*{Abbreviations}
%\section*{Acknowledgements}

\printbibliography[heading=subbibliography,notkeyword=this]

\end{document}
