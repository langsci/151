\documentclass[output=paper,
modfonts
]{LSP/langsci}


%% add all extra packages you need to load to this file 
% \usepackage{todo} %% removed,cna use todonotes instead. % Jason reactivated
% \usepackage{graphicx} % not needed because forest loads tikz, which loads graphicx
\usepackage{tabularx}
\usepackage{amsmath} 
\usepackage{multicol}
\usepackage{lipsum}
\usepackage{longtable}
\usepackage{booktabs}
\usepackage[normalem]{ulem}
%\usepackage{tikz} % not needed because forest loads tikz
\usepackage{phonrule} % for SPE-style phonological rules
\usepackage{pst-all} % loads the main pstricks tools; for arrow diagrams in Hale.tex
%\usepackage{leipzig} % for gloss abbreviations
\usepackage[% for automatic cross-referencing
compress,%
capitalize,% labels are always capitalized in LSP style
noabbrev]% labels are always spelled out in LSP style
{cleveref}

% based on http://tex.stackexchange.com/a/318983/42880 for using gb4e examples with cleveref
\crefname{xnumi}{}{}
\creflabelformat{xnumi}{(#2#1#3)}
\crefrangeformat{xnumi}{(#3#1#4)--(#5#2#6)}
\crefname{xnumii}{}{}
\creflabelformat{xnumii}{(#2#1#3)}
\crefrangeformat{xnumii}{(#3#1#4)--(#5#2#6)}

%\usepackage[notcite,notref]{showkeys} %%removed, not helping CB.
%\usepackage{showidx} %%remove for final compiling - shows index keys at top of page.
 
\usepackage{langsci/styles/langsci-gb4e}  
 \usepackage{pifont}
% % OT tableaux                                                
% \usepackage{pstricks,colortab}  
\usepackage{multirow} % used in OT tableaux
\usepackage{rotating} %needed for angled text%
\usepackage{colortbl} % for cell shading
 
 \usepackage{avm}  
\usepackage[linguistics]{forest} 
\usetikzlibrary{matrix,fit} % for matrix of nodes in Kaisse and Bat-El


\usepackage{hhline}
\newcommand{\cgr}{\cellcolor[gray]{0.8}}
\newcommand{\cn}{\centering}



\newcommand{\reff}[1]{(\ref{#1})}
%\usepackage{newtxtext,newtxmath}


%\usepackage[normalem] {ulem}
\usepackage{qtree}
%\usepackage{natbib}
%\usepackage{tikz}
%\usepackage{gb4e}
\usepackage{phonrule}  
%\bibliographystyle{humannat}



\usepackage{minibox}

%\include{psheader-metr}

\def\bl#1{$_{\textrm{{\footnotesize #1}}}$}

% table formatting package/commands

%%add all your local new commands to this file

\newcommand{\form}[1]{\mbox{\emph{#1}}}
\newcommand{\uf}[1]{\mbox{/#1/}}

% borrowed from expex and converted from plan tex to latex
\newcommand{\judge}[1]{{\upshape #1\hspace{0.1em}}}
\newcommand{\ljudge}[1]{\makebox[0pt][r]{\judge{#1}}}

\newcommand\tikzmark[1]{\tikz[remember picture, baseline=(#1.base)] \node[anchor=base,inner sep=0pt, outer sep=0pt] (#1) {#1};} % for adding decorations, arrows, lines, etc. to text
\newcommand\tikzmarknamed[2]{\tikz[remember picture, baseline=(#1.base)] \node[anchor=base,inner sep=0pt, outer sep=0pt] (#1) {#2};} % for adding decorations, arrows, lines, etc. to text
\newcommand\tikzmarkfullnamed[2]{\tikz[remember picture, baseline=(#1.base)] \node[anchor=base,inner sep=0pt, outer sep=0pt] (#1) {\vphantom{X}#2};} % for adding decorations, arrows, lines, etc. to text; this one works best for decorations above a line of text because it adds in the heigh of a capital letter and takes two arguments - one for the node name and one for the printed text

\newcommand{\sub}[1]{$_{\text{#1}}$} % for non-math subscripts
\newcommand{\subit}[1]{\sub{\textit{#1}}} % for italics non-math subscripts
\newcommand{\1}{\rlap{$'$}\xspace} % for the prime in X' (the \rlap command allows the prime to be ignored for horizontal spacing in trees, and the \xspace command allows you to use this in normal text without adding \ afterwards). This isn't crucial, but it helps the formatting to look a little better.

% Aissen:
\newcommand\tikzmarkfull[1]{\tikz[remember picture, baseline=(#1.base)] \node[anchor=base,inner sep=0pt, outer sep=0pt] (#1) {\vphantom{X}#1};} % for adding decorations, arrows, lines, etc. to text; this one works best for decorations above a line of text because it adds in the heigh of a capital letter and takes one argument that serves as the name and the printed text
\newcommand{\bridgeover}[2]{% for a line that creates a bridge over text, connecting two nodes
	\begin{tikzpicture}[remember picture,overlay]
	\draw[thick,shorten >=3pt,shorten <=3pt] (#1.north) |- +(0ex,2.5ex) -| (#2.north);
	\end{tikzpicture}
}
\newcommand{\bridgeoverht}[3]{% for a line that creates a bridge over text, connecting two nodes and specifing the height of the bridge
	\begin{tikzpicture}[remember picture,overlay]
	\draw[thick,shorten >=3pt,shorten <=3pt] (#2.north) |- +(0ex,#1) -| (#3.north);
	\end{tikzpicture}
}
\newcommand{\bridgeoverex}{\vspace*{3ex}} % place before an example that has a \bridgeover so that there is enough vertical space for it

% Chung:
\newcommand{\lefttabular}[1]{\begin{tabular}{p{0.5in}}#1\end{tabular}}

% Kaisse:
\newcommand{\mgmorph}[1]{|(#1)| {#1}}
\newcommand{\mgone}[2][$\times$]{\node at (#2.base) [above=2ex] (1#2) {\vphantom{X}#1};}
\newcommand{\mgtwo}[2][$\times$]{\mgone{#2} \node at (#2.base) [above=4.5ex] (2#2) {\vphantom{X}#1};}
\newcommand{\mgthree}[2][$\times$]{\mgtwo{#2} \node at (#2.base) [above=7ex] (3#2) {\vphantom{X}#1};}
\newcommand{\mgftl}[1]{\node at (1#1) [left] {(};}
\newcommand{\mgftr}[1]{\node at (1#1) [right] {)};}
\newcommand{\mgfoot}[2]{\mgftl{#1}\mgftr{#2}}
\newcommand{\mgldelim}[2]{\node at (#2.west) [left,inner sep = 0pt, outer sep = 0pt] {#1};}
\newcommand{\mgrdelim}[2]{\node at (#2.east) [right,inner sep = 0pt, outer sep = 0pt] {#1};}

\newcommand{\squish}{\hspace*{-3pt}}

% Kavitskaya:
\newcommand{\assoc}[2]{\draw (#1.south) -- (#2.north);}
\newcolumntype{L}{>{\raggedright\arraybackslash}X}

% Lepic & Padden:
\newcommand{\fitpic}[1]{\resizebox{\hsize}{!}{\includegraphics{#1}}} % from http://tex.stackexchange.com/a/148965/42880
\newcommand{\signpic}[1]{\includegraphics[width=1.36in]{#1}}
\newcolumntype{C}{>{\centering\arraybackslash}X}

% Spencer:

\newcommand{\textex}[1]{\textit{#1}\xspace}
\newcommand{\lxm}[1]{\textsc{#1}\xspace}

% Thrainsson:

\renewcommand{\textasciitilde}{\char`~} % for use with TTF/OTF fonts (see comments on http://tex.stackexchange.com/a/317/42880)
\newcommand{\tikzarrow}[2]{% for an arrow connecting two nodes
\begin{tikzpicture}[remember picture,overlay]
\draw[thick,shorten >=3pt,shorten <=3pt,->,>=stealth] (#1) -- (#2);
\end{tikzpicture}
}

\newlength{\padding}
\setlength{\padding}{0.5em}
\newcommand{\lesspadding}{\hspace*{-\padding}}
\newcommand{\feat}[1]{\lesspadding#1\lesspadding}

% Hammond

\usepackage[]{graphicx}\usepackage[]{xcolor}
%% maxwidth is the original width if it is less than linewidth
%% otherwise use linewidth (to make sure the graphics do not exceed the margin)
\makeatletter
\def\maxwidth{ %
  \ifdim\Gin@nat@width>\linewidth
    \linewidth
  \else
    \Gin@nat@width
  \fi
}
\makeatother

\definecolor{fgcolor}{rgb}{0.345, 0.345, 0.345}
\newcommand{\hlnum}[1]{\textcolor[rgb]{0.686,0.059,0.569}{#1}}%
\newcommand{\hlstr}[1]{\textcolor[rgb]{0.192,0.494,0.8}{#1}}%
\newcommand{\hlcom}[1]{\textcolor[rgb]{0.678,0.584,0.686}{\textit{#1}}}%
\newcommand{\hlopt}[1]{\textcolor[rgb]{0,0,0}{#1}}%
\newcommand{\hlstd}[1]{\textcolor[rgb]{0.345,0.345,0.345}{#1}}%
\newcommand{\hlkwa}[1]{\textcolor[rgb]{0.161,0.373,0.58}{\textbf{#1}}}%
\newcommand{\hlkwb}[1]{\textcolor[rgb]{0.69,0.353,0.396}{#1}}%
\newcommand{\hlkwc}[1]{\textcolor[rgb]{0.333,0.667,0.333}{#1}}%
\newcommand{\hlkwd}[1]{\textcolor[rgb]{0.737,0.353,0.396}{\textbf{#1}}}%
\let\hlipl\hlkwb

\usepackage{framed}
\makeatletter
\newenvironment{kframe}{%
 \def\at@end@of@kframe{}%
 \ifinner\ifhmode%
  \def\at@end@of@kframe{\end{minipage}}%
  \begin{minipage}{\columnwidth}%
 \fi\fi%
 \def\FrameCommand##1{\hskip\@totalleftmargin \hskip-\fboxsep
 \colorbox{shadecolor}{##1}\hskip-\fboxsep
     % There is no \\@totalrightmargin, so:
     \hskip-\linewidth \hskip-\@totalleftmargin \hskip\columnwidth}%
 \MakeFramed {\advance\hsize-\width
   \@totalleftmargin\z@ \linewidth\hsize
   \@setminipage}}%
 {\par\unskip\endMakeFramed%
 \at@end@of@kframe}
\makeatother

\definecolor{shadecolor}{rgb}{.97, .97, .97}
\definecolor{messagecolor}{rgb}{0, 0, 0}
\definecolor{warningcolor}{rgb}{1, 0, 1}
\definecolor{errorcolor}{rgb}{1, 0, 0}
\newenvironment{knitrout}{}{} % an empty environment to be redefined in TeX

\usepackage{alltt}

%revised version started: 12/17/16

%NEEDS: allbib.bib - already added to the master bibliography file.
%cited references only: bibexport -o mhTMP.bib main1-blx.aux
%PLUS sramh-img*, sramh.tex

%added stuff
\newcommand{\add}[1]{\textcolor{blue}{#1}}
%deleted stuff
\newcommand{\del}[1]{\textcolor{red}{(removed: #1)}}
%uncomment these to turn off colors
\renewcommand{\add}[1]{#1}
\renewcommand{\del}[1]{}

%shortcuts
\newcommand{\w}{\ili{Welsh}}
\newcommand{\e}{\ili{English}}
\newcommand{\io}{Input Optimization}




 \newcommand{\hand}{\ding{43}}
% \newcommand{\rot}[1]{\begin{rotate}{90}#1\end{rotate}} %shortcut for angled text%  
% \newcommand{\rotcon}[1]{\raisebox{-5ex}{\hspace*{1ex}\rot{\hspace*{1ex}#1}}}

%% add all extra packages you need to load to this file 
% \usepackage{todo} %% removed,cna use todonotes instead. % Jason reactivated
% \usepackage{graphicx} % not needed because forest loads tikz, which loads graphicx
\usepackage{tabularx}
\usepackage{amsmath} 
\usepackage{multicol}
\usepackage{lipsum}
\usepackage{longtable}
\usepackage{booktabs}
\usepackage[normalem]{ulem}
%\usepackage{tikz} % not needed because forest loads tikz
\usepackage{phonrule} % for SPE-style phonological rules
\usepackage{pst-all} % loads the main pstricks tools; for arrow diagrams in Hale.tex
%\usepackage{leipzig} % for gloss abbreviations
\usepackage[% for automatic cross-referencing
compress,%
capitalize,% labels are always capitalized in LSP style
noabbrev]% labels are always spelled out in LSP style
{cleveref}

% based on http://tex.stackexchange.com/a/318983/42880 for using gb4e examples with cleveref
\crefname{xnumi}{}{}
\creflabelformat{xnumi}{(#2#1#3)}
\crefrangeformat{xnumi}{(#3#1#4)--(#5#2#6)}
\crefname{xnumii}{}{}
\creflabelformat{xnumii}{(#2#1#3)}
\crefrangeformat{xnumii}{(#3#1#4)--(#5#2#6)}

%\usepackage[notcite,notref]{showkeys} %%removed, not helping CB.
%\usepackage{showidx} %%remove for final compiling - shows index keys at top of page.
 
\usepackage{langsci/styles/langsci-gb4e}  
 \usepackage{pifont}
% % OT tableaux                                                
% \usepackage{pstricks,colortab}  
\usepackage{multirow} % used in OT tableaux
\usepackage{rotating} %needed for angled text%
\usepackage{colortbl} % for cell shading
 
 \usepackage{avm}  
\usepackage[linguistics]{forest} 
\usetikzlibrary{matrix,fit} % for matrix of nodes in Kaisse and Bat-El


\usepackage{hhline}
\newcommand{\cgr}{\cellcolor[gray]{0.8}}
\newcommand{\cn}{\centering}



\newcommand{\reff}[1]{(\ref{#1})}
%\usepackage{newtxtext,newtxmath}


%\usepackage[normalem] {ulem}
\usepackage{qtree}
%\usepackage{natbib}
%\usepackage{tikz}
%\usepackage{gb4e}
\usepackage{phonrule}  
%\bibliographystyle{humannat}



\usepackage{minibox}

%\include{psheader-metr}

\def\bl#1{$_{\textrm{{\footnotesize #1}}}$}
\usepackage{arydshln}
\usepackage{rotating}

%%add all your local new commands to this file

\newcommand{\form}[1]{\mbox{\emph{#1}}}
\newcommand{\uf}[1]{\mbox{/#1/}}

% borrowed from expex and converted from plan tex to latex
\newcommand{\judge}[1]{{\upshape #1\hspace{0.1em}}}
\newcommand{\ljudge}[1]{\makebox[0pt][r]{\judge{#1}}}

\newcommand\tikzmark[1]{\tikz[remember picture, baseline=(#1.base)] \node[anchor=base,inner sep=0pt, outer sep=0pt] (#1) {#1};} % for adding decorations, arrows, lines, etc. to text
\newcommand\tikzmarknamed[2]{\tikz[remember picture, baseline=(#1.base)] \node[anchor=base,inner sep=0pt, outer sep=0pt] (#1) {#2};} % for adding decorations, arrows, lines, etc. to text
\newcommand\tikzmarkfullnamed[2]{\tikz[remember picture, baseline=(#1.base)] \node[anchor=base,inner sep=0pt, outer sep=0pt] (#1) {\vphantom{X}#2};} % for adding decorations, arrows, lines, etc. to text; this one works best for decorations above a line of text because it adds in the heigh of a capital letter and takes two arguments - one for the node name and one for the printed text

\newcommand{\sub}[1]{$_{\text{#1}}$} % for non-math subscripts
\newcommand{\subit}[1]{\sub{\textit{#1}}} % for italics non-math subscripts
\newcommand{\1}{\rlap{$'$}\xspace} % for the prime in X' (the \rlap command allows the prime to be ignored for horizontal spacing in trees, and the \xspace command allows you to use this in normal text without adding \ afterwards). This isn't crucial, but it helps the formatting to look a little better.

% Aissen:
\newcommand\tikzmarkfull[1]{\tikz[remember picture, baseline=(#1.base)] \node[anchor=base,inner sep=0pt, outer sep=0pt] (#1) {\vphantom{X}#1};} % for adding decorations, arrows, lines, etc. to text; this one works best for decorations above a line of text because it adds in the heigh of a capital letter and takes one argument that serves as the name and the printed text
\newcommand{\bridgeover}[2]{% for a line that creates a bridge over text, connecting two nodes
	\begin{tikzpicture}[remember picture,overlay]
	\draw[thick,shorten >=3pt,shorten <=3pt] (#1.north) |- +(0ex,2.5ex) -| (#2.north);
	\end{tikzpicture}
}
\newcommand{\bridgeoverht}[3]{% for a line that creates a bridge over text, connecting two nodes and specifing the height of the bridge
	\begin{tikzpicture}[remember picture,overlay]
	\draw[thick,shorten >=3pt,shorten <=3pt] (#2.north) |- +(0ex,#1) -| (#3.north);
	\end{tikzpicture}
}
\newcommand{\bridgeoverex}{\vspace*{3ex}} % place before an example that has a \bridgeover so that there is enough vertical space for it

% Chung:
\newcommand{\lefttabular}[1]{\begin{tabular}{p{0.5in}}#1\end{tabular}}

% Kaisse:
\newcommand{\mgmorph}[1]{|(#1)| {#1}}
\newcommand{\mgone}[2][$\times$]{\node at (#2.base) [above=2ex] (1#2) {\vphantom{X}#1};}
\newcommand{\mgtwo}[2][$\times$]{\mgone{#2} \node at (#2.base) [above=4.5ex] (2#2) {\vphantom{X}#1};}
\newcommand{\mgthree}[2][$\times$]{\mgtwo{#2} \node at (#2.base) [above=7ex] (3#2) {\vphantom{X}#1};}
\newcommand{\mgftl}[1]{\node at (1#1) [left] {(};}
\newcommand{\mgftr}[1]{\node at (1#1) [right] {)};}
\newcommand{\mgfoot}[2]{\mgftl{#1}\mgftr{#2}}
\newcommand{\mgldelim}[2]{\node at (#2.west) [left,inner sep = 0pt, outer sep = 0pt] {#1};}
\newcommand{\mgrdelim}[2]{\node at (#2.east) [right,inner sep = 0pt, outer sep = 0pt] {#1};}

\newcommand{\squish}{\hspace*{-3pt}}

% Kavitskaya:
\newcommand{\assoc}[2]{\draw (#1.south) -- (#2.north);}
\newcolumntype{L}{>{\raggedright\arraybackslash}X}

% Lepic & Padden:
\newcommand{\fitpic}[1]{\resizebox{\hsize}{!}{\includegraphics{#1}}} % from http://tex.stackexchange.com/a/148965/42880
\newcommand{\signpic}[1]{\includegraphics[width=1.36in]{#1}}
\newcolumntype{C}{>{\centering\arraybackslash}X}

% Spencer:

\newcommand{\textex}[1]{\textit{#1}\xspace}
\newcommand{\lxm}[1]{\textsc{#1}\xspace}

% Thrainsson:

\renewcommand{\textasciitilde}{\char`~} % for use with TTF/OTF fonts (see comments on http://tex.stackexchange.com/a/317/42880)
\newcommand{\tikzarrow}[2]{% for an arrow connecting two nodes
\begin{tikzpicture}[remember picture,overlay]
\draw[thick,shorten >=3pt,shorten <=3pt,->,>=stealth] (#1) -- (#2);
\end{tikzpicture}
}

\newlength{\padding}
\setlength{\padding}{0.5em}
\newcommand{\lesspadding}{\hspace*{-\padding}}
\newcommand{\feat}[1]{\lesspadding#1\lesspadding}

% Hammond

\usepackage[]{graphicx}\usepackage[]{xcolor}
%% maxwidth is the original width if it is less than linewidth
%% otherwise use linewidth (to make sure the graphics do not exceed the margin)
\makeatletter
\def\maxwidth{ %
  \ifdim\Gin@nat@width>\linewidth
    \linewidth
  \else
    \Gin@nat@width
  \fi
}
\makeatother

\definecolor{fgcolor}{rgb}{0.345, 0.345, 0.345}
\newcommand{\hlnum}[1]{\textcolor[rgb]{0.686,0.059,0.569}{#1}}%
\newcommand{\hlstr}[1]{\textcolor[rgb]{0.192,0.494,0.8}{#1}}%
\newcommand{\hlcom}[1]{\textcolor[rgb]{0.678,0.584,0.686}{\textit{#1}}}%
\newcommand{\hlopt}[1]{\textcolor[rgb]{0,0,0}{#1}}%
\newcommand{\hlstd}[1]{\textcolor[rgb]{0.345,0.345,0.345}{#1}}%
\newcommand{\hlkwa}[1]{\textcolor[rgb]{0.161,0.373,0.58}{\textbf{#1}}}%
\newcommand{\hlkwb}[1]{\textcolor[rgb]{0.69,0.353,0.396}{#1}}%
\newcommand{\hlkwc}[1]{\textcolor[rgb]{0.333,0.667,0.333}{#1}}%
\newcommand{\hlkwd}[1]{\textcolor[rgb]{0.737,0.353,0.396}{\textbf{#1}}}%
\let\hlipl\hlkwb

\usepackage{framed}
\makeatletter
\newenvironment{kframe}{%
 \def\at@end@of@kframe{}%
 \ifinner\ifhmode%
  \def\at@end@of@kframe{\end{minipage}}%
  \begin{minipage}{\columnwidth}%
 \fi\fi%
 \def\FrameCommand##1{\hskip\@totalleftmargin \hskip-\fboxsep
 \colorbox{shadecolor}{##1}\hskip-\fboxsep
     % There is no \\@totalrightmargin, so:
     \hskip-\linewidth \hskip-\@totalleftmargin \hskip\columnwidth}%
 \MakeFramed {\advance\hsize-\width
   \@totalleftmargin\z@ \linewidth\hsize
   \@setminipage}}%
 {\par\unskip\endMakeFramed%
 \at@end@of@kframe}
\makeatother

\definecolor{shadecolor}{rgb}{.97, .97, .97}
\definecolor{messagecolor}{rgb}{0, 0, 0}
\definecolor{warningcolor}{rgb}{1, 0, 1}
\definecolor{errorcolor}{rgb}{1, 0, 0}
\newenvironment{knitrout}{}{} % an empty environment to be redefined in TeX

\usepackage{alltt}

%revised version started: 12/17/16

%NEEDS: allbib.bib - already added to the master bibliography file.
%cited references only: bibexport -o mhTMP.bib main1-blx.aux
%PLUS sramh-img*, sramh.tex

%added stuff
\newcommand{\add}[1]{\textcolor{blue}{#1}}
%deleted stuff
\newcommand{\del}[1]{\textcolor{red}{(removed: #1)}}
%uncomment these to turn off colors
\renewcommand{\add}[1]{#1}
\renewcommand{\del}[1]{}

%shortcuts
\newcommand{\w}{\ili{Welsh}}
\newcommand{\e}{\ili{English}}
\newcommand{\io}{Input Optimization}




 \newcommand{\hand}{\ding{43}}
% \newcommand{\rot}[1]{\begin{rotate}{90}#1\end{rotate}} %shortcut for angled text%  
% \newcommand{\rotcon}[1]{\raisebox{-5ex}{\hspace*{1ex}\rot{\hspace*{1ex}#1}}}

%% add all extra packages you need to load to this file 
% \usepackage{todo} %% removed,cna use todonotes instead. % Jason reactivated
% \usepackage{graphicx} % not needed because forest loads tikz, which loads graphicx
\usepackage{tabularx}
\usepackage{amsmath} 
\usepackage{multicol}
\usepackage{lipsum}
\usepackage{longtable}
\usepackage{booktabs}
\usepackage[normalem]{ulem}
%\usepackage{tikz} % not needed because forest loads tikz
\usepackage{phonrule} % for SPE-style phonological rules
\usepackage{pst-all} % loads the main pstricks tools; for arrow diagrams in Hale.tex
%\usepackage{leipzig} % for gloss abbreviations
\usepackage[% for automatic cross-referencing
compress,%
capitalize,% labels are always capitalized in LSP style
noabbrev]% labels are always spelled out in LSP style
{cleveref}

% based on http://tex.stackexchange.com/a/318983/42880 for using gb4e examples with cleveref
\crefname{xnumi}{}{}
\creflabelformat{xnumi}{(#2#1#3)}
\crefrangeformat{xnumi}{(#3#1#4)--(#5#2#6)}
\crefname{xnumii}{}{}
\creflabelformat{xnumii}{(#2#1#3)}
\crefrangeformat{xnumii}{(#3#1#4)--(#5#2#6)}

%\usepackage[notcite,notref]{showkeys} %%removed, not helping CB.
%\usepackage{showidx} %%remove for final compiling - shows index keys at top of page.
 
\usepackage{langsci/styles/langsci-gb4e}  
 \usepackage{pifont}
% % OT tableaux                                                
% \usepackage{pstricks,colortab}  
\usepackage{multirow} % used in OT tableaux
\usepackage{rotating} %needed for angled text%
\usepackage{colortbl} % for cell shading
 
 \usepackage{avm}  
\usepackage[linguistics]{forest} 
\usetikzlibrary{matrix,fit} % for matrix of nodes in Kaisse and Bat-El


\usepackage{hhline}
\newcommand{\cgr}{\cellcolor[gray]{0.8}}
\newcommand{\cn}{\centering}



\newcommand{\reff}[1]{(\ref{#1})}
%\usepackage{newtxtext,newtxmath}


%\usepackage[normalem] {ulem}
\usepackage{qtree}
%\usepackage{natbib}
%\usepackage{tikz}
%\usepackage{gb4e}
\usepackage{phonrule}  
%\bibliographystyle{humannat}



\usepackage{minibox}

%\include{psheader-metr}

\def\bl#1{$_{\textrm{{\footnotesize #1}}}$}
\usepackage{arydshln}
\usepackage{rotating}

%%add all your local new commands to this file

\newcommand{\form}[1]{\mbox{\emph{#1}}}
\newcommand{\uf}[1]{\mbox{/#1/}}

% borrowed from expex and converted from plan tex to latex
\newcommand{\judge}[1]{{\upshape #1\hspace{0.1em}}}
\newcommand{\ljudge}[1]{\makebox[0pt][r]{\judge{#1}}}

\newcommand\tikzmark[1]{\tikz[remember picture, baseline=(#1.base)] \node[anchor=base,inner sep=0pt, outer sep=0pt] (#1) {#1};} % for adding decorations, arrows, lines, etc. to text
\newcommand\tikzmarknamed[2]{\tikz[remember picture, baseline=(#1.base)] \node[anchor=base,inner sep=0pt, outer sep=0pt] (#1) {#2};} % for adding decorations, arrows, lines, etc. to text
\newcommand\tikzmarkfullnamed[2]{\tikz[remember picture, baseline=(#1.base)] \node[anchor=base,inner sep=0pt, outer sep=0pt] (#1) {\vphantom{X}#2};} % for adding decorations, arrows, lines, etc. to text; this one works best for decorations above a line of text because it adds in the heigh of a capital letter and takes two arguments - one for the node name and one for the printed text

\newcommand{\sub}[1]{$_{\text{#1}}$} % for non-math subscripts
\newcommand{\subit}[1]{\sub{\textit{#1}}} % for italics non-math subscripts
\newcommand{\1}{\rlap{$'$}\xspace} % for the prime in X' (the \rlap command allows the prime to be ignored for horizontal spacing in trees, and the \xspace command allows you to use this in normal text without adding \ afterwards). This isn't crucial, but it helps the formatting to look a little better.

% Aissen:
\newcommand\tikzmarkfull[1]{\tikz[remember picture, baseline=(#1.base)] \node[anchor=base,inner sep=0pt, outer sep=0pt] (#1) {\vphantom{X}#1};} % for adding decorations, arrows, lines, etc. to text; this one works best for decorations above a line of text because it adds in the heigh of a capital letter and takes one argument that serves as the name and the printed text
\newcommand{\bridgeover}[2]{% for a line that creates a bridge over text, connecting two nodes
	\begin{tikzpicture}[remember picture,overlay]
	\draw[thick,shorten >=3pt,shorten <=3pt] (#1.north) |- +(0ex,2.5ex) -| (#2.north);
	\end{tikzpicture}
}
\newcommand{\bridgeoverht}[3]{% for a line that creates a bridge over text, connecting two nodes and specifing the height of the bridge
	\begin{tikzpicture}[remember picture,overlay]
	\draw[thick,shorten >=3pt,shorten <=3pt] (#2.north) |- +(0ex,#1) -| (#3.north);
	\end{tikzpicture}
}
\newcommand{\bridgeoverex}{\vspace*{3ex}} % place before an example that has a \bridgeover so that there is enough vertical space for it

% Chung:
\newcommand{\lefttabular}[1]{\begin{tabular}{p{0.5in}}#1\end{tabular}}

% Kaisse:
\newcommand{\mgmorph}[1]{|(#1)| {#1}}
\newcommand{\mgone}[2][$\times$]{\node at (#2.base) [above=2ex] (1#2) {\vphantom{X}#1};}
\newcommand{\mgtwo}[2][$\times$]{\mgone{#2} \node at (#2.base) [above=4.5ex] (2#2) {\vphantom{X}#1};}
\newcommand{\mgthree}[2][$\times$]{\mgtwo{#2} \node at (#2.base) [above=7ex] (3#2) {\vphantom{X}#1};}
\newcommand{\mgftl}[1]{\node at (1#1) [left] {(};}
\newcommand{\mgftr}[1]{\node at (1#1) [right] {)};}
\newcommand{\mgfoot}[2]{\mgftl{#1}\mgftr{#2}}
\newcommand{\mgldelim}[2]{\node at (#2.west) [left,inner sep = 0pt, outer sep = 0pt] {#1};}
\newcommand{\mgrdelim}[2]{\node at (#2.east) [right,inner sep = 0pt, outer sep = 0pt] {#1};}

\newcommand{\squish}{\hspace*{-3pt}}

% Kavitskaya:
\newcommand{\assoc}[2]{\draw (#1.south) -- (#2.north);}
\newcolumntype{L}{>{\raggedright\arraybackslash}X}

% Lepic & Padden:
\newcommand{\fitpic}[1]{\resizebox{\hsize}{!}{\includegraphics{#1}}} % from http://tex.stackexchange.com/a/148965/42880
\newcommand{\signpic}[1]{\includegraphics[width=1.36in]{#1}}
\newcolumntype{C}{>{\centering\arraybackslash}X}

% Spencer:

\newcommand{\textex}[1]{\textit{#1}\xspace}
\newcommand{\lxm}[1]{\textsc{#1}\xspace}

% Thrainsson:

\renewcommand{\textasciitilde}{\char`~} % for use with TTF/OTF fonts (see comments on http://tex.stackexchange.com/a/317/42880)
\newcommand{\tikzarrow}[2]{% for an arrow connecting two nodes
\begin{tikzpicture}[remember picture,overlay]
\draw[thick,shorten >=3pt,shorten <=3pt,->,>=stealth] (#1) -- (#2);
\end{tikzpicture}
}

\newlength{\padding}
\setlength{\padding}{0.5em}
\newcommand{\lesspadding}{\hspace*{-\padding}}
\newcommand{\feat}[1]{\lesspadding#1\lesspadding}

% Hammond

\usepackage[]{graphicx}\usepackage[]{xcolor}
%% maxwidth is the original width if it is less than linewidth
%% otherwise use linewidth (to make sure the graphics do not exceed the margin)
\makeatletter
\def\maxwidth{ %
  \ifdim\Gin@nat@width>\linewidth
    \linewidth
  \else
    \Gin@nat@width
  \fi
}
\makeatother

\definecolor{fgcolor}{rgb}{0.345, 0.345, 0.345}
\newcommand{\hlnum}[1]{\textcolor[rgb]{0.686,0.059,0.569}{#1}}%
\newcommand{\hlstr}[1]{\textcolor[rgb]{0.192,0.494,0.8}{#1}}%
\newcommand{\hlcom}[1]{\textcolor[rgb]{0.678,0.584,0.686}{\textit{#1}}}%
\newcommand{\hlopt}[1]{\textcolor[rgb]{0,0,0}{#1}}%
\newcommand{\hlstd}[1]{\textcolor[rgb]{0.345,0.345,0.345}{#1}}%
\newcommand{\hlkwa}[1]{\textcolor[rgb]{0.161,0.373,0.58}{\textbf{#1}}}%
\newcommand{\hlkwb}[1]{\textcolor[rgb]{0.69,0.353,0.396}{#1}}%
\newcommand{\hlkwc}[1]{\textcolor[rgb]{0.333,0.667,0.333}{#1}}%
\newcommand{\hlkwd}[1]{\textcolor[rgb]{0.737,0.353,0.396}{\textbf{#1}}}%
\let\hlipl\hlkwb

\usepackage{framed}
\makeatletter
\newenvironment{kframe}{%
 \def\at@end@of@kframe{}%
 \ifinner\ifhmode%
  \def\at@end@of@kframe{\end{minipage}}%
  \begin{minipage}{\columnwidth}%
 \fi\fi%
 \def\FrameCommand##1{\hskip\@totalleftmargin \hskip-\fboxsep
 \colorbox{shadecolor}{##1}\hskip-\fboxsep
     % There is no \\@totalrightmargin, so:
     \hskip-\linewidth \hskip-\@totalleftmargin \hskip\columnwidth}%
 \MakeFramed {\advance\hsize-\width
   \@totalleftmargin\z@ \linewidth\hsize
   \@setminipage}}%
 {\par\unskip\endMakeFramed%
 \at@end@of@kframe}
\makeatother

\definecolor{shadecolor}{rgb}{.97, .97, .97}
\definecolor{messagecolor}{rgb}{0, 0, 0}
\definecolor{warningcolor}{rgb}{1, 0, 1}
\definecolor{errorcolor}{rgb}{1, 0, 0}
\newenvironment{knitrout}{}{} % an empty environment to be redefined in TeX

\usepackage{alltt}

%revised version started: 12/17/16

%NEEDS: allbib.bib - already added to the master bibliography file.
%cited references only: bibexport -o mhTMP.bib main1-blx.aux
%PLUS sramh-img*, sramh.tex

%added stuff
\newcommand{\add}[1]{\textcolor{blue}{#1}}
%deleted stuff
\newcommand{\del}[1]{\textcolor{red}{(removed: #1)}}
%uncomment these to turn off colors
\renewcommand{\add}[1]{#1}
\renewcommand{\del}[1]{}

%shortcuts
\newcommand{\w}{\ili{Welsh}}
\newcommand{\e}{\ili{English}}
\newcommand{\io}{Input Optimization}




 \newcommand{\hand}{\ding{43}}
% \newcommand{\rot}[1]{\begin{rotate}{90}#1\end{rotate}} %shortcut for angled text%  
% \newcommand{\rotcon}[1]{\raisebox{-5ex}{\hspace*{1ex}\rot{\hspace*{1ex}#1}}}

%\input{localpackages.tex}
\usepackage{arydshln}
\usepackage{rotating}

%\input{localcommands.tex}
\newcommand{\tworow}[1]{\multirow{2}{*}{#1}}


\newcommand{\tworow}[1]{\multirow{2}{*}{#1}}


\newcommand{\tworow}[1]{\multirow{2}{*}{#1}}



\title{Rules and blocks}

\author{%
Gregory Stump\affiliation{University of Kentucky}
}

% \chapterDOI{} %will be filled in at production
% \epigram{}

\abstract{
In a series of publications, Stephen Anderson developed the idea that the definition of a language’s inflectional morphology involves blocks of realization rules such that (i) realization rules’ order of application follows from the ordering of the blocks to which they belong and (ii) realization rules belonging to the same block stand in a relation of paradigmatic opposition. A question that naturally arises from this conception of rule interaction is whether it is possible for the same rule to figure in the application of more than one block.  I discuss two systems of verb inflection exploiting exactly this possibility -- those of Limbu and Southern Sotho.  In order to account for the special properties of such systems, I argue that in the definition of a language’s inflectional morphology, one rule may be dependent upon another, and that in such cases, the dependent rule may figure in the application of more than one block precisely because the “carrier” rules on which it is dependent differ in their block membership.  In formal terms, this means that the definition of a language’s inflectional morphology may draw upon principles of rule conflation by which a dependent realization rule combines with its carrier rule to form a single, more complex rule, typically occupying the same block as the carrier rule.  I further show that there is considerable independent motivation for the postulation of these principles.
}

\begin{document}
\maketitle

\section{Introduction}

In a series\footnote{Key references include \citealt{Anderson1977a,Anderson1982,Anderson1984,Anderson1984b,Anderson1986}.} of articles culminating in his 1992 monograph \is{A-morphous Morphology}\textit{A\nobreakdash-morphous Morphology}, Stephen Anderson developed a model for the precise inferential-realizational definition of complex inflectional\is{inflection} patterns.\footnote{In the typology of morphological theories proposed by \citet{Stump2001}, a theory is inferential if it employs rules to infer the form of a language’s words and stems from that of less complex stems; a theory is realizational if its definition of a language’s morphology takes a word’s content as logically antecedent to its form.}

Two central principles of this model are (1) and (2).  According to (1), the definition of the Latin word form \textit{laud\=a-ba-nt-ur} ‘they were being praised’ involves the realization of a \is{morphosyntax}morphosyntactic property set through the interaction of ordered rule \is{blocking}blocks.  One of these houses a rule realizing the imperfect indicative through the \isi{suffixation} of \textit{{}-b\=a}; this is followed by a block housing a rule realizing third-person plural subject agreement by the suffixation of \textit{{}-nt}; this, in turn, is followed by a block containing a rule realizing \isi{passive} voice through the suffixation of \textit{-ur}.  The successive application of these \isi{rule blocks} infers the word form \textit{laud\=abantur} from the stem \textit{laud\=a-} as the realization of the property set \{3 pl imperf ind pass\} in the \isi{paradigm} of the lexeme \textsc{laud\=are} ‘praise’.

\ea%1
	   A language’s inflectional\is{inflection} rules are organized into ordered blocks such that two rules’ order of application depends on the ordering of the blocks to which they belong.
\z

\ea%2
Rules belonging to the same block are disjunctive:  at most one rule per block applies in the realization of a given word form.  In general, competition between rules is resolved in favor of the rule with the narrower domain of application. 
    \z

  According to (2), the definition of a word form by a sequence of rule blocks involves the application of at most one rule per block.  Two rules belonging to the same block may be defined so as to apply in disjoint contexts; for instance, the rule realizing third-person plural subject agreement through the suffixation of \textit{{}-nt} realizes different property sets from the rule realizing first-person plural subject agreement throught the suffixation of \textit{{}-mus}.  But it can also happen that two rules belonging to the same block are both in principle applicable in the same context; for instance, \textit{{}-\=\i} and \textit{{}-\=o} both realize first-person singlar subject agreement and might therefore be seen as entering into competition in the realization of certain forms.  Given that the \textit{{}-\=\i} rule\textit{} applies only in the first-person singular perfect indicative active (e.g.\textit{} \textit{laud\=av\=\i} ‘I have praised’), its domain of application is narrower than that of the \textit{{}-\=o} rule, which apparently applies as a default, surfacing in the present and future indicative active (\textit{laud\=o} ‘I praise’, \textit{laud\=ab\=o} ‘I will praise’) and passive (\textit{laudor} ‘I am praised’, \textit{laud\=abor} ‘I will be praised’) as well as in the future perfect indicative active (\textit{laud\=aver\=o} ‘I will have praised’); accordingly, the \textit{{}-\=\i} rule overrides the \textit{{}-\=o} rule in the realization of the first person singular perfect indicative active.

Anderson’s model has afforded the most plausible existing accounts of a diverse range of inflectional systems (see, for example, the analyses of Potawatomi and Georgian in \citealt{Anderson1977a}, 1984a, 1986 and that of German in \citealt{Zwicky1985a}), and it continues to raise important theoretical questions. One such question is whether the same rule\footnote{Two rule applications are seen as involving the “same rule” if they realize the same morphosyntactic content by means of the same exponent even if they introduce that exponent into different positions.} may figure in the application of more than one rule block.  I argue here that in a particular class of cases, this is precisely what happens.  

In the cases in question, there is always a relation of dependency among particular rules.  \citet{Harris2017} describes relations of this sort in affixal terms as involving a dependent affix that only appears in the presence of an available carrier affix.  Adopting and extending her terminology, I describe such relations as involving a dependent rule that only applies in combination with an available carrier rule.  As I show, a dependent rule may figure in the application of more than one block if the rules on which it is dependent differ in their block membership.  Instances of this sort are of two kinds.

First, there are instances of multiple exponence in which the same rule of affixation apparently applies in more than one block in a word form’s inflectional realization; an example from the Limbu language [Kiranti; Nepal] is the multiple exponence of certain agent concord properties in the inflection of transitive verbs.  Second, there are instances of polyfunctionality involving a rule of affixation whose function varies systematically according to the block in which it applies; an example is the polyfunctionality of concordial affixes in the verbal inflection of the Bantu languages.  In order to account for cases of these two sorts, it is desirable to supplement (1) and (2) with principles (3) and (4).

\ea%3
    \label{ex:3}
 A dependent rule may be conflated with a carrier rule to produce a more complex rule.  Where a dependent rule R$_{1}$ realizes property set $\sigma $ by means of exponent \textit{x} and its carrier rule R$_{2}$ realizes property set $\tau $ by means of exponent \textit{y}, the conflation of R$_{1}$ with R$_{2}$ is intuitively a rule realizing the property set $\sigma $ ${\cup}$ $\tau $ by means of the combined exponents \textit{x} and \textit{y}.
\z

\ea%4
    \label{ex:4}
   A rule block may contain both simple and conflated rules.
\z

  As I shall show, these principles afford economical accounts of multiple exponence in Limbu verbs and of polyfunctional verbal concord markers in Southern Sotho [Bantu; Lesotho].  After describing the expression of agent properties in Limbu verbs (§2) and the Southern Sotho system of verbal concord (§3), I propose a formal framework for rule conflation in inflectional morphology (§4).  I then present explicit theories of the observed pattern of multiple exponence in Limbu (§5) and that of polyfunctional concordial morphology in Southern Sotho (§6).  I conclude with some observations about the wider importance of rule conflation in an adequate theory of morphology (§7).

\section{Multiple exponence in the expression of agent inflection in Limbu verbs}

In Limbu, two agent concord suffixes participate in relations of multiple exponence:  \textit{\nobreakdash-ŋ}, an expression of first\nobreakdash-person singular agent concord, and \textit{\nobreakdash-m}, an expression of nonthird\nobreakdash-person plural agent concord. \tabref{tab:1} exemplifies the distribution of these suffixes in positive forms of \textsc{huʔmaʔ} ‘teach’.\footnote{The structure of \tabref{tab:1} should be carefully noted. Each row in the table is occupied by a different word form in the paradigm of the~Limbu verb \textsc{huʔmaʔ} ‘teach’. Each word is in exploded form, with its parts arranged in columns corresponding to the affix position classes postulated by van Driem. (I follow him in labeling these classes \textbf{pf1} and \textbf{sf1}–\textbf{sf10}.) Thus, the word form in the 1s~→ 3ns row of the nonpreterite part of the table is \textit{huʔr}{}-\textit{u}{}-\textit{ŋ}{}-\textit{si}{}-\textit{ŋ} ‘I teach them’. This table does not comprise the complete paradigm of \textsc{huʔmaʔ} ‘teach’, but encompasses those forms that involve the~ agent suffixes \textit{{}-ŋ} and \textit{{}-m} (as well as a few other pertinent forms in which the appearance of these suffixes is overridden). The claim that these suffixes appear in two different positions means that they appear in two different columns, since each column defines an affix position class in the traditional sense.}   (In this table, parenthesized segments are superficially elided in prevocalic position by an ordinary phonological process.)  Both suffixes appear in two different affix positions;  \citet{Driem1987} labels these positions \textbf{sf5} and \textbf{sf9}.\footnote{Affix positions \textbf{sf3} and \textbf{sf6} are missing from \tabref{tab:1} because the affixes that appear in these positions don’t occur in forms having a first\nobreakdash-person singular agent or a nonthird\nobreakdash-person plural agent.}

\begin{table}[ht]
\begin{tabular}{lllllllllllll}
\lsptoprule
& agent & \multicolumn{2}{c}{\bfseries pf1} & & \multicolumn{8}{c}{\bfseries sf}\\ \cline{3-4} \cline{6-13}
& → patient  & a & b & stem\textsuperscript{1} & \bfseries 1 & \bfseries 2 & \bfseries 4 & \bfseries 5 & \bfseries 7 & \bfseries 8 & \bfseries 9 & \bfseries 10\\ \hline
& 1s → 2s &  &  & \itshape huʔ & \itshape nɛ &  &  & \cgr &  &  & \cgr  & \\
& 1s → 2d &  &  & \itshape huʔ & \itshape nɛ &  &  & \cgr &  & \textit{ci}\textsuperscript{3} & \cgr \itshape ŋ & \\
\multirow{5}{*}{\rotatebox{90}{Nonpreterite}} & 1s → 2p &  &  & \itshape huʔ & \itshape n(ɛ) &  &  & \cgr &  & \itshape i & \cgr \itshape ŋ & \\
& 1s → 3s &  &  & \itshape huʔr &  &  & \itshape u & \cgr \itshape ŋ &  &  & \cgr & \\
& 1s → 3ns &  &  & \itshape huʔr &  &  & \itshape u & \cgr \itshape ŋ &  & \itshape si & \cgr \itshape ŋ & \\
\hhline{~------------} & 1pi → 3s & \itshape a &  & \itshape huʔr &  &  & \itshape u & \cgr \itshape m &  &  & \cgr & \\
& 1pi → 3ns & \itshape a &  & \itshape huʔr &  &  & \itshape u & \cgr \itshape m &  & \itshape si & \cgr \itshape m & \\
\hhline{~------------} & 1pe → 2 &  &  & \itshape huʔ & \itshape nɛ &  &  & \cgr & \itshape ci &  & \cgr  & \itshape ge\\
& 1pe → 3s &  &  & \itshape huʔr &  &  & \itshape u & \cgr \itshape m &  &  & \cgr  & \textit{be}\textsuperscript{4}\\
& 1pe → 3ns &  &  & \itshape huʔr &  &  & \itshape u & \cgr \itshape m &  & \itshape si & \cgr \itshape m & \textit{be}\textsuperscript{4}\\
\hhline{~------------} & 2 → 1 & \itshape a & \textit{gɛ}\textsuperscript{2} & \itshape huʔ &  &  &  & \cgr &  &  & \cgr  & \\
& 2p → 3s &  & \itshape kɛ & \itshape huʔr &  &  & \itshape u & \cgr \itshape m &  &  &  \cgr & \\
& 2p → 3ns &  & \itshape kɛ & \itshape huʔr &  &  & \itshape u & \cgr \itshape m &  & \itshape si & \cgr \itshape m & \\ \hline
& 1s → 2s &  &  & \itshape huʔ & \itshape n(ɛ) & \itshape ɛ &  & \cgr &  &  & \cgr  & \\
& 1s → 2d &  &  & \itshape huʔ & \itshape n(ɛ) & \itshape ɛ &  & \cgr &  & \textit{ci}\textsuperscript{3} & \cgr \itshape ŋ & \\
\multirow{4}{*}{\rotatebox{90}{Preterite}} & 1s → 2p &  &  & \itshape huʔ & \itshape n(ɛ) & \itshape (ɛ) &  & \cgr &  & \itshape i & \cgr \itshape ŋ & \\
& 1s → 3s &  &  & \itshape huʔr &  & \itshape (ɛ) & \itshape u & \cgr \itshape ŋ &  &  & \cgr  & \\
& 1s → 3ns &  &  & \itshape huʔr &  & \itshape (ɛ) & \itshape u & \cgr \itshape ŋ &  & \itshape si & \cgr \itshape ŋ & \\
\hhline{~------------} & 1pi → 3s & \itshape a &  & \itshape huʔr &  & \itshape (ɛ) & \itshape u & \cgr \itshape m &  &  & \cgr  & \\
& 1pi → 3ns & \itshape a &  & \itshape huʔr &  & \itshape (ɛ) & \itshape u & \cgr \itshape m &  & \itshape si & \cgr \itshape m & \\
\hhline{~------------} & 1pe → 2 &  &  & \itshape huʔ & \itshape n(ɛ) & \itshape ɛ &  & \cgr & \itshape ci &  & \cgr & \itshape ge\\
& 1pe → 3s &  &  & \itshape huʔ &  &  &  & \cgr  & \itshape mʔna &  & \cgr & \\
& 1pe → 3ns &  &  & \itshape huʔ &  &  &  & \cgr  & \itshape mʔna & \itshape si & \cgr  & \\
\hhline{~------------} & 2 → 1 & \itshape a & \textit{gɛ}\textsuperscript{2} & \itshape huʔr &  & \itshape ɛ &  &  \cgr &  &  & \cgr  & \\
& 2p → 3s &  & \itshape kɛ & \itshape huʔr &  &  & \itshape u & \cgr \itshape m &  &  & \cgr & \\
& 2p → 3ns &  & \itshape kɛ & \itshape huʔr &  &  & \itshape u & \cgr \itshape m &  & \itshape si & \cgr \itshape m & \\
\lspbottomrule
\end{tabular}
\caption{The agent suffixes \nobreakdash-ŋ and \nobreakdash-m in positive forms of the Limbu verb \textsc{huʔmaʔ} ‘teach’}
\label{tab:1}
1. \textit{huʔr} is a prevocalic alternant of \textit{huʔ.}

2. \textit{gɛ} is an alternant of \textit{kɛ} \citep[2]{Driem1987}

3. \textit{s} becomes \textit{c} after \textit{ɛ}  \citep[77]{Driem1987}

4. \textit{be} is a phonologically conditioned alternant of \textit{ge}  \citep[102]{Driem1987}
\end{table}\clearpage

The distribution of these suffixes is, in fact, doubly puzzling.  Besides participating in relations of multiple exponence, they also exhibit gaps in their distribution.  Consider first the suffix \textit{\nobreakdash-ŋ}.  Because ten of the forms in \tabref{tab:1} realize first\nobreakdash-person singular agent properties, all ten would be compatible with the appearance of the \textit{\nobreakdash-ŋ} suffix in both the \textbf{sf5} and \textbf{sf9} positions.  Yet, only two of the forms exhibit \nobreakdash-\textit{ŋ} in both positions; two exhibit it only in position \textbf{sf5}; four, only in postion \textbf{sf9}; and two lack \textit{\nobreakdash-ŋ} altogether.  Consider likewise the distribution of the nonthird\nobreakdash-person plural agent suffix \textit{\nobreakdash-m}.  Among the fourteen forms in \tabref{tab:1} that realize nonthird\nobreakdash-person plural agent properties, only five exhibit \textit{\nobreakdash-m} in both the \textbf{sf5} and \textbf{sf9} positions; five have it in the \textbf{sf5} position only; and four lack \textit{\nobreakdash-m} altogether.

A cursory examination reveals the distributional generalization accounting for these results:  \textit{\nobreakdash-ŋ} and \textit{\nobreakdash-m} appear in position \textbf{sf5} only if there is an overt affix in position \textbf{sf4}, and they appear in \textbf{sf9} only if there is an overt affix in \textbf{sf8}.  In other words, the rules that introduce \textit{\nobreakdash-ŋ} and \textit{\nobreakdash-m} in Limbu are dependent rules whose application presumes that of a carrier rule filling position \textbf{sf4} or \textbf{sf8}. Because there are carrier rules in more than one rule block, the \textit{\nobreakdash-ŋ} and \textit{\nobreakdash-m} rules may both figure in the application of more than one block.

\section{Polyfunctional concordial morphology in the verb inflection of Southern Sotho}
Typically of \ili{Bantu} languages, Southern Sotho has a rich noun\nobreakdash-class system one of whose manifestations is the inflection of verbs for the noun class of their subject and object arguments.  In the analysis proposed by  \citet{doke1985}, this system exhibits seven noun classes; these have the effect of subclassifying the third person, so that like the first and second persons, each noun class subsumes both singular and plural forms.  \tabref{tab:2} presents the inventory of prefixes by which verbs inflect for the person, number and noun class of their subject.  By a similar inventory of prefixes, transitive verbs may\footnote{Unlike the subject concords, whose use is obligatory in finite forms, the object concords are optional, generally being use to express a pronominal object rather than to express agreement with an overt object phrase \citep[242]{doke1985}.} inflect for the properties of their object; the examples in \tabref{tab:3} illustrate.  \tabref{tab:4} presents the inventories of subject\nobreakdash-coding and object\nobreakdash-coding prefixes side by side; as this table shows, the two inventories are nearly identical; the only exceptions are in the singular of the first person and of class 1, where the exponents of subject properties differ from those of the corresponding object properties.

\begin{table}[ht]
\begin{tabular}{p{1cm}p{1.5cm}ccccll}
\lsptoprule
  \bfseries Subject & \multicolumn{3}{c}{\bfseries Subject class} &  & \multicolumn{2}{c}{\bfseries Subject number}\\
\hhline{~---~--} \bfseries person & {\bfseries Doke \&} & \multicolumn{2}{c}{\bfseries Meinhof} &  & \bfseries Singular & \bfseries Plural\\
\hhline{~~--~~~} &  \bfseries Mofokeng   & \bfseries sg & \bfseries pl &  &  & \\
\hhline{----~--}
 \cn1 &  &  &  &  & \itshape k\=e\nobreakdash-tla\nobreakdash-bòna & \itshape r\=e\nobreakdash-tla\nobreakdash-bòna\\
 \cn 2 &  &  &  &  & \itshape u\nobreakdash-tla\nobreakdash-bòna & \itshape l\=e\nobreakdash-tla\nobreakdash-bòna\\
\cn 3 & \cn 1 & 1 &  2 &  & \itshape \=o\nobreakdash-tla\nobreakdash-bòna & \itshape ba\nobreakdash-tla\nobreakdash-bòna\\
&  \cn 2 & \cn  3 &  4 &  & \itshape \=o\nobreakdash-tla\nobreakdash-bòna & \itshape \=e\nobreakdash-tla\nobreakdash-bòna\\
& \cn 3 & \cn 5 &  6/10 &  & \itshape l\=e\nobreakdash-tla\nobreakdash-bòna & {\itshape a\nobreakdash-tla\nobreakdash-bòna,} \par

 \itshape      li\nobreakdash-tla\nobreakdash-bòna\\
& \cn 4 &  7 &  8 &  & \itshape s\=e\nobreakdash-tla\nobreakdash-bòna & \itshape li\nobreakdash-tla\nobreakdash-bòna\\
& \cn 5 & 9 &  10 &  & \itshape \=e\nobreakdash-tla\nobreakdash-bòna & \itshape li\nobreakdash-tla\nobreakdash-bòna\\
& \cn 6 &  14 &  6 &  & \itshape b\=o\nobreakdash-tla\nobreakdash-bòna & \itshape a\nobreakdash-tla\nobreakdash-bòna\\
\hhline{~~--~--} & 7 & \multicolumn{2}{c}{15/17}  &  \multicolumn{3}{c}{\itshape    h\=o\nobreakdash-tla\nobreakdash-bòna}  \\
\lspbottomrule
\end{tabular}
\caption{Future\nobreakdash-tense forms of Southern Sotho \textsc{bòna} ‘see’: ‘I / you / etc. will see’ \citep[207ff.]{doke1985}}
\label{tab:2}
\end{table}

\begin{table}[ht]
\begin{footnotesize}
\begin{tabular}{p{1cm}p{1.5cm}ccccll}
\lsptoprule
 {\bfseries Object}  & \multicolumn{3}{c}{\bfseries Object class} &  & \multicolumn{2}{c}{\bfseries Object number}\\
\hhline{~---~--} \bfseries person & \cn {\bfseries Doke \&} & \multicolumn{2}{c}{\bfseries Meinhof} &  & \bfseries Singular & \bfseries Plural\\
\hhline{~~--~~~} & \cn \bfseries Mofokeng  & sg & pl &  &  & \\
\hhline{----~--}
 \cn1 &  &  &  &  & \itshape ba\nobreakdash-tla\nobreakdash-m\nobreakdash-pòna & \itshape ba\nobreakdash-tla\nobreakdash-r\=e\nobreakdash-bòna\\
 \cn 2 &  &  &  &  & \itshape ba\nobreakdash-tla\nobreakdash-u\nobreakdash-bòna & \itshape ba\nobreakdash-tla\nobreakdash-l\=e\nobreakdash-bòna\\
\cn 3 & \cn 1 & 1 & 2 &  & \itshape ba\nobreakdash-tla\nobreakdash-m\=o\nobreakdash-bòna & \itshape ba\nobreakdash-tla\nobreakdash-ba\nobreakdash-bòna\\
& \cn 2 & 3 & 4 &  & \itshape ba\nobreakdash-tla\nobreakdash-\=o\nobreakdash-bòna & \itshape ba\nobreakdash-tla\nobreakdash-\=e\nobreakdash-bòna\\
& \cn 3 & 5 & 6/10 &  & \itshape ba\nobreakdash-tla\nobreakdash-l\=e\nobreakdash-bòna & {\itshape ba\nobreakdash-tla\nobreakdash-a\nobreakdash-bòna,} 

\itshape      ba\nobreakdash-tla\nobreakdash-li\nobreakdash-bòna\\
&\cn  4 & 7 & 8 &  & \itshape ba\nobreakdash-tla\nobreakdash-s\=e\nobreakdash-bòna & \itshape ba\nobreakdash-tla\nobreakdash-li\nobreakdash-bòna\\
& \cn 5 & 9 & 10 &  & \itshape ba\nobreakdash-tla\nobreakdash-\=e\nobreakdash-bòna & \itshape ba\nobreakdash-tla\nobreakdash-li\nobreakdash-bòna\\
& \cn 6 & 14 & 6 &  & \itshape ba\nobreakdash-tla\nobreakdash-b\=o\nobreakdash-bòna & \itshape ba\nobreakdash-tla\nobreakdash-a\nobreakdash-bòna\\
\hhline{~~--~--} & \cn 7 & \multicolumn{2}{c}{15/17} &   \multicolumn{3}{c}{\itshape ba\nobreakdash-tla\nobreakdash-h\=o\nobreakdash-bòna}\\
\lspbottomrule
\end{tabular}
\end{footnotesize}
\caption{Future\nobreakdash-tense forms of Southern Sotho \textsc{bòna} ‘see’: ‘they will see me / you / etc.’  \citep[242ff]{doke1985}}
\label{tab:3}
\end{table}



\begin{table}[ht]
\begin{tabular}{p{1cm}p{1.5cm}ccccllll}
\lsptoprule
  & \multicolumn{3}{c}{\bfseries Class} & &  \multicolumn{2}{c}{\bfseries Subject}  & & \multicolumn{2}{c}{\bfseries Object}\\
\hhline{~---~--~--} \bfseries Person & \cn {\bfseries Doke \&}  & \multicolumn{2}{c}{\bfseries Meinhof}  & & \bfseries sg & \bfseries pl & &  \bfseries sg & \bfseries pl\\
\hhline{~~--~~~~~~} & \cn \bfseries Mofokeng & sg & pl & & &  &  &  &  \\
\hhline{----~--~--}
\cn 1 &  &  & & & \cgr \itshape k\=e\nobreakdash- & \itshape r\=e\nobreakdash-  & & \cgr \itshape N\nobreakdash-* & \itshape r\=e\nobreakdash-\\
\cn 2 &  &  &  & & \itshape u\nobreakdash- & \itshape l\=e\nobreakdash-  & & \itshape u\nobreakdash- & \itshape l\=e\nobreakdash-\\
\cn 3 & \cn 1 & 1 & 2 & & \cgr \itshape \=o\nobreakdash- & \itshape ba\nobreakdash-  & & \cgr \itshape m\=o\nobreakdash- & \itshape ba\nobreakdash-\\
& \cn 2 & 3 & 4 & &  \itshape \=o\nobreakdash- & \itshape \=e\nobreakdash- & & \itshape \=o\nobreakdash- & \itshape \=e\nobreakdash-\\
& \cn 3 & 5 & 6/10 &   & \itshape l\=e\nobreakdash- & \itshape a\nobreakdash-, li\nobreakdash-  & & \itshape l\=e\nobreakdash- & \itshape a\nobreakdash-, li\nobreakdash-\\
& \cn 4 & 7 & 8 & & \itshape s\=e\nobreakdash- & \itshape li\nobreakdash-  & & \itshape s\=e\nobreakdash- & \itshape li\nobreakdash-\\
& \cn 5 & 9 & 10 &  & \itshape \=e\nobreakdash- & \itshape li\nobreakdash-  & & \itshape \=e\nobreakdash- & \itshape li\nobreakdash-\\
& \cn 6 & 14 & 6 &  & \itshape b\=o\nobreakdash- & \itshape a\nobreakdash- & & \itshape b\=o\nobreakdash- & \itshape a\nobreakdash-\\
\hhline{~~--~--~--} & \cn 7 &  \multicolumn{2}{c}{15/17} & &  \multicolumn{2}{c}{\itshape h\=o\nobreakdash-} & &  \multicolumn{2}{c}{\itshape h\=o\nobreakdash-}\\
 \multicolumn{10}{c}{*\textit{N} represents a homorganic nasal} \\
\lspbottomrule
\end{tabular}
\caption{Indicative verbal concords in Southern Sotho \citep[197,243]{doke1985}}

\label{tab:4}
\end{table}


The principal difference between the two inventories is morphotactic:  subject-coding prefixes occupy the position before that of tense prefixes (such as the future-tense prefix \textit{tla\nobreakdash-} in Tables 2 and 3) while object-coding prefixes occupy the position following that of tense prefixes.  Thus, the general pattern is that the prefixes in \tabref{tab:4} express properties of person, number and noun class, and that it is a prefix’s position that determines whether the properties that it expresses are subject or object properties.  Put another way, the rules introducing the noun-class concords in \tabref{tab:4} generally figure in the application of more than one rule block, expressing subject properties in one block and object properties in another.

\section{Rule conflation}
It is clear from the foregoing evidence that in the definition of a language’s inflectional morphology, the same realization rule may figure in the application of more than one rule block.  I propose that this is an effect of the phenomenon of rule conflation; in particular, I propose that when rule R figures in the application of both Blocks A and B, it is because R may conflate both with certain Block A rules and with certain Block B rules.  I represent the conflation of R$_{1}$ with R$_{2}$ as [R$_{1}$~©~R$_{2}$].

I make six essential assumptions about the definition of rule conflation.  

 \subsection{Rule-block membership} 

A conflated rule [R$_{1}$~©~R$_{2}$] belongs to the same rule block as its carrier rule R$_{2}$.  

 \subsection{Forms defined by conflated rules}

Where R$_{1}$ is a rule that affixes \textit{a} by means of operation F and R$_{2}$ is a rule that affixes \textit{b} by means of operation G, the conflated rule [R$_{1}$~©~R$_{2}$] affixes \textit{b′} by means of operation G, where \textit{b′} is the result of affixing \textit{a} to \textit{b} by means of operation F.  According to this definition, there are four logically possible patterns of conflation for rules of affixation; these are represented schematically in part (A) of \tabref{tab:5}.  The conflation of R$_{1}$ with R$_{2}$ is analogous to function composition when R$_{1}$ and R$_{2}$ both effect prefixation or when both effect suffixation.  But when R$_{1}$ is prefixational and R$_{2}$ is suffixational, the application of [R$_{1}$~©~R$_{2}$] to stem X is X\textit{ab} rather than \textit{a}X\textit{b}; and when R$_{1}$ is suffixational and R$_{2}$ is prefixational, the application of [R$_{1}$~©~R$_{2}$] to stem X is \textit{ba}X rather than \textit{b}X\textit{a}.  In these latter cases, the conflation of R$_{1}$ with R$_{2}$ cannot be likened to the mathematical notion of function composition.

\begin{table}[ht]
\begin{tabular}{p{0.2cm}cccc}
\lsptoprule
& Dependent & Carrier & {Conflated} & {[R$_{1}$~©~R$_{2}$] applied} \\
& rule R$_{1}$ & rule R$_{2}$  & rule [R$_{1}$~©~R$_{2}$]  & to stem X \\
\hline
 (A) & \textit{a}\nobreakdash-prefixation & \textit{b}\nobreakdash-prefixation & \textit{ab}\nobreakdash-prefixation & \textit{ab}X\\
& \textit{a}\nobreakdash-prefixation & \textit{b}\nobreakdash-suffixation & \textit{ab}\nobreakdash-suffixation & X\textit{ab}\\
& \textit{a}\nobreakdash-suffixation & \textit{b}\nobreakdash-prefixation & \textit{ba}\nobreakdash-prefixation & \textit{ba}X\\
& \textit{a}\nobreakdash-suffixation & \textit{b}\nobreakdash-suffixation & \textit{ba}\nobreakdash-suffixation & X\textit{ba}\\
\hline
 (B) & \textit{a}\nobreakdash-prefixation & identity function & \textit{a}\nobreakdash-prefixation & \textit{a}X\\
& \textit{a}\nobreakdash-suffixation & identity function & \textit{a}\nobreakdash-suffixation & X\textit{a}\\
\lspbottomrule
\end{tabular}
\caption{Six logical possibilities for the conflation [R$_{1}$~©~R$_{2}$] of a dependent rule R$_{1}$ with a carrier rule R$_{2}$$'$}
\label{tab:5}
\end{table}

\subsection{A conflated rule’s direction of affixation}

Whether [R$_{1}$~©~R$_{2}$] is a rule of prefixation or suffixation is uniquely determined by the properties of R$_{1}$~and R$_{2}$.  If R$_{2}$ is a rule of affixation, then the direction of affixation of [R$_{1}$~©~R$_{2}$] is that of R$_{2}$, as indicated in (ii) above; but if R$_{2}$ is a rule of significative absence,\footnote{A rule of significative absence realizes a particular property set by means of an identity function.  In a realizational theory of morphology, a rule of significative absence realizing a property set $\sigma $ overrides the overt morphology of a competing rule realizing some property set of which $\sigma $ is an extension.  (Cf. the analysis of Bulgarian verb inflection proposed by \citealt[441ff]{Stump2001})}  then the direction of affixation of [R$_{1}$~©~R$_{2}$] is that of R$_{1}$, as in part (B) of \tabref{tab:5}.

\subsection{Content realized by a conflated rule} 

If rule R$_{1}$~realizes the morphosyntactic property set $\alpha $ and rule R$_{2}$ realizes the property set $\beta $, then rule [R$_{1}$~©~R$_{2}$] realizes the combination of $\alpha $ and $\beta $.  In the simplest cases, the relevant mode of combination can simply be seen as set union:  $\alpha $ ${\cup}$ $\beta $.  But in the general case, it is preferable to regard the mode of set combination as unification;\footnote{The assumed definition of unification is as in (i); this definition depends on the assumed definition of extension in (ii).  (Cf. Gazdar et al. 1985: 27; \citealt[41]{Stump2001}.)
\begin{itemize}
\item[(i)] The \textbf{\textit{unification}} of $\rho $ and $\sigma $ [i.e. $\rho $ ${\sqcup}$ $\sigma $ ] is the smallest well\nobreakdash-formed extension of both $\rho $ and $\sigma $ .
	\begin{itemize}
	\item[]
		\begin{itemize}
		\item[\textit{Example:}] \{\textsc{\{}sbj \textsc{3 s}g\textsc{\}, \{}obj pl\}\} ${\sqcup}$ \{prs, \textsc{\{}obj 1\}\} = \{\{sbj 3 sg\}, prs, \{obj 1 pl\}\} 
		\end{itemize}
	\end{itemize}
\item[(ii)] Given two sets $\sigma $, $\tau $:  $\sigma $ is an \textbf{\textit{extension}} of $\tau $ [i.e. $\tau $ ${\sqsubseteq}$ $\sigma $ ] iff for each property x ${\in}$ $\tau $, 
	\begin{itemize}
	\item[either (i)]   x is simple property and x ${\in}$ $\sigma$ 
	\item[or (ii)] x is a complex property (= a set of properties) such that y ${\in}$ $\sigma $ and y is an extension of x. 
		\begin{itemize}
		\item[\textit{Examples:}] \{pl\} ${\sqsubseteq}$ \{1 pl\} \\ \{prs \{obj 1\}\} ${\sqsubseteq}$ \{prs \{obj 1 pl\}\}
		\end{itemize}
	\end{itemize}
\end{itemize}}
for instance, the combination of \{fut, \{sbj 3 sg\}\} with \{\{sbj fem\}\} should be the unification \{fut \{sbj 3 sg fem\}\} rather than the union \{fut, \{sbj 3 sg\}, \{sbj fem\}\}.  That is, if R$_{1}$ realizes $\alpha $ and R$_{2}$ realizes $\beta $, then [R$_{1}$~©~R$_{2}$] realizes the unification $\alpha $~${\sqcup}$~$\beta $.

\subsection{Recursion}

The definition of rule conflation does not exclude the possibility that a conflated rule might itself enter into the conflation of a still more complex rule; that is, rule conflation may be recursive.

\subsection{Nonconcatenative rules and conflation}

A priori, there is no reason why the morphological rules that enter into such conflations must necessarily be rules of affixation or of significative absence.  The most convincing cases, however, do involve rules of these two sorts, and I shall focus exclusively on such cases here.  Nevertheless, nothing that I say here should be seen as excluding the possibility that nonconcatenative rules might also enter into relations of rule conflation.

Rule conflation is an operation on rules rather than on affixes; nevertheless, if R$_{1}$ and R$_{2}$ are rules introducing the respective affixes \textit{a} and \textit{b}, one can, as a kind of shorthand, refer to the affix \textit{ab} (or \textit{ba}) introduced by the conflated rule [R$_{1}$~©~R$_{2}$] as a \textsc{conflated affix.}

As I now show, this conception of rule conflation affords a straightforward account of multiple exponence in the expression of Limbu agent inflection (§5) and of polyfunctional concordial morphology in Southern Sotho (§6).  

\section{Rule conflation and multiple exponence in Limbu}

Consider again the inflection of Limbu verbs for the properties of their agent argument.  As was seen in §2, the suffix \nobreakdash-\textit{ŋ} (expressing first\nobreakdash-person singular agent properties) and the suffix \textit{\nobreakdash-m} (expressing nonthird\nobreakdash-person plural agent properties) may appear in either of two positions—or in both—in a verb form’s inflectional morphotactics; but their appearance in either position is dependent on that of a suffix in the immediately preceding position.  The following analysis of this distributional pattern is based on two key assumptions:\footnote{ The dependent rules at issue in the proposed analyses of Limbu and Southern Sotho are only manifested in conflation with a carrier rule.  But one can also imagine that a rule might be able to function both as a dependent rule and as an independent rule; the rules introducing the Swahili relative affixes are argued to have this status in the analysis proposed by Stump (to appear a).}

\begin{itemize}
\item 
the agent\nobreakdash-coding suffixes \textit{\nobreakdash-ŋ} and \textit{\nobreakdash-m} are introduced by dependent rules that only apply in conflation with another, “carrier” rule, and
\item 
carrier rules for the \textit{\nobreakdash-ŋ} and \textit{\nobreakdash-m} rules exist in more than one block.  
\end{itemize}

This analysis\footnote{The following abbreviations are employed for the morphosyntactic properties in this analysis:  agt = agent, pat = patient, pret = preterite, 1/2/3 = first/second/third person, –3 = nonthird person, excl = exclusive, –incl = noninclusive, ns = nonsingular, sg = singular, pl = plural.} employs independent realization rules that introduce the suffixes\footnote{Concerning the person prefixes \textit{a\nobreakdash-} and \textit{k$\varepsilon $}\nobreakdash- in \tabref{tab:1}, see van \citealt{Driem1987}: 77ff.} in \tabref{tab:1}; these are organized into several rule blocks, each of which fills a particular affix position.  These independent rules and their block membership are given in \tabref{tab:6}.  There are also dependent realization rules; these introduce the agent\nobreakdash-coding suffixes \textit{\nobreakdash-ŋ} and \textit{\nobreakdash-m}, as in (5).  Rule conflation is defined by the conflation rules in (6).  Rule (6a) conflates the dependent rules with the three carrier rules identified in \tabref{tab:6}:  \textbf{4}\nobreakdash-a, \textbf{8}\nobreakdash-a and \textbf{8}\nobreakdash-b.  The resulting conflated rules are listed (redundantly) in \tabref{tab:7}.  

Because a conflated rule belongs to the same rule block as the carrier rule on which it is based, the conflated rule and the carrier rule compete to realize certain morphosyntactic property sets; in any such instance of competition, the conflated rule prevails by virtue of the fact that its domain of application is smaller than that of the carrier rule.\footnote{Concerning each rule in Table~\ref{tab:6}, see \citet{Driem1987}: \textbf{1}\nobreakdash-a, pp.88f; \textbf{2}\nobreakdash-a, 
pp.89ff; \textbf{4}\nobreakdash-a, p.82; \textbf{7}\nobreakdash-a, p.100; \textbf{8}\nobreakdash-a, pp.101f; \textbf{8}\nobreakdash-b, pp.95f; \textbf{10}\nobreakdash-a, 
pp.102f; \textbf{11}\nobreakdash-a, 100f.}

\begin{table}[ht]
\begin{tabular}{ccllc}
\lsptoprule
\multirow{2}{*}{Block} & Rule & \multicolumn{2}{c}{Realization rules} & Carrier \\
\hhline{~~---}
 & label & \multicolumn{1}{c}{Properties realized} & \multicolumn{1}{c}{Operation} & rule?\\
 \hline
 \bfseries sf1 & \textbf{1}\nobreakdash-a & \{\{agt 1\}\{pat 2\}\} & X → X\textit{nɛ} & \multirow{2}{*}{no}\\
 \bfseries sf2 & \textbf{2}\nobreakdash-a & \{pret\} & X → X\textit{ɛ} & \\
 \hline
 \bfseries sf4 & \textbf{4}\nobreakdash-a & \{\{pat\textsc{} 3\}\} & X → X\textit{u} & yes\\
 \hline
 \bfseries sf7 & \textbf{7}\nobreakdash-a & \{\{agt 1 ns\}\{pat 2\}\} & X → X\textit{ci} & no\\
 \hline
  \multirow{2}{*}{\bfseries sf8} & \textbf{8}\nobreakdash-a & \{\{pat\textsc{} ns\}\} & X → X\textit{si} & \multirow{2}{*}{yes}\\
 & \textbf{8}\nobreakdash-b & \{\{pat\textsc{–3} –incl pl\}\} & X → X\textit{i} & \\
 \hline
 \bfseries sf10 & \textbf{10}\nobreakdash-a & \{\{excl\}\} & X → X\textit{ge} & no\\
 \hline
 \bfseries sf11 & \textbf{11}\nobreakdash-a & \{\{agt 1 pl excl\} pret \{pat 3\}\} & X → X\textit{mʔna} & yes, for \textbf{8}\nobreakdash-a\\
\lspbottomrule
\end{tabular}
\caption{Some independent realization rules of Limbu verb inflection}
\label{tab:6}
\end{table}

\ea Dependent realization rules %5
    \label{ex:5}
    \langinfo{lg}{fam}{src}\\
	\begin{tabular}{lll}
	 Ŋ. & \{\{agt\textsc{1} sg\}\}  & : X→ X\textit{ŋ}  (van \citealt{Driem1987}: 99)\\
	M. &  \{\{agt –3 pl\}\} &  : X → X\textit{m} (van \citealt{Driem1987}: 99f)
	\end{tabular}
\z
      
\ea Conflation rules %6
    \label{ex:6}
    \begin{xlist}
    \ex Where R is a rule in Block \textbf{$\alpha $} (\textbf{$\alpha $} ${\in}$\{\textbf{4},\textbf{8}\}), \\
    		[Ŋ~©~R],  [M~©~R]  ${\in}$ Block \textbf{$\alpha $}.
     \ex  {[}\textbf{8}\nobreakdash-a~©~\textbf{11}\nobreakdash-a] ${\in}$ Block \textbf{11}.\footnote{Conflation rule (6b) helps to resolve a conundrum in \tabref{tab:1}.  Notice first that the suffix -\textit{mʔna} introduced by rule \textbf{11}{}-a as an exponent of the property set \{\{agt 1 pl excl\} pret \{pat 3\}\} only combines with one other suffix, namely the suffix \textit{{}-si}  introduced by rule \textbf{8}{}-a as an exponent of \{\{pat\textsc{} ns\}\}; yet, it is featurally compatible with the suffixes introduced by \textbf{4}{}-a and \textbf{10}{}-a.  Moreover, the suffix \textit{{}-si} in the form \textit{huʔ\nobreakdash-mʔna\nobreakdash-si} ‘we (excl) taught them’ does not carry \textit{{}-m}, even though (a) it is a carrier elsewhere and (b) \textit{{}-m} would be featurally appropriate for this word form.  I therefore depart from van Driem in postulating Block \textbf{11} as a portmanteau rule block \citep[141]{Stump2001} that is paradigmatically opposed to and defaults to the sequence of other suffixal blocks.  It houses exactly two rules:  the simple rule \textbf{11}{}-a (which suffixes \nobreakdash-\textit{mʔna}) and the conflated rule [\textbf{8}\nobreakdash-a~©~\textbf{11}\nobreakdash-a] (which suffixes -\textit{mʔna-si}).  Because Block \textbf{11} is paradigmatically opposed to the sequence of rule blocks to which Block \textbf{8} belongs, the application of rule [\textbf{8}\nobreakdash-a~©~\textbf{11}\nobreakdash-a] excludes that of rule [M~©~\textbf{8}\nobreakdash-a], effectively blocking the appearance of \textit{{}-m} in forms such as \textit{huʔ\nobreakdash-mʔna\nobreakdash-si} ‘we (excl) taught them’.}
     \end{xlist}
\z

\begin{table}[ht]
\begin{tabular}{clll}
\lsptoprule
\multirow{2}{*}{Block} & \multirow{2}{*}{Rule label} & \multicolumn{2}{c}{Realization rules}\\
\hhline{~~--} &  & Properties realized & Operation\\
\hline
 \multirow{2}{*}{4} & [Ŋ~©~\textbf{4}\nobreakdash-a] & \{\{agt\textsc{1} sg\}\{pat\textsc{} 3\}\} & X → X\textit{uŋ}\\
& [M~©~\textbf{4}\nobreakdash-a] & \{\{agt –3 pl\}\{pat\textsc{} 3\}\} & X → X\textit{um}\\
\hline
\multirow{3}{*}{8} & [Ŋ~©~\textbf{8}\nobreakdash-a] & \{\{agt\textsc{1} sg\}\{pat\textsc{} ns\}\}: & X → X\textit{siŋ}\\
& [Ŋ~©~\textbf{8}\nobreakdash-b] & \{\{agt\textsc{1} sg\}\{pat\textsc{–3} –incl pl \}\} & X → X\textit{iŋ}\\
& [M~©~\textbf{8}\nobreakdash-a] & \{\{agt –3 pl\}\{pat\textsc{} ns\}\}: & X → X\textit{sim}\\
\hline
 11 & [\textbf{8}\nobreakdash-a~©~\textbf{11}\nobreakdash-a] & \{\{agt 1 pe\} pret\{pat 3 ns\}\} & X → X\textit{mʔnasi}\\
\lspbottomrule
\end{tabular}
\caption{Some conflated realization rules of Limbu verb inflection}
\label{tab:7}
\end{table}

  This analysis correctly defines all of the forms in \tabref{tab:1}.  In particular, it accounts for the superficially erratic distribution of the agent concords \nobreakdash-\textit{ŋ} and \textit{\nobreakdash-m}.  Thus, \tabref{tab:8} presents the manner in which the rules in Tables 6 and 7 define four words:  

\begin{exe}
\sn \begin{itemize}
\item 
\textit{huʔr\nobreakdash-u\nobreakdash-ŋ\nobreakdash-si\nobreakdash-ŋ} ‘I teach them’, in which \nobreakdash-\textit{ŋ} appears twice—after \nobreakdash-\textit{u} and after \textit{\nobreakdash-si};
\item 
\textit{huʔr\nobreakdash-u\nobreakdash-ŋ} ‘I teach him’, in which \nobreakdash-\textit{ŋ} appears after \textit{\nobreakdash-u} only; 
\item 
\textit{huʔ\nobreakdash-nɛ\nobreakdash-ci\nobreakdash-ŋ} ‘I teach you two’, in which \nobreakdash-\textit{ŋ} appears after \textit{\nobreakdash-si} only; and
\item 
\textit{huʔ\nobreakdash-nɛ} ‘I teach you (sg.)’, in which \nobreakdash-\textit{ŋ} fails to appear.
\end{itemize}\end{exe}

As \tabref{tab:8} shows, \textit{\nobreakdash-ŋ} only appears in conflation with an immediately preceding carrier:  in one case, it appears twice because there are two carriers to conflate with; in another, only the carrier \textit{\nobreakdash-u} is available; in yet another, only the carrier \textit{\nobreakdash-si} is available; and sometimes, there is no carrier at all to conflate with.  The proposed analysis provides a similar account of the comparable behavior of the suffix \textit{\nobreakdash-m}.

\begin{table}[ht]
\begin{tabular}{rccc}
\lsptoprule
\hline
\rowcolor[gray]{0.8} \multicolumn{2}{r}{Property set:} & \{\{agt 1 s\}\{pat 3 ns\}\} & \{\{agt 1 s\}\{pat 3 s\}\}\\
\rowcolor[gray]{0.8} \multicolumn{2}{r}{Stem:}  & \textit{huʔ} ‘teach’ & \textit{huʔ} ‘teach’ \\
\rowcolor[gray]{0.8} & & (prevocalically \textit{huʔr}) & (prevocalically \textit{huʔr})\\
\hline
 Rule applying in &  &  & \\
\raggedleft Block: & \textbf{sf1}: & (none) & (none)\\
& \textbf{sf2}: & (none) & (none)\\
& \textbf{sf4}: & \multicolumn{1}{l}{[Ŋ~©~\textbf{4}\nobreakdash-a]: \textit{huʔr\nobreakdash-u\nobreakdash-ŋ}} &
\multicolumn{1}{l}{[Ŋ~©~\textbf{4}\nobreakdash-a]:\textit{huʔr\nobreakdash-u\nobreakdash-ŋ}}\\
& \textbf{sf7}: & (none) & (none)\\
& \textbf{sf8}: & \multicolumn{1}{l}{[Ŋ~©~\textbf{8}\nobreakdash-a]: \textit{huʔr\nobreakdash-u\nobreakdash-ŋ\nobreakdash-si\nobreakdash-ŋ}} & (none)\\
& \textbf{sf10}: & (none) & (none)\\
& \textbf{sf11}: & (none) & (none)\\
\hline
&  & \itshape huʔr\nobreakdash-u\nobreakdash-ŋ\nobreakdash-si\nobreakdash-ŋ & \itshape huʔr\nobreakdash-u\nobreakdash-ŋ\\
&  & ‘I teach them’ & ‘I teach him’\\
\hline
\multicolumn{2}{c}{} &  & \\
\hline
\rowcolor[gray]{0.8} \multicolumn{2}{r}{Property set:} & \{\{agt 1 s\}\{pat 2 de\}\} & \{\{agt 1 s\}\{pat 2 s\}\}\\
\rowcolor[gray]{0.8} \multicolumn{2}{r}{ Stem:}  & \textit{huʔ} ‘teach’ & \textit{huʔ} ‘teach’\\
\hline
 Rule applying in &  &  & \\
\raggedleft Block: & \textbf{sf1}: & \multicolumn{1}{l}{\textbf{1}\nobreakdash-a:  \textit{huʔ\nobreakdash-nɛ}} & 
\multicolumn{1}{l}{\textbf{1}\nobreakdash-a:  \textit{huʔ\nobreakdash-nɛ}}\\
& \textbf{sf2}: & (none) & (none)\\
& \textbf{sf4}: & (none) & (none)\\
& \textbf{sf7}: & (none) & (none)\\
& \textbf{sf8}: & \multicolumn{1}{l}{[Ŋ~©~\textbf{8}\nobreakdash-a]:  \textit{huʔ\nobreakdash-nɛ\nobreakdash-ci\nobreakdash-ŋ}} & (none)\\
& \textbf{sf10}: & (none) & (none)\\
& \textbf{sf11}: & (none) & (none)\\
&  & \itshape huʔ\nobreakdash-nɛ\nobreakdash-ci\nobreakdash-ŋ & \itshape huʔ\nobreakdash-nɛ\\
&  & ‘I teach you two’ & ‘I teach you (sg.)’\\
\lspbottomrule
\end{tabular}
\caption{The definition of four Limbu verb forms in the proposed analysis}
\label{tab:8}
\end{table}

\section{Rule conflation and polyfunctional concord in Southern Sotho}

Return now to the morphology of verbal concord in Southern Sotho.  As we saw in §3, this morphology is largely polyfunctional.  Typically, a verbal concord may appear in either of two positions in a verb’s inflectional morphotactics; but unlike the agent\nobreakdash-coding suffixes in Limbu, which express the same content no matter where they appear, the Southern Sotho verbal concords express subject properties in one position but object properties in another.  The notion of rule conflation makes it possible to account for this difference by assuming that in Southern Sotho, the rules expressing noun\nobreakdash-class concord conflate with a general rule of subject concord in one block and with a general rule of object concord in a different block.   Because the two general rules are formulated as identity functions (realizing subject concord and object concord, respectively), the conflated subject concords have the same phonological form as the conflated object concords.

  Thus, consider the following definition of the Southern Sotho inflectional markings in Tables 2 and 3.  In this analysis, there are three blocks of independent realization rules, as in \tabref{tab:9}.  Block \textbf{a} houses the rules of object concord:  these include the special object\nobreakdash-concord rules for the first\nobreakdash-person singular (\textbf{a}\nobreakdash-i) and third singular class 1 (\textbf{a}\nobreakdash-ii); in addition, it includes a default rule (\textbf{a}\nobreakdash-iii) realizing object concord by means of an identity operation.  Block \textbf{b} houses rules realizing tense properties, here exemplified by the future tense.  Block \textbf{c} houses rules of subject concord, including the special rule (\textbf{c}\nobreakdash-i) of first\nobreakdash-person singular subject concord and a default rule (\textbf{c}\nobreakdash-ii) realizing subject concord by means of an identity operation.  In addition to the independent realization rules in \tabref{tab:9}, the analysis requires the large inventory of dependent rules in \tabref{tab:10}.  The conflation rule in (7) conflates each dependent rule with the default object\nobreakdash-concord rule (\textbf{a}\nobreakdash-iii) and with the default subject\nobreakdash-concord rule (\textbf{c}\nobreakdash-ii).  The resulting conflated rules are listed (redundantly) in \tabref{tab:11}.

\begin{table}[ht]
\begin{tabular}{ccccc}
\lsptoprule
\multirow{2}{*}{Block} & Rule & \multicolumn{2}{c}{Realization rules} & Carrier \\
\hhline{~~--~} &  label & Properties realized & Operation & rule? \\
\hline
\multirow{3}{*}{\bfseries a} & \textbf{a}\nobreakdash-i & \{\{obj 1 sg\}\} & X → \textit{N}X* & \\
& \textbf{a}\nobreakdash-ii & \{\{obj 3 sg \textsc{cl}:1\}\} & X → \textit{m\=o}X & \\
& \textbf{a}\nobreakdash-iii & \{\{obj\}\} & X → X & yes\\
\hline
 \bfseries b & \textbf{b}\nobreakdash-i & \{fut\} & X → \textit{tla}X & \\
 \hline
\multirow{2}{*}{\bfseries c} & \textbf{c}\nobreakdash-i & \{\{sbj 1 sg\}\} & X → \textit{k\=e}X & \\
& \textbf{c}\nobreakdash-ii & \{\{sbj\}\} & X → X & yes\\
\multicolumn{5}{c}{*\textit{N} represents a homorganic nasal.}\\
\lspbottomrule
\end{tabular}
\caption{Three blocks of independent realization rules in Southern Sotho}
\label{tab:9}
\end{table}
\newpage

\begin{table}[ht]
\begin{tabular}{lll}
\lsptoprule
\multicolumn{1}{c}{Rule} & \multicolumn{2}{c}{Realization rules}\\
\hhline{~--} \multicolumn{1}{c}{label} & Properties realized & Operation\\
\textbf{agr}\nobreakdash-i & \{\{2 sg\}\} & X → \textit{u}X\\
\textbf{agr}\nobreakdash-ii & \{\{3 sg\}\} & X → \textit{\=o}X\\
\textbf{agr}\nobreakdash-iii & \{\{3 sg \textsc{cl}:3\}\} & X → \textit{l\=e}X\\
\textbf{agr}\nobreakdash-iv & \{\{3 sg \textsc{cl}:4\}\} & X → \textit{s\=e}X\\
\textbf{agr}\nobreakdash-v & \{\{3 sg \textsc{cl}:5\}\} & X → \textit{\=e}X\\
\textbf{agr}\nobreakdash-vi & \{\{3 sg \textsc{cl}:6\}\} & X → \textit{b\=o}X\\
\textbf{agr}\nobreakdash-vii & \{\{3 \textsc{cl}:7\}\} & X → \textit{h\=o}X\\
\textbf{agr}\nobreakdash-viii & \{\{1 pl\}\} & X → \textit{r\=e}X\\
\textbf{agr}\nobreakdash-ix & \{\{2 pl\}\} & X → \textit{l\=e}X\\
\textbf{agr}\nobreakdash-x & \{\{3 pl \textsc{cl}:1\}\} & X → \textit{ba}X\\
\textbf{agr}\nobreakdash-xi & \{\{3 pl \textsc{cl}:1{\textbar}2\}\} & X → \textit{\=e}X\\
\textbf{agr}\nobreakdash-xii & \{\{3 pl \textsc{cl}:3\}\} & X → \textit{a}X\textit{} {\textbar} \textit{li}X\\
\textbf{agr}\nobreakdash-xiii & \{\{3 pl \textsc{cl}:4{\textbar}5\}\} & X → \textit{li}X\\
\textbf{agr}\nobreakdash-xiv & \{\{3 pl \textsc{cl}:6\}\} & X → \textit{a}X\\
\lspbottomrule
\end{tabular}
\caption{Dependent realization rules for verbal concord in Southern Sotho}
\label{tab:10}
\end{table}

\begin{exe}
\ex Conflation rule\\
Where \textbf{agr}\nobreakdash-\textit{n} is a dependent realization rule and R is a carrier rule in Block $\alpha $, [\textbf{agr}\nobreakdash-\textit{n}~©~R] ${\in}$ Block $\alpha $.
\end{exe}

\newpage

\begin{table}[ht]
\begin{tabular}{clll}
\lsptoprule
\multirow{2}{*}{Block} & \multirow{2}{*}{\multicolumn{1}{c}{Rule label}} & \multicolumn{2}{c}{Realization rules}\\
\hhline{~~--} &  & Properties realized & Operation\\
\hline
 \bfseries a & \textbf{[agr}\nobreakdash-i~©~\textbf{a}\nobreakdash-iii] & \{\{obj 2 sg\}\} & X → \textit{u}X\\
& \textbf{[agr}\nobreakdash-ii~©~\textbf{a}\nobreakdash-iii] & \{\{obj 3 sg \textsc{cl}:1{\textbar}2\}\} & X → \textit{\=o}X\\
& \textbf{[agr}\nobreakdash-iii~©~\textbf{a}\nobreakdash-iii] & \{\{obj 3 sg \textsc{cl}:3\}\} & X → \textit{l\=e}X\\
& \textbf{[agr}\nobreakdash-iv\textbf{~©~}\textbf{a}\nobreakdash-iii] & \{\{obj 3 sg \textsc{cl}:4\}\} & X → \textit{s\=e}X\\
& \textbf{[agr}\nobreakdash-v\textbf{~©~}\textbf{a}\nobreakdash-iii] & \{\{obj 3 sg \textsc{cl}:5\}\} & X → \textit{\=e}X\\
& \textbf{[agr}\nobreakdash-vi\textbf{~©~}\textbf{a}\nobreakdash-iii] & \{\{obj 3 sg \textsc{cl}:6\}\} & X → \textit{b\=o}X\\
& \textbf{[agr}\nobreakdash-vii\textbf{~©~}\textbf{a}\nobreakdash-iii] & \{\{obj 3 \textsc{cl}:7\}\} & X → \textit{h\=o}X\\
& \textbf{[agr}\nobreakdash-viii\textbf{~©~}\textbf{a}\nobreakdash-iii] & \{\{obj 1 pl\}\} & X → \textit{r\=e}X\\
& \textbf{[agr}\nobreakdash-ix\textbf{~©~}\textbf{a}\nobreakdash-iii] & \{\{obj 2 pl\}\} & X → \textit{l\=e}X\\
& \textbf{[agr}\nobreakdash-x\textbf{~©~}\textbf{a}\nobreakdash-iii] & \{\{obj 3 pl \textsc{cl}:1\}\} & X → \textit{ba}X\\
& \textbf{[agr}\nobreakdash-xi\textbf{~©~}\textbf{a}\nobreakdash-iii] & \{\{obj 3 pl \textsc{cl}:1{\textbar}2\}\} & X → \textit{\=e}X\\
& \textbf{[agr}\nobreakdash-xii\textbf{~©~}\textbf{a}\nobreakdash-iii] & \{\{obj 3 pl \textsc{cl}:3\}\} & X → \textit{a}X\textit{} {\textbar} \textit{li}X\\
& \textbf{[agr}\nobreakdash-xiii\textbf{~©~}\textbf{a}\nobreakdash-iii] & \{\{obj 3 pl \textsc{cl}:4{\textbar}5\}\} & X → \textit{li}X\\
& \textbf{[agr}\nobreakdash-xiv\textbf{~©~}\textbf{a}\nobreakdash-iii] & \{\{obj 3 pl \textsc{cl}:6\}\} & X → \textit{a}X\\
\hline
 \bfseries c & \textbf{[agr}\nobreakdash-i~©~\textbf{c}\nobreakdash-ii] & \{\{sbj 2 sg\}\} & X → \textit{u}X\\
& \textbf{[agr}\nobreakdash-ii~©~\textbf{c}\nobreakdash-ii] & \{\{sbj 3 sg \textsc{cl}:1{\textbar}2\}\} & X → \textit{\=o}X\\
& \textbf{[agr}\nobreakdash-iii~©~\textbf{c}\nobreakdash-ii] & \{\{sbj 3 sg \textsc{cl}:3\}\} & X → \textit{l\=e}X\\
& \textbf{[agr}\nobreakdash-iv\textbf{~©~}\textbf{c}\nobreakdash-ii] & \{\{sbj 3 sg \textsc{cl}:4\}\} & X → \textit{s\=e}X\\
& \textbf{[agr}\nobreakdash-v\textbf{~©~}\textbf{c}\nobreakdash-ii] & \{\{sbj 3 sg \textsc{cl}:5\}\} & X → \textit{\=e}X\\
& \textbf{[agr}\nobreakdash-vi\textbf{~©~}\textbf{c}\nobreakdash-ii] & \{\{sbj 3 sg \textsc{cl}:6\}\} & X → \textit{b\=o}X\\
& \textbf{[agr}\nobreakdash-vii\textbf{~©~}\textbf{c}\nobreakdash-ii] & \{\{sbj 3 \textsc{cl}:7\}\} & X → \textit{h\=o}X\\
& \textbf{[agr}\nobreakdash-viii\textbf{~©~}\textbf{c}\nobreakdash-ii] & \{\{sbj 1 pl\}\} & X → \textit{r\=e}X\\
& \textbf{[agr}\nobreakdash-ix\textbf{~©~}\textbf{c}\nobreakdash-ii] & \{\{sbj 2 pl\}\} & X → \textit{l\=e}X\\
& \textbf{[agr}\nobreakdash-x\textbf{~©~}\textbf{c}\nobreakdash-ii] & \{\{sbj 3 pl \textsc{cl}:1\}\} & X → \textit{ba}X\\
& \textbf{[agr}\nobreakdash-xi\textbf{~©~}\textbf{c}\nobreakdash-ii] & \{\{sbj 3 pl \textsc{cl}:1{\textbar}2\}\} & X → \textit{\=e}X\\
& \textbf{[agr}\nobreakdash-xii\textbf{~©~}\textbf{c}\nobreakdash-ii] & \{\{sbj 3 pl \textsc{cl}:3\}\} & X → \textit{a}X\textit{} {\textbar} \textit{li}X\\
& \textbf{[agr}\nobreakdash-xiii\textbf{~©~}\textbf{c}\nobreakdash-ii] & \{\{sbj 3 pl \textsc{cl}:4{\textbar}5\}\} & X → \textit{li}X\\
& \textbf{[agr}\nobreakdash-xiv\textbf{~©~}\textbf{c}\nobreakdash-ii] & \{\{sbj 3 pl \textsc{cl}:6\}\} & X → \textit{a}X\\
\hhline{~---}
\lspbottomrule
\end{tabular}
\caption{Some conflated realization rules of Southern Sotho verb inflection}
\label{tab:11}
\end{table}

Each of the conflated rules in \tabref{tab:11} belongs to the same rule block as the carrier rule on which it is based.  As in the Limbu analysis proposed above, a conflated rule and the carrier rule on which it is based compete to realize certain morphosyntactic property sets, and being the narrower rule, the conflated rule prevails in each such case.   

  This analysis correctly defines all of the forms in Tables 2 and 3.  In particular, it accounts for the fact that in all but a handful of cases, each subject concord has a corresponding object concord that expresses the same person, number and noun class by means of the same prefix.  Thus, \tabref{tab:12} presents the manner in which the rules in Tables 10 and 11 define two words:  

\begin{itemize}
\item 
\textit{ba\nobreakdash-tla\nobreakdash-b\=o\nobreakdash-bòna}  ‘they (\textsc{cl}:1) will see it (\textsc{cl}:6)’, in which \textit{ba\nobreakdash-} is a third\nobreakdash-person plural class 1 subject concord and \textit{b\=o\nobreakdash-} is a singular class 6 object concord; and
\item
\textit{b\=o\nobreakdash-tla\nobreakdash-ba\nobreakdash-bòna}  ‘it (\textsc{cl}:6) will see them (\textsc{cl}:1)’, in which \textit{b}\textit{\=o}\textit{\nobreakdash-} is a singular class 6 subject concord and \textit{ba\nobreakdash-} is a third\nobreakdash-person plural class 1 object concord. 
\end{itemize}

\begin{table}[ht]
\begin{tabular}{rrrl}
\lsptoprule
\hline
\rowcolor[gray]{0.8} \multicolumn{2}{r}{Property set:} & \multicolumn{2}{c}{\{\{sbj 3 pl \textsc{cl:}1\} fut \{obj 3 sg \textsc{cl:}6\}\}}\\
\rowcolor[gray]{0.8} \multicolumn{2}{r}{Stem:} & \multicolumn{2}{c}{\textit{bòna} ‘see’}\\
\hline
\raggedleft Rule applying in &  &  & \\
\raggedleft Block & \bfseries a: & \raggedleft \textbf{[agr}\nobreakdash-vi\textbf{~©~}\textbf{a}\nobreakdash-iii]:\textit{}  & \itshape b\=o\nobreakdash-bòna  \\
& \bfseries b: & \raggedleft \textbf{b}\nobreakdash-i:\textit{}  & \itshape tla\nobreakdash-b\=o\nobreakdash-bòna  \\
& \bfseries c: & \raggedleft \textbf{[agr}\nobreakdash-x\textbf{~©~}\textbf{c}\nobreakdash-ii]:\textit{}  & \itshape ba\nobreakdash-tla\nobreakdash-b\=o\nobreakdash-bòna  \\
\hline
&  & \multicolumn{2}{c}{\itshape ba\nobreakdash-tla\nobreakdash-b\=o\nobreakdash-bòna} \\
&  & \multicolumn{2}{c}{‘they (\textsc{cl}:1) will see it (\textsc{cl}:6)’}\\
\hline 
\rowcolor[gray]{0.8} \multicolumn{2}{r}{Property set:} & \multicolumn{2}{c}{\{\{sbj 3 sg \textsc{cl:}6\} fut \{obj 3 pl \textsc{cl:}1\}\}}\\
\rowcolor[gray]{0.8} \multicolumn{2}{r}{Stem:} & \multicolumn{2}{c}{\textit{bòna} ‘see’}\\
\hline
\raggedleft Rule applying in &  &  & \\
\raggedleft Block & \bfseries a: & \raggedleft \textbf{[agr}\nobreakdash-x\textbf{~©~}\textbf{a}\nobreakdash-iii]:\textit{}  & \textit{ba\nobreakdash-bòna}  \\
& \bfseries b: & \raggedleft \textbf{b}\nobreakdash-i:\textit{}  & \textit{tla\nobreakdash-ba\nobreakdash-bòna}  \\
& \bfseries c: & \raggedleft \textbf{[agr}\nobreakdash-vi\textbf{~©~}\textbf{c}\nobreakdash-ii]:\textit{}  & \textit{b\=o\nobreakdash-tla\nobreakdash-ba\nobreakdash-bòna}  \\
\hline
&  & \multicolumn{2}{c}{\textit{b\=o\nobreakdash-tla\nobreakdash-ba\nobreakdash-bòna}} \\
&  & \multicolumn{2}{c}{‘it (\textsc{cl}:6) will see them (\textsc{cl}:1)’}\\
\lspbottomrule
\end{tabular}
\caption{The definition of two Southern Sotho verb forms in the proposed analysis}
\label{tab:12}
\end{table}

As \tabref{tab:12} shows, the dependent rules introducing \textit{b\=o\nobreakdash-} (\textbf{agr}\nobreakdash-vi\textbf{~}in \tabref{tab:10})\textit{} and \textit{\nobreakdash-ba} (\textbf{agr}\nobreakdash-x\textbf{~}in \tabref{tab:10}) both conflate with the carrier rule \textbf{a}\nobreakdash-iii (\tabref{tab:9}) to produce rules of object concord in Block \textbf{a} and both conflate with the carrier rule \textbf{c}\nobreakdash-ii (\tabref{tab:9}) to produce a rule of subject concord in Block \textbf{c}.  

\section{Wider evidence for rule conflation}
The analyses proposed here for multiple exponence in Limbu agent concord and for polyfunctional verbal concords in Southern Sotho both depend on the notion that morphological rules may conflate to produce more complex rules (= principle (3)) and the notion that conflated rules may compete with simple rules as members of the same rule block (= principle (4)).  

These principles of rule conflation are motivated independently of the need to account for multiple exponence and polyfunctionality.  First, they make it possible to account for apparent anomalies in the interaction of inflectional rule applications.  For example, a rule’s order of application may seem to depend on whether or not another rule applies. In Fula, a pronominal object suffix on a verb in the relative past tense ordinarily follows that verb’s subject suffx, as in (8a,b); but in the particular case in which a verb has both a singular personal object suffix (2sg \textit{\nobreakdash-mA} or 3sg \textit{\nobreakdash-mO}) and the first\nobreakdash-person singular subject suffix \nobreakdash-\textit{mi}, the expected order is reversed, as in (8c,d).  Principles (3) and (4) allow one to say that the rules realizing the subject and object suffixes in the relative past tense belong to a single rule block; that the object rules ordinarily conflate with the subject rules; but that the \textit{\nobreakdash-mi} rule instead conflates with the \textit{\nobreakdash-mA} and \textit{\nobreakdash-mO} rules.  


\ea
	\ea
	\gll  {\itshape mball\nobreakdash-u\nobreakdash-mi\nobreakdash-ɓe\nobreakdash-’}\\
	{help\nobreakdash-\textsc{rel.pst.act}\textsc{\nobreakdash-1sg.sbj\nobreakdash-3pl.}\textsc{cl.2.obj}\textsc{\nobreakdash-fg}}\\
	\glt `I helped them'
	
	\ex {\itshape mball\nobreakdash-u\nobreakdash-ɗaa\nobreakdash-mO\nobreakdash-’}\\
	{help\textit{\nobreakdash-}\textsc{rel.pst.act}\textsc{\nobreakdash-}\textsc{2sg.sbj}\textsc{\nobreakdash-}\textsc{3sg.cl.1.obj}\textsc{\nobreakdash-fg}}\\
	\glt {‘you (sg.) helped him’}
	
	\ex {\itshape mball\nobreakdash-u\nobreakdash-mA\nobreakdash-mi\nobreakdash-’}\\
	{help\textit{\nobreakdash-}\textsc{rel.pst.act}\textsc{\nobreakdash-}\textsc{2}\textsc{sg.obj}\textsc{\nobreakdash-}\textsc{1sg}\textsc{.}\textsc{sbj}\textsc{\nobreakdash-fg}}\\
	\glt {‘I helped you (sg.)’} 
	
	\ex {\itshape mball\nobreakdash-u\nobreakdash-mO\nobreakdash-mi\nobreakdash-’}\\
	{help\textit{\nobreakdash-}\textsc{rel.pst.act}\textsc{\nobreakdash-}\textsc{3sg.cl.1.obj}\textsc{\nobreakdash-}\textsc{1sg.sbj}\textsc{\nobreakdash-fg}}\\
	\glt ‘I helped him’ \hfill (\citealt{Arnott1970}, Appendix 15)\\
	
	\z
\z

In another apparently anomalous interaction of inflectional rules,  an affix either precedes the stem with which it joins or follows it, with the choice of position being conditioned by the presence or absence of some other affix.  In Swahili, the verbal concord coding the properties of a relative verb form’s relativized argument appears postverbally in tenseless affirmative forms, but preverbally in forms that are prefixally marked for tense or negation; thus, the class 8 relative concord \textit{vyo} is postverbal in (9a) but preverbal in (9b).  The principles of rule conflation make it possible to say that the relative affix is suffixed to the verb stem by default, but is suffixed to an overt prefixal exponent of tense or negation (Stump to appear a).

\ea%9
    \label{ex:9}
    	\ea
	\gll \textit{a\nobreakdash-vi\nobreakdash-soma\nobreakdash-vyo} \\ 
	\textsc{sbj:cl.}1\textit{\nobreakdash-}\textsc{obj:cl.}8\textit{\nobreakdash-}read\textit{\nobreakdash-}\textsc{rel:cl.}8 \\
	 \glt ‘(books) which he reads’
	 
	 \ex 
	 \gll \textit{a\nobreakdash-si\nobreakdash-vyo\nobreakdash-vi\nobreakdash-soma}\\
	 \textsc{sbj:cl.}1\textit{\nobreakdash-}\textsc{neg}\textsc{\nobreakdash-}\textsc{rel:cl.}8\textit{\nobreakdash-}\textsc{obj:cl.}8\textit{\nobreakdash-}read]\\
	 \glt ‘(books) which he doesn’t read’

	\z
\z        



As Stump (to appear b) shows, the principles of rule conflation afford simple  solutions to a number of other apparent anomalies in the interaction of inflection rules. These include the incidence of variable affix order \citep{Bickel2007} and of Wackernagel affixes (\citealt{Nevis1992},  \citealt{Bonami2008}) as well as the superficially puzzling fact that affix sequences may preserve the same internal order whether the sequence as a whole is prefixal or suffixal, as in European Portuguese verb inflection (\citealt{Luis2005}).

Second, the principles of rule conflation in (3) and (4) make it possible to account for nonmonotonic interactions among inflectional rules.  The usual expectation is that a realization rule possesses the same intrinsic properties whether it applies alone or in combination with other rules.  But there are anomalous cases in which this expectation is not met. Once the definition of a language’s morphology includes a conflated rule [R$_{1}$~©~R$_{2}$], this rule may evolve independently, taking on properties not directly stemming from either R$_{1}$~or R$_{2}$.  In this way, the properties exhibited by a rule applying in isolation may not always be preserved when it is conflated with other rules.  In view of this fact, the content attributed to conflated rules in §4.4 above should be seen as their \textit{default} content, subject to modification by processes of grammaticalization.  That is, the content expressed by rule [R$_{1}$~©~R$_{2}$] is, in the default case, a monotonic function of the content expressed by rules R$_{1}$~and~R$_{2}$; but this default is subject to override.  

There are at least three ways in which the resulting nonmonotonicity may be manifested.  One reflection of this fact is the phenomenon of ‘potentiation’ \citep{Williams1981}, by which an unproductive rule becomes productive when applying in combination with another rule (as the unproductive \textit{\nobreakdash-ity} rule becomes productive in combination with the \textit{\nobreakdash-able} rule; cf. \citealt{Aronoff1976}, \citealt{Bochner1992}).  

Another reflection is the fact that the domain of rule R$_{1}$ apparently depends on whether a particular rule R$_{2}$ applies subsequently.  By principles (3) and (4), such cases arise when a conflated rule [R$_{1}$~©~R$_{2}$] evolves a domain application distinct from that of R$_{2}$.  Thus, a stem may be in the domain of R$_{2}$ but not that of [R$_{1}$~©~R$_{2}$], as in the case of \textit{base} → \textit{basic}, *\textit{basical}; at the same time, a stem may be in the domain of [R$_{1}$~©~R$_{2}$] but not that of R$_{2}$, as in the case of \textit{whimsy} → \textit{whimsical}, *\textit{whimsic}.  A third reflection arises in cases in which two rules apparently realize less content separately than they do together. In Latin \textit{reg\=emus} ‘we shall rule’, the conflation of the rules that suffix \textit{\nobreakdash-\=e} and \textit{\nobreakdash-mus} expresses the first\nobreakdash-person plural future active even though neither rule by itself is an expression of future tense.\footnote{See Stump (to appear b) for discussion of a similar case from Old English.}  These nonmonotonic phenomena have never before been seen as manifestations of a single overarching principle; the principles of rule conflation, however, facilitate precisely such a perspective.

Third, the principles of rule conflation make it possible to account for  parallelisms between the application of a single rule and that of a sequence of rules.  A word form’s inflectional morphology is sometimes informally conceived of as instantiating a sequence of “slots” each of which corresponds to a set of rules available to fill it.  Andersonian rule blocks are a kind of formal reconstruction of this idea, whose simplest interpretation involves individual rules providing alternative ways of filling the same slot.  There are, however, apparent deviations from this pattern, in which successive slots are ordinarily filled by successive rule applications but may in some instances be simultaneously filled by a single rule application introducing a “wide” affix that somehow straddles two or more slots. The Swahili portmanteau prefix \textit{si\nobreakdash-} is an example.  In Swahili negative indicative verb forms,  the usual pattern is for the negative \textit{ha\nobreakdash-} rule to fill slot 1 and a subject\nobreakdash-concord rule to fill slot 2, e.g. \textit{ha\nobreakdash-tu\nobreakdash-ta\nobreakdash-taka} [\textsc{neg\nobreakdash-1pl\nobreakdash-fut\nobreakdash-}want] ‘we will not want’.  But in first\nobreakdash-person singular negative verb forms, the application of the negative first\nobreakdash-person singular \textit{si\nobreakdash-} rule seems to straddle slots 1 and 2.  The principles of rule conflation resolve this conundrum by allowing a rule block to contain both conflated rules (e.g. the first\nobreakdash-person plural negative \textit{ha\nobreakdash-tu\nobreakdash-} rule) and simple rules (e.g. the \textit{si\nobreakdash-} rule) in paradigmatic opposition; in this way, the behavior of portmanteau rules is reconciled with the natural assumption that paradigmatic opposition is a relation between two rules rather than a relation between a rule and a sequence of rules. 

The principles of rule conflation in (3) and (4) are a simple and natural extension of the principles of realization-rule interaction developed by Anderson (see again (1) and (2)).  Rule conflation allows a variety of apparently recalcitrant phenomena to be reconciled with a general scheme of rule interaction based on ordered blocks of realization rules in which the members of a given block are mutually exclusive in their application.  

%\section*{Abbreviations}
%\section*{Acknowledgements}

%\todos

\printbibliography[heading=subbibliography,notkeyword=this]
\end{document}
