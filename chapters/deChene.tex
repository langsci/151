\documentclass[output=paper,
modfonts
]{LSP/langsci}


%% add all extra packages you need to load to this file 
% \usepackage{todo} %% removed,cna use todonotes instead. % Jason reactivated
% \usepackage{graphicx} % not needed because forest loads tikz, which loads graphicx
\usepackage{tabularx}
\usepackage{amsmath} 
\usepackage{multicol}
\usepackage{lipsum}
\usepackage{longtable}
\usepackage{booktabs}
\usepackage[normalem]{ulem}
%\usepackage{tikz} % not needed because forest loads tikz
\usepackage{phonrule} % for SPE-style phonological rules
\usepackage{pst-all} % loads the main pstricks tools; for arrow diagrams in Hale.tex
%\usepackage{leipzig} % for gloss abbreviations
\usepackage[% for automatic cross-referencing
compress,%
capitalize,% labels are always capitalized in LSP style
noabbrev]% labels are always spelled out in LSP style
{cleveref}

% based on http://tex.stackexchange.com/a/318983/42880 for using gb4e examples with cleveref
\crefname{xnumi}{}{}
\creflabelformat{xnumi}{(#2#1#3)}
\crefrangeformat{xnumi}{(#3#1#4)--(#5#2#6)}
\crefname{xnumii}{}{}
\creflabelformat{xnumii}{(#2#1#3)}
\crefrangeformat{xnumii}{(#3#1#4)--(#5#2#6)}

%\usepackage[notcite,notref]{showkeys} %%removed, not helping CB.
%\usepackage{showidx} %%remove for final compiling - shows index keys at top of page.
 
\usepackage{langsci/styles/langsci-gb4e}  
 \usepackage{pifont}
% % OT tableaux                                                
% \usepackage{pstricks,colortab}  
\usepackage{multirow} % used in OT tableaux
\usepackage{rotating} %needed for angled text%
\usepackage{colortbl} % for cell shading
 
 \usepackage{avm}  
\usepackage[linguistics]{forest} 
\usetikzlibrary{matrix,fit} % for matrix of nodes in Kaisse and Bat-El


\usepackage{hhline}
\newcommand{\cgr}{\cellcolor[gray]{0.8}}
\newcommand{\cn}{\centering}



\newcommand{\reff}[1]{(\ref{#1})}
%\usepackage{newtxtext,newtxmath}


%\usepackage[normalem] {ulem}
\usepackage{qtree}
%\usepackage{natbib}
%\usepackage{tikz}
%\usepackage{gb4e}
\usepackage{phonrule}  
%\bibliographystyle{humannat}



\usepackage{minibox}

%\include{psheader-metr}

\def\bl#1{$_{\textrm{{\footnotesize #1}}}$}

%%add all your local new commands to this file

\newcommand{\form}[1]{\mbox{\emph{#1}}}
\newcommand{\uf}[1]{\mbox{/#1/}}

% borrowed from expex and converted from plan tex to latex
\newcommand{\judge}[1]{{\upshape #1\hspace{0.1em}}}
\newcommand{\ljudge}[1]{\makebox[0pt][r]{\judge{#1}}}

\newcommand\tikzmark[1]{\tikz[remember picture, baseline=(#1.base)] \node[anchor=base,inner sep=0pt, outer sep=0pt] (#1) {#1};} % for adding decorations, arrows, lines, etc. to text
\newcommand\tikzmarknamed[2]{\tikz[remember picture, baseline=(#1.base)] \node[anchor=base,inner sep=0pt, outer sep=0pt] (#1) {#2};} % for adding decorations, arrows, lines, etc. to text
\newcommand\tikzmarkfullnamed[2]{\tikz[remember picture, baseline=(#1.base)] \node[anchor=base,inner sep=0pt, outer sep=0pt] (#1) {\vphantom{X}#2};} % for adding decorations, arrows, lines, etc. to text; this one works best for decorations above a line of text because it adds in the heigh of a capital letter and takes two arguments - one for the node name and one for the printed text

\newcommand{\sub}[1]{$_{\text{#1}}$} % for non-math subscripts
\newcommand{\subit}[1]{\sub{\textit{#1}}} % for italics non-math subscripts
\newcommand{\1}{\rlap{$'$}\xspace} % for the prime in X' (the \rlap command allows the prime to be ignored for horizontal spacing in trees, and the \xspace command allows you to use this in normal text without adding \ afterwards). This isn't crucial, but it helps the formatting to look a little better.

% Aissen:
\newcommand\tikzmarkfull[1]{\tikz[remember picture, baseline=(#1.base)] \node[anchor=base,inner sep=0pt, outer sep=0pt] (#1) {\vphantom{X}#1};} % for adding decorations, arrows, lines, etc. to text; this one works best for decorations above a line of text because it adds in the heigh of a capital letter and takes one argument that serves as the name and the printed text
\newcommand{\bridgeover}[2]{% for a line that creates a bridge over text, connecting two nodes
	\begin{tikzpicture}[remember picture,overlay]
	\draw[thick,shorten >=3pt,shorten <=3pt] (#1.north) |- +(0ex,2.5ex) -| (#2.north);
	\end{tikzpicture}
}
\newcommand{\bridgeoverht}[3]{% for a line that creates a bridge over text, connecting two nodes and specifing the height of the bridge
	\begin{tikzpicture}[remember picture,overlay]
	\draw[thick,shorten >=3pt,shorten <=3pt] (#2.north) |- +(0ex,#1) -| (#3.north);
	\end{tikzpicture}
}
\newcommand{\bridgeoverex}{\vspace*{3ex}} % place before an example that has a \bridgeover so that there is enough vertical space for it

% Chung:
\newcommand{\lefttabular}[1]{\begin{tabular}{p{0.5in}}#1\end{tabular}}

% Kaisse:
\newcommand{\mgmorph}[1]{|(#1)| {#1}}
\newcommand{\mgone}[2][$\times$]{\node at (#2.base) [above=2ex] (1#2) {\vphantom{X}#1};}
\newcommand{\mgtwo}[2][$\times$]{\mgone{#2} \node at (#2.base) [above=4.5ex] (2#2) {\vphantom{X}#1};}
\newcommand{\mgthree}[2][$\times$]{\mgtwo{#2} \node at (#2.base) [above=7ex] (3#2) {\vphantom{X}#1};}
\newcommand{\mgftl}[1]{\node at (1#1) [left] {(};}
\newcommand{\mgftr}[1]{\node at (1#1) [right] {)};}
\newcommand{\mgfoot}[2]{\mgftl{#1}\mgftr{#2}}
\newcommand{\mgldelim}[2]{\node at (#2.west) [left,inner sep = 0pt, outer sep = 0pt] {#1};}
\newcommand{\mgrdelim}[2]{\node at (#2.east) [right,inner sep = 0pt, outer sep = 0pt] {#1};}

\newcommand{\squish}{\hspace*{-3pt}}

% Kavitskaya:
\newcommand{\assoc}[2]{\draw (#1.south) -- (#2.north);}
\newcolumntype{L}{>{\raggedright\arraybackslash}X}

% Lepic & Padden:
\newcommand{\fitpic}[1]{\resizebox{\hsize}{!}{\includegraphics{#1}}} % from http://tex.stackexchange.com/a/148965/42880
\newcommand{\signpic}[1]{\includegraphics[width=1.36in]{#1}}
\newcolumntype{C}{>{\centering\arraybackslash}X}

% Spencer:

\newcommand{\textex}[1]{\textit{#1}\xspace}
\newcommand{\lxm}[1]{\textsc{#1}\xspace}

% Thrainsson:

\renewcommand{\textasciitilde}{\char`~} % for use with TTF/OTF fonts (see comments on http://tex.stackexchange.com/a/317/42880)
\newcommand{\tikzarrow}[2]{% for an arrow connecting two nodes
\begin{tikzpicture}[remember picture,overlay]
\draw[thick,shorten >=3pt,shorten <=3pt,->,>=stealth] (#1) -- (#2);
\end{tikzpicture}
}

\newlength{\padding}
\setlength{\padding}{0.5em}
\newcommand{\lesspadding}{\hspace*{-\padding}}
\newcommand{\feat}[1]{\lesspadding#1\lesspadding}

% Hammond

\usepackage[]{graphicx}\usepackage[]{xcolor}
%% maxwidth is the original width if it is less than linewidth
%% otherwise use linewidth (to make sure the graphics do not exceed the margin)
\makeatletter
\def\maxwidth{ %
  \ifdim\Gin@nat@width>\linewidth
    \linewidth
  \else
    \Gin@nat@width
  \fi
}
\makeatother

\definecolor{fgcolor}{rgb}{0.345, 0.345, 0.345}
\newcommand{\hlnum}[1]{\textcolor[rgb]{0.686,0.059,0.569}{#1}}%
\newcommand{\hlstr}[1]{\textcolor[rgb]{0.192,0.494,0.8}{#1}}%
\newcommand{\hlcom}[1]{\textcolor[rgb]{0.678,0.584,0.686}{\textit{#1}}}%
\newcommand{\hlopt}[1]{\textcolor[rgb]{0,0,0}{#1}}%
\newcommand{\hlstd}[1]{\textcolor[rgb]{0.345,0.345,0.345}{#1}}%
\newcommand{\hlkwa}[1]{\textcolor[rgb]{0.161,0.373,0.58}{\textbf{#1}}}%
\newcommand{\hlkwb}[1]{\textcolor[rgb]{0.69,0.353,0.396}{#1}}%
\newcommand{\hlkwc}[1]{\textcolor[rgb]{0.333,0.667,0.333}{#1}}%
\newcommand{\hlkwd}[1]{\textcolor[rgb]{0.737,0.353,0.396}{\textbf{#1}}}%
\let\hlipl\hlkwb

\usepackage{framed}
\makeatletter
\newenvironment{kframe}{%
 \def\at@end@of@kframe{}%
 \ifinner\ifhmode%
  \def\at@end@of@kframe{\end{minipage}}%
  \begin{minipage}{\columnwidth}%
 \fi\fi%
 \def\FrameCommand##1{\hskip\@totalleftmargin \hskip-\fboxsep
 \colorbox{shadecolor}{##1}\hskip-\fboxsep
     % There is no \\@totalrightmargin, so:
     \hskip-\linewidth \hskip-\@totalleftmargin \hskip\columnwidth}%
 \MakeFramed {\advance\hsize-\width
   \@totalleftmargin\z@ \linewidth\hsize
   \@setminipage}}%
 {\par\unskip\endMakeFramed%
 \at@end@of@kframe}
\makeatother

\definecolor{shadecolor}{rgb}{.97, .97, .97}
\definecolor{messagecolor}{rgb}{0, 0, 0}
\definecolor{warningcolor}{rgb}{1, 0, 1}
\definecolor{errorcolor}{rgb}{1, 0, 0}
\newenvironment{knitrout}{}{} % an empty environment to be redefined in TeX

\usepackage{alltt}

%revised version started: 12/17/16

%NEEDS: allbib.bib - already added to the master bibliography file.
%cited references only: bibexport -o mhTMP.bib main1-blx.aux
%PLUS sramh-img*, sramh.tex

%added stuff
\newcommand{\add}[1]{\textcolor{blue}{#1}}
%deleted stuff
\newcommand{\del}[1]{\textcolor{red}{(removed: #1)}}
%uncomment these to turn off colors
\renewcommand{\add}[1]{#1}
\renewcommand{\del}[1]{}

%shortcuts
\newcommand{\w}{\ili{Welsh}}
\newcommand{\e}{\ili{English}}
\newcommand{\io}{Input Optimization}




 \newcommand{\hand}{\ding{43}}
% \newcommand{\rot}[1]{\begin{rotate}{90}#1\end{rotate}} %shortcut for angled text%  
% \newcommand{\rotcon}[1]{\raisebox{-5ex}{\hspace*{1ex}\rot{\hspace*{1ex}#1}}}

%% add all extra packages you need to load to this file 
% \usepackage{todo} %% removed,cna use todonotes instead. % Jason reactivated
% \usepackage{graphicx} % not needed because forest loads tikz, which loads graphicx
\usepackage{tabularx}
\usepackage{amsmath} 
\usepackage{multicol}
\usepackage{lipsum}
\usepackage{longtable}
\usepackage{booktabs}
\usepackage[normalem]{ulem}
%\usepackage{tikz} % not needed because forest loads tikz
\usepackage{phonrule} % for SPE-style phonological rules
\usepackage{pst-all} % loads the main pstricks tools; for arrow diagrams in Hale.tex
%\usepackage{leipzig} % for gloss abbreviations
\usepackage[% for automatic cross-referencing
compress,%
capitalize,% labels are always capitalized in LSP style
noabbrev]% labels are always spelled out in LSP style
{cleveref}

% based on http://tex.stackexchange.com/a/318983/42880 for using gb4e examples with cleveref
\crefname{xnumi}{}{}
\creflabelformat{xnumi}{(#2#1#3)}
\crefrangeformat{xnumi}{(#3#1#4)--(#5#2#6)}
\crefname{xnumii}{}{}
\creflabelformat{xnumii}{(#2#1#3)}
\crefrangeformat{xnumii}{(#3#1#4)--(#5#2#6)}

%\usepackage[notcite,notref]{showkeys} %%removed, not helping CB.
%\usepackage{showidx} %%remove for final compiling - shows index keys at top of page.
 
\usepackage{langsci/styles/langsci-gb4e}  
 \usepackage{pifont}
% % OT tableaux                                                
% \usepackage{pstricks,colortab}  
\usepackage{multirow} % used in OT tableaux
\usepackage{rotating} %needed for angled text%
\usepackage{colortbl} % for cell shading
 
 \usepackage{avm}  
\usepackage[linguistics]{forest} 
\usetikzlibrary{matrix,fit} % for matrix of nodes in Kaisse and Bat-El


\usepackage{hhline}
\newcommand{\cgr}{\cellcolor[gray]{0.8}}
\newcommand{\cn}{\centering}



\newcommand{\reff}[1]{(\ref{#1})}
%\usepackage{newtxtext,newtxmath}


%\usepackage[normalem] {ulem}
\usepackage{qtree}
%\usepackage{natbib}
%\usepackage{tikz}
%\usepackage{gb4e}
\usepackage{phonrule}  
%\bibliographystyle{humannat}



\usepackage{minibox}

%\include{psheader-metr}

\def\bl#1{$_{\textrm{{\footnotesize #1}}}$}
\usepackage{arydshln}
\usepackage{rotating}

%%add all your local new commands to this file

\newcommand{\form}[1]{\mbox{\emph{#1}}}
\newcommand{\uf}[1]{\mbox{/#1/}}

% borrowed from expex and converted from plan tex to latex
\newcommand{\judge}[1]{{\upshape #1\hspace{0.1em}}}
\newcommand{\ljudge}[1]{\makebox[0pt][r]{\judge{#1}}}

\newcommand\tikzmark[1]{\tikz[remember picture, baseline=(#1.base)] \node[anchor=base,inner sep=0pt, outer sep=0pt] (#1) {#1};} % for adding decorations, arrows, lines, etc. to text
\newcommand\tikzmarknamed[2]{\tikz[remember picture, baseline=(#1.base)] \node[anchor=base,inner sep=0pt, outer sep=0pt] (#1) {#2};} % for adding decorations, arrows, lines, etc. to text
\newcommand\tikzmarkfullnamed[2]{\tikz[remember picture, baseline=(#1.base)] \node[anchor=base,inner sep=0pt, outer sep=0pt] (#1) {\vphantom{X}#2};} % for adding decorations, arrows, lines, etc. to text; this one works best for decorations above a line of text because it adds in the heigh of a capital letter and takes two arguments - one for the node name and one for the printed text

\newcommand{\sub}[1]{$_{\text{#1}}$} % for non-math subscripts
\newcommand{\subit}[1]{\sub{\textit{#1}}} % for italics non-math subscripts
\newcommand{\1}{\rlap{$'$}\xspace} % for the prime in X' (the \rlap command allows the prime to be ignored for horizontal spacing in trees, and the \xspace command allows you to use this in normal text without adding \ afterwards). This isn't crucial, but it helps the formatting to look a little better.

% Aissen:
\newcommand\tikzmarkfull[1]{\tikz[remember picture, baseline=(#1.base)] \node[anchor=base,inner sep=0pt, outer sep=0pt] (#1) {\vphantom{X}#1};} % for adding decorations, arrows, lines, etc. to text; this one works best for decorations above a line of text because it adds in the heigh of a capital letter and takes one argument that serves as the name and the printed text
\newcommand{\bridgeover}[2]{% for a line that creates a bridge over text, connecting two nodes
	\begin{tikzpicture}[remember picture,overlay]
	\draw[thick,shorten >=3pt,shorten <=3pt] (#1.north) |- +(0ex,2.5ex) -| (#2.north);
	\end{tikzpicture}
}
\newcommand{\bridgeoverht}[3]{% for a line that creates a bridge over text, connecting two nodes and specifing the height of the bridge
	\begin{tikzpicture}[remember picture,overlay]
	\draw[thick,shorten >=3pt,shorten <=3pt] (#2.north) |- +(0ex,#1) -| (#3.north);
	\end{tikzpicture}
}
\newcommand{\bridgeoverex}{\vspace*{3ex}} % place before an example that has a \bridgeover so that there is enough vertical space for it

% Chung:
\newcommand{\lefttabular}[1]{\begin{tabular}{p{0.5in}}#1\end{tabular}}

% Kaisse:
\newcommand{\mgmorph}[1]{|(#1)| {#1}}
\newcommand{\mgone}[2][$\times$]{\node at (#2.base) [above=2ex] (1#2) {\vphantom{X}#1};}
\newcommand{\mgtwo}[2][$\times$]{\mgone{#2} \node at (#2.base) [above=4.5ex] (2#2) {\vphantom{X}#1};}
\newcommand{\mgthree}[2][$\times$]{\mgtwo{#2} \node at (#2.base) [above=7ex] (3#2) {\vphantom{X}#1};}
\newcommand{\mgftl}[1]{\node at (1#1) [left] {(};}
\newcommand{\mgftr}[1]{\node at (1#1) [right] {)};}
\newcommand{\mgfoot}[2]{\mgftl{#1}\mgftr{#2}}
\newcommand{\mgldelim}[2]{\node at (#2.west) [left,inner sep = 0pt, outer sep = 0pt] {#1};}
\newcommand{\mgrdelim}[2]{\node at (#2.east) [right,inner sep = 0pt, outer sep = 0pt] {#1};}

\newcommand{\squish}{\hspace*{-3pt}}

% Kavitskaya:
\newcommand{\assoc}[2]{\draw (#1.south) -- (#2.north);}
\newcolumntype{L}{>{\raggedright\arraybackslash}X}

% Lepic & Padden:
\newcommand{\fitpic}[1]{\resizebox{\hsize}{!}{\includegraphics{#1}}} % from http://tex.stackexchange.com/a/148965/42880
\newcommand{\signpic}[1]{\includegraphics[width=1.36in]{#1}}
\newcolumntype{C}{>{\centering\arraybackslash}X}

% Spencer:

\newcommand{\textex}[1]{\textit{#1}\xspace}
\newcommand{\lxm}[1]{\textsc{#1}\xspace}

% Thrainsson:

\renewcommand{\textasciitilde}{\char`~} % for use with TTF/OTF fonts (see comments on http://tex.stackexchange.com/a/317/42880)
\newcommand{\tikzarrow}[2]{% for an arrow connecting two nodes
\begin{tikzpicture}[remember picture,overlay]
\draw[thick,shorten >=3pt,shorten <=3pt,->,>=stealth] (#1) -- (#2);
\end{tikzpicture}
}

\newlength{\padding}
\setlength{\padding}{0.5em}
\newcommand{\lesspadding}{\hspace*{-\padding}}
\newcommand{\feat}[1]{\lesspadding#1\lesspadding}

% Hammond

\usepackage[]{graphicx}\usepackage[]{xcolor}
%% maxwidth is the original width if it is less than linewidth
%% otherwise use linewidth (to make sure the graphics do not exceed the margin)
\makeatletter
\def\maxwidth{ %
  \ifdim\Gin@nat@width>\linewidth
    \linewidth
  \else
    \Gin@nat@width
  \fi
}
\makeatother

\definecolor{fgcolor}{rgb}{0.345, 0.345, 0.345}
\newcommand{\hlnum}[1]{\textcolor[rgb]{0.686,0.059,0.569}{#1}}%
\newcommand{\hlstr}[1]{\textcolor[rgb]{0.192,0.494,0.8}{#1}}%
\newcommand{\hlcom}[1]{\textcolor[rgb]{0.678,0.584,0.686}{\textit{#1}}}%
\newcommand{\hlopt}[1]{\textcolor[rgb]{0,0,0}{#1}}%
\newcommand{\hlstd}[1]{\textcolor[rgb]{0.345,0.345,0.345}{#1}}%
\newcommand{\hlkwa}[1]{\textcolor[rgb]{0.161,0.373,0.58}{\textbf{#1}}}%
\newcommand{\hlkwb}[1]{\textcolor[rgb]{0.69,0.353,0.396}{#1}}%
\newcommand{\hlkwc}[1]{\textcolor[rgb]{0.333,0.667,0.333}{#1}}%
\newcommand{\hlkwd}[1]{\textcolor[rgb]{0.737,0.353,0.396}{\textbf{#1}}}%
\let\hlipl\hlkwb

\usepackage{framed}
\makeatletter
\newenvironment{kframe}{%
 \def\at@end@of@kframe{}%
 \ifinner\ifhmode%
  \def\at@end@of@kframe{\end{minipage}}%
  \begin{minipage}{\columnwidth}%
 \fi\fi%
 \def\FrameCommand##1{\hskip\@totalleftmargin \hskip-\fboxsep
 \colorbox{shadecolor}{##1}\hskip-\fboxsep
     % There is no \\@totalrightmargin, so:
     \hskip-\linewidth \hskip-\@totalleftmargin \hskip\columnwidth}%
 \MakeFramed {\advance\hsize-\width
   \@totalleftmargin\z@ \linewidth\hsize
   \@setminipage}}%
 {\par\unskip\endMakeFramed%
 \at@end@of@kframe}
\makeatother

\definecolor{shadecolor}{rgb}{.97, .97, .97}
\definecolor{messagecolor}{rgb}{0, 0, 0}
\definecolor{warningcolor}{rgb}{1, 0, 1}
\definecolor{errorcolor}{rgb}{1, 0, 0}
\newenvironment{knitrout}{}{} % an empty environment to be redefined in TeX

\usepackage{alltt}

%revised version started: 12/17/16

%NEEDS: allbib.bib - already added to the master bibliography file.
%cited references only: bibexport -o mhTMP.bib main1-blx.aux
%PLUS sramh-img*, sramh.tex

%added stuff
\newcommand{\add}[1]{\textcolor{blue}{#1}}
%deleted stuff
\newcommand{\del}[1]{\textcolor{red}{(removed: #1)}}
%uncomment these to turn off colors
\renewcommand{\add}[1]{#1}
\renewcommand{\del}[1]{}

%shortcuts
\newcommand{\w}{\ili{Welsh}}
\newcommand{\e}{\ili{English}}
\newcommand{\io}{Input Optimization}




 \newcommand{\hand}{\ding{43}}
% \newcommand{\rot}[1]{\begin{rotate}{90}#1\end{rotate}} %shortcut for angled text%  
% \newcommand{\rotcon}[1]{\raisebox{-5ex}{\hspace*{1ex}\rot{\hspace*{1ex}#1}}}

%% add all extra packages you need to load to this file 
% \usepackage{todo} %% removed,cna use todonotes instead. % Jason reactivated
% \usepackage{graphicx} % not needed because forest loads tikz, which loads graphicx
\usepackage{tabularx}
\usepackage{amsmath} 
\usepackage{multicol}
\usepackage{lipsum}
\usepackage{longtable}
\usepackage{booktabs}
\usepackage[normalem]{ulem}
%\usepackage{tikz} % not needed because forest loads tikz
\usepackage{phonrule} % for SPE-style phonological rules
\usepackage{pst-all} % loads the main pstricks tools; for arrow diagrams in Hale.tex
%\usepackage{leipzig} % for gloss abbreviations
\usepackage[% for automatic cross-referencing
compress,%
capitalize,% labels are always capitalized in LSP style
noabbrev]% labels are always spelled out in LSP style
{cleveref}

% based on http://tex.stackexchange.com/a/318983/42880 for using gb4e examples with cleveref
\crefname{xnumi}{}{}
\creflabelformat{xnumi}{(#2#1#3)}
\crefrangeformat{xnumi}{(#3#1#4)--(#5#2#6)}
\crefname{xnumii}{}{}
\creflabelformat{xnumii}{(#2#1#3)}
\crefrangeformat{xnumii}{(#3#1#4)--(#5#2#6)}

%\usepackage[notcite,notref]{showkeys} %%removed, not helping CB.
%\usepackage{showidx} %%remove for final compiling - shows index keys at top of page.
 
\usepackage{langsci/styles/langsci-gb4e}  
 \usepackage{pifont}
% % OT tableaux                                                
% \usepackage{pstricks,colortab}  
\usepackage{multirow} % used in OT tableaux
\usepackage{rotating} %needed for angled text%
\usepackage{colortbl} % for cell shading
 
 \usepackage{avm}  
\usepackage[linguistics]{forest} 
\usetikzlibrary{matrix,fit} % for matrix of nodes in Kaisse and Bat-El


\usepackage{hhline}
\newcommand{\cgr}{\cellcolor[gray]{0.8}}
\newcommand{\cn}{\centering}



\newcommand{\reff}[1]{(\ref{#1})}
%\usepackage{newtxtext,newtxmath}


%\usepackage[normalem] {ulem}
\usepackage{qtree}
%\usepackage{natbib}
%\usepackage{tikz}
%\usepackage{gb4e}
\usepackage{phonrule}  
%\bibliographystyle{humannat}



\usepackage{minibox}

%\include{psheader-metr}

\def\bl#1{$_{\textrm{{\footnotesize #1}}}$}
\usepackage{arydshln}
\usepackage{rotating}

%%add all your local new commands to this file

\newcommand{\form}[1]{\mbox{\emph{#1}}}
\newcommand{\uf}[1]{\mbox{/#1/}}

% borrowed from expex and converted from plan tex to latex
\newcommand{\judge}[1]{{\upshape #1\hspace{0.1em}}}
\newcommand{\ljudge}[1]{\makebox[0pt][r]{\judge{#1}}}

\newcommand\tikzmark[1]{\tikz[remember picture, baseline=(#1.base)] \node[anchor=base,inner sep=0pt, outer sep=0pt] (#1) {#1};} % for adding decorations, arrows, lines, etc. to text
\newcommand\tikzmarknamed[2]{\tikz[remember picture, baseline=(#1.base)] \node[anchor=base,inner sep=0pt, outer sep=0pt] (#1) {#2};} % for adding decorations, arrows, lines, etc. to text
\newcommand\tikzmarkfullnamed[2]{\tikz[remember picture, baseline=(#1.base)] \node[anchor=base,inner sep=0pt, outer sep=0pt] (#1) {\vphantom{X}#2};} % for adding decorations, arrows, lines, etc. to text; this one works best for decorations above a line of text because it adds in the heigh of a capital letter and takes two arguments - one for the node name and one for the printed text

\newcommand{\sub}[1]{$_{\text{#1}}$} % for non-math subscripts
\newcommand{\subit}[1]{\sub{\textit{#1}}} % for italics non-math subscripts
\newcommand{\1}{\rlap{$'$}\xspace} % for the prime in X' (the \rlap command allows the prime to be ignored for horizontal spacing in trees, and the \xspace command allows you to use this in normal text without adding \ afterwards). This isn't crucial, but it helps the formatting to look a little better.

% Aissen:
\newcommand\tikzmarkfull[1]{\tikz[remember picture, baseline=(#1.base)] \node[anchor=base,inner sep=0pt, outer sep=0pt] (#1) {\vphantom{X}#1};} % for adding decorations, arrows, lines, etc. to text; this one works best for decorations above a line of text because it adds in the heigh of a capital letter and takes one argument that serves as the name and the printed text
\newcommand{\bridgeover}[2]{% for a line that creates a bridge over text, connecting two nodes
	\begin{tikzpicture}[remember picture,overlay]
	\draw[thick,shorten >=3pt,shorten <=3pt] (#1.north) |- +(0ex,2.5ex) -| (#2.north);
	\end{tikzpicture}
}
\newcommand{\bridgeoverht}[3]{% for a line that creates a bridge over text, connecting two nodes and specifing the height of the bridge
	\begin{tikzpicture}[remember picture,overlay]
	\draw[thick,shorten >=3pt,shorten <=3pt] (#2.north) |- +(0ex,#1) -| (#3.north);
	\end{tikzpicture}
}
\newcommand{\bridgeoverex}{\vspace*{3ex}} % place before an example that has a \bridgeover so that there is enough vertical space for it

% Chung:
\newcommand{\lefttabular}[1]{\begin{tabular}{p{0.5in}}#1\end{tabular}}

% Kaisse:
\newcommand{\mgmorph}[1]{|(#1)| {#1}}
\newcommand{\mgone}[2][$\times$]{\node at (#2.base) [above=2ex] (1#2) {\vphantom{X}#1};}
\newcommand{\mgtwo}[2][$\times$]{\mgone{#2} \node at (#2.base) [above=4.5ex] (2#2) {\vphantom{X}#1};}
\newcommand{\mgthree}[2][$\times$]{\mgtwo{#2} \node at (#2.base) [above=7ex] (3#2) {\vphantom{X}#1};}
\newcommand{\mgftl}[1]{\node at (1#1) [left] {(};}
\newcommand{\mgftr}[1]{\node at (1#1) [right] {)};}
\newcommand{\mgfoot}[2]{\mgftl{#1}\mgftr{#2}}
\newcommand{\mgldelim}[2]{\node at (#2.west) [left,inner sep = 0pt, outer sep = 0pt] {#1};}
\newcommand{\mgrdelim}[2]{\node at (#2.east) [right,inner sep = 0pt, outer sep = 0pt] {#1};}

\newcommand{\squish}{\hspace*{-3pt}}

% Kavitskaya:
\newcommand{\assoc}[2]{\draw (#1.south) -- (#2.north);}
\newcolumntype{L}{>{\raggedright\arraybackslash}X}

% Lepic & Padden:
\newcommand{\fitpic}[1]{\resizebox{\hsize}{!}{\includegraphics{#1}}} % from http://tex.stackexchange.com/a/148965/42880
\newcommand{\signpic}[1]{\includegraphics[width=1.36in]{#1}}
\newcolumntype{C}{>{\centering\arraybackslash}X}

% Spencer:

\newcommand{\textex}[1]{\textit{#1}\xspace}
\newcommand{\lxm}[1]{\textsc{#1}\xspace}

% Thrainsson:

\renewcommand{\textasciitilde}{\char`~} % for use with TTF/OTF fonts (see comments on http://tex.stackexchange.com/a/317/42880)
\newcommand{\tikzarrow}[2]{% for an arrow connecting two nodes
\begin{tikzpicture}[remember picture,overlay]
\draw[thick,shorten >=3pt,shorten <=3pt,->,>=stealth] (#1) -- (#2);
\end{tikzpicture}
}

\newlength{\padding}
\setlength{\padding}{0.5em}
\newcommand{\lesspadding}{\hspace*{-\padding}}
\newcommand{\feat}[1]{\lesspadding#1\lesspadding}

% Hammond

\usepackage[]{graphicx}\usepackage[]{xcolor}
%% maxwidth is the original width if it is less than linewidth
%% otherwise use linewidth (to make sure the graphics do not exceed the margin)
\makeatletter
\def\maxwidth{ %
  \ifdim\Gin@nat@width>\linewidth
    \linewidth
  \else
    \Gin@nat@width
  \fi
}
\makeatother

\definecolor{fgcolor}{rgb}{0.345, 0.345, 0.345}
\newcommand{\hlnum}[1]{\textcolor[rgb]{0.686,0.059,0.569}{#1}}%
\newcommand{\hlstr}[1]{\textcolor[rgb]{0.192,0.494,0.8}{#1}}%
\newcommand{\hlcom}[1]{\textcolor[rgb]{0.678,0.584,0.686}{\textit{#1}}}%
\newcommand{\hlopt}[1]{\textcolor[rgb]{0,0,0}{#1}}%
\newcommand{\hlstd}[1]{\textcolor[rgb]{0.345,0.345,0.345}{#1}}%
\newcommand{\hlkwa}[1]{\textcolor[rgb]{0.161,0.373,0.58}{\textbf{#1}}}%
\newcommand{\hlkwb}[1]{\textcolor[rgb]{0.69,0.353,0.396}{#1}}%
\newcommand{\hlkwc}[1]{\textcolor[rgb]{0.333,0.667,0.333}{#1}}%
\newcommand{\hlkwd}[1]{\textcolor[rgb]{0.737,0.353,0.396}{\textbf{#1}}}%
\let\hlipl\hlkwb

\usepackage{framed}
\makeatletter
\newenvironment{kframe}{%
 \def\at@end@of@kframe{}%
 \ifinner\ifhmode%
  \def\at@end@of@kframe{\end{minipage}}%
  \begin{minipage}{\columnwidth}%
 \fi\fi%
 \def\FrameCommand##1{\hskip\@totalleftmargin \hskip-\fboxsep
 \colorbox{shadecolor}{##1}\hskip-\fboxsep
     % There is no \\@totalrightmargin, so:
     \hskip-\linewidth \hskip-\@totalleftmargin \hskip\columnwidth}%
 \MakeFramed {\advance\hsize-\width
   \@totalleftmargin\z@ \linewidth\hsize
   \@setminipage}}%
 {\par\unskip\endMakeFramed%
 \at@end@of@kframe}
\makeatother

\definecolor{shadecolor}{rgb}{.97, .97, .97}
\definecolor{messagecolor}{rgb}{0, 0, 0}
\definecolor{warningcolor}{rgb}{1, 0, 1}
\definecolor{errorcolor}{rgb}{1, 0, 0}
\newenvironment{knitrout}{}{} % an empty environment to be redefined in TeX

\usepackage{alltt}

%revised version started: 12/17/16

%NEEDS: allbib.bib - already added to the master bibliography file.
%cited references only: bibexport -o mhTMP.bib main1-blx.aux
%PLUS sramh-img*, sramh.tex

%added stuff
\newcommand{\add}[1]{\textcolor{blue}{#1}}
%deleted stuff
\newcommand{\del}[1]{\textcolor{red}{(removed: #1)}}
%uncomment these to turn off colors
\renewcommand{\add}[1]{#1}
\renewcommand{\del}[1]{}

%shortcuts
\newcommand{\w}{\ili{Welsh}}
\newcommand{\e}{\ili{English}}
\newcommand{\io}{Input Optimization}




 \newcommand{\hand}{\ding{43}}
% \newcommand{\rot}[1]{\begin{rotate}{90}#1\end{rotate}} %shortcut for angled text%  
% \newcommand{\rotcon}[1]{\raisebox{-5ex}{\hspace*{1ex}\rot{\hspace*{1ex}#1}}}

%\input{localpackages.tex}
\usepackage{arydshln}
\usepackage{rotating}

%\input{localcommands.tex}
\newcommand{\tworow}[1]{\multirow{2}{*}{#1}}


\newcommand{\tworow}[1]{\multirow{2}{*}{#1}}


\newcommand{\tworow}[1]{\multirow{2}{*}{#1}}


 

\title{Root-based syntax and Japanese derivational morphology}

\author{%
Brent de Chene\affiliation{Waseda University}
}

% \sectionDOI{} %will be filled in at production
% \epigram{}

\abstract{
 This paper argues that the formation of transitive and
intransitive verb stems in Japanese, a process that has been widely seen
as supporting the Distributed Morphology view of derivational
stem-formation as performed by the syntax, cannot in fact be analyzed as
syntactic. The Japanese data are thus consistent with \citegen{anderson1982a}
claim that it is precisely that morphology traditionally classified as
inflectional that reflects syntactic operations.
}

\begin{document}
\maketitle

\section{Introduction}

In a well-known paper, \citet[587]{anderson1982a} proposes that ``\isi{Inflectional
morphology} is what is relevant to the \isi{syntax},'' where syntactically
relevant properties are those ``assigned to words by principles which
make essential reference to larger \isi{syntactic structures}.'' He claims
further that a delimitation of \isi{inflection} on this basis closely mirrors
the traditional understanding of where the boundary between inflection
and \isi{derivation} lies. In contrast, the \isi{Distributed Morphology} literature,
in treating syntax as root-based and stem formation of all types as
syntactic, denies significance to the traditional distinction between
inflection and derivation and renders vacuous the claim that inflection
is just that portion of morphology that realizes elements and properties
manipulated by the syntax.\footnote{The founding paper of the DM
  framework, \citet{halle1993a}, presents DM as a theory of
  inflection and makes no explicit claims about derivation, but the
  adoption of root-based syntax and the rejection of the
  inflection/derivation distinction are clear at least by \citet{marantz1997a,marantz2001a}. See below for further references.} The present paper
takes up the formation of transitive and intransitive verb stems in
\ili{Japanese}, a case that has been widely seen as supporting the DM view of
stem-formation as performed by the syntax, and argues that a closer look
reveals that the derivational processes in question cannot in fact be
analyzed as syntactic. In the end, then, the Japanese data is consistent
with Anderson's view that there is a fundamental distinction between
\isi{inflection} and \isi{derivation} and that the criterion of syntactic relevance
picks out just that morphology traditionally classified as
inflectional.\footnote{On a personal note, while I have taken the idea
  that inflection is precisely the syntactically relevant morphology as
  a guiding principle for many years, it was anything but obvious to me
  at the time Steve proposed it. It ranks high in my personal inventory
  of the many things I have learned from Steve, and I am happy to have
  this opportunity to reaffirm it in a volume dedicated to him.}

In recent years, the derivational morphology of the Japanese verb has
become a standard example (as in \citealt{harley2012a}) illustrating the DM claim
that syntax is \isi{root}-based -- the claim, that is, that along with
\isi{functional morphemes}, the atoms of syntactic computation are \isi{roots}
rather than (inflectable) stems or (inflected) words \citep[5]{embick2008a}. In particular, it has become widely accepted \citep[106]{marantz2013a}
that the Japanese \isi{suffixes} that create transitive and intransitive verb
stems are instances of little v, causative and inchoative, that attach
to roots and thus that the verb stems themselves are syntactic
constructions -- much like, say, the combination of a verb stem with a
tense element or a main verb with an auxiliary. Here, I note first that
these claims about the constituency of Japanese verb stems rest on a
restricted database that masks the fact that a significant number of
stems involve sequences of two transitivity-determining suffixes. I then
present the failure of two nested suffixes to interact in the way
expected of syntactic elements -- in particular, the fact that an inner
suffix must be taken as invisible for purposes of semantic
interpretation and argument structure -- as the first of several related
arguments casting doubt on the proposal to generate Japanese verb stems
syntactically.

The data on which DM theorists base their claim that the verbal
derivational suffixes of Japanese are instances of little v attaching to
roots is the appendix of \citet{jacobsen1992a}, which represents a light
revision of the appendix of \citet{jacobsen1982a}, and in turn appears lightly
revised as Appendix I in \citet{volpe2005a}. That appendix consists of roughly
350 pairs of isoradical intransitive and transitive verbs presented in
their citation forms (Imperfect/Nonpast Conclusive) and sorted into
sixteen classes depending on the derivational suffixes that appear at
the right edge of their stems. The fact that the Jacobsen/Volpe appendix
is limited to verb stems presented pairwise means that using it as a
basis for the identification of roots requires assuming for each
transitivity pair that there are neither stems of other lexical
categories nor verb stems outside the transitivity pair that provide
information about the relevant root. Section~\ref{background} below, in the context of
presenting background information on Japanese derivation, introduces a
number of cases in which this assumption is unjustified. The following
three sections, building on the observations of section~\ref{background}, present
reasons for doubting that verb stems are syntactically derived. While
for concreteness I refer throughout to the DM literature cited above and
related work, the argumentation is intended to apply to any proposal to
generate Japanese verb stems syntactically.

Section 3, first, shows that a substantial minority of verb stems
involve two transitivizing (T) or intransitivizing (I) suffixes (with
the four orders TT, TI, IT, II all attested), but that an outer suffix
must be taken to render an inner one null and void for purposes of
argument structure and semantic interpretation. Section 4 shows that the
same is true for the suffix pair \textit{-m-} (verbal) and \textit{-si-}
(adjectival), with the additional complication that the order in which
those two suffixes appear relative to a root R is an idiosyncratic
function of R. Section 5, finally, argues against a syntactic account of
stem formation on the basis of semantic change, claiming, for lexical
causatives in particular, that the diachronic instability of the
putatively compositional causative interpretation (much as if a phrase
like \textit{kick the bucket} were to lose its compositional
interpretation, retaining only the idiosyncratic one) shows that that
interpretation cannot have been based on a syntactic derivation in the
first place. In all of these cases, the behavior of the derivational
suffixes under consideration is contrasted with that of inflectional and
uncontroversially syntactic elements. Section 6, a brief conclusion,
sketches two possible non-syntactic approaches to derivational
morphology and speaker knowledge thereof and suggests that the choice
between them for cases like the one considered here remains a topic for
further research.

\section{Background}\label{background}

In considering the shortcomings of Jacobsen's \citeyearpar{jacobsen1982a,jacobsen1992a} appendix as a
database for Japanese verbal derivation, the first thing to note is that
the pairwise presentation of the data does not always adequately
represent the relations of isoradicality that hold among verb stems.
This is because a number of roots underlie three or (in at least one
case) four verb stems rather than two; in such cases, Jacobsen either
lists two pairs in separate places or, as we will see below, omits one
of the stems. In several cases involving three stems on a single root,
there are two pairs of stems differentiated at least roughly by root
alloseme, with a formal contrast for either transitives or intransitives
but not both. For example, the difference between the allosemes `solve'
and `dissolve, melt' of the root \textit{tok-} corresponds to a formal
distinction for transitives but not for intransitives, as shown in (\ref{solve})
and (\ref{solve2}).\footnote{Below, taking the distinction between inflection and
derivation in Japanese to be uncontroversial, I use \textit{stem} in the
traditional meaning `morpheme (sequence) subject to inflection' and cite
bare stems rather than inflected forms; ``(i)'' and ``(t)'' in glosses
indicate intransitive and transitive meanings, respectively.}


\ea\label{solve}
	\ea \textit{tok-e-} `be solved' 
	\ex \textit{tok-} `solve'
	\z

\ex\label{solve2} 
	\ea \textit{tok-e-} `melt (i)'
	\ex \textit{tok-as-} `melt (t)'
	\z
\z
In other cases, as in (\ref{connect}) and (\ref{float}), there is no alloseme-dependent
pairing, simply a triplet of isoradical stems.
\ea\label{connect}
	\ea \textit{tunag-ar-} `be connected'
	\ex \textit{tunag-e-} `connect (t)'
	\ex \textit{tunag-} `connect (t)'
	\z
\ex\label{float} 
	\ea \textit{uk-} `float (i)'
	\ex \textit{uk-ab-} `float (i)'
	\ex \textit{uk-ab-e-} `float (t)'
	\z
\z

\noindent In these last two cases, the policy of pairwise listing results in one
stem of each isoradical set (specifically, (3b) and (4a)) being left out
of the database.

In fairness to Jacobsen, it must be noted that morphological analysis
was not his aim in compiling his appendix. Most crucially for our
purposes, he nowhere refers to the notion ``root'', and it is only with
\citegen{volpe2005a} DM treatment that the root becomes a central concept in
the interpretation of the appendix data. Volpe's \citeyearpar[121 (note 27)]{volpe2005a} procedure for root extraction, however, amounts to simply peeling off
the outermost derivational suffix and labeling the residue a root, and
he has been followed implicitly in this practice by other DM theorists.

We should observe before proceeding that there are many cases,
illustrated by (5) below, in which Volpe's procedure does in fact yield
a root.

\ea
	\ea \textit{nao-r-} `get better (illness, injury); get repaired'
	\ex \textit{nao-s-} `cure; repair'
	\z
\z
(5) is clearly the kind of case \citet[106]{marantz2013a} has in mind when he
says about Japanese that ``there seems overwhelming support for
analyzing the suffixes signaling either the lexical causative as opposed
to the inchoative or the inchoative as opposed to the lexical causative
as realizations of a little \textit{v} head attaching to the root.'' As we
will now see, however, there are a number of respects in which the
properties of (5) do not generalize to the Japanese derivational system
as a whole. Most crucially, there is reliable evidence for a number of
Volpe's ``roots'' that they are actually morphologically complex, with
the result that many verb stems contain two derivational suffixes rather
than one. Given that, as we have already noted, Volpe's procedure for
root extraction involves no attempt to compare verb stems with stems of
other lexical classes or with verb stems outside the transitivity pair
under consideration, this result is unsurprising. Let us examine a few
representative cases.

Consider the sequence \textit{tunag-} of (3) above. Comparison of that
sequence, roughly meaning `connect', with the noun \textit{tuna} `rope'
suggests that the former is undersegmented, and in particular that the
transitive stem \textit{tunag-} consists of the noun stem \textit{tuna} (or
the root that underlies it) suffixed with \textit{-g-.} This suggestion is
confirmed when we observe that \textit{-g-} is suffixal in a number of
other stems as well, with a core subset ((6)-(7) below and the three of
note 3) displaying a very specific semantics: \textit{-g-} takes as input
a noun stem denoting a tool T and returns a verb stem with the meaning
``to make typical use of T''. Three examples that occasion
resegmentation of entries of the Jacobsen/Volpe appendix are given in
(6) through (8), with both a transitive and an intransitive stem noted
in each case.\footnote{Three further examples whose status in the
  contemporary language might be thought questionable are \textit{tumu-g-}
  `spin (thread)' (\textit{tumu} `spindle'), \textit{ha-g-} `fletch (arrow)'
  (\textit{ha} `feather'), and, with an irregular alternation of \textit{t}
  with \textit{s,} \textit{husa-g-} `cover, stop up' (\textit{huta} `cover').}
\ea
	 \ea \textit{tuna} `rope'
	 \ex \textit{tuna-g-} `tie together, tie up'
	 \ex \textit{tuna-g-ar-} `get connected'\footnote{Kunio Nishiyama (personal
  	communication) suggests the possibility that \textit{-g-} in (6) is a
  	(transitivity-neutral) verbalizer, with the transitivity of (6b)
  	resulting from a null transitivizer parallel to the intransitive
  	\textit{-ar-} of (6c). A fully general form of this proposal will
  	require the postulation of a very large number of morphological zeros.}
	\z
\ex 
	 \ea \textit{to(-isi)} `whetstone'
	 \ex \textit{to-g-} `whet'
	 \ex \textit{to-g-ar-} `become pointed'
	\z
\ex 
	 \ea \textit{mata} `crotch, fork'
	 \ex \textit{mata-g-} `step over, straddle (t)'
	 \ex \textit{mata-g-ar-} `straddle (i)'
	\z
\z
The derivational relationships postulated in (6)-(8) appear
unimpeachable in both formal and semantic terms: the roots are
nonalternating, and the semantic relationship between nominal and verbal
meanings is unmistakable.

More common as a stem-forming suffix than \textit{-g-} is \textit{-m-,}
which can be shown to be a stem formant in several dozen verbs. (9)-(11)
display three cases in which recognition of suffixal \textit{-m-} forces
resegmentation of strings that Volpe takes to be roots (the (a) items of
(9) and (10) are adjective stems, and that of (11) is an adjectival
noun, a stem with adjectival meaning but essentially nominal
inflection).
\ea
	 \ea \textit{ita-} `painful'
	 \ex \textit{ita-m-} `be painful, get injured'
	 \ex \textit{ita-m-e-} `injure'
	\z
\ex 
	 \ea \textit{yuru-} `slack'
	 \ex \textit{yuru-m-} `slacken (i)'
	 \ex \textit{yuru-m-e-} `slacken (t)'
	\z
\ex
	 \ea \textit{hiso-ka} `stealthy, secret'
	 \ex \textit{hiso-m-} `be hidden, lurk'
	 \ex \textit{hiso-m-e-} `conceal, mask'
	\z
\z
We have seen that in addition to verb stems formed with the common
suffixes \textit{-r-} and \textit{-s-,} illustrated in (5), there are verb
stems formed with \textit{-g-} and \textit{-m-.} In fact, of the nine
occurring stem-final consonants, all but \textit{n} can be shown to be
suffixal in some stems. Suffixal \textit{-b-} has been illustrated in (4b)
above; (12) through (14) display one example each for \textit{-k-,}
-\textit{t-,} and \textit{-w-} (\textit{w} deletes in the phrasal phonology
before nonlow vowels; here and below, I take reference to a suffix
-C(V)- to subsume reference to its post-consonantal allomorph -aC(V)-).

\ea
	 \ea \textit{na-k-} `make characteristic sound' (animal); `weep' (human)
	 \ex \textit{na-r-} `sound (i)' (inanimate subject)
	 \ex \textit{na-r-as-} `sound (t)'
	\z
\ex 
	 \ea \textit{hana-re-} `move (i) away (from); be released'
	 \ex \textit{hana-s-} `move (t) away (from); release'
	 \ex \textit{hana-t-} `release forcefully, discharge'
	\z
\ex 
	 \ea \textit{muk-} `face, look (in a direction)'
	 \ex \textit{muk-e-} `cause to face, turn (t) (in a direction)'
	 \ex \textit{muk-aw-} `face, proceed toward'
	 \ex \textit{muk-aw-e-} `(go to) meet, receive (a visitor)'
	\z
\z
We see, then, that the inventory of suffixes that create verb stems of
determinate transitivity is a good deal larger than envisioned in the
Jacobsen/Volpe appendix, where, apart from idiosyncratic formations, the
relevant set is essentially limited to \textit{-r-}, \textit{-s-},
\textit{-re-}, \textit{-se-}, \textit{-e-}, \textit{-i-}, and zero. In closing
this introductory section, let us consider two semantic issues that
arise with respect to the Jacobsen/Volpe appendix data. The first
involves the interpretation of roots, the second the interpretation of
suffixes.

Quite apart from the question of whether or not roots are taken to be
elements that are manipulated by the syntax, no attempt to segment stems
into roots and suffixes synchronically is a fully grounded project in
the absence of a criterion for isoradicality -- a criterion, that is, for
determining when two given stems share a root and when they do not. The
semantic lability of individual stems over time that will be illustrated
in section 5 makes this by no means an idle question. It is, however, a
question that neither Jacobsen nor Volpe engage with seriously; \citet[38]{jacobsen1982a}\footnote{See also note 5, p.34 and the corresponding note 30 of
  \citealt{jacobsen1992a} (pp. 248--249).} says only that the members of a transitivity
pair must exhibit ``a certain degree of semantic affinity'', and \citet[32]{volpe2005a} confines himself to observing that ``Root semantics is a
wide-open area for further research''. The question of isoradicality is
essentially coextensive with the traditional problem of distinguishing
homophony from polysemy, a problem that may ultimately be illuminated by
psycholinguistic and neurolinguistic research (see \citealt[103]{marantz2013a}). It
is worth keeping in mind, however, that any program involving the
synchronic identification of roots requires innumerable provisional
decisions on this matter.

Turning now to the interpretation of the stem-forming suffixes of which
we have seen a number of examples, let us note first that while \citet{volpe2005a} follows \citet{jacobsen1982a,jacobsen1992a} in referring to the two members of
a transitivity pair as ``intransitive'' and ``transitive'', more recent
literature such as \citet{harley2008a,harley2012a} and \citet{marantz2013a} use the more
specific ``inchoative'' and ``causative''. In fact, cases like
\textit{ka-r-} (Western Japan; cf. Eastern \textit{ka-ri-}) `borrow' versus
\textit{ka-s-} `lend' and \textit{azuk-ar-} `take on deposit' versus
\textit{azuk-e-} `deposit' show that even the former pair of terms is too
specific to be accurate in general. This is because the first member of
each of those pairs shows ``intransitive'' morphology in spite of
displaying what, under Burzio's generalization, are the twin hallmarks
of causative little v, namely an agentive external argument and
accusative case-marking. Cross-linguistic parallels\footnote{See \citet[59, 84--85, 107]{kuo2015a} for the Taiwanese languages Amis, Puyama, and Seediq, respectively; other languages for which the relationship can
  be easily verified include Tagalog and Swahili.} suggest that the
treatment of `borrow' as the intransitive counterpart of `lend' is by no
means accidental or exceptional. The phenomenon of a stem with causative
meaning but ``intransitive'' morphology appears to show that if the
semantics of the two morphological types are specified separately, they
will have to overlap. Let us briefly note another type of example that
suggests the same conclusion.

The stems \textit{too-r-} `pass through' and \textit{mata-g-} `step over,
pass over, straddle' ((8b) above) are closely parallel in both their
semantics and their case-marking. When the subject is animate, as in
(15) (where stem-internal segmentation is suppressed), that subject
(marked nominative but omitted in the examples) is both an agent and a
theme moving along a path, and the accusative object is an intermediate
point on that path.
\ea
	\ea 
	\gll Syootengai o toot-te eki ni modot-ta.\\
	shopping.district \textsc{acc} pass.through-\textsc{cj} station \textsc{dat} return-\textsc{pf}\\
	\glt ``I passed through the shopping district and returned to the station.''

	\ex 
	\gll Saku o matai-de hodoo ni hait-ta. \\
	barrier \textsc{acc} step.over-\textsc{cj} sidewalk \textsc{dat} enter-\textsc{pf}\\
	\glt ``I stepped over the barrier and onto the sidewalk.''
	\z
\z
In other uses, the agent of examples (15) may be replaced by an
inanimate theme, with \textit{matag-} in the meaning ``pass over'', or by
a path argument, as in \textit{The road passes through the tunnel/over the
train tracks.}

In spite of the close semantic parallelism between \textit{too-r-} and
\textit{mata-g-}, however, the two stems differ in their transitivity
status: \textit{too-r-} is the intransitive corresponding to transitive
\textit{too-s-} `pass though (t)', while \textit{mata-g-} is the transitive
corresponding to intransitive \textit{mata-g-ar-} `straddle' ((8c) above),
the latter differing from \textit{mata-g-} in taking a dative rather than
an accusative object. Unless \textit{too-r-} and \textit{mata-g-} are
semantically distinct in a way we have failed to identify, this fact
shows that the transitivity status of a stem cannot be a function of
that stem's semantics alone, and a fortiori cannot be a function of the
semantics of that stem's suffix. An alternative possibility, which
considerations of space preclude developing here, is that there is a
continuum of degrees of transitivity, as suggested by \citet{hopper1980a} and subsequent work, and that what transitivity pairs
have in common is that the ``transitive'' member has a higher degree of
transitivity than the ``intransitive'' member.\footnote{\citet[73--74]{jacobsen1992a} develops a scalar concept of transitivity but does not suggest that the common point of transitivity pairs is a transitivity
  differential in favor of the morphologically transitive member.} In
any case, however, the evidence we have seen here is sufficient to
establish that there is no simple, general account of the semantics of
the suffixes that create transitivity-specific Japanese verb stems, and
that, as was the case regarding the question of a criterion for
isoradicality, much work remains to be done in this area.

Above, we have seen that the data of the Jacobsen/Volpe appendix is a
good deal more complex and irregular, both formally and semantically,
than consideration of examples like (5) might suggest. Nothing in the
present section, however, is intended as an argument for or against any
particular treatment of that data. Taking our discussion of the
Jacobsen/Volpe appendix as a starting point, we now turn, in sections 3
through 5, to arguments against proposals to generate Japanese verb
stems syntactically.

\section{Sequences of verbal suffixes}\label{sequences}

As we have already noted, one consequence of the resegmentations that
are entailed by comparing the stems that participate in transitivity
pairs with stems of other lexical categories (as well as with other verb
stems) is that many stems can be seen to display a sequence of two
suffixes attached successively to a root rather than a single
transitivity-determining suffix. For example, the (c) examples of (6)
through (8) above all involve the sequence \textit{-g-ar-,} where the
first suffix creates a transitive stem and the second an intransitive.
Similarly, the (c) examples of (9) through (11) all involve
\textit{-m-e-,} where the first suffix creates an intransitive stem and
the second a transitive. Suffix sequences are also observed in (12c) and
(14d).

Sequences of two transitivizing suffixes and two intransitivizing
suffixes are observed as well. For example, (16d) below, where (16) is
an expansion of (6), involves the sequence \textit{-g-e-,} where both
suffixes create transitive stems, and (17c) involves the sequence
\textit{-m-ar-,} where both suffixes create intransitive stems.
\ea
	 \ea \textit{tuna} `rope'
	 \ex \textit{tuna-g-} `tie together, tie up'
	 \ex \textit{tuna-g-ar-} `get connected'
	 \ex \textit{tuna-g-e-} `tie together, connect'
	\z
\ex 
	 \ea \textit{yasu-raka} `peaceful, calm'
	 \ex \textit{yasu-m-} `rest (i)'
	 \ex \textit{yasu-m-ar-} `become rested, at ease'
	 \ex \textit{yasu-m-e-} `rest (t)'
	\z
\z
Recall now the DM claim that Japanese transitivity-determining suffixes
are instances of little v, with at least an inchoative and a causative
``flavor'' \citep[107]{marantz2013a} to be distinguished. Abstracting away from
the fact that (at a minimum) both types of little v will have to be
polysemous, and writing the inchoative version as ``v\textsubscript{i}''
and the causative version as ``v\textsubscript{c}'', the structure of
the two stems of (5), for example, will be as shown in (18) (simplified
glosses given) .
\ea
	\ea \textit{nao-r-} {[}{[}R{]}v\textsubscript{i}{]} `get better'
	\ex \textit{nao-s-} {[}{[}R{]}v\textsubscript{c}{]} `make better'
	\z
\z
In the same way, the structure of the stems (16c-d) will be as in
(19), and that of the stems (17c-d) will be as in (20). (Here and
below, I take the fact that \textit{-g-} and \textit{-m-} (and also
\textit{-b-, -k-, -t-, -w-}) in isolation are entirely parallel in
function to the suffixes the DM literature treats as little v (notably
\textit{-r-, -s-,} and \textit{-e-} (see e.g. \citealt[108]{marantz2013a})) to license
a parallel treatment for them in the DM framework we are taking as
representative of syntactic treatments of derivation.)
\ea
	\ea \textit{tuna-g-ar-} {[}{[}{[}R{]}v\textsubscript{c}{]}v\textsubscript{i}{]} `connect (i)'
	\ex \textit{tuna-g-e-} {[}{[}{[}R{]}v\textsubscript{c}{]}v\textsubscript{c}{]} `connect (t)'
	\z
\ex 
	\ea \textit{yasu-m-ar-} {[}{[}{[}R{]}v\textsubscript{i}{]}v\textsubscript{i}{]} `get rested'
	\ex \textit{yasu-m-e-} {[}{[}{[}R{]}v\textsubscript{i}{]}v\textsubscript{c}{]} `rest (t)'
	\z
\z
If the representations of (19)-(20) are constructed in the syntax, in
line with the proposal that roots and functional morphemes are the
primitives of syntactic derivation, we will expect them to be
interpreted compositionally, with the meaning of the outer little v
combining with the result of composing the meaning of the inner little v
with that of the root. In fact, no verb stem has an interpretation that
involves two units of ``little v meaning'', either two instances of
`inchoative' or two instances of `causative' or one of each; for
interpretive purposes, the only little v that matters in representations
like those of (19)-(20) is the outer one.\footnote{While the
  v\textsubscript{i} of (20b) could be taken to be semantically active,
  the meaning of such causatives would have to coincide with that of
  causatives derived from roots, as in (18b). The semantic inertness of
  the inner little v thus follows for this case as for the others. (In
  DM, identification of category-determining elements with phase heads
  requires that lexical causatives, being monophasal, be root-based
  \citep{marantz2007a}.)} This is as if, when the Perfect auxiliary occurs
outside of the Progressive in English or the Passive outside of the
(productive) Causative in Japanese, as illustrated in (21), the outer
auxiliary were to nullify the interpretation of the inner one rather
than composing with it semantically.
\ea
	\ea have been eating { } {[}PERF{[}PROG{[}V{]}{]}{]}
	\ex \textit{tabe-sase-rare-} { } {[}{[}{[}V{]}CAUS{]}PASS{]} `be made to eat'
	\z
\z
It would seem that in uncontroversially syntactic constructions like
those of (21), this kind of nullification never occurs, and thus that we
can assume that the syntactic computational system includes no mechanism
for opting out of compositional interpretation in this way. The
structures of (19)-(20) therefore pose a major problem for the idea that
the suffixes deriving Japanese verb stems are syntactic elements.

We have seen that the syntactic status of constructions like (19)-(20)
is called into question by their interpretive properties. The
representations of (19) pose a second problem as well, namely that the
internal v\textsubscript{c} will introduce an external argument that
must ultimately remain unrealized.\footnote{The causative interpretation
  and the external argument may in fact be introduced by separate heads \citep[chapter 3]{pylkka2008a}; what is important for our purposes is that in the data at hand they are both present when a transitivizing
  suffix appears alone but absent when it appears inside another
  transitivity-determining suffix.} In the remainder of this section, I
concentrate on documenting further instances of the construction (19a),
verb stems that introduce no external argument in spite of containing a
transitivizing suffix.

Consider first the isoradical sets (22)-(25), all of which illustrate
the suffix sequence \textit{-r-e-.}\footnote{Taking the root to be
  \textit{maku-} in (22) obviates postulating a new suffix allomorph for
  the b. and c. examples but requires a rule deleting a root-final vowel
  in a zero-derived verb stem for (22a). Given also a rule \textit{a} +
  \textit{i} \textgreater{} \textit{e,} mirroring the presumed historical
  development (see \citealt{ono1953a}  and subsequent literature), many apparently
  consonant-final roots could be reanalyzed along parallel lines; for
  example, the stems of (1)-(2) above could be \textit{tok-,}
  \textit{toka-i-, toka-s-} (√toka) rather than \textit{tok-, tok-e-,}
  \textit{tok-as-} (√tok).}
\ea
	 \ea \textit{mak-} `roll up, wind around'
	 \ex \textit{maku-r-} `roll up, tuck up'
	 \ex \textit{maku-r-e-} `get turned up, ride up'
	\z
\ex 
	 \ea \textit{nezi} `screw'
	 \ex \textit{nezi-r-} `twist'
	 \ex \textit{nezi-r-e-} `get twisted'
	\z
\ex  
	 \ea \textit{yabu-k-} `rip (t)'
	 \ex \textit{yabu-r-} `rip (t)'
	 \ex \textit{yabu-r-e-} `rip (i)'
	\z
\ex 
	 \ea \textit{kasu-ka} `faint, at the limits of perception'
	 \ex \textit{kasu-m-} `become hazy, dim'
	 \ex \textit{kasu-m-e-} `cloud (the vision of), deceive; graze, skim over; skim off, steal'
	 \ex \textit{kasu-r-} `graze (touch lightly in passing)'
	 \ex \textit{kasu-r-e-} `become faint or discontinuous (printing, writing); become hoarse (voice)'
	\z
\z
The stems of (22)-(25) are all in common use in contemporary Japanese; a
final parallel set that is particularly transparent semantically but for
which the verb stems are obsolete is \textit{kubi} `neck', \textit{kubi-r-}
`strangle', \textit{kubi-r-e-} `die by hanging oneself'.

Examples of the construction (19a) involving the suffix sequence
\textit{-m-ar-} can also be cited, as in (26)-(28). (26a) reflects the
fact, not previously exemplified, that bare roots not infrequently occur
reduplicated as adverbial items of the mimetic vocabulary.
\ea
	 \ea \textit{kurukuru} `round and round (rotation, winding)'
	 \ex \textit{kur-} `reel in, wind'
	 \ex \textit{kuru-m-} `wrap by rolling'
	 \ex \textit{kuru-m-ar-} `be rolled up, wrapped up'
	 \ex \textit{kuru-m-e-} `lump together'
	\z
\ex 
	 \ea \textit{tuka} `hilt, handle'
	 \ex \textit{tuka-m-} `grasp' (accusative object)
	 \ex \textit{tuka-m-ar-} `be caught, captured'; `hold on to' (dative object)
	 \ex \textit{tuka-m-aw-e-} `catch, capture'
	\z
\ex
	 \ea \textit{haza-ma} `gap, interstice' (\textless{} \textit{hasa-ma} (\textit{ma} `interval'))
	 \ex \textit{hasa-m-} `insert between'
	 \ex \textit{hasa-m-ar-} `get caught between'
	\z
\z
In (6)-(8) and (22)-(28), then, we have seen examples in which
intransitivizing suffixes appear outside transitivizing suffixes,
resulting in stems of the shape (19a). These are structures for which,
as a result of the internal v\textsubscript{c}, both a causative
interpretation and an external argument are predicted, but do not
materialize. We have already argued that the syntactic status of all
four constructions (19)-(20) is called into question by the fact that
the inner little v of those constructions is never interpreted.
Regarding the unrealized external argument of stems of the shape (19a),
similarly, it is clear that there is no way, in a system of syntactic
derivation based on selectional features and the Merge operation and
restricted by a ``no tampering'' condition \citep[138]{chomsky2008a}, for a
specifier introduced by one head to be deleted or ignored as a
consequence of merger of a higher head. The conclusion seems
inescapable, then, that a system of stem-formation that allows stems of
the form (19a), and stems of the form (19)-(20) more generally, cannot
be the result of the syntactic computational system.

\section{Verbal \textit{-m-} and adjectival \textit{-si-}}\label{verbal}

In (19)-(20) above, we saw that transitivizing and intransitivizing
suffixes, characterized as v\textsubscript{c} and v\textsubscript{i}
respectively, can occur in any of the four logically possible orders
following a root. We have not seen any examples, however, in which the
members of an individual pair of suffixes appear in a given order after
one set of roots but in the opposite order after another set. For
example, the suffixes of the sequence \textit{-g-e-} always occur in that
order regardless of their status as transitivizing or intransitivizing.
In fact, there are three possibilities in that regard: both suffixes can
be transitivizing, as in (16d), the first can be intransitivizing and
the second transitivizing, as in \textit{yawa-ra-g-e}- `soften (t)' (cf.
\textit{yawa-ra-g-} `soften (i)'), or the first can be transitivizing and
the second intransitivizing, as in \textit{hisya-g-e-} $\sim$ \textit{hisi-g-e-}
`be crushed' (cf. \textit{hisya-g-} $\sim$ \textit{hisi-g-} `crush'). In this
section we will observe two suffixes, one deriving verb stems and the
other adjective stems, for which there are four modes of attachment to a
root: direct affixation of each suffix, verbal suffix preceding
adjectival, adjectival suffix preceding verbal, and both orders with the
same root. It will be argued that both the fact that only the outer
suffix is interpreted, parallel with what we saw in section 3, and the
fact that the relative position of the suffixes is an idiosyncratic
function of the individual root militate against treating the suffixes
as syntactic elements.

Many Japanese roots support both a verb stem in \textit{-m-,} exemplified
in section 3, and an adjective stem formed with the suffix \textit{-si-.}
While adjective stems in \textit{-si-} are not treated in the DM
literature on Japanese derivation, that suffix has a natural DM analysis
as a category-determining little a, where the latter is a stative
counterpart of inchoative v\textsubscript{i} and causative
v\textsubscript{c} \citep[103]{marantz2013a}. In the examples of (29)-(30),
both suffixes attach directly to a root, making those examples parallel,
as the displayed structure shows, to the verb stems \textit{nao-r-} and
\textit{nao-s-} that we saw in (5) and (18) (the root of (30) also
supports a stem \textit{kuy-i-} that is a close synonym of (30b); \textit{y}
deletes before a front vowel in the phrasal phonology).
\ea
	\ea \textit{suzu-si-} {[}{[}R{]}a{]} `cool, refreshing'
	\ex \textit{suzu-m-} {[}{[}R{]}v\textsubscript{i}{]} `cool off, refresh oneself'
	\z
\ex 
	\ea \textit{kuy-asi-} {[}{[}R{]}a{]} `causing chagrin, regret'
	\ex \textit{kuy-am-} {[}{[}R{]}v\textsubscript{c}{]} `rue, regret'
	\z
\z
There are a number of roots supporting both types of stem seen in
(29)-(30), however, for which the verb stem in \textit{-m-} is derived
from the adjective stem in \textit{-si-.} This is illustrated in (31)-(32)
(I take \textit{-si-} to be suffixal in an otherwise unsegmentable
CVCV\textit{si-} adjective stem).
\ea
	\ea \textit{kuru-si-} {[}{[}R{]}a{]} `painful, uncomfortable, difficult'
	\ex \textit{kuru-si-m-} {[}{[}{[}R{]}a{]}v\textsubscript{i}{]} `suffer'
	\z
\ex 
	\ea \textit{kana-si-} {[}{[}R{]}a{]} `sad'
	\ex \textit{kana-si-m-} {[}{[}{[}R{]}a{]}v\textsubscript{i}{]} `grieve, sorrow'
	\z
\z
And there are roots for which, in contrast, the verb stem in \textit{-m-,}
whether transitive (as in (33b)) or intransitive (as in (34b)) serves as
the base for derivation of the adjective stem in \textit{-si-}:\footnote{For
  an English parallel to the three types (29)-(30), (31)-(32), (33)-(34),
  consider \textit{ambigu-ous/ity, duplic-it-ous, monstr-os-ity.}}
\ea
	\ea \textit{uto-} {[}{[}R{]}a{]} `distant, ill-informed'
	\ex \textit{uto-m-} {[}{[}R{]}v\textsubscript{c}{]} `shun, ostracize'
	\ex \textit{uto-m-asi-} {[}{[}{[}R{]}v\textsubscript{c}{]}a{]} `unpleasant, repugnant'
	\z
\ex 
	\ea \textit{ita-} {[}{[}R{]}a{]} `painful'
	\ex \textit{ita-m-} {[}{[}R{]}v\textsubscript{i}{]} `be painful; get damaged'
	\ex \textit{ita-m-asi-} {[}{[}{[}R{]}v\textsubscript{i}{]}a{]} `pitiable, pathetic'
	\z
\z
Finally, there is at least one root for which both the verb stem in
\textit{-m-} and the adjective stem in \textit{-si-} contain both suffixes,
in the opposite order in the two cases:
\ea
	\ea \textit{tutu-m-asi-} {[}{[}{[}R{]}v\textsubscript{c}{]}a{]} `modest, unpretentious'
	\ex \textit{tutu-si-m-} {[}{[}{[}R{]}a{]}v\textsubscript{c}{]} `be cautious regarding; abstain from'
	\z
\z
What conclusions can we draw from the data of (29)-(35)? First of all,
with regard to interpretation, those examples support the same
observation that was made in section 3 for stems of the four types in
(19)-(20), namely that when a stem contains two derivational suffixes,
the inner one is interpretively inert.\footnote{While one might imagine
  for some of the doubly suffixed stems of (31)-(35) that the
  interpretation of the whole depends in some way on that of the inner
  suffix, there is evidence against this idea in some cases. With
  respect to (34), for example, the root-reduplicated adjective
  \textit{itaita-si-} `pitiable, pathetic' shows that the occurrence of
  that meaning for the stem \textit{ita-m-asi-} has nothing to do with the
  inner suffix \textit{-m-.}} The semantic relations of the two stems to
each other and to the root in (35), for example, are roughly the same as
in (29)-(30), even though the stems of (35) each contain two suffixes
and the stems of (29)-(30) only one. This observation, as we have seen,
casts doubt on the proposal that the suffixes in question are syntactic
elements.

A parallel argument can be made regarding the relative position of
suffixes. (19)-(20) have already shown, of course, that if suffixes are
divided into transitivizing (``causative'') and intransitivizing
(``inchoative'') types, there are no constraints on their relative order
when two of them occur in the same stem, so that their actual order in
particular cases becomes a function of the individual root. As suggested
by the discussion of the suffix sequence \textit{-g-e-} at the beginning
of this section, though, if we classify suffixes on strictly
distributional grounds, without reference to transitivity value, it is
possible to set up two position classes that will obviate conditioning
of suffix order by roots in the great majority of cases: roughly
speaking, the suffixes recognized by the Jacobsen/Volpe segmentation of
stems will belong to the outer layer, with the inner layer being
composed of suffixes such as \textit{-g-, -m-, -w-,} and
(transitivity-neutral) \textit{-r-.}

For the data of (29)-(35), however, conditioning of suffix order by
individual roots is inescapable. This, then, constitutes a second way,
independent of the interpretive inertness of the inner suffix, in which
the behavior of \textit{-m-} and \textit{-si-} fails to conform to what we
would expect of syntactic elements. Returning to the analogy with
auxiliary verbs that we appealed to in section 3 (see (21) above), the
positional relations of those two suffixes are as if the Perfect and the
Progressive auxiliaries (say) both appeared adjacent to the stem for one
class of verbs, but the Perfect was formed by placing the Perfect
auxiliary outside the Progressive for a second class of verbs, and the
Progressive was formed by placing the Progressive auxiliary outside the
Perfect for a third class. The reason, of course, that this is difficult
to imagine is that we expect unambiguously syntactic elements to appear
in a fixed order with respect to a verbal or nominal stem. Indeed, since
the 1990s, a great deal of work in cartographic syntax (notably \citealt{cinque1999a}) has developed the idea that the (hierarchical) ordering of
syntactic functional heads is fixed not only internally to a single
language, but universally. From that perspective, the radical failure of
Japanese verbal \textit{-m-} and adjectival \textit{-si-} to display a
consistent ordering makes it extremely difficult to view them as
syntactic heads.

\section{Compositional meanings and semantic change}\label{composition}

We have claimed that the syntactic computational system includes no
mechanism for opting out of compositional interpretation, in particular
by allowing a higher head to nullify the interpretation of a lower one.
More generally, it seems reasonable to assume that the compositional
interpretation of structures generated by the syntax is automatic, so
that there is no way to block the compositional interpretation of a
syntactic constituent.\footnote{I will assume that this principle is not
  compromised by the delayed transfer to the interfaces characteristic
  of phase-based derivation \citep{chomsky2001b}.} We expect it to be true, in
other words, that no instance of a syntactically generated structure or
construction can idiosyncratically fail to display the compositional
semantic interpretation associated with that structure or
construction.\footnote{Correspondingly, establishing that some phrase P
  is a counterexample to this principle will require (a) displaying P's
  syntactic structure; (b) displaying the rule of interpretation
  associated with that structure; and (c) showing that P
  idiosyncratically lacks the predicted interpretation.} As a result, a
phrase like \textit{kick the bucket} that is demonstrably generated by the
syntax will automatically have the compositional interpretation
predicted by its lexical items and its syntactic structure,
independently of whether it has one or more listed interpretations as
well. As a diachronic corollary, we can infer that loss of the
compositional interpretation of a syntactically generated constituent is
not a possible change, assuming that the grammar and the lexicon have
remained stable in the relevant respects. Thus, it would not be possible
for \textit{kick the bucket} to lose its compositional interpretation over
time, retaining only the idiomatic one. When a phrase that was once
generated by the syntax does have only a listed interpretation, it is
either because the component words have dropped out of the lexicon, as
is probably the case for the phrase \textit{to plight one's troth} for
most contemporary English speakers, or because the grammar no longer
generates phrases of the type in question, as is the case for the phrase
\textit{till death do us part.}

What is true for manifestly phrasal constituents is true for inflected
forms as well. Lexicalization (i.e. idiomatization) of \textit{guts} in
the meaning `courage' and \textit{balls} in the meaning `audacity' has no
effect on the status of those forms as regular plurals as long as the
relevant stems and the rules for forming and interpreting plurals are
diachronically stable. In Japanese, many verbal Gerund forms in
\textit{\nobreakdash-te} are lexicalized as adverbs: \textit{sitagatte, yotte}
`consequently' (\textit{sitagaw-} `obey', \textit{yor-} `be due to'),
\textit{kiwamete, itatte} `extremely' (\textit{kiwame-} `reach, carry to
extremity', \textit{itar-} `reach'). As long as the relevant verb stems
remain in the lexicon and \textit{-te} remains an inflectional suffix,
however, there is no way that these idiomatic meanings can replace the
compositional meanings that the forms have by virtue of their
inflectional (ultimately, syntactic) status. The same is true of verbal
Conjunctive forms that have been lexicalized as nouns: \textit{nagasi}
`sink' (\textit{naga-s-} `make flow'), \textit{nagare} `flow, course of
events' (\textit{naga-re-} `flow').\footnote{The semantics of these nouns
  has been treated in the DM literature since \citet{volpe2005a} as involving
  selection of root allosemes by a noun-forming suffix (``special
  meanings of the root triggered across the little \textit{v} head'' \citep[107]{marantz2013a}). The extreme semantic distance that separates many of the nouns from their corresponding roots (abundantly documented by
  Volpe), however, makes idiom-formation a more plausible basis for the
  nominal meanings than alloseme choice (for the distinction between the
  two mechanisms, see \citealt[105]{marantz2013a}).}

If loss of a compositional interpretation is not a possible semantic
change, assuming stability of grammar and lexicon, then demonstrating
that the predicted compositional meaning of a putatively syntactic
construction is subject to loss over time will support the conclusion
that the construction in question is not syntactic after all, since if
it were, its compositional meaning should be diachronically stable. In
the present section, I will make this argument with respect to the
Japanese lexical causative in \textit{-s-,} exemplified by stems like
\textit{nao-s-} `cure, repair', seen in (5b) and (18b) above.
Specifically, I will document a number of cases in which the
construction {[}R{[}\textit{s}{]}{]} can be shown to have had the predicted
interpretation CAUS(ǁRǁ) (ǁRǁ the interpretation of R) originally but
later to have lost that interpretation in spite of the fact that ǁRǁ
itself has remained constant.

As a first example, consider the stem \textit{yurus-} `allow, forgive'. In
Old Japanese (see \citealp{omodaka1967a}), the primary meaning of this stem
is `slacken (t)', with secondary meanings `let go of'; `allow, comply
with, tolerate'; and `forgive, exempt'. \textit{Yurus-,} in other words,
is historically the causative in \textit{-s-} on √yuru `slack' (see (10)
above), a root that in modern Japanese underlies the adjective stem
\textit{yuru-} `slack', the nominal adjective \textit{yuru-yaka} `slack,
gradual', and the verb stems \textit{yuru-m-} `slacken (i)' and
\textit{yuru-m-e-} `slacken (t)'. As is clear from these four stems, the
root has been completely stable semantically over thirteen centuries,
and the same can be assumed for causative \textit{-s-.} There is no trace
in the modern meaning of \textit{yurus-,} however, of the original
concrete primary meaning `slacken'. That meaning, in other words, has
been completely replaced by the originally secondary or extended
meanings `allow' and `forgive'. If \textit{yuru-s-} had been a syntactic
construction, with the meaning `slacken (t)' the compositional result of
a semantic rule of interpretation, this replacement should have been
impossible, just as we have suggested that it would be impossible for
\textit{kick the bucket} to lose its compositional meaning and retain only
the idiomatic one.

The history of the stem \textit{itas-} `do (humble)' is broadly parallel.
In Old Japanese, it is the causative corresponding to \textit{itar-}
`reach a limit', as explicitly noted in \citet{omodaka1967a}, and thus
means `bring to a limit'. In the modern language, while intransitive
\textit{itar-} has retained its original meaning, \textit{itas-} is for the
most part, bleached of concrete content, simply a suppletive humble
variant of \textit{suru} `do'. A third case in which a \textit{s-}stem has
lost a putatively compositional causative meaning involves \textit{konas-}
`deal with, take care of; be skilled at', whose primary meaning was
originally `break up, pulverize' and which is based historically on
\textit{ko} `powder' \citep{ono1974a}. Like many other original monosyllables,
\textit{ko} has been replaced as a freestanding noun by a bisyllabic form,
in this case \textit{kona,} which is attested starting around 1700. The
only serious proposal for the origin of \textit{kona} (see NKD) appears to
be that it is a backformation based on \textit{konas-}. If the
backformation theory is correct, \textit{kona} and \textit{konas-} were
unquestionably isoradical at the relevant point in time, so that
\textit{konas-} consisted of √kona `powder' plus causative \textit{-s-.}
Today, however, while the root noun remains in the language, the meaning
`break up, pulverize' for the verb is extinct.\footnote{While
  dictionaries retain examples like \textit{tuti o konasu} `break up dirt
  (clods)', the speakers I have consulted deny knowledge of such a
  usage.}

Two further stems in \textit{-s-} for which the predicted causative
meaning appears to have been lost over time are \textit{hatas-} `carry
out, perform, accomplish' and \textit{kuras-} `make a living; live, spend
(time)'. The roots appear in the zero-derived noun \textit{hata} `edge,
perimeter; outside' and the zero-derived adjective stem \textit{kura-}
`dark', respectively, and are semantically identifiable in the
intransitives \textit{hate-} `end (i)' and \textit{kure-} `darken (day), end
(i)' (for the \textit{a} $\sim$ \textit{e} alternation, see note 11 above). The
expected primary meaning `end (t)' of \textit{hatas-} appears in the gloss
`bring to a conclusion' in \citet{omodaka1967a}; for \textit{kuras-,}
similarly, \citeauthor{omodaka1967a} record the expected primary meaning `spend the
time until evening' (i.e. `let the day darken'). In both cases, however,
this compositional meaning is absent from the modern stems, neither of
which stands in a purely causative relation to the corresponding
intransitive or to the root. The meaning of \textit{hatas-,} as the above
definition indicates, inherently includes an element of purposive
activity (carrying out a command, achieving a goal, fulfilling an
obligation) that is absent from that of \textit{hate-.} While the semantic
difference between \textit{kuras-} and \textit{kure-} is more subtle, the
basic fact preventing the former from functioning as the causative of
the latter is that, unlike \textit{kure-} (`come to an end'),
\textit{kuras-} (`spend (time)') is atelic. Both \textit{hatas-} and
\textit{kuras-,} then, like \textit{yurus-, itas-, and konas-,} are cases in
which the predicted interpretation CAUS(ǁRǁ) of the construction
{[}R{[}\textit{s}{]}{]} has been lost over time.

In this section, we have seen an argument against the syntactic
derivation of Japanese verb stems based on semantic change, using
causatives in \textit{-s-} as a representative stem-type. It goes without
saying, we should emphasize, that perhaps the most common type of
semantic change, the addition of idiomatic or extended meanings, does
not count against the hypothesis of syntactic generation: as is well
known, linguistic units of any size can be idiomatized, with the
tendency to undergo idiomatization inversely proportional, roughly
speaking, to size \citep[14]{sciullo1987a}. But loss of a
putatively compositional meaning, we have claimed, does count against
syntactic generation, because there is no reason to take the
compositional interpretation of syntactic structure to be anything but
automatic and exceptionless. In order for a compositional meaning M to
be lost, the syntactic structure underlying it would first have to be
exempted from compositional interpretation, with M being lexicalized at
the same time; M could then be lost from the lexicon. If this sequence
of events is impossible because exemptions of the required type are
never granted, however, a putatively compositional meaning that is in
fact subject to loss cannot have been based on a syntactic derivation in
the first place.

\section{Conclusion}\label{conclusion}

Above, I have attempted to evaluate the proposal that the derivational
suffixes that create transitive and intransitive verb stems in Japanese
are syntactic heads, in particular varieties of little v. Crucial
evidence in this regard has come from identifying an inner layer of
derivational suffixation (\textit{-g-, -m-,} etc.) in addition to the
well-known outer layer whose main members are -\textit{r-}, \textit{-s-},
\textit{-re-}, \textit{-se-}, \textit{-e-}, \textit{-i-}, and zero, since this
has allowed us to raise the question of how two derivational suffixes
interact when they occur together in the same stem. We saw in section~\ref{sequences}
that in such a case, the inner suffix is always inert for purposes of
argument structure and semantic interpretation, casting doubt on the
position that the suffixes are syntactic elements. In section~\ref{verbal}, we saw
that the same is true for combinations of the verbal suffix \textit{-m-}
and the adjectival suffix \textit{-si-,} with the added complication that
the order in which those two suffixes occur is an idiosyncratic function
of the root. Finally, in section~\ref{composition}, we argued, without reference to
suffix sequences, that the combination of a root and a
transitivity-determining suffix, taking causative \textit{-s-} as a
representative example, cannot be a syntactic construction because its
putatively compositional interpretation is unstable over time. All the
evidence we have seen, then, points toward the conclusion that the
derivational suffixes under consideration are not syntactic elements.
Equivalently, if one wishes in the face of this evidence to generate
Japanese verb and adjective stems syntactically, one will require
relaxation of otherwise well-motivated constraints on structure-building
and interpretation precisely for the domain of the stem. As suggested at
the outset, our conclusions in this regard support Anderson's \citeyearpar[594]{anderson1982a} position on the place of morphology in the grammar: derivation is pre-syntactic, and the units of lexical storage are inflectable stems;
inflection, in contrast, is the post-syntactic spellout of morphological
elements and morphosyntactic properties that are treated by syntactic
operations.

The conclusion that Japanese derivational suffixes, in contrast with
suffixes like the Passive and the productive Causative, are not
syntactic elements is supported at a more impressionistic level by the
fact that, as is easily confirmed, the two sets of suffixes differ
sharply in their degree of regularity, both formal and semantic.
Formally, while variation in the shape of the Passive suffix
\textit{-(r)are-} is limited to phonologically conditioned alternation of
\textit{r} with zero at the left edge, and variation in the shape of the
Causative suffix \textit{-(s)as(e)-} is limited to phonologically
conditioned alternation of \textit{s} with zero at the left edge and
non-phonological alternation of \textit{e} with zero at the right,
variation in the realization of what under a DM analysis will be
v\textsubscript{i} and v\textsubscript{c} is highly unconstrained, with
multiple unrelated allomorphs for each of the suffixes and almost
complete overlap between the two allomorph sets. Semantically, while the
meaning of Passive stems in \textit{-(r)are-} and (apart from occasional
idioms) Causative stems in \textit{-(s)as(e)-} is both regular and
relatively straightforward to characterize, the meaning of stems in
v\textsubscript{i} and v\textsubscript{c} is in most cases multiply
polysemous and highly idiosyncratic; the glosses we have given above,
while aiming at a marginal increase in accuracy over the labels in
\citet{jacobsen1992a} and \citet{volpe2005a}, in many cases only scratch the surface of
the problem of specifying stem meaning. With regard to semantics, it
should also be remembered that, as we noted in section~\ref{background}, morphological
analysis internal to the stem proceeds on the basis of an unredeemed
promissory note regarding the criterion for isoradicality and that
equally serious questions arise about how the meaning of transitivity
suffixes is to be specified, given the apparent semantic overlap between
transitivizing and intransitivizing morphology.

If Japanese verb and adjective stems are not, then, created by the
syntactic computational system, how should we conceive of their
structure and, crucially, the knowledge that speakers have about that
structure? Broadly speaking, there are two types of answer that could be
given to this question. On one of them, derivational morphology of the
type we have seen here would be the result of a combinatory system
roughly parallel to syntax but less regular both in terms of the
hierarchical relationships holding among grammatical elements and the
semantic interpretation of complex structures. From the standpoint of
theoretical parsimony, of course, this would seem like an unattractive
proposal; surely, if possible, we would prefer to maintain that the
language faculty involves a ``single generative engine'' \citep{marantz2001a,marantz2005a}. Viewing language as a biological object, however, there would appear to be no grounds for excluding a priori the possibility that our linguistic capacities include a combinatory stem-formation module of the sort in question. In evolutionary terms, such a module might have
provided a vastly expanded repertory of named concepts in advance of the
emergence of a fully regular and productive syntax, representing a sort
of half-way house on the road to discrete infinity.

The second type of answer that could be given to the question of the
form taken by speaker knowledge of the relations among isoradical stems,
assuming that those relations are not mediated by the syntactic
computational system, is that that knowledge is frankly
non-generative -- that is, non-combinatory. In this case, all stems will
be lexically listed, with relations among them captured by redundancy
rules, for example, those of the type pioneered by \citet{jackendoff1975a} (see also \citealt[53]{jackendoff2002a}). What is unsatisfying about this type of answer is that it provides no insight into why derivational morphology should
exist at all -- why, that is, stems (setting aside compounds) are not all
atomic. While we have seen evidence that at least some derivational
morphology cannot be syntactic, then, there is no unambiguously
attractive alternative account of the structure of speaker knowledge in
this area. As a result, the place of derivational morphology in our
linguistic competence remains very much an open question.

\subsection*{Acknowledgments} I would like to express my appreciation to
Takayuki Ikezawa, in conversations with whom the idea for this paper
emerged. I am also grateful to Kunio Nishiyama and Yoko Sugioka for
comments on a presentation of some of this material at the 152nd meeting
of the Linguistic Society of Japan (Tokyo, June 2016). Finally, I am
indebted to several reviewers for comments that have resulted in a
number of clarifications and improvements. Remaining errors of fact or
interpretation are my responsibility.

\subsection*{Abbreviations}
\textsc{cj} `(second or perfective) conjunctive', \textsc{pf} `perfective'

\nocite{unknown-a,unknown-b}

\printbibliography[heading=subbibliography,notkeyword=this]


\end{document}


