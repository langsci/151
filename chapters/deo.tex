\documentclass[output=paper,
modfonts
]{LSP/langsci}
% \bibliography{localbibliography}

% % add all extra packages you need to load to this file 
% \usepackage{todo} %% removed,cna use todonotes instead. % Jason reactivated
% \usepackage{graphicx} % not needed because forest loads tikz, which loads graphicx
\usepackage{tabularx}
\usepackage{amsmath} 
\usepackage{multicol}
\usepackage{lipsum}
\usepackage{longtable}
\usepackage{booktabs}
\usepackage[normalem]{ulem}
%\usepackage{tikz} % not needed because forest loads tikz
\usepackage{phonrule} % for SPE-style phonological rules
\usepackage{pst-all} % loads the main pstricks tools; for arrow diagrams in Hale.tex
%\usepackage{leipzig} % for gloss abbreviations
\usepackage[% for automatic cross-referencing
compress,%
capitalize,% labels are always capitalized in LSP style
noabbrev]% labels are always spelled out in LSP style
{cleveref}

% based on http://tex.stackexchange.com/a/318983/42880 for using gb4e examples with cleveref
\crefname{xnumi}{}{}
\creflabelformat{xnumi}{(#2#1#3)}
\crefrangeformat{xnumi}{(#3#1#4)--(#5#2#6)}
\crefname{xnumii}{}{}
\creflabelformat{xnumii}{(#2#1#3)}
\crefrangeformat{xnumii}{(#3#1#4)--(#5#2#6)}

%\usepackage[notcite,notref]{showkeys} %%removed, not helping CB.
%\usepackage{showidx} %%remove for final compiling - shows index keys at top of page.
 
\usepackage{langsci/styles/langsci-gb4e}  
 \usepackage{pifont}
% % OT tableaux                                                
% \usepackage{pstricks,colortab}  
\usepackage{multirow} % used in OT tableaux
\usepackage{rotating} %needed for angled text%
\usepackage{colortbl} % for cell shading
 
 \usepackage{avm}  
\usepackage[linguistics]{forest} 
\usetikzlibrary{matrix,fit} % for matrix of nodes in Kaisse and Bat-El


\usepackage{hhline}
\newcommand{\cgr}{\cellcolor[gray]{0.8}}
\newcommand{\cn}{\centering}



\newcommand{\reff}[1]{(\ref{#1})}
%\usepackage{newtxtext,newtxmath}


%\usepackage[normalem] {ulem}
\usepackage{qtree}
%\usepackage{natbib}
%\usepackage{tikz}
%\usepackage{gb4e}
\usepackage{phonrule}  
%\bibliographystyle{humannat}



\usepackage{minibox}

%\include{psheader-metr}

\def\bl#1{$_{\textrm{{\footnotesize #1}}}$}

% %add all your local new commands to this file

\newcommand{\form}[1]{\mbox{\emph{#1}}}
\newcommand{\uf}[1]{\mbox{/#1/}}

% borrowed from expex and converted from plan tex to latex
\newcommand{\judge}[1]{{\upshape #1\hspace{0.1em}}}
\newcommand{\ljudge}[1]{\makebox[0pt][r]{\judge{#1}}}

\newcommand\tikzmark[1]{\tikz[remember picture, baseline=(#1.base)] \node[anchor=base,inner sep=0pt, outer sep=0pt] (#1) {#1};} % for adding decorations, arrows, lines, etc. to text
\newcommand\tikzmarknamed[2]{\tikz[remember picture, baseline=(#1.base)] \node[anchor=base,inner sep=0pt, outer sep=0pt] (#1) {#2};} % for adding decorations, arrows, lines, etc. to text
\newcommand\tikzmarkfullnamed[2]{\tikz[remember picture, baseline=(#1.base)] \node[anchor=base,inner sep=0pt, outer sep=0pt] (#1) {\vphantom{X}#2};} % for adding decorations, arrows, lines, etc. to text; this one works best for decorations above a line of text because it adds in the heigh of a capital letter and takes two arguments - one for the node name and one for the printed text

\newcommand{\sub}[1]{$_{\text{#1}}$} % for non-math subscripts
\newcommand{\subit}[1]{\sub{\textit{#1}}} % for italics non-math subscripts
\newcommand{\1}{\rlap{$'$}\xspace} % for the prime in X' (the \rlap command allows the prime to be ignored for horizontal spacing in trees, and the \xspace command allows you to use this in normal text without adding \ afterwards). This isn't crucial, but it helps the formatting to look a little better.

% Aissen:
\newcommand\tikzmarkfull[1]{\tikz[remember picture, baseline=(#1.base)] \node[anchor=base,inner sep=0pt, outer sep=0pt] (#1) {\vphantom{X}#1};} % for adding decorations, arrows, lines, etc. to text; this one works best for decorations above a line of text because it adds in the heigh of a capital letter and takes one argument that serves as the name and the printed text
\newcommand{\bridgeover}[2]{% for a line that creates a bridge over text, connecting two nodes
	\begin{tikzpicture}[remember picture,overlay]
	\draw[thick,shorten >=3pt,shorten <=3pt] (#1.north) |- +(0ex,2.5ex) -| (#2.north);
	\end{tikzpicture}
}
\newcommand{\bridgeoverht}[3]{% for a line that creates a bridge over text, connecting two nodes and specifing the height of the bridge
	\begin{tikzpicture}[remember picture,overlay]
	\draw[thick,shorten >=3pt,shorten <=3pt] (#2.north) |- +(0ex,#1) -| (#3.north);
	\end{tikzpicture}
}
\newcommand{\bridgeoverex}{\vspace*{3ex}} % place before an example that has a \bridgeover so that there is enough vertical space for it

% Chung:
\newcommand{\lefttabular}[1]{\begin{tabular}{p{0.5in}}#1\end{tabular}}

% Kaisse:
\newcommand{\mgmorph}[1]{|(#1)| {#1}}
\newcommand{\mgone}[2][$\times$]{\node at (#2.base) [above=2ex] (1#2) {\vphantom{X}#1};}
\newcommand{\mgtwo}[2][$\times$]{\mgone{#2} \node at (#2.base) [above=4.5ex] (2#2) {\vphantom{X}#1};}
\newcommand{\mgthree}[2][$\times$]{\mgtwo{#2} \node at (#2.base) [above=7ex] (3#2) {\vphantom{X}#1};}
\newcommand{\mgftl}[1]{\node at (1#1) [left] {(};}
\newcommand{\mgftr}[1]{\node at (1#1) [right] {)};}
\newcommand{\mgfoot}[2]{\mgftl{#1}\mgftr{#2}}
\newcommand{\mgldelim}[2]{\node at (#2.west) [left,inner sep = 0pt, outer sep = 0pt] {#1};}
\newcommand{\mgrdelim}[2]{\node at (#2.east) [right,inner sep = 0pt, outer sep = 0pt] {#1};}

\newcommand{\squish}{\hspace*{-3pt}}

% Kavitskaya:
\newcommand{\assoc}[2]{\draw (#1.south) -- (#2.north);}
\newcolumntype{L}{>{\raggedright\arraybackslash}X}

% Lepic & Padden:
\newcommand{\fitpic}[1]{\resizebox{\hsize}{!}{\includegraphics{#1}}} % from http://tex.stackexchange.com/a/148965/42880
\newcommand{\signpic}[1]{\includegraphics[width=1.36in]{#1}}
\newcolumntype{C}{>{\centering\arraybackslash}X}

% Spencer:

\newcommand{\textex}[1]{\textit{#1}\xspace}
\newcommand{\lxm}[1]{\textsc{#1}\xspace}

% Thrainsson:

\renewcommand{\textasciitilde}{\char`~} % for use with TTF/OTF fonts (see comments on http://tex.stackexchange.com/a/317/42880)
\newcommand{\tikzarrow}[2]{% for an arrow connecting two nodes
\begin{tikzpicture}[remember picture,overlay]
\draw[thick,shorten >=3pt,shorten <=3pt,->,>=stealth] (#1) -- (#2);
\end{tikzpicture}
}

\newlength{\padding}
\setlength{\padding}{0.5em}
\newcommand{\lesspadding}{\hspace*{-\padding}}
\newcommand{\feat}[1]{\lesspadding#1\lesspadding}

% Hammond

\usepackage[]{graphicx}\usepackage[]{xcolor}
%% maxwidth is the original width if it is less than linewidth
%% otherwise use linewidth (to make sure the graphics do not exceed the margin)
\makeatletter
\def\maxwidth{ %
  \ifdim\Gin@nat@width>\linewidth
    \linewidth
  \else
    \Gin@nat@width
  \fi
}
\makeatother

\definecolor{fgcolor}{rgb}{0.345, 0.345, 0.345}
\newcommand{\hlnum}[1]{\textcolor[rgb]{0.686,0.059,0.569}{#1}}%
\newcommand{\hlstr}[1]{\textcolor[rgb]{0.192,0.494,0.8}{#1}}%
\newcommand{\hlcom}[1]{\textcolor[rgb]{0.678,0.584,0.686}{\textit{#1}}}%
\newcommand{\hlopt}[1]{\textcolor[rgb]{0,0,0}{#1}}%
\newcommand{\hlstd}[1]{\textcolor[rgb]{0.345,0.345,0.345}{#1}}%
\newcommand{\hlkwa}[1]{\textcolor[rgb]{0.161,0.373,0.58}{\textbf{#1}}}%
\newcommand{\hlkwb}[1]{\textcolor[rgb]{0.69,0.353,0.396}{#1}}%
\newcommand{\hlkwc}[1]{\textcolor[rgb]{0.333,0.667,0.333}{#1}}%
\newcommand{\hlkwd}[1]{\textcolor[rgb]{0.737,0.353,0.396}{\textbf{#1}}}%
\let\hlipl\hlkwb

\usepackage{framed}
\makeatletter
\newenvironment{kframe}{%
 \def\at@end@of@kframe{}%
 \ifinner\ifhmode%
  \def\at@end@of@kframe{\end{minipage}}%
  \begin{minipage}{\columnwidth}%
 \fi\fi%
 \def\FrameCommand##1{\hskip\@totalleftmargin \hskip-\fboxsep
 \colorbox{shadecolor}{##1}\hskip-\fboxsep
     % There is no \\@totalrightmargin, so:
     \hskip-\linewidth \hskip-\@totalleftmargin \hskip\columnwidth}%
 \MakeFramed {\advance\hsize-\width
   \@totalleftmargin\z@ \linewidth\hsize
   \@setminipage}}%
 {\par\unskip\endMakeFramed%
 \at@end@of@kframe}
\makeatother

\definecolor{shadecolor}{rgb}{.97, .97, .97}
\definecolor{messagecolor}{rgb}{0, 0, 0}
\definecolor{warningcolor}{rgb}{1, 0, 1}
\definecolor{errorcolor}{rgb}{1, 0, 0}
\newenvironment{knitrout}{}{} % an empty environment to be redefined in TeX

\usepackage{alltt}

%revised version started: 12/17/16

%NEEDS: allbib.bib - already added to the master bibliography file.
%cited references only: bibexport -o mhTMP.bib main1-blx.aux
%PLUS sramh-img*, sramh.tex

%added stuff
\newcommand{\add}[1]{\textcolor{blue}{#1}}
%deleted stuff
\newcommand{\del}[1]{\textcolor{red}{(removed: #1)}}
%uncomment these to turn off colors
\renewcommand{\add}[1]{#1}
\renewcommand{\del}[1]{}

%shortcuts
\newcommand{\w}{\ili{Welsh}}
\newcommand{\e}{\ili{English}}
\newcommand{\io}{Input Optimization}




 \newcommand{\hand}{\ding{43}}
% \newcommand{\rot}[1]{\begin{rotate}{90}#1\end{rotate}} %shortcut for angled text%  
% \newcommand{\rotcon}[1]{\raisebox{-5ex}{\hspace*{1ex}\rot{\hspace*{1ex}#1}}}

%% add all extra packages you need to load to this file 
% \usepackage{todo} %% removed,cna use todonotes instead. % Jason reactivated
% \usepackage{graphicx} % not needed because forest loads tikz, which loads graphicx
\usepackage{tabularx}
\usepackage{amsmath} 
\usepackage{multicol}
\usepackage{lipsum}
\usepackage{longtable}
\usepackage{booktabs}
\usepackage[normalem]{ulem}
%\usepackage{tikz} % not needed because forest loads tikz
\usepackage{phonrule} % for SPE-style phonological rules
\usepackage{pst-all} % loads the main pstricks tools; for arrow diagrams in Hale.tex
%\usepackage{leipzig} % for gloss abbreviations
\usepackage[% for automatic cross-referencing
compress,%
capitalize,% labels are always capitalized in LSP style
noabbrev]% labels are always spelled out in LSP style
{cleveref}

% based on http://tex.stackexchange.com/a/318983/42880 for using gb4e examples with cleveref
\crefname{xnumi}{}{}
\creflabelformat{xnumi}{(#2#1#3)}
\crefrangeformat{xnumi}{(#3#1#4)--(#5#2#6)}
\crefname{xnumii}{}{}
\creflabelformat{xnumii}{(#2#1#3)}
\crefrangeformat{xnumii}{(#3#1#4)--(#5#2#6)}

%\usepackage[notcite,notref]{showkeys} %%removed, not helping CB.
%\usepackage{showidx} %%remove for final compiling - shows index keys at top of page.
 
\usepackage{langsci/styles/langsci-gb4e}  
 \usepackage{pifont}
% % OT tableaux                                                
% \usepackage{pstricks,colortab}  
\usepackage{multirow} % used in OT tableaux
\usepackage{rotating} %needed for angled text%
\usepackage{colortbl} % for cell shading
 
 \usepackage{avm}  
\usepackage[linguistics]{forest} 
\usetikzlibrary{matrix,fit} % for matrix of nodes in Kaisse and Bat-El


\usepackage{hhline}
\newcommand{\cgr}{\cellcolor[gray]{0.8}}
\newcommand{\cn}{\centering}



\newcommand{\reff}[1]{(\ref{#1})}
%\usepackage{newtxtext,newtxmath}


%\usepackage[normalem] {ulem}
\usepackage{qtree}
%\usepackage{natbib}
%\usepackage{tikz}
%\usepackage{gb4e}
\usepackage{phonrule}  
%\bibliographystyle{humannat}



\usepackage{minibox}

%\include{psheader-metr}

\def\bl#1{$_{\textrm{{\footnotesize #1}}}$}
\usepackage{arydshln}
\usepackage{rotating}

%%add all your local new commands to this file

\newcommand{\form}[1]{\mbox{\emph{#1}}}
\newcommand{\uf}[1]{\mbox{/#1/}}

% borrowed from expex and converted from plan tex to latex
\newcommand{\judge}[1]{{\upshape #1\hspace{0.1em}}}
\newcommand{\ljudge}[1]{\makebox[0pt][r]{\judge{#1}}}

\newcommand\tikzmark[1]{\tikz[remember picture, baseline=(#1.base)] \node[anchor=base,inner sep=0pt, outer sep=0pt] (#1) {#1};} % for adding decorations, arrows, lines, etc. to text
\newcommand\tikzmarknamed[2]{\tikz[remember picture, baseline=(#1.base)] \node[anchor=base,inner sep=0pt, outer sep=0pt] (#1) {#2};} % for adding decorations, arrows, lines, etc. to text
\newcommand\tikzmarkfullnamed[2]{\tikz[remember picture, baseline=(#1.base)] \node[anchor=base,inner sep=0pt, outer sep=0pt] (#1) {\vphantom{X}#2};} % for adding decorations, arrows, lines, etc. to text; this one works best for decorations above a line of text because it adds in the heigh of a capital letter and takes two arguments - one for the node name and one for the printed text

\newcommand{\sub}[1]{$_{\text{#1}}$} % for non-math subscripts
\newcommand{\subit}[1]{\sub{\textit{#1}}} % for italics non-math subscripts
\newcommand{\1}{\rlap{$'$}\xspace} % for the prime in X' (the \rlap command allows the prime to be ignored for horizontal spacing in trees, and the \xspace command allows you to use this in normal text without adding \ afterwards). This isn't crucial, but it helps the formatting to look a little better.

% Aissen:
\newcommand\tikzmarkfull[1]{\tikz[remember picture, baseline=(#1.base)] \node[anchor=base,inner sep=0pt, outer sep=0pt] (#1) {\vphantom{X}#1};} % for adding decorations, arrows, lines, etc. to text; this one works best for decorations above a line of text because it adds in the heigh of a capital letter and takes one argument that serves as the name and the printed text
\newcommand{\bridgeover}[2]{% for a line that creates a bridge over text, connecting two nodes
	\begin{tikzpicture}[remember picture,overlay]
	\draw[thick,shorten >=3pt,shorten <=3pt] (#1.north) |- +(0ex,2.5ex) -| (#2.north);
	\end{tikzpicture}
}
\newcommand{\bridgeoverht}[3]{% for a line that creates a bridge over text, connecting two nodes and specifing the height of the bridge
	\begin{tikzpicture}[remember picture,overlay]
	\draw[thick,shorten >=3pt,shorten <=3pt] (#2.north) |- +(0ex,#1) -| (#3.north);
	\end{tikzpicture}
}
\newcommand{\bridgeoverex}{\vspace*{3ex}} % place before an example that has a \bridgeover so that there is enough vertical space for it

% Chung:
\newcommand{\lefttabular}[1]{\begin{tabular}{p{0.5in}}#1\end{tabular}}

% Kaisse:
\newcommand{\mgmorph}[1]{|(#1)| {#1}}
\newcommand{\mgone}[2][$\times$]{\node at (#2.base) [above=2ex] (1#2) {\vphantom{X}#1};}
\newcommand{\mgtwo}[2][$\times$]{\mgone{#2} \node at (#2.base) [above=4.5ex] (2#2) {\vphantom{X}#1};}
\newcommand{\mgthree}[2][$\times$]{\mgtwo{#2} \node at (#2.base) [above=7ex] (3#2) {\vphantom{X}#1};}
\newcommand{\mgftl}[1]{\node at (1#1) [left] {(};}
\newcommand{\mgftr}[1]{\node at (1#1) [right] {)};}
\newcommand{\mgfoot}[2]{\mgftl{#1}\mgftr{#2}}
\newcommand{\mgldelim}[2]{\node at (#2.west) [left,inner sep = 0pt, outer sep = 0pt] {#1};}
\newcommand{\mgrdelim}[2]{\node at (#2.east) [right,inner sep = 0pt, outer sep = 0pt] {#1};}

\newcommand{\squish}{\hspace*{-3pt}}

% Kavitskaya:
\newcommand{\assoc}[2]{\draw (#1.south) -- (#2.north);}
\newcolumntype{L}{>{\raggedright\arraybackslash}X}

% Lepic & Padden:
\newcommand{\fitpic}[1]{\resizebox{\hsize}{!}{\includegraphics{#1}}} % from http://tex.stackexchange.com/a/148965/42880
\newcommand{\signpic}[1]{\includegraphics[width=1.36in]{#1}}
\newcolumntype{C}{>{\centering\arraybackslash}X}

% Spencer:

\newcommand{\textex}[1]{\textit{#1}\xspace}
\newcommand{\lxm}[1]{\textsc{#1}\xspace}

% Thrainsson:

\renewcommand{\textasciitilde}{\char`~} % for use with TTF/OTF fonts (see comments on http://tex.stackexchange.com/a/317/42880)
\newcommand{\tikzarrow}[2]{% for an arrow connecting two nodes
\begin{tikzpicture}[remember picture,overlay]
\draw[thick,shorten >=3pt,shorten <=3pt,->,>=stealth] (#1) -- (#2);
\end{tikzpicture}
}

\newlength{\padding}
\setlength{\padding}{0.5em}
\newcommand{\lesspadding}{\hspace*{-\padding}}
\newcommand{\feat}[1]{\lesspadding#1\lesspadding}

% Hammond

\usepackage[]{graphicx}\usepackage[]{xcolor}
%% maxwidth is the original width if it is less than linewidth
%% otherwise use linewidth (to make sure the graphics do not exceed the margin)
\makeatletter
\def\maxwidth{ %
  \ifdim\Gin@nat@width>\linewidth
    \linewidth
  \else
    \Gin@nat@width
  \fi
}
\makeatother

\definecolor{fgcolor}{rgb}{0.345, 0.345, 0.345}
\newcommand{\hlnum}[1]{\textcolor[rgb]{0.686,0.059,0.569}{#1}}%
\newcommand{\hlstr}[1]{\textcolor[rgb]{0.192,0.494,0.8}{#1}}%
\newcommand{\hlcom}[1]{\textcolor[rgb]{0.678,0.584,0.686}{\textit{#1}}}%
\newcommand{\hlopt}[1]{\textcolor[rgb]{0,0,0}{#1}}%
\newcommand{\hlstd}[1]{\textcolor[rgb]{0.345,0.345,0.345}{#1}}%
\newcommand{\hlkwa}[1]{\textcolor[rgb]{0.161,0.373,0.58}{\textbf{#1}}}%
\newcommand{\hlkwb}[1]{\textcolor[rgb]{0.69,0.353,0.396}{#1}}%
\newcommand{\hlkwc}[1]{\textcolor[rgb]{0.333,0.667,0.333}{#1}}%
\newcommand{\hlkwd}[1]{\textcolor[rgb]{0.737,0.353,0.396}{\textbf{#1}}}%
\let\hlipl\hlkwb

\usepackage{framed}
\makeatletter
\newenvironment{kframe}{%
 \def\at@end@of@kframe{}%
 \ifinner\ifhmode%
  \def\at@end@of@kframe{\end{minipage}}%
  \begin{minipage}{\columnwidth}%
 \fi\fi%
 \def\FrameCommand##1{\hskip\@totalleftmargin \hskip-\fboxsep
 \colorbox{shadecolor}{##1}\hskip-\fboxsep
     % There is no \\@totalrightmargin, so:
     \hskip-\linewidth \hskip-\@totalleftmargin \hskip\columnwidth}%
 \MakeFramed {\advance\hsize-\width
   \@totalleftmargin\z@ \linewidth\hsize
   \@setminipage}}%
 {\par\unskip\endMakeFramed%
 \at@end@of@kframe}
\makeatother

\definecolor{shadecolor}{rgb}{.97, .97, .97}
\definecolor{messagecolor}{rgb}{0, 0, 0}
\definecolor{warningcolor}{rgb}{1, 0, 1}
\definecolor{errorcolor}{rgb}{1, 0, 0}
\newenvironment{knitrout}{}{} % an empty environment to be redefined in TeX

\usepackage{alltt}

%revised version started: 12/17/16

%NEEDS: allbib.bib - already added to the master bibliography file.
%cited references only: bibexport -o mhTMP.bib main1-blx.aux
%PLUS sramh-img*, sramh.tex

%added stuff
\newcommand{\add}[1]{\textcolor{blue}{#1}}
%deleted stuff
\newcommand{\del}[1]{\textcolor{red}{(removed: #1)}}
%uncomment these to turn off colors
\renewcommand{\add}[1]{#1}
\renewcommand{\del}[1]{}

%shortcuts
\newcommand{\w}{\ili{Welsh}}
\newcommand{\e}{\ili{English}}
\newcommand{\io}{Input Optimization}




 \newcommand{\hand}{\ding{43}}
% \newcommand{\rot}[1]{\begin{rotate}{90}#1\end{rotate}} %shortcut for angled text%  
% \newcommand{\rotcon}[1]{\raisebox{-5ex}{\hspace*{1ex}\rot{\hspace*{1ex}#1}}}

%% add all extra packages you need to load to this file 
% \usepackage{todo} %% removed,cna use todonotes instead. % Jason reactivated
% \usepackage{graphicx} % not needed because forest loads tikz, which loads graphicx
\usepackage{tabularx}
\usepackage{amsmath} 
\usepackage{multicol}
\usepackage{lipsum}
\usepackage{longtable}
\usepackage{booktabs}
\usepackage[normalem]{ulem}
%\usepackage{tikz} % not needed because forest loads tikz
\usepackage{phonrule} % for SPE-style phonological rules
\usepackage{pst-all} % loads the main pstricks tools; for arrow diagrams in Hale.tex
%\usepackage{leipzig} % for gloss abbreviations
\usepackage[% for automatic cross-referencing
compress,%
capitalize,% labels are always capitalized in LSP style
noabbrev]% labels are always spelled out in LSP style
{cleveref}

% based on http://tex.stackexchange.com/a/318983/42880 for using gb4e examples with cleveref
\crefname{xnumi}{}{}
\creflabelformat{xnumi}{(#2#1#3)}
\crefrangeformat{xnumi}{(#3#1#4)--(#5#2#6)}
\crefname{xnumii}{}{}
\creflabelformat{xnumii}{(#2#1#3)}
\crefrangeformat{xnumii}{(#3#1#4)--(#5#2#6)}

%\usepackage[notcite,notref]{showkeys} %%removed, not helping CB.
%\usepackage{showidx} %%remove for final compiling - shows index keys at top of page.
 
\usepackage{langsci/styles/langsci-gb4e}  
 \usepackage{pifont}
% % OT tableaux                                                
% \usepackage{pstricks,colortab}  
\usepackage{multirow} % used in OT tableaux
\usepackage{rotating} %needed for angled text%
\usepackage{colortbl} % for cell shading
 
 \usepackage{avm}  
\usepackage[linguistics]{forest} 
\usetikzlibrary{matrix,fit} % for matrix of nodes in Kaisse and Bat-El


\usepackage{hhline}
\newcommand{\cgr}{\cellcolor[gray]{0.8}}
\newcommand{\cn}{\centering}



\newcommand{\reff}[1]{(\ref{#1})}
%\usepackage{newtxtext,newtxmath}


%\usepackage[normalem] {ulem}
\usepackage{qtree}
%\usepackage{natbib}
%\usepackage{tikz}
%\usepackage{gb4e}
\usepackage{phonrule}  
%\bibliographystyle{humannat}



\usepackage{minibox}

%\include{psheader-metr}

\def\bl#1{$_{\textrm{{\footnotesize #1}}}$}
\usepackage{arydshln}
\usepackage{rotating}

%%add all your local new commands to this file

\newcommand{\form}[1]{\mbox{\emph{#1}}}
\newcommand{\uf}[1]{\mbox{/#1/}}

% borrowed from expex and converted from plan tex to latex
\newcommand{\judge}[1]{{\upshape #1\hspace{0.1em}}}
\newcommand{\ljudge}[1]{\makebox[0pt][r]{\judge{#1}}}

\newcommand\tikzmark[1]{\tikz[remember picture, baseline=(#1.base)] \node[anchor=base,inner sep=0pt, outer sep=0pt] (#1) {#1};} % for adding decorations, arrows, lines, etc. to text
\newcommand\tikzmarknamed[2]{\tikz[remember picture, baseline=(#1.base)] \node[anchor=base,inner sep=0pt, outer sep=0pt] (#1) {#2};} % for adding decorations, arrows, lines, etc. to text
\newcommand\tikzmarkfullnamed[2]{\tikz[remember picture, baseline=(#1.base)] \node[anchor=base,inner sep=0pt, outer sep=0pt] (#1) {\vphantom{X}#2};} % for adding decorations, arrows, lines, etc. to text; this one works best for decorations above a line of text because it adds in the heigh of a capital letter and takes two arguments - one for the node name and one for the printed text

\newcommand{\sub}[1]{$_{\text{#1}}$} % for non-math subscripts
\newcommand{\subit}[1]{\sub{\textit{#1}}} % for italics non-math subscripts
\newcommand{\1}{\rlap{$'$}\xspace} % for the prime in X' (the \rlap command allows the prime to be ignored for horizontal spacing in trees, and the \xspace command allows you to use this in normal text without adding \ afterwards). This isn't crucial, but it helps the formatting to look a little better.

% Aissen:
\newcommand\tikzmarkfull[1]{\tikz[remember picture, baseline=(#1.base)] \node[anchor=base,inner sep=0pt, outer sep=0pt] (#1) {\vphantom{X}#1};} % for adding decorations, arrows, lines, etc. to text; this one works best for decorations above a line of text because it adds in the heigh of a capital letter and takes one argument that serves as the name and the printed text
\newcommand{\bridgeover}[2]{% for a line that creates a bridge over text, connecting two nodes
	\begin{tikzpicture}[remember picture,overlay]
	\draw[thick,shorten >=3pt,shorten <=3pt] (#1.north) |- +(0ex,2.5ex) -| (#2.north);
	\end{tikzpicture}
}
\newcommand{\bridgeoverht}[3]{% for a line that creates a bridge over text, connecting two nodes and specifing the height of the bridge
	\begin{tikzpicture}[remember picture,overlay]
	\draw[thick,shorten >=3pt,shorten <=3pt] (#2.north) |- +(0ex,#1) -| (#3.north);
	\end{tikzpicture}
}
\newcommand{\bridgeoverex}{\vspace*{3ex}} % place before an example that has a \bridgeover so that there is enough vertical space for it

% Chung:
\newcommand{\lefttabular}[1]{\begin{tabular}{p{0.5in}}#1\end{tabular}}

% Kaisse:
\newcommand{\mgmorph}[1]{|(#1)| {#1}}
\newcommand{\mgone}[2][$\times$]{\node at (#2.base) [above=2ex] (1#2) {\vphantom{X}#1};}
\newcommand{\mgtwo}[2][$\times$]{\mgone{#2} \node at (#2.base) [above=4.5ex] (2#2) {\vphantom{X}#1};}
\newcommand{\mgthree}[2][$\times$]{\mgtwo{#2} \node at (#2.base) [above=7ex] (3#2) {\vphantom{X}#1};}
\newcommand{\mgftl}[1]{\node at (1#1) [left] {(};}
\newcommand{\mgftr}[1]{\node at (1#1) [right] {)};}
\newcommand{\mgfoot}[2]{\mgftl{#1}\mgftr{#2}}
\newcommand{\mgldelim}[2]{\node at (#2.west) [left,inner sep = 0pt, outer sep = 0pt] {#1};}
\newcommand{\mgrdelim}[2]{\node at (#2.east) [right,inner sep = 0pt, outer sep = 0pt] {#1};}

\newcommand{\squish}{\hspace*{-3pt}}

% Kavitskaya:
\newcommand{\assoc}[2]{\draw (#1.south) -- (#2.north);}
\newcolumntype{L}{>{\raggedright\arraybackslash}X}

% Lepic & Padden:
\newcommand{\fitpic}[1]{\resizebox{\hsize}{!}{\includegraphics{#1}}} % from http://tex.stackexchange.com/a/148965/42880
\newcommand{\signpic}[1]{\includegraphics[width=1.36in]{#1}}
\newcolumntype{C}{>{\centering\arraybackslash}X}

% Spencer:

\newcommand{\textex}[1]{\textit{#1}\xspace}
\newcommand{\lxm}[1]{\textsc{#1}\xspace}

% Thrainsson:

\renewcommand{\textasciitilde}{\char`~} % for use with TTF/OTF fonts (see comments on http://tex.stackexchange.com/a/317/42880)
\newcommand{\tikzarrow}[2]{% for an arrow connecting two nodes
\begin{tikzpicture}[remember picture,overlay]
\draw[thick,shorten >=3pt,shorten <=3pt,->,>=stealth] (#1) -- (#2);
\end{tikzpicture}
}

\newlength{\padding}
\setlength{\padding}{0.5em}
\newcommand{\lesspadding}{\hspace*{-\padding}}
\newcommand{\feat}[1]{\lesspadding#1\lesspadding}

% Hammond

\usepackage[]{graphicx}\usepackage[]{xcolor}
%% maxwidth is the original width if it is less than linewidth
%% otherwise use linewidth (to make sure the graphics do not exceed the margin)
\makeatletter
\def\maxwidth{ %
  \ifdim\Gin@nat@width>\linewidth
    \linewidth
  \else
    \Gin@nat@width
  \fi
}
\makeatother

\definecolor{fgcolor}{rgb}{0.345, 0.345, 0.345}
\newcommand{\hlnum}[1]{\textcolor[rgb]{0.686,0.059,0.569}{#1}}%
\newcommand{\hlstr}[1]{\textcolor[rgb]{0.192,0.494,0.8}{#1}}%
\newcommand{\hlcom}[1]{\textcolor[rgb]{0.678,0.584,0.686}{\textit{#1}}}%
\newcommand{\hlopt}[1]{\textcolor[rgb]{0,0,0}{#1}}%
\newcommand{\hlstd}[1]{\textcolor[rgb]{0.345,0.345,0.345}{#1}}%
\newcommand{\hlkwa}[1]{\textcolor[rgb]{0.161,0.373,0.58}{\textbf{#1}}}%
\newcommand{\hlkwb}[1]{\textcolor[rgb]{0.69,0.353,0.396}{#1}}%
\newcommand{\hlkwc}[1]{\textcolor[rgb]{0.333,0.667,0.333}{#1}}%
\newcommand{\hlkwd}[1]{\textcolor[rgb]{0.737,0.353,0.396}{\textbf{#1}}}%
\let\hlipl\hlkwb

\usepackage{framed}
\makeatletter
\newenvironment{kframe}{%
 \def\at@end@of@kframe{}%
 \ifinner\ifhmode%
  \def\at@end@of@kframe{\end{minipage}}%
  \begin{minipage}{\columnwidth}%
 \fi\fi%
 \def\FrameCommand##1{\hskip\@totalleftmargin \hskip-\fboxsep
 \colorbox{shadecolor}{##1}\hskip-\fboxsep
     % There is no \\@totalrightmargin, so:
     \hskip-\linewidth \hskip-\@totalleftmargin \hskip\columnwidth}%
 \MakeFramed {\advance\hsize-\width
   \@totalleftmargin\z@ \linewidth\hsize
   \@setminipage}}%
 {\par\unskip\endMakeFramed%
 \at@end@of@kframe}
\makeatother

\definecolor{shadecolor}{rgb}{.97, .97, .97}
\definecolor{messagecolor}{rgb}{0, 0, 0}
\definecolor{warningcolor}{rgb}{1, 0, 1}
\definecolor{errorcolor}{rgb}{1, 0, 0}
\newenvironment{knitrout}{}{} % an empty environment to be redefined in TeX

\usepackage{alltt}

%revised version started: 12/17/16

%NEEDS: allbib.bib - already added to the master bibliography file.
%cited references only: bibexport -o mhTMP.bib main1-blx.aux
%PLUS sramh-img*, sramh.tex

%added stuff
\newcommand{\add}[1]{\textcolor{blue}{#1}}
%deleted stuff
\newcommand{\del}[1]{\textcolor{red}{(removed: #1)}}
%uncomment these to turn off colors
\renewcommand{\add}[1]{#1}
\renewcommand{\del}[1]{}

%shortcuts
\newcommand{\w}{\ili{Welsh}}
\newcommand{\e}{\ili{English}}
\newcommand{\io}{Input Optimization}




 \newcommand{\hand}{\ding{43}}
% \newcommand{\rot}[1]{\begin{rotate}{90}#1\end{rotate}} %shortcut for angled text%  
% \newcommand{\rotcon}[1]{\raisebox{-5ex}{\hspace*{1ex}\rot{\hspace*{1ex}#1}}}

%\input{localpackages.tex}
\usepackage{arydshln}
\usepackage{rotating}

%\input{localcommands.tex}
\newcommand{\tworow}[1]{\multirow{2}{*}{#1}}


\newcommand{\tworow}[1]{\multirow{2}{*}{#1}}


\newcommand{\tworow}[1]{\multirow{2}{*}{#1}}



\title{On mechanisms by which languages become [nominative-]accusative}

\author{%
Ashwini Deo\affiliation{The Ohio State University}}

\ChapterDOI{10.5281/zenodo.495453}
% \chapterDOI{} %will be filled in at production
% \epigram{}

\abstract{%
New Indo-Aryan languages are characterized by accusative (DOM) objects in ergative, perfective clauses. This paper traces the emergence of this ergative—accusative marking pattern with the goal of determining whether it is to be considered part of a  single ``de-ergativization” trajectory, in which languages gradually lose aspects of their ergative orientation in analogy to the non-ergative portion of the grammar. Data from Middle Indo-Aryan
suggests that accusative marked objects — a deviation
from the classic ergatively-oriented sub-system —  cannot be analyzed
 in terms of the analogical extension of any existing nominative-accusative
model or as a reduction of markedness. In contrast, the empirical facts of Indo-Aryan diachrony
align better with the possibility that such deviations have to do with independent
changes in the broader argument realization options for the language. This is consistent with
Anderson's \citeyearpar{anderson77,anderson2004}  claim that a significant part of the explanation for ergativity-related patterns lies
in patterns of diachronic change rather than abstract structural considerations of Universal
Grammar.
}

\begin{document}
\maketitle


\section{Introduction}

The term \textit{ergative} is used to refer to a grammatical relation marking pattern in  which the object of
a transitive\is{transitivity} verb patterns with the single argument\is{arguments} of an intransitive verb (surfacing with absolutive case),\is{case} while the transitive\is{transitivity}
subject patterns distinctly (surfacing with \isi{ergative} case)\is{case} \citep{dixon79,dixon94,comrie78,plank79ed}.  It has sometimes been claimed  that there is a clear asymmetry between the pervasiveness of ergative--absolutive vs. nominative--accusative marking systems  across sub-domains of grammars  in languages. 

\begin{quote}
No \isi{ergative} language is fully consistent in carrying through the \isi{ergative} principle throughout
its entire morphology, syntax, and lexicon: all languages that exhibit \isi{ergative} patterning in their
commonest \isi{case}-marking system also exhibit some accusative pattern somewhere in the rest of
their grammar. (Moravcsik 1978, p.237)
\end{quote}

A possible way of interpreting this stated generalization is to take it to refer to the presence of accusative \isi{case}-marking in \isi{ergative} languages -- that is, that every language with an ergative-nominative \isi{case} marking or agreement pattern also exhibits a nominative--accusative  pattern in some subsystem of the grammar.  However, this interpretation is clearly not borne out since several languages exist that   have \isi{ergative} \isi{case} but lack  accusative \isi{case} marking altogether.\footnote{An anonymous reviewer points to  languages  like Chukchi,  Tabassaran, Chamalal, Tzutujil, Central Yupik Eskimo, and  Burushaski that lack an accusative case, and therefore lack nominative-accusative ``patterning" in terms of case marking.}  Coon \& Preminger (to appear)  interpret the above claim  to mean that even in languages  which show a high number of \isi{ergative} characteristics,  there can generally be found some portion of the grammar in which the
\isi{ergative} pattern is lost, and transitive\is{transitivity} and intransitive subjects are treated alike. In this case, the term ``\isi{ergative} pattern" seems to refer, not to surface morphological  properties, but more broadly  to syntactic properties like control and \isi{binding} with respect to which the highest \isi{arguments} of a clause may pattern alike.  Split-\isi{ergativity} is a term reserved specifically for morphological\is{morphology} marking patterns and  refers to the systematized occurrence of  a mixed indexing system, which is ergatively organized  in  well-defined 
syntactic-semantic configurations with nominative--accusative marking elsewhere in the language.   The question of how such systems arise in natural languages and change (or persist) through time, as well as the possible diachronic reasons for the \isi{parameters} on which the split is based,    can only be answered  by an investigation of split-\isi{ergative} languages for which 
we have some clear diachronic\is{diachrony} record available.  


Anderson (\citeyear{anderson77}, and later in \citeyear{anderson2004})%\todo{Anderson refs unclear}
has suggested  that to the extent we have such information,  changes involving \isi{ergative} orientation seem to be ``consequences of relatively superficial phenomena."   According to him, \isi{ergative} patterning is not a deep syntactic property of linguistic systems  but rather an emergent effect arising  from several distinct trajectories in the morphological\is{morphology} systems  of languages. In effect, there is no principle that determines  an ``\isi{ergative}" or ``accusative" pattern; rather languages may innovate\is{innovation} or lose specific cases\is{case} such as \isi{ergative} or accusative, with such patterns arising more as emergent effects of the change and not   as abstractly determined invariant objects. 
This paper examines one such emergent effect in trajectories associated with systems containing \isi{ergative} case  -- the emergence of overt accusative (object) marking in \isi{ergative} clauses. New  data from Late Middle Indo-Aryan (MIA)  and Early New Indo-Aryan (NIA) suggests that transitions resulting in deviations from the classic ergatively-oriented sub-system in a split \isi{ergative} language cannot be analyzed uniformly in terms of  the \isi{analogical} extension of  any existing  nominative-accusative model or as a reduction of \isi{markedness}.  In contrast,  the empirical facts of Indo-Aryan \isi{diachrony} align better with  the possibility that such deviations have to do with independent changes in the broader argument\is{arguments} realization options for the language. 
This is consistent with  Anderson's claim that a significant part of the explanation for ergativity-related patterns lies in  patterns of diachronic\is{diachrony} change rather than abstract structural considerations  of \isi{Universal Grammar} \citep[contra][]{delancey1981,dixon94,tsunoda1985}.

\section{Morphosyntactic changes in Middle  Indo-Aryan } 
\subsection{The emergence of ergativity} 
One well-discussed source for \isi{ergative} marking in natural languages is a \isi{passive} clausal structure that gets reanalyzed as active. Oblique marking on the optionally surfacing  \isi{agent} is reanalyzed as \isi{ergative} case while the unmarked subject of the \isi{passive} clause surfaces as absolutive object, identical to the subjects of intransitive clauses.  Indo-Aryan languages  bear the most concrete diachronic\is{diachrony} record for such a \isi{passive}--to-\isi{ergative} shift scenario. In the history of these languages, a \isi{passive} construction with resultative \isi{semantics} was reanalyzed as an active, \isi{ergative}
clause with perfective aspectual reference at least by the time of Epic \ili{Sanskrit} (Old Indo-Aryan (OIA)) and Early MIA (\citealt{andersen86,peterson98,condoravdideo14} a.o.).\footnote{The Indo-Aryan branch of \ili{Indo-European} inherits the deverbal
result \isi{stative}  form with the affix \textit{-ta}
(\isi{allomorph} \textit{-na}) (reconstructed for \ili{Indo-European} as *\textit{-to/-no}). \textit{-ta}, attested at all stages of  OIA and MIA, attaches directly to the root, and the resulting stem is adjectival,  inflecting for number and gender like any other adjectival forms.} In the oldest \ili{Vedic} texts, the \textit{-ta}-affixed form of the verb serves to describe a result-state
brought about by a preceding event when it is used predicatively in
an adjectival \isi{passive} construction. The \textit{-ta} forms (bold-faced) in \cref{ex:stirnam} agree
with the nominative patient while the \isi{agent} remains unexpressed. In
\cref{ex:nrbhir}, the agents and instruments are overtly expressed in the instrumental
case.


\begin{exe}
	\ex\label{ex:abc}
	\begin{xlist}
		\ex\label{ex:stirnam}\gll \textbf{stīr-ṇáṃ} te barhíḥ \textbf{su-tá}  \\
		strew-\textsc{perf.n.sg} you.\textsc{dat.sg} Barhis.\textsc{nom.n.sg} press-\textsc{perf.m.sg} \\
		
		\gll indra sóma-ḥ  \textbf{kṛ-tā́} dhānā́ át-tave  te hári-bhyāṃ\\
Indra.\textsc{voc.sg} Soma-\textsc{nom.m.sg}  do-\textsc{perf.m.pl}  barley.\textsc{nom.m.pl} eat-\textsc{inf}  you.\textsc{gen.sg} horse-\textsc{dat.sg}\\
\glt `The Barhis has been strewn for thee, O Indra; the Soma  has been pressed (into an extract). The barley grains  have been prepared for thy two bay-horses to eat.' (Ṛgveda 3.35.7)

		\ex\label{ex:nrbhir}\gll nṛ-bhir \textbf{dhū-táḥ}  \textbf{su-tó}  áśna-iḥ áv-yo vā́ra-iḥ \textbf{páripū-taḥ}  \\
		man-\textsc{inst.pl}  wash-\textsc{perf.m.sg}  press-\textsc{perf.m.sg}  stone-\textsc{inst.pl}  wool-\textsc{gen.sg}  filter-\textsc{inst.pl}  strain-\textsc{perf.m.sg} \\
		\glt `It (the Soma) has been washed  by men, pressed with the help of stones, strained with  wool-filters.' (Ṛgveda 8.2.2)
	\end{xlist}
\end{exe}


As shown in \cref{ex:def}, the \textit{-ta} form agrees with the sole
(nominative) argument\is{arguments} of intransitive verbs. This results in a
difference in the marking of the subject \isi{arguments} of transitive\is{transitivity} and
intransitive verbs. In \cref{ex:abc} the verb does not agree with the
instrumental agentive arguments.\is{arguments}  In \cref{ex:def}, in contrast, the
verb \textit{śri-taḥ} has a nominative subject \textit{soma}  and
agrees with it in number and gender. 

\begin{exe}
\ex\label{ex:def}\gll div-i somo adhi \textbf{śri-taḥ} \\
heaven-\textsc{loc.sg} soma.\textsc{nom.m.sg} on rest-\textsc{perf.m.sg} \\
\glt `Soma rests (is supported) in the heaven.' (Ṛgveda 10.85.1)
\end{exe}

This  resultative \textit{-ta} construction (sometimes in \isi{periphrasis} with tense auxiliaries)  is the source of
the \isi{ergative} pattern observed in the perfective \isi{aspect} in the later
languages.  In later stages of OIA, the construction was extended to
marking the perfect \isi{aspect} and it exhibited existential as well as
universal perfect readings \citep{condoravdideo14}.  By the time of
Epic \ili{Sanskrit} (late stage of OIA), the \textit{-ta}
construction became a frequently used device for marking past
perfective reference. The \isi{agent} argument\is{arguments} in these cases is most
frequently overt and marked with instrumental case.   Past eventive reference is indicated
by the presence of past referring frame adverbials like \textit{purā}
`formerly' and \textit{tadā} `then'. Perfective clauses containing intransitive verbs occur with nominative subjects \cref{ex:jaratkaruh}. All the examples below are from the \textit{Mahābhārata}, one of two epics that constitute the record for this stage of the language.

\begin{exe}
\ex\label{ex:ghi}
\begin{xlist}
\ex\label{ex:pura} \gll \textbf{purā}  \textbf{ devayug-e}   ca  eva  \textbf{dṛṣ-ṭaṃ}    sarvaṃ  mayā  vibho\\
formerly  god.age-\textsc{loc.sg}  and  \textsc{ptcl}  see-\textsc{perf.n.sg}  everything I-\textsc{inst.sg}  lord-\textsc{voc.sg}\\
\glt `Lord, formerly, in the age of the Deva (Gods), I \textit{saw} everything.' (\textit{Mahābhārata} 3.92.6a; \citealt{deo2012})
\ex\label{ex:hrta} \gll \textbf{hṛ-tā}  gau-ḥ   sā  \textbf{tadā} t-ena  \\
steal-\textsc{perf.f.sg}  cow-\textsc{nom.f.sg}  that-\textsc{nom.f.sg}  then   he-\textsc{inst.3.sg} \\

\gll prapāta-s   tu  na  \textbf{tark-itaḥ}\\
fall-\textsc{nom.m.sg}  \textsc{ptcl}  \textsc{neg}  consider-\textsc{perf.m.sg}\\
\glt `Then he \textit{stole} that cow, but \textit{did} not \textit{consider} the
fall (consequences).' (\textit{Mahābhārata} 1.93.27e; \citealt{deo2012})

\ex\label{ex:jaratkaruh}\gll jaratkāruḥ \textbf{ga-taḥ} svarga-ṃ sahitaḥ \\
Jaratkāru.\textsc{nom.m.sg} go-\textsc{perf.m.sg} heaven-\textsc{acc.sg} accompanied \\

\gll sva-iḥ pitāmaha-iḥ \\
self-\textsc{inst.m.pl}  ancestor-\textsc{inst.m.pl} \\
\glt `Jaratkāru went to heaven accompanied by his ancestors.' (\textit{Mahābhārata} 1.130.43c)
\end{xlist}
\end{exe}

\largerpage%longdistance
The main change between Epic \ili{Sanskrit} (OIA) and the later MIA  stage of the
language concerns the erosion and simplification of the rich
\isi{tense}-\isi{aspect} system \citep{pischel00,bloch65}.  Inflectional\is{inflection} past
referring forms such as the aorist, the inflectional perfect, and the
imperfect disappeared from the language, leaving the \textit{-ta}
construction as the only past referring device.\footnote{Traditional
  grammarians do provide instances of the inflectional perfect and the
  aorist during this period, but they only occur as isolated,
  unanalyzed forms for a few verbs like \textit{āha}-`say-\textsc{aor}'
  and \textit{akāshi} -`do-\textsc{aor}'.}  This loss of the
inflectional\is{inflection} system has often been cited as a reason for the increase
in the frequency and \isi{scope} of the participial construction, which in turn
led to the unmarking of the \isi{stative} nature of the construction. The change to an \isi{ergative} alignment was certainly complete at the  Mid to Late MIA stage  \citep{hock1986,bubenik98}.
The examples below from an archaic MIA
Mahāraṣṭrī text \textit{Vasudevahiṃḍī}
(ca.\ 500 AD) shows this \isi{ergative} alignment. The verb agrees with the
nominative subject in \cref{ex:patto}. In \cref{ex:tena} the verb agrees with
the nominative marked object while the agentive argument\is{arguments} (`that running
one') appears in the instrumental. 

\begin{exe}
\ex\label{ex:jkl}
\begin{xlist}
\ex\label{ex:patto}\gll \textbf{pat-to}  ya  seṇiyo  rāyā   ta-m  \\
reach-\textsc{perf.m.sg}  and  Seṇiya.\textsc{nom.m.sg}  king.\textsc{nom.m.sg}  that-\textsc{acc.sg}  \\

\gll paesa-m \\
place-\textsc{acc.sg} \\

\glt `And King  Seṇiya \textit{reached} that place.' (\textit{Vasudevahiṃḍī} KH. 17.1)

\ex\label{ex:tena}\gll t-eṇa  palāyamāṇ-eṇa  purāṇakuv-o \\
that-\textsc{inst.sg}  running-\textsc{inst.sg}  old.well-\textsc{nom.m.sg} \\

\gll taṇadabbhaparichinn-o  \textbf{diṭ-ṭho} \\
grass.covered-\textsc{nom.m.sg}  notice-\textsc{perf.m.sg}\\

\glt `That running one \textit{noticed} an old well covered with grass.' (\textit{Vasudevahiṃḍī} KH. 8.6)
\end{xlist}
\end{exe}


 Indo-Aryan  \isi{diachrony} after the MIA stage  has often been  characterized as involving a progressive loss of  \isi{ergative} alignment and gradual drift towards a \isi{nominative-accusative marking} in perfective clauses.  There are three observed ways in which the  descendent systems deviate from the proto-\isi{ergative} system of MIA: (a) Loss of \isi{ergative} \isi{morphology} in pronominal and nominal paradigms\is{paradigm}\footnote{In fact, data from some Early NIA languages, e.g. Hindi, reveals that the original instrumental
marking observed on transitive\is{transitivity} subjects for the MIA
\isi{ergative} system is entirely lost for \textit{all} nominal and pronominal expressions in some stages of Indo-Aryan. The \isi{ergative} pattern of agreement is nevertheless retained. The example in \cref{ex:jo} is from the work of Kabir, a poet from the 15th century CE. There is no overt \isi{ergative} marking on the 3rd person subject but the agreement on the verb is  with the
\isi{feminine} object argument\is{arguments} (explicit or unpronounced) \textit{chādar}
`sheet'.

\begin{exe}
\ex\label{ex:jo}\gll jo \textbf{chādar} sura-nara-muni \textbf{oḍh-i} \\ 
which sheet.\textsc{nom.f.sg} gods-men-sages.\textsc{$\emptyset$erg}
wrap-\textsc{perf.f.sg}\\
\glt `Which sheet the  Gods, men, and sages, all wore, (that sheet)...'
\end{exe} 

 }; (b) Subject agreement (replacing or in addition to object agreement); (c) Accusative marking on a privileged class of objects, i.e. the spread of \isi{differential object marking}.
 
\largerpage[2]%longdistance
It is logical to think of  the implementation of any of these changes independently or together as the ``de-ergativization'' of  an \isi{ergative} system in \isi{analogy} to the non-\isi{ergative} portion of the grammar.  Indeed, the patterns seen in individual  NIA languages, such as suppression of overt \isi{ergative} case (e.g.\ in Old Hindi and Marathi); nominative subjects (e.g.\ in Bangla) and  agreement with overt \isi{ergative} subject (e.g.\ in Nepali) are all analogizable\is{analogical} to existing marking patterns in the language such as unmarked subjects,  nominative subjects, and subject agreement. However, the emergence of accusative marking on objects of transitive,\is{transitivity} perfective clauses poses a puzzle for a straightforward \isi{analogical} extension narrative for de-ergativization. The puzzle  arises from the \isi{evolution} of \isi{case} marking in MIA, to which we now turn. 



\subsection{Syncretism in nominal case marking}\label{case change}
A critical change between the OIA and MIA stages, particularly in the Late MIA period, 
is the \isi{restructuring} of the nominal \isi{case} system. Notable here is the loss of
morphological contrast between nominative and accusative as well
as between the genitive and the dative cases.\is{case} The syncretized
set of \isi{case}-endings for full nouns are  given in \is{case|(} \cref{mia-table}. 

\begin{table}
\begin{tabular}[t]{lll}
\lsptoprule
& \multicolumn{1}{c}{Singular} & \multicolumn{1}{c}{Plural} \\
\midrule
Nominative/Accusative & \textit{-u, a, aṃ} & \textit{-a, aĩ} \\
Instrumental/Ergative &\textit{-eṃ}, \textit{iṃ}, \textit{he, hi} & \textit{-e(h)ĩ, ehi, ahĩ} \\
Ablative & \textit{-hu, ahu, aho} & \textit{-hũ, ahũ} \\
Genitive/Dative & \textit{-ho, aho, ha, su, ssu} & \textit{-na, hã} \\
Locative & \textit{-i, hi, hiṃ} & \textit{-hĩ} \\
\lspbottomrule
\end{tabular}
\caption{Case-endings for full nouns.}
\label{mia-table}
\end{table} \is{case|)}


\cref{tab:putta} contains an example of inflected \textit{-a} stems with the noun \textit{putta} `son'. 
\begin{table}
\begin{tabular}[t]{llll}
\lsptoprule
\multicolumn{1}{c}{Stem} & \multicolumn{1}{c}{Case} & \multicolumn{1}{c}{Singular} & \multicolumn{1}{c}{Plural} \\
\midrule
\textit{a}-stems &  Nominative/Accusative  &  \textit{putt-u} & \textit{putt-a}   \\
 & Instrumental/Ergative & \textit{putt-eṃ} & \textit{putta-hiṃ/ehiṃ}  \\
 & Genitive/Dative & \textit{putt-aho/ahu}  & \textit{putta-haṃ}\\
 \lspbottomrule
\end{tabular} 
\caption{Inflected \form{a}-stems with \form{putta} `son'.}
\label{tab:putta}
\end{table}
 
The pronominal system retains more contrasts and \isi{syncretism} between the nominative and accusative is observed only in the \isi{plural} sub-part of most pronominal paradigms.\is{paradigm} \cref{tab:pron} (culled from \citealt{declercq2010}) provides inflectional\is{inflection} forms for some pronominal expressions to illustrate.



\begin{table}
\begin{tabular}[t]{llll}
\lsptoprule
\multicolumn{1}{c}{Stem} & \multicolumn{1}{c}{Case} & \multicolumn{1}{c}{Singular} & \multicolumn{1}{c}{Plural} \\
\midrule
 1st pronoun & Nominative  & \textit{hauṃ} &  \textit{amhẽ, amhaiṃ} \\
 & Accusative& \textit{mai(ṃ)} & \textit{amhẽ, amhaiṃ} \\ 
% & \textsc{inst/erg} &  mai(ṃ) & amhe-hiṃ \\ 
 & Genitive/Dative& \textit{mahu, majjhu} & \textit{amha, amhaha} \\[2ex]
 2nd pronoun & Nominative  & \textit{tuhuṃ} & \textit{tumhẽ} \\ 
 & Accusative  & \textit{paiṃ}, \textit{taiṃ} & \textit{tumhẽ}\\
 & Genitive/Dative & \textit{tahu}, \textit{tujjha} & \textit{tumha}, \textit{tumhaha} \\[2ex]
 3rd pronoun  & Nominative &  \textit{so}, \textit{su}; \textit{sā} & \textit{te}, \textit{tāu}  \\
\textsc{masc;fem} & Accusative & \textit{taṃ}; \textit{sā} &  \textit{te};  \textit{tāu} \\
  & Genitive/Dative & \textit{taho}, \textit{tahu}; \textit{tāhe} & \textit{tāhaṃ}; \textit{tāhaṃ} \\
 \lspbottomrule
\end{tabular}
\caption{Inflectional forms for pronominals.}
\label{tab:pron}
\end{table}



 
 The loss of contrast between the nominative and accusative cases\is{case} in most paradigms\is{paradigm} in a relatively free-word order\is{word order} language leads to heavy reliance on semantic cues from the linguistic material to determine grammatical relations. Consider the following examples from the \textit{Paumacariu}, an 8th century text in verse, to illustrate the syncretic \isi{nominative-accusative marking} (glossed \textsc{nom}).\footnote{This is a Jaina rendition
of the Epic \ili{Sanskrit} text  \textit{Rāmāyana}. The edition used is the H.C. Bhayani edition published by the Bharatiya Vidya Bhavan
between 1953 and 1960. The text is available in searchable electronic format, input by Eva De
Clercq at Ghent University. The reason for using a late MIA text is to identify properties of the system that is as  close to the grammars of the Early NIA system as possible.} In \cref{ex:mno}, a sequence of parallel clauses, whether the first-occurring nominative expression realizes the grammatical subject or the grammatical  object  is determined by the meaning of the clause.\footnote{The \#...\# marks clause boundaries in the sequence.} In \cref{ex:pqr}, the relative pronoun, which refers to a human participant, disambiguates the grammatical structure. 

\begin{exe}
\ex\label{ex:mno}\gll \#kiṃ \textbf{tamu} haṇ-ai ṇa vālu \textbf{ravi}\#  \\
\textsc{ques} darkness.\textsc{nom.sg} destroy-\textsc{impf.3.sg} \textsc{neg}  young sun.\textsc{nom.sg} \\

\gll \#kiṃ vālu \textbf{davaggi} ṇa  ḍah-ai \textbf{vaṇu}\# \\
 \textsc{ques }  young fire.\textsc{nom.sg} \textsc{neg} burn-\textsc{impf.3.sg} forest.\textsc{nom.sg} \\

\gll \#kiṃ kari dal-ai ṇa vālu \textbf{hari}\# \\
\textsc{ques} elephant.\textsc{nom.sg}  shatter-\textsc{impf.3.sg} \textsc{neg} young lion.\textsc{nom.sg} \\

\gll  \#kiṃ vālu ṇa ḍaĩk-ai uragamaṇu\#   \\
\textsc{ques} young \textsc{neg} bite-\textsc{impf.3.sg} snake.\textsc{nom.sg} \\
\glt `Does the young (rising) sun not destroy darkness? Does the young fire (spark) not burn down the forest? Does a young lion (cub) not shatter the elephant? Does the young snake not bite?' (\textit{Paumacariu}  2.21.6.9)

\ex\label{ex:pqr}\gll \textbf{jo} ghañ \textbf{ṇisi-bhoyaṇu} ummah-ai\\
who.\textsc{rel.nom.m.sg} \textsc{ptcl} night.\textsc{loc}-meal.\textsc{nom.m.sg} give.up-\textsc{impf.3.sg}\\

\gll \textbf{vimalattaṇu} \textbf{vimala-gottu} lah-ai \\
spotless.body.\textsc{nom.m.sg} spotless.name.\textsc{nom.m.sg} attain-\textsc{impf.3.sg} \\
\glt `One who gives up eating in the evening (he) attains a spotless body and name.'  (\textit{Paumacariu} 2.34.8.8)
\end{exe}



Accusative marking is clearly visible only on first and second person singular pronouns in \isi{imperfective} clauses as shown in the examples in \cref{ex:stu}. 

\begin{exe}
\ex\label{ex:stu}
\begin{xlist}
\ex \gll suggīu deva \textbf{paiṃ} sambhar-ai \\
Suggiu.\textsc{nom.m.sg} deva.\textsc{nom.m.sg} you.\textsc{acc.sg} remember-\textsc{impf.3.sg} \\
\glt `Lord Suggiu remembers you.'  (\textit{Paumacariu} 3.45.10.8)
\ex \gll jai ṇa vihāṇa-e \textbf{paiṃ} vandhāv-ami \\
if \textsc{neg} tomorrow.\textsc{loc.sg} you.\textsc{acc.sg} bind-\textsc{impf.1.sg} \\
\glt `If I do not capture you tomorrow\dots{}' (\textit{Paumacariu} 3.49.20.3)
\ex \gll jo \textbf{maiṃ} muevi aṇṇu \\
who.\textsc{rel.nom.m.sg} I.\textsc{acc.sg} besides another.\textsc{nom.m.sg} \\

\gll jayakār-ai \\
adore-\textsc{impf.3.sg} \\
\glt `(The one) who adores another one besides me\dots{}'  (\textit{Paumacariu}  2.25.1.9)
\end{xlist}
\end{exe}



Syncretism rooted in \isi{sound change} is also observed
between  the nominative and instrumental forms (the \isi{case} form that gets re-interpreted as \isi{ergative} when appearing with agentive \isi{arguments} in perfective clauses) of the first and second
person \isi{plural} pronouns as in \cref{tab:nominst}.

\begin{table}
\begin{tabular}[t]{llll}
\lsptoprule
\multicolumn{1}{c}{\multirow{2}{*}{Aspect}} & \multicolumn{1}{c}{\multirow{2}{*}{Person}} & \multicolumn{2}{c}{Number}\\ \cmidrule{3-4}
& & \multicolumn{1}{c}{Singular} & \multicolumn{1}{c}{Plural} \\
\midrule
Non-perf & 1   & hauṃ & amhaĩ/amhẽ\\
Perf & 1  &  maiṃ &  amhaĩ/amhẽ/amhe-hiṃ \\
Non-perf & 2  & tuhuṃ & tumhaĩ/tumhẽ \\
Perf & 2 & taiṃ & tumhaĩ/tumhẽ/tumhehiṃ\\
Non-perf & 3  & so & te \\
Perf & 3  & teṃ, teṇẽ  & tehĩ/tāhaṃ \\
\lspbottomrule
\end{tabular}
\caption{Nominative and instrumental pronominal forms.}
\label{tab:nominst}
\end{table}



Despite this \isi{syncretism}, agreement is uniformly with the nominative
argument\is{arguments} -- with the nominative object in constructions based on the
\textit{-ta} form and with the nominative subject elsewhere. The examples
in \cref{ex:vwx} illustrate this pattern with the first and second person \isi{plural} pronouns
\textit{amhẽ} and  \textit{tumhẽ}.  \cref{ex:amhe}
contains the syncretized pronoun \textit{amhẽ} which triggers
agreement in the \isi{imperfective} \isi{aspect} while the same form fails to
trigger agreement in \cref{ex:kiu}. In \cref{ex:jiha} the second person \isi{plural}
syncretic form used in an imperative clause triggers agreement while it fails to trigger verb agreement in the perfective \cref{ex:tumhe}.


\begin{exe}
	\ex\label{ex:vwx}
\begin{xlist}
\ex\label{ex:amhe}\gll \textbf{amhẽ} jāe-va vaṇavāsa-ho \\
we.\textsc{syncr} go-\textsc{impf.1.pl} forest.dwelling-\textsc{dat.sg} \\
\glt `We are going to our forest-exile.' (\textit{Paumacariu} 2.23.14.3)

\ex\label{ex:kiu}\gll ki-u \textbf{amhẽ} ko avarāh-o \\
do-\textsc{perf.m.sg} we.\textsc{syncr} what crime-\textsc{nom.m.sg} \\
\glt `What crime have we done?' (\textit{Paumacariu} 1.2.13.9)

\ex\label{ex:jiha}\gll jiha sakk-aho tiha utthar-aho \textbf{tumhẽ}\\
in.which.way can-\textsc{imp.2.pl} in.that.way save-\textsc{imp.2.pl} you.\textsc{syncr.pl} \\
`Save yourselves in the way that you can.' (\textit{Paumacariu} 5.82.12.4)
%The referent here is singular but the respectful 2nd \isi{plural} form has been used. Agreement follows the choice of form. -aho as one of the imperative 2nd \isi{plural} endings according to de clercq's grammar. 
\ex\label{ex:tumhe}\gll \textbf{tumhẽ} jaṃ cint-iu taṃ \\
you.\textsc{syncr.sg} what.\textsc{rel.m.sg} think-\textsc{perf.m.sg} that.\textsc{correl.m.sg} \\

\gll hū-a \\
happen-\textsc{perf.m.sg} \\
\glt `That, which  you thought (would happen),  happened.' (\textit{Paumacariu} 3.47.9.6)
\end{xlist}
\end{exe}



These patterns of syncretization within the nominal inflectional\is{inflection} system of MIA  
are difficult to reconcile with a story in which there is a straightforward extension of an existing alignment pattern in the language to a marked sub-system of the grammar. 
Although there is a   contrast between the nominative and accusative cases\is{case} in MIA, it is exhibited only  in selected parts of the pronominal system (a subset of the singular pronouns) and therefore seems to be rather weak evidence for extending the accusative marking pattern to \isi{ergative} clauses. A reviewer argues that the regular presence of such a \isi{case}-marking pattern in \isi{imperfective} clauses, however limited in terms of its application,  should not be seen as ``weak" evidence for a nominative accusative pattern.  I concede  that it is indeed theoretically possible that the pattern observed in a small  subset of \isi{imperfective} non-\isi{ergative} clauses gets extended to  perfective, \isi{ergative} clauses. However,
neither existing grammars of MIA \citep{pischel00,vale48,declercq2010} nor an examination of the textual data indicate  any  presence of accusative marked object \isi{arguments} in perfective transitive\is{transitivity} clauses at this stage in the language. Even  pronominal objects \crefrange{ex:haum}{ex:cakkesarena} and human-denoting full noun phrase objects \crefrange{ex:vinivariu}{ex:ditthu} of canonical transitive verbs, which obligatorily appear with overt accusative marking in the NIA languages,  are uniformly marked nominative at this stage.\footnote{Thus, there are no positive  instances with pronominal forms  \textit{maiṃ}, \textit{taiṃ},  \textit{taṃ} etc.\ being used instead of \textit{hauṃ}, \textit{tuhuṃ}, or \textit{so/su} etc.\  in \isi{ergative} clauses with pronominal objects at even the latest stages of Middle Indo-Aryan.} 

\begin{exe}
\ex\label{ex:yza}
\begin{xlist}
\ex\label{ex:haum}\gll \textbf{hauṃ} ṇikkāraṇe ghall-iya  rām-eṃ \\
I.\textsc{nom.sg} without.reason drive.out-\textsc{perf.f.sg} Rām-\textsc{erg.sg} \\
\glt `Rām drove me out (of Ayodhya) without any reason.' (\textit{Paumacariu} 5.81.13.8) 
\ex\label{ex:cakkesarena}\gll cakkesar-eṇa kema \textbf{tuhũ} di-ṭṭhī \\ Cakkesara-\textsc{erg.m.sg} how you.\textsc{nom.sg} see-\textsc{perf.f.sg} \\
\glt `How were you noticed by Cakkesara (Rāvaṇa)?' (\textit{Paumacariu} 2.4.2.1.5) 
\ex\label{ex:vinivariu}\gll viṇivār-iu \textbf{rāvaṇu} rāhav-eṇa \\ dissuade-\textsc{perf.m.sg} rāvaṇa.\textsc{nom.m.sg}  rāhava-\textsc{erg.m.sg} \\
\glt `Rāhava (Rāma)  dissuaded Rāvaṇa' (\textit{Paumacariu} 4.66.14.6)
\ex\label{ex:ditthu}\gll di-ṭṭhu jaṇaddaṇu rāhavacand-eṃ \\
see-\textsc{perf.m.sg} jaṇaddaṇa.\textsc{nom.m.sg} rāhavacanda-\textsc{ins.sg} \\
\glt `Rāhavacanda saw Jaṇaddaṇa.' (\textit{Paumacariu} 2.29.8.1)
\end{xlist}
\end{exe}

Moreover, no language of the later stage (Early NIA)  has an ergative-accusative marking pattern which uses the pronominal forms of late MIA in \isi{ergative} clauses that accusative marking on objects. While the issue needs to be more closely investigated, it seems reasonable to look for an alternative source for accusative marking in \isi{ergative} clauses than the template\is{templates} offered by MIA. 



\section{Differential object marking: A New Indo-Aryan innovation}  
The previous subsection established that accusative  marking  of the MIA variety is both weakly present and shows no evidence of being extended to perfective \isi{ergative} clauses at later stages of Indo-Aryan. This leaves the possibility that the incidence of object marking in \isi{ergative} clauses -- a pervasive phenomenon in the Modern NIA languages -- begins with the Differential Object Marking  pattern -- which is considered to be an NIA innovation.\is{innovation} Differential Object Marking (henceforth DOM) in Indo-Aryan languages is sensitive to animacy and referentiality \isi{features} of arguments.\is{arguments} It is obligatory on 1st and 2nd pronominal objects, and  on 3rd person animate-denoting pronominals. It is optional with animate-denoting full NPs where the absence of object marking correlates with a non-referential  interpretation of the NP. In the Modern NIA languages, this semantically driven pattern of object marking does not distinguish between \isi{ergative} and non-\isi{ergative} clauses; i.e.\ the \isi{case} marking on objects is entirely independent of any overt or covert presence of \isi{case} on the subject.  

Logically, one can imagine two ways in which an ergativity-insensitive object marking pattern can emerge in a system. It could be that the DOM pattern first emerges in 
 Late MIA or Early NIA in non-\isi{ergative} clauses. Such a pattern is then later extended analogically\is{analogical} to \isi{ergative} clauses as part of the de-ergativization trajectory characterizing Indo-Aryan diachrony.\is{diachrony} The second possibility is for the DOM pattern to emerge simultaneously in both \isi{ergative} and non-\isi{ergative} clauses and gradually extend to different classes of verbs. On this latter scenario, the presence of DOM in \isi{ergative} clauses is not part of the larger de-ergativization trajectory that characterizes NIA diachrony,\is{diachrony} but rather attributable to independent developments  that introduce overt marking on direct objects into the \isi{case} system.\footnote{The effects on agreement in languages which exhibit such  a changed \isi{case}-marking pattern may be different. In Modern NIA we see both default agreement in \isi{ergative} clauses  when both subject and object are \isi{case}-marked (e.g. in Hindi, Marathi) or   continued object agreement despite overt accusative marking on the object (e.g. in Gujarati, Marwari).} The empirical facts of Late MIA and Early NIA texts support the second scenario.  In what follows, I will suggest that the  emergence of  DOM in both \isi{ergative} and non-\isi{ergative} clause types of MIA  amounts to  the extension of an inherited OIA  marking pattern observed with the class of so-called ``double object'' verbs.


 
\subsection{Double object verbs in Old Indo-Aryan}
A class of verbs in OIA exhibits a double object pattern in  which the theme or goal and another participant of the denoted event are marked in the accusative case. Semantically, this is a diverse class and includes at least the subclasses in \cref{tab:dooia}. 

\begin{table}
\begin{tabularx}{\linewidth}[t]{p{3cm}L}
\lsptoprule
\multicolumn{1}{c}{Class} & \multicolumn{1}{c}{Verbs}\\
\midrule
Verbs of speaking & \textit{brū} `speak', \textit{vac} `say', \textit{kath} `tell' \\[2ex]
Verbs of asking & \textit{pṛcch} `ask', \textit{yāc} `request, solicit',  \textit{bhikṣ} `beg', \textit{prārth} `plead' \\[2ex]
Verbs of teaching & \textit{upa-diś} `teach', \textit{anu-śās"} `teach', \textit{ā-diś} `direct'  \\[2ex]
Causatives of some transitives & \textit{khād-aya} `cause to eat', \textit{pā-yaya} `cause to drink', \textit{darś-aya}  `cause to see',    \textit{śrāv-aya} `cause to hear'  \\[2ex]
% \textit{sam-īkṣ-aya} `cause to perceive' 
Miscellaneous ditransitives & \textit{jī} `win, \textit{duh} `milk', \textit{daṇḍ} `punish', \textit{nī} `lead'  \\
\lspbottomrule
\end{tabularx}
\caption{Double object verbs in Old Indo-Aryan.}
\label{tab:dooia}
\end{table}



\cref{ex:bcd} contains examples from OIA (Epic \ili{Sanskrit}) involving verbs of speaking in \isi{imperfective}, non-\isi{ergative} clauses. In \cref{ex:atas}, the pronominal denoting the addressee if the verb of speaking event \textit{tvāṃ} is accusative as is the information communicated, \textit{nidarśanam} `the teaching'. \cref{ex:salyo}, from a proximal location in the text,  is similar. 

\begin{exe}
\ex\label{ex:bcd}
\begin{xlist}
\ex\label{ex:atas}\gll atas tvā-ṃ  kathay-e karṇa nidarśan-am \\
hence you-\textsc{acc.sg} tell-\textsc{impf.1.sg} karṇa.\textsc{voc.sg} teaching-\textsc{acc.n.sg} \\

\gll idaṃ punaḥ \\
this.\textsc{acc.n.sg} again  \\
\glt `Hence, O Karṇa, I tell you this teaching (advice) again.' (\textit{Mahābhārata} 8.28.8e)

\ex\label{ex:salyo}\gll śalyo 'brav-īt punaḥ karṇ-aṃ \\
śalya.\textsc{nom.m.sg} speak-\textsc{impfct.3.sg} again karṇa-\textsc{acc.m.sg} \\

\gll nidarśan-am udāhar-an \\  
teaching-\textsc{acc.n.sg} announce-\textsc{part.nom.m.sg}  \\
\glt `Śalya again spoke out  his advice to Karṇa' (\textit{Mahābhārata} 8.28.1c)
\end{xlist}
\end{exe}

An alternative realization for pronominal  animate-denoting higher \isi{arguments} of double object verbs is as \textsc{dat/gen} clitics. 


\begin{exe}
\ex\label{nonergdat}
\begin{xlist}
\ex\label{ex:hanta}\gll hanta \textbf{te} kathay-iṣy-āmi nām-āni iha \\
\textsc{ptcl}  you.\textsc{dat/gen.cl} tell-\textsc{fut-1.sg}  name-\textsc{acc.n.pl} here  \\

\gll manīṣi-ṇām \\ 
wise-one-\textsc{gen.m.pl} \\
\glt `Ah, I will tell you the names of the wise ones.' (\textit{Mahābhārata} 1.48.4a)

\ex\label{ex:isate}\gll \dots{} īś-ate bhagavān ekaḥ saty-am \\
{} reign-\textsc{impf.3.sg} Lord.\textsc{nom.m.sg} alone.\textsc{nom.m.sg} truth.\textsc{acc.n.sg} \\

\gll etad brav-īmi \textbf{te} \\
this.\textsc{acc.n.sg} speak-\textsc{impf.1.sg} you.\textsc{dat/gen.cl} \\
\glt `The Lord alone reigns [over time and death and this universe of mobile and immobile objects], this truth I tell you.' (\textit{Mahābhārata} 5.66.13c)\footnote{The previous line of verse completes the translation: \textit{kālasya ca hi mṛtyoś ca jaṇgamasthāvarasya ca}  (\textit{Mahābhārata} 5.66.13a)}
\end{xlist}
\end{exe} 



In \isi{ergative}, perfective clauses, this higher argument\is{arguments} may  surface variably: either as the nominative subject of the passivized verb form (examples in \cref{ex:efg})  or as a \textsc{dat/gen} marked \isi{clitic} pronoun (examples in \cref{ex:hij}).\footnote{In \cref{ex:ukto}, the passivized subject is covert and the nominative case marking of the pro-dropped subject  is inferred from the agreement on  the auxiliary verb \textit{asmi}.}
  
\begin{exe}
\ex\label{ex:efg}
\begin{xlist}
\ex\label{ex:ukto}\gll uk-to rātr-au mṛg-air as-mi \\
	speak-\textsc{perf.m.sg} night-\textsc{loc.sg} animal-\textsc{inst.pl} be-\textsc{impf.1.sg} \\
	\glt `I was spoken to by the beasts at night.' (\textit{Mahābhārata}  3.244.11a)
	
\ex\label{ex:taya}\gll ta-yā...  śr-āv-ito vacan-āni \textbf{saḥ} \\  
she-\textsc{ins.sg} hear-\textsc{caus-perf.m.sg} word-\textsc{acc.n.sg} he.\textsc{nom.sg} \\
\glt `He was made to hear (these) words by her.' (\textit{Mahābhārata}  2.2.6a)

\ex\label{ex:sa}\gll \textbf{sa} mayā varadaḥ  kām-aṃ \\
he.\textsc{nom.m.sg} I.\textsc{ins.sg} boon.granting.\textsc{nom.m.sg} desire-\textsc{acc.m.sg} \\

\gll yāc-ito dharmasaṃhit-am \\ 
solicit-\textsc{perf.m.sg} virtue.bound-\textsc{acc.m.sg} \\
\glt `He, the boon-granting one, was solicited by me for (fulfilling my) virtuous desire.' (\textit{Mahābhārata} 1.78.3c)
\end{xlist}
\end{exe}

\begin{exe}
\ex\label{ex:hij}
\begin{xlist}
\ex\label{ex:samkh}\gll sāṃkhyadarśan-am etāvad uk-taṃ \textbf{te} \\
sāṃkhyadarśan-\textsc{nom.n.sg} {so far} speak-\textsc{perf.n.sg} you.\textsc{dat/gen.sg}  \\

\gll nṛpasattama \\ 
best.king.\textsc{voc.sg} \\
\glt `Thus far, the Sāṃkhyadarśana was spoken to you, O best of kings.' (\textit{Mahābhārata} 12.295.1a)

\ex\label{ex:tad}\gll tad etat kath-itaṃ sarv-aṃ mayā \textbf{vo} munisattamāḥ \\
thus, this.\textsc{nom.n.sg} tell-\textsc{perf.n.sg} all-\textsc{nom.n.sg} I.\textsc{ins.sg} you.\textsc{dat/gen.pl} great.sage.\textsc{voc.pl} \\
\glt `Thus, I have told you all this, O great sages.'  (\textit{Mahābhārata} 1.20.12a)

\ex\label{ex:upadisto}\gll upadiṣ-ṭo hi \textbf{me} pitr-ā yogo 'nīka-sya bhedan-e \\ teach-\textsc{perf.m.sg} \textsc{ptcl} I.\textsc{dat/gen.cl} father-\textsc{inst.3.sg} method.\textsc{nom.m.sg} array-\textsc{gen.m.sg} penetration-\textsc{loc.n.sg} \\
\glt `The method of penetrating into this (military) array has been taught to me by my father.'   (\textit{Mahābhārata}  7.34.19a)

\ex\label{ex:brahmacaryam}\gll brahmacary-am idaṃ bhadr-e \textbf{mama} \\
celibacy-\textsc{nom.n.sg} this good.lady-\textsc{voc.sg} I.\textsc{gen.sg}\\

\gll dvādaśavārṣik-am dharmarāj-ena ca ādiṣ-ṭaṃ \\
twelve.years-\textsc{nom.n.sg} Dharmarāja-\textsc{ins.sg} and command-\textsc{perf.n.sg} \\  \glt `Good lady, this  twelve-year celibacy  has been commanded of me by Dharmarāja.' (\textit{Mahābhārata} 1.206.21a-c)
\end{xlist}
\end{exe}


The argument realization pattern illustrated in \cref{nonergdat} and  \cref{ex:hij}, where the higher argument\is{arguments} of a double object verb surfaces  with  dative or genitive marking in both \isi{ergative} and non-\isi{ergative} clauses, is fairly robust in OIA. The alterations to the nominal \isi{case} system in MIA  described in \cref{case change}, have no effect on this pattern, since the syncretized \textsc{dat/gen}  remains available for overt marking throughout the period. Crucially, given the organization of the MIA \isi{case} system, this dative/genitive marking is the only reliably present  overt marking on non-subject \isi{arguments} in both \isi{ergative} and non-\isi{ergative} clauses at this later stage.  Based on the data from MIA, it seems most reasonable to conjecture  that this template\is{templates} triggers the \isi{reanalysis} of \textsc{dat/gen}  as accusative marking on a subset of  direct objects. 



\subsection{Double object verbs in  Middle Indo-Aryan}
In \cref{ex:klm} are given examples of  the OIA double object  verbs in their MIA incarnations. Notice that themes surface with the syncretized nominative--accusative case (glossed \textsc{nom}) while the non-theme higher argument\is{arguments} (the addressee of the speech verb in \crefrange{ex:sabbhavem}{ex:marui} and the causee in \cref{ex:taho}) appear with the syncretized \textsc{dat/gen} marking.\footnote{\cref{ex:sabbhavem} and \cref{ex:taho} have subject pro-drop.} 
 
\begin{exe}
\ex\label{ex:klm}
\begin{xlist}
\ex\label{ex:sabbhavem}\gll sabbhāv-eṃ \textbf{rāma-ho} kah-ai ema \\ goodwill-\textsc{ins.sg} rāma-\textsc{dat/gen.sg}  tell-\textsc{impf.3.sg} this.\textsc{nom.n.sg} \\
\glt `He said this to Rāma with goodwill.' (\textit{Paumacariu}  2.40.13.7)

\ex\label{ex:marui}\gll mārui kah-ai vatta \textbf{valadeva-ho} \\ Mārui.\textsc{nom.sg}  tell-\textsc{impf.3.sg}  news.\textsc{nom.sg} valadeva-\textsc{dat/gen.sg} \\
\glt `Māruti told the news to Valadeva.' (\textit{Paumacariu} 3.55.9.1)

\ex\label{ex:taho}\gll \textbf{ta-ho} daris-āv-ami ajju jamattaṇu \\ he-\textsc{dat/gen.sg} see-\textsc{caus.impf.1.sg} now yama.prowess.\textsc{nom.n.sg} \\
\glt `Now, I will show him the prowess of Yama (the god of death).' (\textit{Paumacariu} 1.11.10.6)
\end{xlist}
\end{exe}
  

  
 A look at perfective, \isi{ergative} clauses in MIA  containing double object verbs  reveals overt \textsc{dat/gen} marking on the non-theme argument\is{arguments} and unmarked themes. \cref{ex:pade} contains the \isi{causative} of a perception verb, while \crefrange{ex:kahiu}{ex:tahomaim} contain verbs of speaking. Just like OIA, there is no  difference between \isi{ergative} and non-\isi{ergative} clauses vis-à-vis the realization of non-subject arguments.\is{arguments}  
 

  
\begin{exe}
\ex\label{ex:nop}
\begin{xlist}
\ex\label{ex:pade}\gll paḍ-e paḍima... sīya-he... daris-āv-iya \\
screen-\textsc{loc.sg} image.\textsc{nom.f.sg} Sita-\textsc{dat/gen.sg} see-\textsc{caus-perf.f.sg} \\

\gll bhāmaṇḍala-ho \\
Bhāmaṇḍala-\textsc{dat/gen.sg} \\
\glt `(He) showed the image of Sita on a screen (painting) to Bhāmaṇḍala.' (\textit{Paumacariu} 2.21.8.9)

\ex\label{ex:kahiu}\gll kah-iu āsi \textbf{ma-hu} parama-jiṇind-eṃ \\ tell-\textsc{perf.m.sg} be.\textsc{pst.3.sg} I.\textsc{dat/gen.sg} great-Jinendra-\textsc{erg.m.sg} \\
\glt `The great Jinendra told (this) to me.' (\textit{Paumacariu}  1.1.12.8)

\ex\label{ex:tahomaim}\gll \textbf{ta-ho}  maiṃ  parama-bheu ehu\\
you.\textsc{dat/gen.sg} I.\textsc{erg.sg} great.secret.\textsc{nom.sg} this.\textsc{nom.sg}\\

\gll akkh-iya \\
tell-\textsc{perf.n.sg} \\
\glt `I have told you this great secret.' (\textit{Paumacariu}  1.16.8.9)
\end{xlist}
\end{exe}


 In addition to the non-theme \isi{arguments} of double object verbs, the syncretized \textsc{dat/ gen}  marking also  appears on possessor and goal \isi{arguments} of standard ditransitives (examples in \cref{ex:qrs}) and on themes of  verbs that describe a reciprocal experience (examples in \cref{ex:tuv}).  
\begin{exe}
\ex\label{ex:qrs}
\begin{xlist}
\ex\label{ex:kikkindhaho}\gll \textbf{kikkindha-ho} ghall-iya māla tāe \\
kikkindha-\textsc{dat/gen.sg} put-\textsc{perf.f.sg} garland.\textsc{f.sg} she.\textsc{erg.sg} \\
\glt `She garlanded Kikkindha (lit.\ put a garland on)' (\textit{Paumacariu} 1.7.4.1)

\ex\label{ex:paripesiu}\gll paripes-iu lehu \textbf{pahāṇā-ho}   aṇaraṇṇa-ho  ujjha-he rāṇā-ho \\
send-\textsc{perf.m.sg} letter.\textsc{nom.m.sg}  chief-\textsc{dat/gen.sg} Anaraṇya-\textsc{dat/gen.sg} Ayodhyā-\textsc{dat/gen.sg} king-\textsc{dat/gen.sg} \\
\glt `(He) sent a letter to  Anaraṇya, the king of Ayodhya' (\textit{Paumacariu} 1.15.8.4)

\ex\label{ex:angutthala}\gll aṅgutthala ṇav-evi samapp-iu   tāvahñ \textbf{mahu} \\
finger.ring.\textsc{nom.m.sg} bow-\textsc{ger} hand-\textsc{perf.m.sg} then  I.\textsc{dat/gen.sg}\\

\gll cūḍāmaṇi app-iu \\
precious.gem.\textsc{nom.m.sg} give-\textsc{perf.m.sg} \\
\glt `(After) I handed her the finger ring, having bowed to her, (she) gave me this precious gem.'   (\textit{Paumacariu} 3.55.9.7)

\ex\label{ex:dinna}\gll diṇṇa kaṇṇa maiṃ  \textbf{dasaraha-taṇay-aho} \\
give-\textsc{perf.f.sg} daughter.\textsc{nom.f.sg} I.\textsc{erg.sg} dasaraha-son-\textsc{dat/gen.sg} \\
\glt `I have given my daughter to the son of Dasaraha (Daśaratha).' (\textit{Paumacariu} 2.21.11.4)
\end{xlist}

\ex\label{ex:tuv}
\begin{xlist}
\ex\label{ex:salilu}\gll salil-u \textbf{samudd-aho} jiha milai \\ water-\textsc{nom.sg} ocean-\textsc{dat/gen.sg} as meet-\textsc{impf.3.sg} \\
\glt `Just as the water meets the ocean' (\textit{Paumacariu} 3.56.1.12)

\ex\label{ex:tavehn}\gll tāvehñ gayaṇa-ho oar-evi  \textbf{añjaṇa-he} vasantamāla mil-iya \\
then, sky-\textsc{abl} descend-\textsc{ger}  Añjanā-\textsc{dat/gen.sg} Vasantamāla.\textsc{nom} meet-\textsc{perf.f.sg} \\
\glt `Then, having descended from the sky, Vasantamālā met Añjanā.' (\textit{Paumacariu} 1.19.8.10)
\end{xlist}
\end{exe}


Critically, the syncretized \textsc{dat/gen}  marking  is the only reliable signal of non-subject \isi{arguments} in MIA and it appears without discernible difference in distribution in both \isi{ergative} and non-\isi{ergative} clauses. It does not however appear, for the most part,  on theme/patient \isi{arguments} of canonical transitive\is{transitivity} or ditransitive verbs -- animate or otherwise. \crefrange{ex:ha}{ex:niu} are examples of \isi{ergative} clauses with animate-denoting subjects while \cref{ex:minvara} contains a non-\isi{ergative} clause. 

\begin{exe}
\ex\label{nompatients}
\begin{xlist}
\ex\label{ex:ha}\gll hā vahue-vahue mañ bhantiy-ae    \textbf{tuhũ}\\
alas  bride.\textsc{voc} I.\textsc{erg.sg} unthinking-\textsc{inst.sg}  you.\textsc{nom.sg}\\

\gll ghall-iya aparikkhantiy-ae \\
drive.out-\textsc{perf.f.sg} without.testing-\textsc{erg.f.sg} \\ \glt `Alas, O bride, I drove you out without testing you in any way.' (\textit{Paumacariu} 1.19.15.7)

\ex\label{ex:niu}\gll ṇi-u \textbf{tihuaṇa-paramesaru} tettahe    sapparivāru purandaru jettahe \\
take-\textsc{perf.m.sg} three.worlds.lord.\textsc{nom.m.sg} there with.family.\textsc{nom.m.sg} purandara.\textsc{nom.m.sg} where \\ \glt `(She) took the lord of the three worlds there where Purandara was with his family.' (\textit{Paumacariu} 1.2.2.8)

\ex\label{ex:minvara}\gll \textbf{muṇivara} ghall-es-ai rajjesar-u \\
sage.\textsc{nom.m.pl} drive.out-\textsc{fut-3.sg}  king-\textsc{nom.sg} \\
\glt `The king will drive out the sages.' (\textit{Paumacariu} 2.35.9.1)
\end{xlist}
\end{exe}

%{Pc_15,9.10} etthu silAsaNE  $  attAvaNE  $  acchiu vAli-bhaDAra?
%jasu paYa-bhArENa  $  garuYArENa  $  hau~ kiu kummAYAra?"
% Pc_46,12.2} vaDDAra? kiu uvaYAru teNa  $  mAriu mAYAsuggIu jeNa
%{Pc_46,12.3} ko sakkai tahO pesaNu karevi  $  milu rAmahO maccharu pariharevi






\subsection{The emergence of DOM} 
The key suggestion I make here is that the Indo-Aryan \isi{differential object marking} pattern emerging between late MIA and Early NIA amounts to the generalizing \isi{reanalysis} of  syncretic \textsc{dat/gen} marking on non-subject non-theme \isi{arguments}  as accusative marking on (a privileged class of) objects. The data that provide evidence to enable such a \isi{reanalysis} are clauses containing  double object and other ditransitive verbs which either have implicit (non-overt) theme \isi{arguments} or where the \isi{arguments} (in the case of verbs of speech) are propositional. Such clauses are not very frequent but they do occur  quite reliably in MIA. Examples of non-\isi{ergative} clauses are given  in  \cref{ex:wxy} and \isi{ergative} clauses are in \cref{ex:zab}. 


\begin{exe}
\ex\label{ex:wxy}
\begin{xlist}
\ex\label{ex:akkhai}\gll akkh-ai sīya \textbf{samīraṇa-putt-aho} \\ tell-\textsc{impf.3.sg} Sita.\textsc{nom.sg} Samīraṇaputta-\textsc{dat/gen.sg} \\
\glt `Sita told Samīraṇa-putta (this).' (\textit{Paumacariu} 3.50.10.7)

\ex\label{ex:kahai}\gll kahai mahārisi gayaṇa-gai  \\
say-\textsc{impf.3.sg} great.sage.\textsc{nom.m.sg} sky.traveling.\textsc{nom.m.sg} \\

\gll \textbf{taho} \textbf{lavaṇ-aho} samar-e \textbf{samatth-aho} \\ 
that.\textsc{dat/gen.sg} Lavaṇa-\textsc{dat/gen.sg} battle-\textsc{loc.m.sg} capable-\textsc{dat/gen.sg} \\
\glt `The great sage said to that Lavaṇa, who was capable in battle (thus).' (\textit{Paumacariu} 5.82.8.9) 
\end{xlist}

\ex\label{ex:zab}
\begin{xlist}
\ex\gll \textbf{aṭṭhāvaya-giri-kampāvaṇ-aho} paḍihār-eṃ  akkh-iu \textbf{rāvaṇ-aho} \\
eight.regions.trembling-\textsc{dat/gen.sg} messenger-\textsc{erg.sg} tell-\textsc{perf.m.sg} rāvaṇa-\textsc{dat/gen.sg}  \\
\glt `The messenger told (this) to Ravana, who was capable of causing the eight territories (\textit{aṣṭapada}) to tremble.' (\textit{Paumacariu} 1.15.4.1)

\ex\gll to \textbf{pamiṇipura-paramesar-aho}    daris-āv-iya\\
then pamiṇipura-lord-\textsc{dat/gen.sg} see-\textsc{caus-perf.m.pl}\\

\gll \textbf{vijaya-mahīhar-aho} \\
 vijaya-king-\textsc{dat/gen.sg} \\
\glt `They showed (the boys) to the lord of Pamiṇipura (Padminipura), the king Vijayaparvata.' (\textit{Paumacariu} 2.33.2.1)

\ex\gll \textbf{añjaṇ-ahe} samapp-iu jāya dih-i \\ Añjanā-\textsc{dat/gen.sg} hand-\textsc{perf.m.sg} birth day-\textsc{loc.sg} \\
\glt `They handed him (the baby Hanumān) to Añjanā on the day of his birth.' (\textit{Paumacariu} 1.19.11.6)
\end{xlist}
\end{exe}


In clauses such as those  in \cref{ex:wxy} and \cref{ex:zab},  the only overt non-subject argument\is{arguments} carries \textsc{dat/gen} marking. Moreover, this pattern of marking does not differentiate between whether the subject carries \isi{ergative} marking or is unmarked (nominative). 

Consider a learner that must   arrive upon  the \isi{case} inventory of a language based on the observable input.\is{input} The MIA system provides reliably present morphological\is{morphology} evidence   for nominative,  \isi{ergative}, and dative/genitive \isi{case} but no reliable evidence for accusative case.\is{case} It also provides robust data in which the only non-subject argument\is{arguments} overtly expressed in a clause carries \isi{case} marking (the syncretic \textsc{dat/gen} marking). It is possible that the learner takes this subset of data as evidence for extending the \textsc{dat/gen} marking, reserved for  non-theme \isi{arguments}, to theme and patient \isi{arguments} as well. The Differential Object Marking pattern evidenced in Early NIA emerges because  the \isi{analogical} extension of the overt \textsc{dat/gen}  marking is constrained by the semantic properties associated with the original class of \isi{arguments} marked by it -- animacy and referentiality.
If this hypothesis is correct, then we expect that there may be early data supporting this extension of \textsc{dat/gen} \isi{case} marking to direct objects -- in effect, the \isi{reanalysis} of dative marking  as accusative case,\is{case} restricted to \isi{arguments} meeting the criteria of high animacy and referentiality.


In the previous subsection, it was claimed that    as far as the MIA stage is concerned,  direct \isi{arguments} of canonical transitive\is{transitivity} verbs do not, \textit{for the most part}, surface with  \textsc{dat/gen} marking (examples in \cref{nompatients}). The caveat was provided precisely because the MIA stage itself seems to exhibit some  data which is possibly analyzable as emergent DOM. The tentativeness with which this claim can be made emerges from three uncertainties about the data: (a) Although the lexical verbs appearing with the \textsc{dat/gen} marked objects arguably have an \isi{argument structure} \is{arguments}corresponding to transitive\is{transitivity} verbs and their translational equivalents in \ili{English} are realized as  canonical transitives, given the \isi{semantics} of these verbs,  it seems possible that they pattern either with ditransitives or with ``reciprocal" verbs" or with intransitives having accusative goal \isi{arguments} in \ili{Sanskrit}. Thus, it is necessary to investigate more closely whether these cases are early DOM-instances or whether they should be reclassified as exhibiting previously occurring patterns (b) The object-marking pattern   is very infrequent outside of the class of double-object verbs, other ditransitives, and ``reciprocal verbs". (c) There is absolutely no example of perfective clauses with \isi{ergative} subjects in which the object appears with \textsc{dat/gen} marking. 

It is possible therefore that the human-denoting \textsc{dat/gen} marked NPs in the data below are not the theme/patient  \isi{arguments} in a standard transitive\is{transitivity} template\is{templates} as they appear to be; they may be  better analyzed as recipient or goal arguments.\is{arguments} I will leave the adjudication of this issue for further research. But regardless of their status, they provide further surface evidence to the language acquirer for an object marking \isi{case} ``accusative'' in the language.



In \cref{ex:cde} and \cref{ex:fgh}, we see that the human-denoting non-subject \isi{arguments} of the transitive\is{transitivity} verbs \textit{khama} `forgive', \textit{pekkha} `look at', \textit{garaha} `denounce, curse', \textit{abhiṭṭa} `attack',   \textit{ḍhukka} `approach' and \textit{bhiḍ} `battle'  appear with \textsc{dat/gen} marking. The examples in \cref{ex:cde} contain non-perfective  clauses (\cref{ex:sundari} is an imperative) while those in \cref{ex:fgh} illustrate the argument\is{arguments} realization pattern in perfective clauses.


\begin{exe}
\ex\label{ex:cde}
\begin{xlist}
\ex\gll ekkavāra \textbf{ma-hu} khama-hi bhaḍār-ā \\
one.time I-\textsc{dat/gen.sg} forgive-\textsc{imp.2.g} warrior-\textsc{voc.sg} \\
\glt `O warrior (Lakshmana), please forgive me one time' (\textit{Paumacariu} 3.44.4.7)

\ex\label{ex:sundari}\gll sundari pekkhu pekkhu \textbf{jujjh-ant-aho} \\
beautiful.one.\textsc{voc.sg} see.\textsc{imp.2.sg} see.\textsc{imp.2.sg} fight-\textsc{part.dat/gen.sg} \\
\glt `O beautiful one, look at the battle.' (\textit{Paumacariu} 2.31.12.3)

\ex\gll ema jāma garah-anti \textbf{jiṇind-aho}  āsaṇu \\
thus when denounce-\textsc{impf.3.pl} jiṇinda-\textsc{dat/gen.sg} seat.\textsc{nom.m.sg} \\

\gll cal-iu tāma dharaṇind-aho \\
shake-\textsc{perf.m.sg} then dharaṇinda-\textsc{dat/gen.sg} \\
\glt `When they were denouncing Jiṇinda thus, the seat of Dharaṇinda started to shake.' (\textit{Paumacariu} 1.2.14.5)

\ex\gll ham abbhiṭṭ-ami \textbf{dūsaṇ-aho} \\
I.\textsc{nom.sg} attack-\textsc{impf.1.sg} Dūsaṇa-\textsc{dat/gen.sg} \\
\glt `I will attack Dūsaṇa' (\textit{Paumacariu} 2.40.4.10)
\end{xlist}


\ex\label{ex:fgh}
\begin{xlist}
\ex\gll dhā-iu aṅkusu lakkhaṇ-aho    abbhi-ṭṭu lavaṇu raṇ-e \textbf{rām-aho} \\
run-\textsc{perf.m.sg} aṅkusu.\textsc{nom.m.sg} lakṣmaṇa-\textsc{dat/gen.sg} attack-\textsc{perf.m.sg} lavaṇa.\textsc{nom.m.sg} battlefield-\textsc{loc.sg} rāma-\textsc{dat/gen.sg} \\
\glt `Aṅkuṣa ran to Lakshmaṇa (while) Lavaṇa attacked Rāma' (\textit{Paumacariu} 5.82.14.13)

\ex\gll kattha vi \textbf{bhaḍ-aho} sivaṅgaṇa \\
some.place \textsc{ptcl} warrior-\textsc{dat/gen.sg} she-jackal-group.\textsc{nom.m.pl} \\

\gll ḍhukk-iya \\
approach-\textsc{perf.m.pl} \\
\glt `At some places (on the battlefield), she-jackals approached the (dead) warriors.' (\textit{Paumacariu} 1.17.13.8)

\ex\gll indai bhiḍ-iu samar-e \\
battle-\textsc{perf.m.sg} battlefield-\textsc{loc.sg} \\

\gll \textbf{haṇuvant-aho}\\ Indai.\textsc{nom.m.sg} 
haṇuvant-\textsc{dat/gen.sg} \\
\glt `Indai (Indrajit)  battled with Haṇuvanta in the battlefield.'  (\textit{Paumacariu} 3.53.10.9)
\end{xlist}
\end{exe}



    \subsection{The DOM pattern in Early New Indo-Aryan} 
Turning to the Early New Indo-Aryan stage (illustrated here with Old Marathi), we see a clearly established  animacy- and referentiality-sensitive DOM pattern in both \isi{ergative} and non-\isi{ergative} clauses from the earliest period.\footnote{This  period is represented here  by two texts --  Līḷācharitra (ca. 1286 CE, prose) and  the Dnyāneśvarī (ca. 1287 CE, verse).
} The syncretic \textsc{dat/gen} marking of MIA appears as a generalized oblique \isi{case} and it is augmented with innovated\is{innovation} postpositions that correspond to accusative and dative \isi{case} markers. This trajectory, in which the MIA \isi{case}-system reduces to a nominative/oblique contrast and new postpositions are innovated\is{innovation} to convey the semantic and structural information  associated with the older cases,  is shared across Indo-Aryan languages (\citealt{masica91,bubenik96,bubenik98} a.o). 

Direct objects in Old Marathi surface with an innovated\is{innovation} postpositional accusative \isi{clitic},  \textit{-teṃ}, attached to the oblique stem (the reflex of the MIA \textsc{dat/gen} marker).   The examples selected for presentation here contain transitive\is{transitivity} verbs whose animate-denoting theme \isi{arguments} in both \isi{ergative} and non-\isi{ergative} clauses appear with   overt accusative marking in \crefrange{take}{dhar}. 

\begin{exe}
\ex\label{take}
\begin{xlist}
\ex\gll āmhīṃ \textbf{tuma=teṃ} ne-unuṃ \\
I.\textsc{nom.pl} you.\textsc{pl-acc} take-\textsc{fut.1.pl} \\ 
\glt `We will take you (to Varanasi).'   (\textit{Līḷācaritra} 1.25)

\ex\gll aiseṃ  mhaṇ-auni  \textbf{yā=teṃ}  śrīkarī-ṃ  dhar-ūni  āpuleyā  \\
thus speak-\textsc{ger} this.\textsc{obl=acc} hand-\textsc{ins.sg} hold-\textsc{ger} self.\textsc{obi} \\

\gll gharā=si  ne-leṃ \\
house.\textsc{obl=dat} take-\textsc{perf.n.sg} \\
\glt `Having spoken thus, taking him by the hand, she took him to her house.' (\textit{Līḷācaritra} 1.34) 
\end{xlist}



 

\ex\label{dekh}
\begin{xlist}
	\ex\gll mhaṇoni prakāśā=ce=ni=hi dehabaḷ-eṃ na dekh-atī \\
therefore light.\textsc{obl=of=by=ptcl} strength-\textsc{ins.sg} \textsc{neg} see-\textsc{impf.3.pl} \\

\gll \textbf{mā=teṃ} \\
I.\textsc{obl=acc} \\
\glt `Therefore, even by the strength of light, they do not see me.' (\textit{Dnyāneśvarī} 7.25.158)

\ex\gll tehīṃ  \textbf{yāṃ=teṃ}  uparīye-varauni  dekh-ileṃ \\ he-\textsc{erg.sg} this.\textsc{m.sg-acc}      upper.storey.\textsc{obl}-from.top see-\textsc{perf.n.sg} \\
\glt `He saw this one from the upper story (of the house).' (\textit{Līḷācaritra } 1.6)
\end{xlist}

\ex\label{dhar}
\begin{xlist}
\ex\gll āṇi te  \textbf{āma=teṃ } dhari-tī \\
And they.\textsc{nom.pl} we-\textsc{acc} catch-\textsc{impf.3.pl} \\ \glt `And they (honorific) would catch us.' (\textit{Līḷācaritra} 1.18)

\ex\gll ekī-ṃ ākāś-īṃ \textbf{sūryā=teṃ} dhar-ileṃ \\ one-\textsc{erg.sg} sky-\textsc{loc.sg} sun.\textsc{obl=acc} catch-\textsc{perf.n.sg} \\
\glt `Someone (might) catch  the sun in the sky.'  (\textit{Dnyāneśvarī} 10.0.37)
\end{xlist}
\end{exe}
 
 The examples in \cref{referential} contain the same non-animate denoting but referential argument\is{arguments} \textit{jaga} `world'  that also receives accusative marking in both \isi{imperfective} and perfective, \isi{ergative} clauses (\cref{ex:maga} and \cref{ex:prabam} respectively). 
 
\begin{exe}
	\ex\label{referential} 
	\begin{xlist}
	\ex\label{ex:maga}\gll maga āpu-leṃ keleṃ phokār-itī āṇī  \textbf{jagā=teṃ} dhikkār-itī  \\
	then self-\textsc{gen.n.sg} deed.\textsc{nom.n.sg} proclaim-\textsc{impf.3.pl} and world-\textsc{acc} denounce-\textsc{impf.3.pl} \\
	\glt `Then they proclaim their own deeds and denounce the world.' (\textit{Dnyāneśvarī} 16.10.328)
	
	\ex\label{ex:prabam}\gll prabaṃdhavyāj-eṃ \textbf{jagā=teṃ}
rakṣ-ileṃ jāṇa \\
literary.work.\textsc{inst.sg} world-\textsc{acc} save-\textsc{perf.3.n.sg} know.\textsc{imp.2.sg} \\
\glt `Know that (the Guru) has saved the world through this literary work.'  (\textit{Dnyāneśvarī} 18.78.1765)
\end{xlist}
\end{exe}

 It is necessary to take a much closer  look at the pattern of DOM seen in Old Marathi  languages and compare it on a verb-by-verb and argument-type by argument-type\is{arguments} basis with the MIA pattern. It is only such an investigation that can accurately establish the nuanced differences   between the impoverished accusative marking of Late MIA and the innovated accusative marking of  Old Marathi. Noteworthy is the fact that no reflexes of the MIA accusative marking survive in  the pronominal system  of Old Marathi; only traces of the syncretized \textsc{gen/dat} marking remain.
 

 
 \section{Conclusion}
At first glance, the presence of accusative marking (DOM)  in NIA \isi{ergative} clauses could  be considered to be a case in  which  an existing template\is{template} from the \isi{imperfective} domain is extended by \isi{analogy} to the perfective \isi{ergative} domain. However, a closer study of the \isi{case}-marking patterns of Late MIA reveals that there is no evidence for any direct  extension of the MIA accusative marking to \isi{ergative} clauses. It is more likely the case that the DOM pattern emerges in NIA languages as a \isi{reanalysis} of the MIA \textsc{dat/gen} marking that appears systematically  on a specific  subset of  non-subject \isi{arguments} into a marker of  accusative case. This reanalyzed accusative case is attested in  both \isi{ergative} and non-\isi{ergative} clauses in the earliest texts of Old Marathi,  supporting the hypothesis that accusative marking in \isi{ergative} clauses  is not part of any ``de-ergativization" trajectory 
 in the history of Indo-Aryan  but rather an emergent effect of across-the-board changes in argument\is{arguments} realization options for the languages. 
 



\nocite{anderson77, moravcsik1978, condoravdideo14, deo2012, coonpreminger, dixon79, dixon94, plank79ed, delancey1981, tsunoda1985, andersen86, peterson98, pischel00, bubenik98, bubenik96, bloch65, vale48, masica91, declercq2010, comrie78, anderson2004}



\section*{Abbreviations}
Glosses are as follows.  ``-'' stands  for a   morpheme boundary, ``='' for a \isi{clitic} boundary. 
\begin{multicols}{2}
 \begin{tabbing}
 \textsc{impfct}~~ \= imperfective (Old Indo-Aryan Present) \kill
 \textsc{abl} \> ablative\\
 \textsc{acc} \> accusative\\
 \textsc{aor} \> aorist\\
 \textsc{dat} \> dative\\
 \textsc{erg} \> \isi{ergative}\\
 \textsc{f} \> \isi{feminine}\\
 \textsc{fut} \> future\\
 \textsc{gen}  \> genitive\\
 \textsc{ger} \> \isi{gerund}\\
 \textsc{imp} \> imperative\\
 \textsc{impf} \>  \isi{imperfective}\\
	      \>  (Old Indo-Aryan Present)\\
 \textsc{impfct} \> Old Indo-Aryan Imperfect\\
 \textsc{inf} \> infinitive\\
 \textsc{inst} \> instrumental\\
 \textsc{loc} \> locative\\
 \textsc{m} \> masculine\\
 \textsc{n} \> neuter\\
 \textsc{neg}  \> negation\\
 \textsc{nom} \> nominative\\
 \textsc{pass} \> passive\\
 \textsc{perf}  \> perfective\\
 \textsc{pfct} \> perfect\\
 \textsc{pl} \> \isi{plural}\\
 \textsc{ptcl} \> discourse  particle\\
 \textsc{ptcpl} \> participle\\
 \textsc{prog} \> progressive\\
 \textsc{pv} \> verb particle\\
 \textsc{sg} \> singular\\
 \textsc{syncr} \> syncretic (\textsc{nom/inst})\\
 \textsc{voc} \> vocative\\
 \end{tabbing}
\end{multicols}

\section*{Acknowledgements}
Support from the National Science Foundation (NSF BCS-1255547/BCS-1660959.) is gratefully acknowledged. 

\printbibliography[heading=subbibliography,notkeyword=this]



\end{document}


 
