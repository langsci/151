% !TEX encoding = UTF-8 Unicode

\documentclass[output=paper,
modfonts
]{LSP/langsci}


%% add all extra packages you need to load to this file 
% \usepackage{todo} %% removed,cna use todonotes instead. % Jason reactivated
% \usepackage{graphicx} % not needed because forest loads tikz, which loads graphicx
\usepackage{tabularx}
\usepackage{amsmath} 
\usepackage{multicol}
\usepackage{lipsum}
\usepackage{longtable}
\usepackage{booktabs}
\usepackage[normalem]{ulem}
%\usepackage{tikz} % not needed because forest loads tikz
\usepackage{phonrule} % for SPE-style phonological rules
\usepackage{pst-all} % loads the main pstricks tools; for arrow diagrams in Hale.tex
%\usepackage{leipzig} % for gloss abbreviations
\usepackage[% for automatic cross-referencing
compress,%
capitalize,% labels are always capitalized in LSP style
noabbrev]% labels are always spelled out in LSP style
{cleveref}

% based on http://tex.stackexchange.com/a/318983/42880 for using gb4e examples with cleveref
\crefname{xnumi}{}{}
\creflabelformat{xnumi}{(#2#1#3)}
\crefrangeformat{xnumi}{(#3#1#4)--(#5#2#6)}
\crefname{xnumii}{}{}
\creflabelformat{xnumii}{(#2#1#3)}
\crefrangeformat{xnumii}{(#3#1#4)--(#5#2#6)}

%\usepackage[notcite,notref]{showkeys} %%removed, not helping CB.
%\usepackage{showidx} %%remove for final compiling - shows index keys at top of page.
 
\usepackage{langsci/styles/langsci-gb4e}  
 \usepackage{pifont}
% % OT tableaux                                                
% \usepackage{pstricks,colortab}  
\usepackage{multirow} % used in OT tableaux
\usepackage{rotating} %needed for angled text%
\usepackage{colortbl} % for cell shading
 
 \usepackage{avm}  
\usepackage[linguistics]{forest} 
\usetikzlibrary{matrix,fit} % for matrix of nodes in Kaisse and Bat-El


\usepackage{hhline}
\newcommand{\cgr}{\cellcolor[gray]{0.8}}
\newcommand{\cn}{\centering}



\newcommand{\reff}[1]{(\ref{#1})}
%\usepackage{newtxtext,newtxmath}


%\usepackage[normalem] {ulem}
\usepackage{qtree}
%\usepackage{natbib}
%\usepackage{tikz}
%\usepackage{gb4e}
\usepackage{phonrule}  
%\bibliographystyle{humannat}



\usepackage{minibox}

%\include{psheader-metr}

\def\bl#1{$_{\textrm{{\footnotesize #1}}}$}

%%add all your local new commands to this file

\newcommand{\form}[1]{\mbox{\emph{#1}}}
\newcommand{\uf}[1]{\mbox{/#1/}}

% borrowed from expex and converted from plan tex to latex
\newcommand{\judge}[1]{{\upshape #1\hspace{0.1em}}}
\newcommand{\ljudge}[1]{\makebox[0pt][r]{\judge{#1}}}

\newcommand\tikzmark[1]{\tikz[remember picture, baseline=(#1.base)] \node[anchor=base,inner sep=0pt, outer sep=0pt] (#1) {#1};} % for adding decorations, arrows, lines, etc. to text
\newcommand\tikzmarknamed[2]{\tikz[remember picture, baseline=(#1.base)] \node[anchor=base,inner sep=0pt, outer sep=0pt] (#1) {#2};} % for adding decorations, arrows, lines, etc. to text
\newcommand\tikzmarkfullnamed[2]{\tikz[remember picture, baseline=(#1.base)] \node[anchor=base,inner sep=0pt, outer sep=0pt] (#1) {\vphantom{X}#2};} % for adding decorations, arrows, lines, etc. to text; this one works best for decorations above a line of text because it adds in the heigh of a capital letter and takes two arguments - one for the node name and one for the printed text

\newcommand{\sub}[1]{$_{\text{#1}}$} % for non-math subscripts
\newcommand{\subit}[1]{\sub{\textit{#1}}} % for italics non-math subscripts
\newcommand{\1}{\rlap{$'$}\xspace} % for the prime in X' (the \rlap command allows the prime to be ignored for horizontal spacing in trees, and the \xspace command allows you to use this in normal text without adding \ afterwards). This isn't crucial, but it helps the formatting to look a little better.

% Aissen:
\newcommand\tikzmarkfull[1]{\tikz[remember picture, baseline=(#1.base)] \node[anchor=base,inner sep=0pt, outer sep=0pt] (#1) {\vphantom{X}#1};} % for adding decorations, arrows, lines, etc. to text; this one works best for decorations above a line of text because it adds in the heigh of a capital letter and takes one argument that serves as the name and the printed text
\newcommand{\bridgeover}[2]{% for a line that creates a bridge over text, connecting two nodes
	\begin{tikzpicture}[remember picture,overlay]
	\draw[thick,shorten >=3pt,shorten <=3pt] (#1.north) |- +(0ex,2.5ex) -| (#2.north);
	\end{tikzpicture}
}
\newcommand{\bridgeoverht}[3]{% for a line that creates a bridge over text, connecting two nodes and specifing the height of the bridge
	\begin{tikzpicture}[remember picture,overlay]
	\draw[thick,shorten >=3pt,shorten <=3pt] (#2.north) |- +(0ex,#1) -| (#3.north);
	\end{tikzpicture}
}
\newcommand{\bridgeoverex}{\vspace*{3ex}} % place before an example that has a \bridgeover so that there is enough vertical space for it

% Chung:
\newcommand{\lefttabular}[1]{\begin{tabular}{p{0.5in}}#1\end{tabular}}

% Kaisse:
\newcommand{\mgmorph}[1]{|(#1)| {#1}}
\newcommand{\mgone}[2][$\times$]{\node at (#2.base) [above=2ex] (1#2) {\vphantom{X}#1};}
\newcommand{\mgtwo}[2][$\times$]{\mgone{#2} \node at (#2.base) [above=4.5ex] (2#2) {\vphantom{X}#1};}
\newcommand{\mgthree}[2][$\times$]{\mgtwo{#2} \node at (#2.base) [above=7ex] (3#2) {\vphantom{X}#1};}
\newcommand{\mgftl}[1]{\node at (1#1) [left] {(};}
\newcommand{\mgftr}[1]{\node at (1#1) [right] {)};}
\newcommand{\mgfoot}[2]{\mgftl{#1}\mgftr{#2}}
\newcommand{\mgldelim}[2]{\node at (#2.west) [left,inner sep = 0pt, outer sep = 0pt] {#1};}
\newcommand{\mgrdelim}[2]{\node at (#2.east) [right,inner sep = 0pt, outer sep = 0pt] {#1};}

\newcommand{\squish}{\hspace*{-3pt}}

% Kavitskaya:
\newcommand{\assoc}[2]{\draw (#1.south) -- (#2.north);}
\newcolumntype{L}{>{\raggedright\arraybackslash}X}

% Lepic & Padden:
\newcommand{\fitpic}[1]{\resizebox{\hsize}{!}{\includegraphics{#1}}} % from http://tex.stackexchange.com/a/148965/42880
\newcommand{\signpic}[1]{\includegraphics[width=1.36in]{#1}}
\newcolumntype{C}{>{\centering\arraybackslash}X}

% Spencer:

\newcommand{\textex}[1]{\textit{#1}\xspace}
\newcommand{\lxm}[1]{\textsc{#1}\xspace}

% Thrainsson:

\renewcommand{\textasciitilde}{\char`~} % for use with TTF/OTF fonts (see comments on http://tex.stackexchange.com/a/317/42880)
\newcommand{\tikzarrow}[2]{% for an arrow connecting two nodes
\begin{tikzpicture}[remember picture,overlay]
\draw[thick,shorten >=3pt,shorten <=3pt,->,>=stealth] (#1) -- (#2);
\end{tikzpicture}
}

\newlength{\padding}
\setlength{\padding}{0.5em}
\newcommand{\lesspadding}{\hspace*{-\padding}}
\newcommand{\feat}[1]{\lesspadding#1\lesspadding}

% Hammond

\usepackage[]{graphicx}\usepackage[]{xcolor}
%% maxwidth is the original width if it is less than linewidth
%% otherwise use linewidth (to make sure the graphics do not exceed the margin)
\makeatletter
\def\maxwidth{ %
  \ifdim\Gin@nat@width>\linewidth
    \linewidth
  \else
    \Gin@nat@width
  \fi
}
\makeatother

\definecolor{fgcolor}{rgb}{0.345, 0.345, 0.345}
\newcommand{\hlnum}[1]{\textcolor[rgb]{0.686,0.059,0.569}{#1}}%
\newcommand{\hlstr}[1]{\textcolor[rgb]{0.192,0.494,0.8}{#1}}%
\newcommand{\hlcom}[1]{\textcolor[rgb]{0.678,0.584,0.686}{\textit{#1}}}%
\newcommand{\hlopt}[1]{\textcolor[rgb]{0,0,0}{#1}}%
\newcommand{\hlstd}[1]{\textcolor[rgb]{0.345,0.345,0.345}{#1}}%
\newcommand{\hlkwa}[1]{\textcolor[rgb]{0.161,0.373,0.58}{\textbf{#1}}}%
\newcommand{\hlkwb}[1]{\textcolor[rgb]{0.69,0.353,0.396}{#1}}%
\newcommand{\hlkwc}[1]{\textcolor[rgb]{0.333,0.667,0.333}{#1}}%
\newcommand{\hlkwd}[1]{\textcolor[rgb]{0.737,0.353,0.396}{\textbf{#1}}}%
\let\hlipl\hlkwb

\usepackage{framed}
\makeatletter
\newenvironment{kframe}{%
 \def\at@end@of@kframe{}%
 \ifinner\ifhmode%
  \def\at@end@of@kframe{\end{minipage}}%
  \begin{minipage}{\columnwidth}%
 \fi\fi%
 \def\FrameCommand##1{\hskip\@totalleftmargin \hskip-\fboxsep
 \colorbox{shadecolor}{##1}\hskip-\fboxsep
     % There is no \\@totalrightmargin, so:
     \hskip-\linewidth \hskip-\@totalleftmargin \hskip\columnwidth}%
 \MakeFramed {\advance\hsize-\width
   \@totalleftmargin\z@ \linewidth\hsize
   \@setminipage}}%
 {\par\unskip\endMakeFramed%
 \at@end@of@kframe}
\makeatother

\definecolor{shadecolor}{rgb}{.97, .97, .97}
\definecolor{messagecolor}{rgb}{0, 0, 0}
\definecolor{warningcolor}{rgb}{1, 0, 1}
\definecolor{errorcolor}{rgb}{1, 0, 0}
\newenvironment{knitrout}{}{} % an empty environment to be redefined in TeX

\usepackage{alltt}

%revised version started: 12/17/16

%NEEDS: allbib.bib - already added to the master bibliography file.
%cited references only: bibexport -o mhTMP.bib main1-blx.aux
%PLUS sramh-img*, sramh.tex

%added stuff
\newcommand{\add}[1]{\textcolor{blue}{#1}}
%deleted stuff
\newcommand{\del}[1]{\textcolor{red}{(removed: #1)}}
%uncomment these to turn off colors
\renewcommand{\add}[1]{#1}
\renewcommand{\del}[1]{}

%shortcuts
\newcommand{\w}{\ili{Welsh}}
\newcommand{\e}{\ili{English}}
\newcommand{\io}{Input Optimization}




 \newcommand{\hand}{\ding{43}}
% \newcommand{\rot}[1]{\begin{rotate}{90}#1\end{rotate}} %shortcut for angled text%  
% \newcommand{\rotcon}[1]{\raisebox{-5ex}{\hspace*{1ex}\rot{\hspace*{1ex}#1}}}

%% add all extra packages you need to load to this file 
% \usepackage{todo} %% removed,cna use todonotes instead. % Jason reactivated
% \usepackage{graphicx} % not needed because forest loads tikz, which loads graphicx
\usepackage{tabularx}
\usepackage{amsmath} 
\usepackage{multicol}
\usepackage{lipsum}
\usepackage{longtable}
\usepackage{booktabs}
\usepackage[normalem]{ulem}
%\usepackage{tikz} % not needed because forest loads tikz
\usepackage{phonrule} % for SPE-style phonological rules
\usepackage{pst-all} % loads the main pstricks tools; for arrow diagrams in Hale.tex
%\usepackage{leipzig} % for gloss abbreviations
\usepackage[% for automatic cross-referencing
compress,%
capitalize,% labels are always capitalized in LSP style
noabbrev]% labels are always spelled out in LSP style
{cleveref}

% based on http://tex.stackexchange.com/a/318983/42880 for using gb4e examples with cleveref
\crefname{xnumi}{}{}
\creflabelformat{xnumi}{(#2#1#3)}
\crefrangeformat{xnumi}{(#3#1#4)--(#5#2#6)}
\crefname{xnumii}{}{}
\creflabelformat{xnumii}{(#2#1#3)}
\crefrangeformat{xnumii}{(#3#1#4)--(#5#2#6)}

%\usepackage[notcite,notref]{showkeys} %%removed, not helping CB.
%\usepackage{showidx} %%remove for final compiling - shows index keys at top of page.
 
\usepackage{langsci/styles/langsci-gb4e}  
 \usepackage{pifont}
% % OT tableaux                                                
% \usepackage{pstricks,colortab}  
\usepackage{multirow} % used in OT tableaux
\usepackage{rotating} %needed for angled text%
\usepackage{colortbl} % for cell shading
 
 \usepackage{avm}  
\usepackage[linguistics]{forest} 
\usetikzlibrary{matrix,fit} % for matrix of nodes in Kaisse and Bat-El


\usepackage{hhline}
\newcommand{\cgr}{\cellcolor[gray]{0.8}}
\newcommand{\cn}{\centering}



\newcommand{\reff}[1]{(\ref{#1})}
%\usepackage{newtxtext,newtxmath}


%\usepackage[normalem] {ulem}
\usepackage{qtree}
%\usepackage{natbib}
%\usepackage{tikz}
%\usepackage{gb4e}
\usepackage{phonrule}  
%\bibliographystyle{humannat}



\usepackage{minibox}

%\include{psheader-metr}

\def\bl#1{$_{\textrm{{\footnotesize #1}}}$}
\usepackage{arydshln}
\usepackage{rotating}

%%add all your local new commands to this file

\newcommand{\form}[1]{\mbox{\emph{#1}}}
\newcommand{\uf}[1]{\mbox{/#1/}}

% borrowed from expex and converted from plan tex to latex
\newcommand{\judge}[1]{{\upshape #1\hspace{0.1em}}}
\newcommand{\ljudge}[1]{\makebox[0pt][r]{\judge{#1}}}

\newcommand\tikzmark[1]{\tikz[remember picture, baseline=(#1.base)] \node[anchor=base,inner sep=0pt, outer sep=0pt] (#1) {#1};} % for adding decorations, arrows, lines, etc. to text
\newcommand\tikzmarknamed[2]{\tikz[remember picture, baseline=(#1.base)] \node[anchor=base,inner sep=0pt, outer sep=0pt] (#1) {#2};} % for adding decorations, arrows, lines, etc. to text
\newcommand\tikzmarkfullnamed[2]{\tikz[remember picture, baseline=(#1.base)] \node[anchor=base,inner sep=0pt, outer sep=0pt] (#1) {\vphantom{X}#2};} % for adding decorations, arrows, lines, etc. to text; this one works best for decorations above a line of text because it adds in the heigh of a capital letter and takes two arguments - one for the node name and one for the printed text

\newcommand{\sub}[1]{$_{\text{#1}}$} % for non-math subscripts
\newcommand{\subit}[1]{\sub{\textit{#1}}} % for italics non-math subscripts
\newcommand{\1}{\rlap{$'$}\xspace} % for the prime in X' (the \rlap command allows the prime to be ignored for horizontal spacing in trees, and the \xspace command allows you to use this in normal text without adding \ afterwards). This isn't crucial, but it helps the formatting to look a little better.

% Aissen:
\newcommand\tikzmarkfull[1]{\tikz[remember picture, baseline=(#1.base)] \node[anchor=base,inner sep=0pt, outer sep=0pt] (#1) {\vphantom{X}#1};} % for adding decorations, arrows, lines, etc. to text; this one works best for decorations above a line of text because it adds in the heigh of a capital letter and takes one argument that serves as the name and the printed text
\newcommand{\bridgeover}[2]{% for a line that creates a bridge over text, connecting two nodes
	\begin{tikzpicture}[remember picture,overlay]
	\draw[thick,shorten >=3pt,shorten <=3pt] (#1.north) |- +(0ex,2.5ex) -| (#2.north);
	\end{tikzpicture}
}
\newcommand{\bridgeoverht}[3]{% for a line that creates a bridge over text, connecting two nodes and specifing the height of the bridge
	\begin{tikzpicture}[remember picture,overlay]
	\draw[thick,shorten >=3pt,shorten <=3pt] (#2.north) |- +(0ex,#1) -| (#3.north);
	\end{tikzpicture}
}
\newcommand{\bridgeoverex}{\vspace*{3ex}} % place before an example that has a \bridgeover so that there is enough vertical space for it

% Chung:
\newcommand{\lefttabular}[1]{\begin{tabular}{p{0.5in}}#1\end{tabular}}

% Kaisse:
\newcommand{\mgmorph}[1]{|(#1)| {#1}}
\newcommand{\mgone}[2][$\times$]{\node at (#2.base) [above=2ex] (1#2) {\vphantom{X}#1};}
\newcommand{\mgtwo}[2][$\times$]{\mgone{#2} \node at (#2.base) [above=4.5ex] (2#2) {\vphantom{X}#1};}
\newcommand{\mgthree}[2][$\times$]{\mgtwo{#2} \node at (#2.base) [above=7ex] (3#2) {\vphantom{X}#1};}
\newcommand{\mgftl}[1]{\node at (1#1) [left] {(};}
\newcommand{\mgftr}[1]{\node at (1#1) [right] {)};}
\newcommand{\mgfoot}[2]{\mgftl{#1}\mgftr{#2}}
\newcommand{\mgldelim}[2]{\node at (#2.west) [left,inner sep = 0pt, outer sep = 0pt] {#1};}
\newcommand{\mgrdelim}[2]{\node at (#2.east) [right,inner sep = 0pt, outer sep = 0pt] {#1};}

\newcommand{\squish}{\hspace*{-3pt}}

% Kavitskaya:
\newcommand{\assoc}[2]{\draw (#1.south) -- (#2.north);}
\newcolumntype{L}{>{\raggedright\arraybackslash}X}

% Lepic & Padden:
\newcommand{\fitpic}[1]{\resizebox{\hsize}{!}{\includegraphics{#1}}} % from http://tex.stackexchange.com/a/148965/42880
\newcommand{\signpic}[1]{\includegraphics[width=1.36in]{#1}}
\newcolumntype{C}{>{\centering\arraybackslash}X}

% Spencer:

\newcommand{\textex}[1]{\textit{#1}\xspace}
\newcommand{\lxm}[1]{\textsc{#1}\xspace}

% Thrainsson:

\renewcommand{\textasciitilde}{\char`~} % for use with TTF/OTF fonts (see comments on http://tex.stackexchange.com/a/317/42880)
\newcommand{\tikzarrow}[2]{% for an arrow connecting two nodes
\begin{tikzpicture}[remember picture,overlay]
\draw[thick,shorten >=3pt,shorten <=3pt,->,>=stealth] (#1) -- (#2);
\end{tikzpicture}
}

\newlength{\padding}
\setlength{\padding}{0.5em}
\newcommand{\lesspadding}{\hspace*{-\padding}}
\newcommand{\feat}[1]{\lesspadding#1\lesspadding}

% Hammond

\usepackage[]{graphicx}\usepackage[]{xcolor}
%% maxwidth is the original width if it is less than linewidth
%% otherwise use linewidth (to make sure the graphics do not exceed the margin)
\makeatletter
\def\maxwidth{ %
  \ifdim\Gin@nat@width>\linewidth
    \linewidth
  \else
    \Gin@nat@width
  \fi
}
\makeatother

\definecolor{fgcolor}{rgb}{0.345, 0.345, 0.345}
\newcommand{\hlnum}[1]{\textcolor[rgb]{0.686,0.059,0.569}{#1}}%
\newcommand{\hlstr}[1]{\textcolor[rgb]{0.192,0.494,0.8}{#1}}%
\newcommand{\hlcom}[1]{\textcolor[rgb]{0.678,0.584,0.686}{\textit{#1}}}%
\newcommand{\hlopt}[1]{\textcolor[rgb]{0,0,0}{#1}}%
\newcommand{\hlstd}[1]{\textcolor[rgb]{0.345,0.345,0.345}{#1}}%
\newcommand{\hlkwa}[1]{\textcolor[rgb]{0.161,0.373,0.58}{\textbf{#1}}}%
\newcommand{\hlkwb}[1]{\textcolor[rgb]{0.69,0.353,0.396}{#1}}%
\newcommand{\hlkwc}[1]{\textcolor[rgb]{0.333,0.667,0.333}{#1}}%
\newcommand{\hlkwd}[1]{\textcolor[rgb]{0.737,0.353,0.396}{\textbf{#1}}}%
\let\hlipl\hlkwb

\usepackage{framed}
\makeatletter
\newenvironment{kframe}{%
 \def\at@end@of@kframe{}%
 \ifinner\ifhmode%
  \def\at@end@of@kframe{\end{minipage}}%
  \begin{minipage}{\columnwidth}%
 \fi\fi%
 \def\FrameCommand##1{\hskip\@totalleftmargin \hskip-\fboxsep
 \colorbox{shadecolor}{##1}\hskip-\fboxsep
     % There is no \\@totalrightmargin, so:
     \hskip-\linewidth \hskip-\@totalleftmargin \hskip\columnwidth}%
 \MakeFramed {\advance\hsize-\width
   \@totalleftmargin\z@ \linewidth\hsize
   \@setminipage}}%
 {\par\unskip\endMakeFramed%
 \at@end@of@kframe}
\makeatother

\definecolor{shadecolor}{rgb}{.97, .97, .97}
\definecolor{messagecolor}{rgb}{0, 0, 0}
\definecolor{warningcolor}{rgb}{1, 0, 1}
\definecolor{errorcolor}{rgb}{1, 0, 0}
\newenvironment{knitrout}{}{} % an empty environment to be redefined in TeX

\usepackage{alltt}

%revised version started: 12/17/16

%NEEDS: allbib.bib - already added to the master bibliography file.
%cited references only: bibexport -o mhTMP.bib main1-blx.aux
%PLUS sramh-img*, sramh.tex

%added stuff
\newcommand{\add}[1]{\textcolor{blue}{#1}}
%deleted stuff
\newcommand{\del}[1]{\textcolor{red}{(removed: #1)}}
%uncomment these to turn off colors
\renewcommand{\add}[1]{#1}
\renewcommand{\del}[1]{}

%shortcuts
\newcommand{\w}{\ili{Welsh}}
\newcommand{\e}{\ili{English}}
\newcommand{\io}{Input Optimization}




 \newcommand{\hand}{\ding{43}}
% \newcommand{\rot}[1]{\begin{rotate}{90}#1\end{rotate}} %shortcut for angled text%  
% \newcommand{\rotcon}[1]{\raisebox{-5ex}{\hspace*{1ex}\rot{\hspace*{1ex}#1}}}

%% add all extra packages you need to load to this file 
% \usepackage{todo} %% removed,cna use todonotes instead. % Jason reactivated
% \usepackage{graphicx} % not needed because forest loads tikz, which loads graphicx
\usepackage{tabularx}
\usepackage{amsmath} 
\usepackage{multicol}
\usepackage{lipsum}
\usepackage{longtable}
\usepackage{booktabs}
\usepackage[normalem]{ulem}
%\usepackage{tikz} % not needed because forest loads tikz
\usepackage{phonrule} % for SPE-style phonological rules
\usepackage{pst-all} % loads the main pstricks tools; for arrow diagrams in Hale.tex
%\usepackage{leipzig} % for gloss abbreviations
\usepackage[% for automatic cross-referencing
compress,%
capitalize,% labels are always capitalized in LSP style
noabbrev]% labels are always spelled out in LSP style
{cleveref}

% based on http://tex.stackexchange.com/a/318983/42880 for using gb4e examples with cleveref
\crefname{xnumi}{}{}
\creflabelformat{xnumi}{(#2#1#3)}
\crefrangeformat{xnumi}{(#3#1#4)--(#5#2#6)}
\crefname{xnumii}{}{}
\creflabelformat{xnumii}{(#2#1#3)}
\crefrangeformat{xnumii}{(#3#1#4)--(#5#2#6)}

%\usepackage[notcite,notref]{showkeys} %%removed, not helping CB.
%\usepackage{showidx} %%remove for final compiling - shows index keys at top of page.
 
\usepackage{langsci/styles/langsci-gb4e}  
 \usepackage{pifont}
% % OT tableaux                                                
% \usepackage{pstricks,colortab}  
\usepackage{multirow} % used in OT tableaux
\usepackage{rotating} %needed for angled text%
\usepackage{colortbl} % for cell shading
 
 \usepackage{avm}  
\usepackage[linguistics]{forest} 
\usetikzlibrary{matrix,fit} % for matrix of nodes in Kaisse and Bat-El


\usepackage{hhline}
\newcommand{\cgr}{\cellcolor[gray]{0.8}}
\newcommand{\cn}{\centering}



\newcommand{\reff}[1]{(\ref{#1})}
%\usepackage{newtxtext,newtxmath}


%\usepackage[normalem] {ulem}
\usepackage{qtree}
%\usepackage{natbib}
%\usepackage{tikz}
%\usepackage{gb4e}
\usepackage{phonrule}  
%\bibliographystyle{humannat}



\usepackage{minibox}

%\include{psheader-metr}

\def\bl#1{$_{\textrm{{\footnotesize #1}}}$}
\usepackage{arydshln}
\usepackage{rotating}

%%add all your local new commands to this file

\newcommand{\form}[1]{\mbox{\emph{#1}}}
\newcommand{\uf}[1]{\mbox{/#1/}}

% borrowed from expex and converted from plan tex to latex
\newcommand{\judge}[1]{{\upshape #1\hspace{0.1em}}}
\newcommand{\ljudge}[1]{\makebox[0pt][r]{\judge{#1}}}

\newcommand\tikzmark[1]{\tikz[remember picture, baseline=(#1.base)] \node[anchor=base,inner sep=0pt, outer sep=0pt] (#1) {#1};} % for adding decorations, arrows, lines, etc. to text
\newcommand\tikzmarknamed[2]{\tikz[remember picture, baseline=(#1.base)] \node[anchor=base,inner sep=0pt, outer sep=0pt] (#1) {#2};} % for adding decorations, arrows, lines, etc. to text
\newcommand\tikzmarkfullnamed[2]{\tikz[remember picture, baseline=(#1.base)] \node[anchor=base,inner sep=0pt, outer sep=0pt] (#1) {\vphantom{X}#2};} % for adding decorations, arrows, lines, etc. to text; this one works best for decorations above a line of text because it adds in the heigh of a capital letter and takes two arguments - one for the node name and one for the printed text

\newcommand{\sub}[1]{$_{\text{#1}}$} % for non-math subscripts
\newcommand{\subit}[1]{\sub{\textit{#1}}} % for italics non-math subscripts
\newcommand{\1}{\rlap{$'$}\xspace} % for the prime in X' (the \rlap command allows the prime to be ignored for horizontal spacing in trees, and the \xspace command allows you to use this in normal text without adding \ afterwards). This isn't crucial, but it helps the formatting to look a little better.

% Aissen:
\newcommand\tikzmarkfull[1]{\tikz[remember picture, baseline=(#1.base)] \node[anchor=base,inner sep=0pt, outer sep=0pt] (#1) {\vphantom{X}#1};} % for adding decorations, arrows, lines, etc. to text; this one works best for decorations above a line of text because it adds in the heigh of a capital letter and takes one argument that serves as the name and the printed text
\newcommand{\bridgeover}[2]{% for a line that creates a bridge over text, connecting two nodes
	\begin{tikzpicture}[remember picture,overlay]
	\draw[thick,shorten >=3pt,shorten <=3pt] (#1.north) |- +(0ex,2.5ex) -| (#2.north);
	\end{tikzpicture}
}
\newcommand{\bridgeoverht}[3]{% for a line that creates a bridge over text, connecting two nodes and specifing the height of the bridge
	\begin{tikzpicture}[remember picture,overlay]
	\draw[thick,shorten >=3pt,shorten <=3pt] (#2.north) |- +(0ex,#1) -| (#3.north);
	\end{tikzpicture}
}
\newcommand{\bridgeoverex}{\vspace*{3ex}} % place before an example that has a \bridgeover so that there is enough vertical space for it

% Chung:
\newcommand{\lefttabular}[1]{\begin{tabular}{p{0.5in}}#1\end{tabular}}

% Kaisse:
\newcommand{\mgmorph}[1]{|(#1)| {#1}}
\newcommand{\mgone}[2][$\times$]{\node at (#2.base) [above=2ex] (1#2) {\vphantom{X}#1};}
\newcommand{\mgtwo}[2][$\times$]{\mgone{#2} \node at (#2.base) [above=4.5ex] (2#2) {\vphantom{X}#1};}
\newcommand{\mgthree}[2][$\times$]{\mgtwo{#2} \node at (#2.base) [above=7ex] (3#2) {\vphantom{X}#1};}
\newcommand{\mgftl}[1]{\node at (1#1) [left] {(};}
\newcommand{\mgftr}[1]{\node at (1#1) [right] {)};}
\newcommand{\mgfoot}[2]{\mgftl{#1}\mgftr{#2}}
\newcommand{\mgldelim}[2]{\node at (#2.west) [left,inner sep = 0pt, outer sep = 0pt] {#1};}
\newcommand{\mgrdelim}[2]{\node at (#2.east) [right,inner sep = 0pt, outer sep = 0pt] {#1};}

\newcommand{\squish}{\hspace*{-3pt}}

% Kavitskaya:
\newcommand{\assoc}[2]{\draw (#1.south) -- (#2.north);}
\newcolumntype{L}{>{\raggedright\arraybackslash}X}

% Lepic & Padden:
\newcommand{\fitpic}[1]{\resizebox{\hsize}{!}{\includegraphics{#1}}} % from http://tex.stackexchange.com/a/148965/42880
\newcommand{\signpic}[1]{\includegraphics[width=1.36in]{#1}}
\newcolumntype{C}{>{\centering\arraybackslash}X}

% Spencer:

\newcommand{\textex}[1]{\textit{#1}\xspace}
\newcommand{\lxm}[1]{\textsc{#1}\xspace}

% Thrainsson:

\renewcommand{\textasciitilde}{\char`~} % for use with TTF/OTF fonts (see comments on http://tex.stackexchange.com/a/317/42880)
\newcommand{\tikzarrow}[2]{% for an arrow connecting two nodes
\begin{tikzpicture}[remember picture,overlay]
\draw[thick,shorten >=3pt,shorten <=3pt,->,>=stealth] (#1) -- (#2);
\end{tikzpicture}
}

\newlength{\padding}
\setlength{\padding}{0.5em}
\newcommand{\lesspadding}{\hspace*{-\padding}}
\newcommand{\feat}[1]{\lesspadding#1\lesspadding}

% Hammond

\usepackage[]{graphicx}\usepackage[]{xcolor}
%% maxwidth is the original width if it is less than linewidth
%% otherwise use linewidth (to make sure the graphics do not exceed the margin)
\makeatletter
\def\maxwidth{ %
  \ifdim\Gin@nat@width>\linewidth
    \linewidth
  \else
    \Gin@nat@width
  \fi
}
\makeatother

\definecolor{fgcolor}{rgb}{0.345, 0.345, 0.345}
\newcommand{\hlnum}[1]{\textcolor[rgb]{0.686,0.059,0.569}{#1}}%
\newcommand{\hlstr}[1]{\textcolor[rgb]{0.192,0.494,0.8}{#1}}%
\newcommand{\hlcom}[1]{\textcolor[rgb]{0.678,0.584,0.686}{\textit{#1}}}%
\newcommand{\hlopt}[1]{\textcolor[rgb]{0,0,0}{#1}}%
\newcommand{\hlstd}[1]{\textcolor[rgb]{0.345,0.345,0.345}{#1}}%
\newcommand{\hlkwa}[1]{\textcolor[rgb]{0.161,0.373,0.58}{\textbf{#1}}}%
\newcommand{\hlkwb}[1]{\textcolor[rgb]{0.69,0.353,0.396}{#1}}%
\newcommand{\hlkwc}[1]{\textcolor[rgb]{0.333,0.667,0.333}{#1}}%
\newcommand{\hlkwd}[1]{\textcolor[rgb]{0.737,0.353,0.396}{\textbf{#1}}}%
\let\hlipl\hlkwb

\usepackage{framed}
\makeatletter
\newenvironment{kframe}{%
 \def\at@end@of@kframe{}%
 \ifinner\ifhmode%
  \def\at@end@of@kframe{\end{minipage}}%
  \begin{minipage}{\columnwidth}%
 \fi\fi%
 \def\FrameCommand##1{\hskip\@totalleftmargin \hskip-\fboxsep
 \colorbox{shadecolor}{##1}\hskip-\fboxsep
     % There is no \\@totalrightmargin, so:
     \hskip-\linewidth \hskip-\@totalleftmargin \hskip\columnwidth}%
 \MakeFramed {\advance\hsize-\width
   \@totalleftmargin\z@ \linewidth\hsize
   \@setminipage}}%
 {\par\unskip\endMakeFramed%
 \at@end@of@kframe}
\makeatother

\definecolor{shadecolor}{rgb}{.97, .97, .97}
\definecolor{messagecolor}{rgb}{0, 0, 0}
\definecolor{warningcolor}{rgb}{1, 0, 1}
\definecolor{errorcolor}{rgb}{1, 0, 0}
\newenvironment{knitrout}{}{} % an empty environment to be redefined in TeX

\usepackage{alltt}

%revised version started: 12/17/16

%NEEDS: allbib.bib - already added to the master bibliography file.
%cited references only: bibexport -o mhTMP.bib main1-blx.aux
%PLUS sramh-img*, sramh.tex

%added stuff
\newcommand{\add}[1]{\textcolor{blue}{#1}}
%deleted stuff
\newcommand{\del}[1]{\textcolor{red}{(removed: #1)}}
%uncomment these to turn off colors
\renewcommand{\add}[1]{#1}
\renewcommand{\del}[1]{}

%shortcuts
\newcommand{\w}{\ili{Welsh}}
\newcommand{\e}{\ili{English}}
\newcommand{\io}{Input Optimization}




 \newcommand{\hand}{\ding{43}}
% \newcommand{\rot}[1]{\begin{rotate}{90}#1\end{rotate}} %shortcut for angled text%  
% \newcommand{\rotcon}[1]{\raisebox{-5ex}{\hspace*{1ex}\rot{\hspace*{1ex}#1}}}

%\input{localpackages.tex}
\usepackage{arydshln}
\usepackage{rotating}

%\input{localcommands.tex}
\newcommand{\tworow}[1]{\multirow{2}{*}{#1}}


\newcommand{\tworow}[1]{\multirow{2}{*}{#1}}


\newcommand{\tworow}[1]{\multirow{2}{*}{#1}}



\title{Saussure's Dilemma: Parole and its potential}

\author{%
Alan Timberlake\affiliation{Columbia University}
}

\ChapterDOI{10.5281/zenodo.495466}
% \chapterDOI{} %will be filled in at production
% \epigram{}

\abstract{%
Saussure's account of the transformation of Latin to French stress leads to the unintended conclusion that \emph{parole} has a life of its own:  parole persists even after it is no longer dictated by \emph{langue}; \emph{parole} can prevent change or, conversely, presage potential change.  Saussure's example is paralleled by intrusive \emph{r} (\emph{Cuba[r]against your friends}, not \emph{*day[r]and}) and the competition of \emph{napron} and its near twin, and eventual successor, \emph{apron}.  \emph{Parole} lives.    
}

\begin{document}
\maketitle


\section{Saussure's Dilemma}\label{saussures-dilemma}

It seems only fitting to begin this tribute to Steve Anderson, friend
and erstwhile UCLA colleague, historian of \isi{linguistics} and confirmed
\emph{helvétophile}, with Ferdinand de Saussure and his discussion of
the history of \isi{stress} from \ili{Latin} to \ili{French} (Chapter III.4--6 of the
\emph{Course}). That history presents a dilemma for Saussure's
separation of \isi{synchrony} from \isi{diachrony} and linguistic activity
(\emph{parole}) from system (\emph{système}, \emph{langue}).

Here is Saussure's account of the transition from (ante)penultimate
\isi{stress} in \ili{Latin} to final \isi{stress} in \ili{French}:

\begin{quote}
In \ili{French}, the accent always falls on the last \isi{syllable} unless this
\isi{syllable} contains a mute \emph{e} (\emph{ə}). This is a synchronic fact,
a relation between the whole set of \ili{French} words and accent. What is its
source? A previous state. \ili{Latin} had a different and more complicated
system of accentuation: the accent was on the penultimate \isi{syllable} when
the latter was long; when short, the accent fell back on the antepenult
(cf. \emph{amī́cus, ánĭma}). The \ili{Latin} law suggests relations that
are in no way analogous to the \ili{French} law. Doubtless the accent is the
same in the sense that it remained in the same position; in \ili{French} words
it always falls on the \isi{syllable} that had it in \ili{Latin}: \textit{amī́cum → ami, ánimam →âme}. But the two formulas are
different for the two moments because the forms of the words changed. We
know that everything after the accent either disappeared or was reduced
to mute \emph{e}. As a result of the alteration of the word, the
position of the accent with respect to the whole was no longer the same;
subsequently speakers, conscious of the new relation, instinctively put
the accent on the last syllable,\is{syllable} even in borrowed words introduced in
their written forms (\emph{facile, consul, ticket, burgrave}, etc.).
Speakers obviously did not try to change systems, to apply a new
formula, since in words like \emph{amī́cum → ami}, the accent
always remained on the same syllable;\is{syllable} but speakers changed the position
of the accent without having a hand in it. A law of accentuation, like
everything that pertains to the linguistic system, is an arrangement of
terms, a fortuitous and involuntary result of evolution.\is{evolution} \citep[86]{DESb}
\end{quote}

To explain modern \ili{French} final \isi{stress}, Saussure goes back to its source
in \ili{Latin}, or rather to a stage of Romance subsequent to classical \ili{Latin}
but much earlier than modern \ili{French}. He first defines the \isi{stress} rules
for \ili{French} (final) and \ili{Latin} ((ante)penultimate, depending on the
quantity of the penult). That done, Saussure mentions in passing that
the position of \isi{stress} in \ili{French} preserves the original position of
\isi{stress} from \ili{Latin}, and illustrates the claim with examples of the two
subcases of \ili{Latin} stress,\is{stress} on a long penultimate (\emph{amī́cum → ami}) and 
on the antepenultimate when the penult is short
(\emph{ánimam →âme}).\is{prosody} It is worth mentioning that his formulation
``the accent is the same in the sense that it remained in the same
position'' is not original; it repeats a standard observation from
\ili{French} philology in the middle of the nineteenth century. Thus in 1862
Gaston Paris (and seven other scholars he mentions, p. 11) stated the
observation that \isi{stress} falls on the same \isi{syllable} in \ili{French} as it did
in \ili{Latin}; for that phenomenon Paris in particular uses the apt term
``persistence'' (\emph{persistance}) \citep[28]{PA}. Saussure does not use
that term, but his formulation echoes this earlier tradition. Saussure
then mentions the familiar fact that syllables after the \isi{syllable} with
the ``persistent'' \isi{stress} are subject to \isi{stress} reduction. At this point one
might think he is preparing to explain how final \isi{stress} in \ili{French} arose,
for example, perhaps by generalizing the word-final \isi{stress} of words like
\emph{amī́cum → ami} that had undergone apocope. Saussure does not
go in that direction. Instead, he declines to give a linguistic
explanation for the development of final \isi{stress} and places the burden on
speakers acting ``instinctively'' and then on the vague assertion that
``a diachronic\is{diachrony} fact was interposed.'' So although Saussure at the outset
seemed prepared to explain modern \ili{French} \isi{stress} in terms of its source
in \ili{Latin}, the source is not relevant to Saussure's interpretation. He
ends with a summary blaming the chaotic nature of change: ``A law of
accentuation, like everything that pertains to the linguistic system, is
an arrangement of terms, a fortuitous and involuntary result of
evolution.''\is{evolution}

Saussure seemed to recognize that the \ili{Latin} rule of (ante)penultimate
\isi{stress} was lost as a rule. Further, it cannot have been a rule of the
\emph{système}; had it been rule of the \emph{système}, it would have
been maintained. Saussure's response was this:

\begin{quote}
The synchronic law is general but not imperative. Doubtless it is
imposed on individuals by the weight of collective usage\ldots{}, but
here I do not have in mind an obligation on the part of speakers. I mean
that in language no force guarantees the maintenance of a \isi{regularity}
when established on some point. Being a simple expression of an existing
arrangement, the synchronic law reports a state of affairs; it is like a
law that states that trees in a certain orchard are arranged in the
shape of a quincunx. And the arrangement that the law defines is
precarious precisely because it is not imperative. Nothing is more
regular than the synchronic law that governs \ili{Latin} accentuation\ldots{};
but the accentual rule did not resist the forces of alteration and gave
way to a new law, the one of \ili{French}\ldots{} In short, if one speaks of
law in \isi{synchrony}, it is in the sense of an arrangement, a principle of
\isi{regularity}. \citep[92--93]{DESb}
\end{quote}

This analysis, developed in connection with a discussion of six facts of
\ili{Indo-European}, is here applied to the \ili{Latin} \isi{stress} rule, which was
downgraded to a descriptive observation about behavior maintained by
social convention, analogous to the rule stating that ``the trees in a
certain orchard are arranged in the shape of a quincunx.''

The phenomenon of \ili{Latin} stress,\is{stress} with its properties of persistence,
precariousness, and \isi{regularity} presented a dilemma for Saussure. The
phenomenon of persistence implies that usage (\emph{parole}) has a life
of its own; \emph{parole} is maintained as parole, as habit, transmitted
by imitation from one generation to the next. To invoke a metaphor,
substituting ``language'' for ``body'' in Newton's first law, we could
say: ``a language at rest remains at rest unless it is acted on by an
external force.'' Usage, such as the \ili{Latin} \isi{stress} rule, can exhibit
coherent patterns (such as the elegant parallelism of length of the
penultimate and antepenultimate position). Moreover, the patterns of
\emph{parole} are capable of defining the conditions for change (such as
the ``persistent'' location of \isi{stress} after \ili{Latin} which conditions
post-tonic apocope), and in this way patterns of \emph{parole} act like
elements of \emph{système}.

Saussure's dilemma was that the more he insisted on the dominant,
special, pure exclusionary status of \emph{système}, the more ethereal
and abstract system became, and that had the paradoxical effect of
elevating \emph{parole} to be the object of investigation.

\section{{``Law and Order''} and \isi{sandhi} doublets}

To document the fate of /r/ in weak position (after a vowel, not before
a vowel), \citet{KUR} divided the eastern seaboard into
four discontinuous zones, two northern and two southern, in which /r/
becomes {[}ə̯{]} in weak position: northern -- including New England
(Connecticut River east) and metropolitan New York; southern -- including
Upper South (Virginia, into northern North Carolina) and Lower South
(South Carolina, Georgia). The four zones are separated by transitional
belts which retain some form of \emph{r} (presumably American {[}ɹ{]} or
``velarized constricted'' \emph{r} {[}ɚ{]}). The largest of the
transitional belts is Pennsylvania, from which rhoticism spread to
Midwest and Midland dialects.

The discussion here focuses on the northern zones, which \citet{KUR} treated as a single zone. As shown in Table 1, in the north /r/
is reflected in weak position as {[}ə̯{]} after mid and high vowels
({[}i, u, e, o{]}). After low vowels ({[}a/ɐ, ɒ/ɔ{]} and here {[}ə{]} as
well) what must have been the earlier reflex {[}ə̯{]} was lost or
absorbed by the vowel.

\begin{table}[ht]
\centering\caption{Postvocalic rhotic reflexes, North}
\begin{tabular}{ll}
\lsptoprule
high/mid V & low V\\
\midrule
{[}ir{]}, {[}ur{]}, {[}er{]}, {[}or{]} \textgreater{} &  [ar/ɐr], [ɒr/ɔr], [ɘr]
\textgreater{}\\
{[}iə̯{]}, {[}uə̯{]}, {[}eə̯{]}, {[}oə̯{]} & {[}aə̯{]}, {[}ɒə̯/ɔə̯{]}, {[}ɘ{]}
\textgreater{}\\
\emph{ear} {[}iə̯{]} & [a/ɐ], [ɒ/ɔ], [ɘ] \\
\emph{poor} {[}puə̯{]} & \emph{far} {[}fa/fɐ{]}\\
\emph{care} {[}keə̯{]} & \emph{for} {[}fɒ/fɔ{]}\\
\emph{four} {[}foə̯{]} & \emph{father} [faðɘ]\\
\lspbottomrule
\end{tabular}
\end{table}

\newpage 
Like northern dialects, southern dialects also absorbed {[}ə̯{]} after
low-mid [a/ɐ, ɒ/ɔ, ə]. Furthermore: ``In Southern folk speech, /ə̯/
is often lost, \emph{door}, \emph{four}, \emph{poor} /doə̯, foə̯, poə̯/
thus becoming /do, fo, po/'' \citep[171a and map \#156]{KUR}.\footnote{The
  lexeme \emph{poor} is treated once as having a mid vowel (171a) and
  otherwise as a high vowel (170b, 171b, 172a, 172b). (One, also 171b,
  is ambiguous.)}

Kurath and McDavid devoted special attention to sandhi
contexts -- contexts in which the word-final vowel which once had /r/ is
used in a phrase with a following word. Then the word with original /r/
can be said to have two ``sandhi \isi{doublets},'' depending on whether the
second word begins with a consonant (when the original /r/ would have
been in weak position) or with a vowel (when the original /r/ would have
been prevocalic). They stated: ``\ldots{}\emph{ear, poor, care, four}
have\ldots{} the positional allomorphs /iə̯ \textasciitilde\ iə̯r, puə̯ \textasciitilde\ puə̯r, kæə̯ (keə̯) \textasciitilde\ kæə̯r (keə̯r), foə̯ (fɔə̯) 
\textasciitilde\ foə̯r (fɔə̯r)/, and \textit{car, for} (stressed), \textit{father} 
the allomorphs /ka (kɐ) \textasciitilde\ kar (kɐr), fɒ (fə) \textasciitilde\ fɒr (fɔr), fað \textasciitilde\ faðər/.'' Note
the difference between non-low vowels, in which the reflex is
{[}Və̯(r){]}, and low vowels, in which the reflex {[}V(r){]} lacks
{[}ə̯{]}, since {[}ə̯{]} had been absorbed by the preceding low vowel. The
idea of calling these \isi{doublets} (and they provide a notation for
\isi{doublets}) suggests a model of the \isi{lexicon} in which a lexical item is
composed of multiple subunits, which could be written as an ordered pair
such as \{{[}iə̯{]}/sandhi before consonant; {[}iə̯r{]} / sandhi before
vowel\}. One might, for example, write a doublet for the noun
\emph{Cuba} as pronounced by John F. Kennedy in his speech ``in the
Cuban Missile Crisis'' generally as {[}kubə{]}, as in \emph{and then
shall Cuba{[}ə{]} be welcomed} (16:07), but {[}kubər{]} in phrases such
as \emph{Soviet assistance to Cuba{[}r{]} and I quote} (4:32) and
\emph{turned Cuba{[}r{]} against your friends} (15:05).\footnote{\url{http://www.historyplace.com/speeches/jfk-cuban.htm}.}

Examples constructed in the spirit of \citet{KUR} are given in
Table 2, top.

\begin{table}[ht]
\centering\caption{\isi{Sandhi} /r/ and Intrusive /r/, North}
\begin{tabular}{lll}
\lsptoprule
context 		  & 		high, mid V 		& low V\\
\midrule
sandhi 	  	  &  {[}iə̯{]}, {[}uə̯{]}, {[}eə̯{]}, {[}oə̯{]}	&	[a/ɐ], [ɒ/ɔ], [ɘ]\\
{[}V(ə̯)r͡V{]} & \emph{ear and} {[}iə̯rænd{]}		&	\emph{far and} {[}farænd{]}\\
			  & \emph{poor and} {[}puə̯rænd{]}		&	\emph{for all} {[}fɒral{]} \\
			  & \emph{care and} {[}keə̯rænd{]}		&	\emph{father and} {[}faðɘrænd{]}\\
			  & \emph{four and} {[}foə̯rænd{]}	 	& 	\\
intrusive		  & {[}i{]}, {[}u{]}, {[}e{]}, {[}o{]}			&	[a/ɐ], [ɒ/ɔ], [ɘ] \\
{[}Vr͡V{]}	  & \emph{three and} *{[}θrirænd{]}		&	\emph{ma and} {[}marænd{]} \\
			  & \emph{two and} *{[}turænd{]}		&	\emph{law and} [lɒr\underline{ænd}] \\
			  & \emph{day and} *{[}derænd{]}		&	\emph{Martha and} {[}maθɘrænd{]} \\
			  & \emph{know it} *{[}norɪt{]}			&	\\
\lspbottomrule
\end{tabular}
\end{table}

\largerpage
It is worth drawing attention to the fact that prevocalic sandhi
examples with non-low vowels have the sequence {[}ə̯r{]} (p. 171b); as in
{[}iə̯r, puə̯r, kæə̯r (keə̯r), foə̯r (fɔə̯r){]} from the list above. The
sandhi sequence {[}Və̯rV{]} has in effect two segments -- {[}ə̯{]} and
{[}r{]} -- which reflect earlier /r/. Both cannot be original. There must
have been an antecedent stage of {[}*irV, *purV, *kærV (*kerV), *forV
(*fɔrV){]} in sandhi position before a vowel. The {[}ə̯{]} we see now in
{[}iə̯r{]}, etc., had to have been introduced by \isi{analogy}\is{analogical} from other forms
to the sandhi forms before vowel.

  
Analogy is relevant to history in another respect \citep{SOS}. Not
uncommonly, words that ended originally in low vowels without /r/
acquired a non-etymological, or ``intrusive,'' /r/ in sandhi, as in the
familiar \emph{law and order} {[}lɒrəndɒdə{]} and other examples in
Table 2. As \citet[172a]{KUR} state,

\begin{quote}
On the \isi{analogy} of such \isi{doublets} as \emph{for} /fɒ (fɔ) \textasciitilde{}
fɒr (fɔr)/, \emph{car} /ka (kɐ) \textasciitilde{} kar (kɐr/, and
\emph{father} /faðə \textasciitilde{} faðər/, positional allomorphs
ending in /r/ are often created in Eastern New England and Metropolitan
New York for words that historically end in the vowels /ɒ \textasciitilde\ ɔ, a \textasciitilde\ ɐ, ə/, as \emph{law},
\emph{ma}, \emph{Martha}. Thus one hears \emph{law and order} /lɒr ənd
ɒdə, lɔr ənd ɔdə/, \emph{ma and pa} /mar ən(d) pa, mɐ ən(d) pɐ/,
\emph{Martha and I} /maθər (mɐθər) ənd ai/.
\end{quote}

The examples of \isi{intrusive /r/} just cited involved only words which end
in a low vowel -- that is, they have the same vocalism in the non-sandhi
environments as words that originally ended in /r/ but which absorbed
the {[}ə̯{]} reflex of /r/; thus \emph{law} /lɒ (lɔ)/ has the same
vocalism as originally rhotic words like \emph{for} /fɒ (fɔ)/. But words
like \emph{three}, \emph{two}, \emph{day}, \emph{know}, which end in mid
and high vowels, differ. \citet[172b]{KUR} state:

\begin{quote}
It is worth noting that after the normally upgliding free vowels /i, u,
e, o/, as in \emph{three}, \emph{two}, \emph{day}, \emph{know}, an
\isi{analogical} ``intrusive" /r/ never occurs. The reason for this is clear:
since /θri, tu, de, no/ do not end like the phrase-final /r/-less
allomorphs of \emph{ear}, \emph{poor}, \emph{care}, \emph{four} /iə̯,
puə̯, keə̯, foə̯/, the basis for creating allomorphs ending in /r/ is
lacking.
\end{quote}

Thus according to \citet{KUR}, the development of \isi{intrusive /r/}
involves the comparison of \isi{stem} shapes, for example {[}lɒ{]} with
{[}fɒ{]}, which are similar and permit \isi{analogy},\is{analogical} as opposed to
\emph{three} {[}θri{]} with {[}iə̯{]}, which are dissimilar and do not
permit analogy.\is{analogy}

\newpage 
This distribution is interesting. What determines whether \isi{analogical}
\isi{intrusive /r/} develops is an arbitrary division of vowels inherited from
the previous history of derhoticism; that is to say, a distinction in
vowels involved in the earlier history of reflexes of /r/ in weak
position continues to have an effect on later developments. Thus
\emph{parole} has the property of inertia (\emph{persistance}), so that
later changes (such as the \isi{analogical} development of \isi{intrusive /r/}) can
be sensitive to properties of \emph{parole} that persist. At the same
time as \emph{parole} is inertial and conservative, \emph{parole}
nevertheless carries with it the possibility of change. Thus original
\emph{r-}less words ending in low vowels have the potential to develop
an intrusive sandhi /r/, as happened in northern dialects. Conversely,
original \emph{r-}full words had the potential to eliminate the second
member of the ``doublet'' in which /r/ reappears in sandhi before a
vowel; this is what happened in southern dialects (especially Upper
South but even in the Lower South sandhi forms with /r/ are ``only half
as frequent as the variants without /r/'', \citealt[171b]{KUR}).

\emph{Parole}, then, is inertial but carries the potential for change.
This example is similar to what Saussure said about \ili{Latin} stress,\is{stress} that
it remained on the \isi{syllable} where it had always been -- by convention, or
memory, or inertia -- but eventually the \isi{stress} was repositioned.

It might be objected that it would be easy to state a rule inserting /r/
that is sensitive to vowel height; insertion would happen only in
position after low vowels. But why low vowels? Low vowels are not
universally more likely than other vowels to adopt a phonotactic
sequence {[}VCV{]} that other vowels. Intrusive /r/ develops only after
low vowels because it is only low vowels that offered a model for
\isi{analogical} extension, and that is a distribution that goes back to a
prior change; the restriction to low vowels can only be understood by
viewing it as the hangover from a previous stage. Moreover, it is not
just any consonant that reappears; it is just the one sound /r/. The /r/
can participate in ``intrusive'' \isi{analogy} because the /r/, and only the
/r/, was carried over from earlier history. The fact that /r/ is
involved in \isi{analogy}\is{analogical} at all is a further instance of persistence of
\emph{parole}.

\section{{``Watergate''} and its ilk}

Against this background I want to discuss how innovations\is{innovation} can arise
directly out of speech. The word \emph{Watergate}\is{Watergate} and its derivatives
can serve as an illustration. As is familiar, \emph{Watergate} is the
name of a complex of five buildings built in Washington, D.C., over the
period 1963--1971. An office building in this complex was used by the
Democratic National Committee as headquarters leading up to the 1972
election. The Democratic offices suffered a break-in, for which staff
members of the Republican administration were later discovered to be
responsible. The break-in triggered an embarrassing scandal and, because
of the attempt to cover up the original crime, led to the resignation of
President Richard Nixon.

A modification of the name for this location keeps being applied to more
events, which, like the original \isi{Watergate}, include at least two events,
layers of agency, times, places. The core is the pairing of two events:
first, an event carried out in secret and, second, the fallout,
including the embarrassment caused by the event for the participants and
perhaps further developments (cover-up, disclosure). The whole scenario
is a rich instance of the familiar trope of \isi{metonymy}, which points to
one event -- here, the original transgression -- which can invoke
associated events (here, the fall-out) and the constituents of those
events (locus, agents, patients). The name for this complex of events
and constituents, which occurred in 1972, is of course
\emph{Watergate} -- the name for the place is applied to the whole
package of events, by the trope of \emph{pars (locus) pro toto} (complex of
events -- crime, scandal, cover-up, further fall-out). The semantic
operations involved in \emph{Watergate} are familiar, banal tropes.

Event complexes similar to the original \emph{Watergate} scenario can be
named by the new \isi{compound} \{\emph{x+gate}\}, where \{\emph{gate}\}
refers to the existence of a scandalous event (and fall-out) and
\emph{x} refers to a \isi{focus} -- a constituent that is central to the
events -- such as the \isi{agent} (\emph{Billygate}) or causal entity
(\emph{nannygate}) or the patient (\emph{contragate}).

The morphological\is{morphology} structure and \isi{semantics} of the new \isi{compound} \{\emph{x}
``focus''+\emph{gate}\} ``event(s) leading to scandal'' seem clear, and it
seems clear that the \isi{compound} is related to the origin \emph{Watergate}.
How? Given the apparent overlap of \{\emph{gate}\} in both, one might
imagine that the word \emph{Watergate} was decomposed into two
morphemes, \{\emph{water}\} and \{\emph{gate}\}, and that \isi{reanalysis}
provided the model for \is{neologism}neologisms. But this cannot be: by itself
``\emph{water}'' does not mean anything in this context; it is not the
focus. And for that matter, \emph{gate} doesn't mean scandal here in
the \isi{compound} \emph{\isi{Watergate}.} In the original word \emph{Watergate},
there is no division; \emph{Watergate} is the name for the complex as a
whole, not for any of its constituents.

And yet \emph{Watergate} was self-evidently the source for the formula
\{\emph{x+gate}\} and novel applications of the formula. What this means
is that ``\isi{Watergate}'' -- the name for a whole complex of agents and
events -- allowed speakers to imagine a new structure \{\emph{x+gate}\}
whose \isi{semantics} give overall \isi{semantics} analogous to the meaning of
\emph{Watergate} (secret event and subsequent scandal, specific place or
agents, etc.) but in which the event complex is broken into two
constituents; one of them, \{\emph{x}\}, refers to the focus of events,
and the other part, \{\emph{gate}\}, establishes the existence of a
secret event and its attendant scandal involving the \isi{focus} \{\emph{x}\},
whereas in the source \emph{Watergate}, the whole included all the
components.

Two aspects of this \is{language change}change are significant. First, the new \isi{structure} is
motivated by the inherited word, but it is not a copy; it cannot be
generated by a proportional \isi{analogy}.\is{analogical} Instead, what the example shows is
that speech has the potential of providing motivation for creating new
speech directly. To say it another way, speech is not just speech;
speech invites modal possibilities. The second point is that the source
here really is speech that actually occurred in real time:
\emph{Watergate} started as a single event complex that occurred at some
time in history; it did not start as a pattern. That is, a singular
event and the accompanying speech give rise to an innovation;\is{innovation} speech
creates speech. This new \{\emph{x+gate}\} is a virtual structure which
might exist indefinitely. We cannot verify its existence until it is
acted on. Therein is a property of language that has eluded description:
the fact that speech happens, that activity matters, it happens when a
novel formation is used, and it happens to the extent that neologisms\is{neologism}
are created and used in speech.

This example, then, suggests a more active role for \emph{parole}
(performance, speech) than has usually been assumed. In this instance
actual speech from a very specific historical time (1972) provided the
model and created the potential for new speech, and that is what
resulted. It is worth stating that speech is not just blind activity;
speech comes with implicit patterns, whether firmly established or -- as
in this case -- potential, possible, modal speech.

It could be mentioned that this formation, along with similar neologisms
motivated by \emph{alcoholic}, have distinctive stylistic overtones and
spheres of usage -- in the personal sphere, gentle mockery
(\emph{shopaholic}, \emph{chocoholic}) and not-so-gentle journalistic
irony for the former (\emph{Camillagate}). The News History Gallery at
the Newseum\footnote{\url{http://www.newseum.org/}} in Washington, D.C., devoted
to the history of journalism, has an exhibit called ``The `Gate'
Syndrome,'' illustrated by five examples, starting with \emph{Koreagate}
(1976). \footnote{
  Arnold Zwicky calls \{gate\} a ``\isi{libfix}'' -- ``lib'' in the sense of
  ``liberated'' -- which captures the idea that a mental operation
  extracts a new \isi{affix}
  (\url{https://arnoldzwicky.org/2010/01/23/libfixes/}). The author
  wishes to thank the editors for this and many other valuable and droll
  comments and corrections.}

\section{{``(N)apron''} as dynamic doublet}

A somewhat similar change is the change from \emph{napron} to
\emph{apron} in Middle \ili{English}. As is familiar, a dozen or so nouns
which had once begun with an initial consonant \emph{n} lost the
\emph{n} and came to begin with the vowel of the first syllable.\is{syllable}
According to the standard analysis, this happened because when such
nouns were used with the indefinite article \emph{a(n)}, a sequence of
{[}{anV}{]} would result, and then it is unclear whether the
intervocalic {[n]} belongs to the \isi{stem} of the noun or to the
article. The ambiguity opened up the possibility that the {[}n{]} could
be attributed to the article and the noun could be reanalyzed as
beginning with a vowel. Subsequently the \isi{stem} shape without the vowel
could be extended to all contexts; thus \{a+napron\} \textgreater{}
{[}anapron{]} was analyzed as \{an+apron\} \textgreater{} {[}anapron{]},
leading to the use of \{apron\} elsewhere. As is well known, the
converse also occurs, where nouns beginning with an initial vowel
(\emph{an ewt}) acquired an initial \emph{n} from the indefinite article
(\textgreater{} \emph{a newt}). It is not clear why the change of
\isi{metanalysis} should be able to go in either direction.

This standard analysis discusses only the end-points of this
change -- prior to \isi{metanalysis}, after \isi{metanalysis} -- but does not
describe how the change progressed. To get a sense of how this change
actually proceeded, I attempted to trace the history of spellings
\emph{(n)apron} in Middle \ili{English} with an eye to variation in the choice
of the word form in different contexts. The task was rather more
challenging than I had expected. The word \emph{(n)apron} is quite
specific. It occurs infrequently, primarily in wills and inventories of
good to be bequeathed. (And also, as will be noted below, in a
description of the rules of the household of Edward IV.) The item is
mentioned only in a minority of the wills or inventories available, and
usually when the deceased is a woman. For example, the extensive
\emph{Wills and Inventories of Bury St. Edmunds} has approximately 150
printed pages of wills from the beginning of the fifteenth century (one
will from 1370, then 1418, etc.) to the late sixteenth century (1570),
and has no instances of the word in either variant, \emph{napron} or
\emph{apron}. That, despite instances such as the will of one Agas Herte
(a. 1522, pp. 114--18), who bequeathed about 50 distinct household
objects to her son, including ``\emph{ij tabyll clothes, vj napkyns,
iiij pleyne and to of diap, a salte saler of pewter}\ldots{}'' and about
the same number to her daughter, including ``\emph{ij tabell clothes, vi
napkyns, iiij pleyn and ij of diap, and a pleyn towel}\ldots{}'' Among
all the items she bequeathed, including the items made of cloth just
mentioned, no \emph{(n)apron} was mentioned. This might because this
household, and other households as well, did not use \emph{(n)aprons};
it might be they were considered too insignificant to be mentioned in
bequests (though towels and napkins and sheets are recorded regularly).
In any event, the frequency with which \emph{(n)apron} appears is
modest. In short, it has proven difficult to find document sets in which
\emph{(n)apron} is mentioned multiple times; examples are isolated. To
maximize the range of texts examined, I used Hathitrust/Google scans
subjected to OCR. I searched for both \emph{napron} and \emph{apron},
both singular and \isi{plural}, in variant spellings.

We can first take a quick look at chronology, using a ledger
(\emph{Fabric rolls}) kept by the York Minster which recorded
miscellaneous expenses annually. The entries are written in \ili{Latin},
though names for some items specific to the contemporary realia appear
in \ili{English}. Half a dozen times the rolls record payment for the costs of
masonry, both for wages and equipment -- aprons and gloves for masons
(called ``setters''). The earliest record from 1371 surprisingly has
\emph{n-}less \emph{aprons} (\emph{ij aprons et cirotecis} `two aprons
and gloves'\emph{,} 1371). Then at the beginning of the fifteenth
century come two instances of \emph{naprons: In remuneracione data
cementariis vocatis setters ad parietes \textbf{cum naprons} et
cirotecis, per annum 9s. 10d.} `as compensation given to the masons
known as setters at the wall with aprons and gloves, annually 9s. 10d.'
(1404); \emph{In ij pellibus emptis et datis eisdem \textbf{pro
naporons}}, `two hides were bought and given to them to serve as aprons'
(1423). At the end of the fifteenth century there are two examples of
\ili{Latin} \emph{limas} (\emph{\textbf{duobus limatibus,}} 1497--98;
\emph{\textbf{Pro ij limatibus},} 1499), and shortly thereafter,
\emph{aprons} (\emph{\textbf{pro ij le aprons} de correo pro les setters
per spacium ij mensium, 12d.} `for two aprons of hide for the setters
for the period of two months, 12 shillings' (1504). The use of
\emph{aprons} in 1371 seems anomalously early (could it be an error in
transcribing the text?). This anomaly aside, the examples suggest a
chronology: \emph{napron} was used in the fifteenth century (1404, 1423)
and shifted to \emph{apron} the beginning of the fifteenth century
(1504). Other texts suggest there was still some variation in the
sixteenth century. By 1600 \emph{apron} had taken over.

Against the background of generally skimpy attestation of
\emph{(n)apron} in the fifteenth and sixteenth century, there are two
texts which offer enough examples to allow us to say something about
usage. One is a single text, the so-called Liber Niger Domus Regis,
which specifies the duties and compensation of the staff of King Edward
IV's household in the last quarter of the fifteenth century (c. 1480). A
modern edition compiles three manuscripts (discussion, \citealt[51--60]{myers}). The oldest is a manuscript from the end of the fifteenth
century, which served as the basis for the famous \citeyear{XEDWARD} publication by
the Society of Antiquaries (abbreviated ``A''); however, text A is now
defective, and it also appears that the 1790 edition took some
liberties, so the printed 1790 edition cannot be trusted to represented
the oldest text A. In the accompanying Table 3 I've cited the location
of readings from the 1790 reading edition in \textbar{}\textbar{}. The
next oldest is a sixteenth-century manuscript (preserved in the Public
Records Office, the Exchequer, abbreviated E), from the era of Henry
VIII, is similar to A but fuller. Third, the youngest of the three
manuscripts, known as Harleian 642 (here H), is a seventeenth-century
copy made by Sir Simonds d'Ewers. In fact, differences recorded in
footnotes by Myers in his edition are minimal and affect the analysis
here in only one respect, mentioned below.

There are basically two contexts (with one additional outlier). Examples
repeat over the descriptions of many different servants. The twelve
examples of \emph{(n)apron} are given in abbreviated form Table 3.

\begin{table}[ht]
\centering\caption{\emph{(n)apron} in Liber Niger of Edward IV}
\begin{tabular}{lll}
\lsptoprule
& type & text {[}variants{]}\\
\midrule
§55	&	\textsc{mod}	&	they have part of the
\textsuperscript{α}yeftes\textsuperscript{α} geuvn to the  \\
|49.7d| & &  houshold\ldots{} but \textbf{\emph{none aprons}} {[}1790: \\
	 & & 	\textsuperscript{α}gyftes\textsuperscript{α}{]} \\

§62 &	\textsc{do}		&	{[}takith{]} at euery of the iiij festes of the \\
|52.28d| & & yere, \emph{\textbf{naprons}} of the great spycery \\

§62 & 	\textsc{do}		&	take \emph{\textbf{naprons}} also at euerych of the iiij \\
|52.28d| &	& festes \\

§80 &	\textsc{do}		&	etithe in the halle; taking for wages\ldots{} and \\
|71.6| & & nyʒt lyuereye, \emph{\textbf{napors}}, and parte of the\\
	& & generall giftes \\

§80 & 	\textsc{do}		&	taking for his wynter clothing chaunces, \\
	& & \emph{\textbf{napors}}, parte of the giftes generall \\
	& &  {[}extended passage, absent in 1790{]}\\

§74 &	\textsc{1do}	&	 Eche of them takethe\ldots{} \emph{\textbf{j napron}} of lynyn \\
|61.26d| & & cloth of ij ellez {[}1790: \emph{\textbf{a} \textbf{naperon}}{]} \\

§77 &	\textsc{1do}	 &	At euery of the iiij festes, \emph{\textbf{j napron}} of j \\
|65.19d| & & elle, price vjd. {[}1790: \emph{\textbf{one napron}} of one\\
	& & 	elle{]}\\

§33 &	\textsc{prp}	&	{[}he takith{]}\ldots{} ij elles of lynen clothe \emph{\textbf{for}}  \\
|36.10u| & & \emph{\textbf{aprons}}, price the elle, xijd. \\

§77 &	\textsc{prp}	&	ij ellez of lynyn cloth\ldots{} \textbf{\emph{for naprons}} \\
|64.25d| & & \\

§77 &	\textsc{prp}	&	 j elle of lynnyn clothe \emph{\textbf{for naprons}}	\\
|65.8d| & & \\

§77 &	\textsc{prp}	&	j elle \emph{\textbf{for naprons}} of lynyn cloth \\
|64.41| & & \\

§80 &	\textsc{prp}	&	\textsuperscript{α}and for chaunces iiijs. viijd.\textsuperscript{α} of \emph{\textbf{napors}} at \\
|71.26d| & &  euerye {[}H\textsuperscript{αα}; H \emph{\textbf{aprons}}{]} \\
\lspbottomrule
\end{tabular}
\\
§ = section in Myers, || page in 1790 edition
\end{table}

In one isolated instance, the noun is preceded by \emph{none}; the nasal
might have elicited the following \emph{apron}. The other eleven tokens
are split between two contexts. In six instances the noun is the direct
object of the verb `take' (listed as ``\textsc{do}''). The entities
taken have already been formed into garments; all have \emph{napron}.
Within this group of six examples, in two of these six, marked here as
``1\textsc{do},'' the word \emph{napron} is preceded in texts E and H by
\emph{j}, that is, Roman numeral `one'. (The published 1790 version has
the indefinite article \emph{a} \emph{napron} in one instance and the
written word \emph{one napron} in the other, rather than the numeral.)
Both E and H use numerals consistently in discussions of compensation;
prices of elles of linen cloth are cited with numerals, such as \emph{j
elle}. The numeral here must be original, thus \emph{j napron} in both
examples. I will return to these in a moment.

The second group of six examples involves the statement that servants
receive compensation in the form of linen cloth which is supposed to be
turned into \emph{(n)aprons}, expressed by a \isi{preposition}, usually
\emph{for}, once \emph{of}. In this context the entity referred to as
\emph{(n)apron} does not yet exist; the noun has a future attributive
sense: ``the speaker wishes to assert something about whatever or
whoever fits that description'' of being (or becoming) an apron
\citep[285]{DON}. An example is: \emph{at eueryche of the iiij
festes of the yere, of the clerk of greete spycery, ij elles of lynen
clothe for \textbf{aprons}}, \emph{price the elle,
xij}\textsuperscript{d}. This one sentence has \emph{aprons}, which
seems to suggest that an attributive reading implies \emph{aprons}. But
this sentence is the only instance with \emph{aprons} among the five
tokens of this attributive context, so an attributive reading by itself
can't explain \emph{aprons} in this specific example. I return to this
token below.

Let us turn to the second text, namely \emph{Durham wills and inventories}. The second text is not, strictly speaking,
a single text but a series of wills; still, they are all from one locale
and one tradition over a short interval, from 1562 to 1570, and can be
treated as a single text. (In fact, there is a string of four tokens of
\emph{(n)apron} in a row.) The tokens are given in Table 4.

%\begin{longtable}[]{@{}ll@{}}
\begin{table}[ht]
\centering\caption{\emph{(n)apron} in \textit{Durham Wills and Inventories}}
\begin{tabular}{ll}
\toprule
type & text\\
\midrule

\textsc{mod} & It' I bequith to Agnes Carter \emph{\textbf{a linn
Apron}}. (I.277,\\
& 1567)\\

\textsc{mod} & It'm I gyve to Helenor Huntley iiij\textsuperscript{or} blake patletts iiij\textsuperscript{or}\\
& cherches \emph{\textbf{a blewe apron}} \& ij\textsuperscript{o} velvett pattletts (I.343, \\
& 1570)\\

\textsc{art} & to Thomas Burdon a busshell of wheat  -- to Jane\\
& Brantinga' a line kyrcheff \emph{\textbf{an apron}} \& a pair of hoose\\
& (I.198--99, 1562)\\

\textsc{art} & I geve unto Elizabeth Hackforth a kerchif, a raill, a\\
& smock, \emph{\textbf{an apron}} and all my workday rayment and in \\
& mony 3\emph{s}. 4\emph{d}. (III.56--57, 1570)\\
& It'm I gyve to katheryn barnes ij\textsuperscript{s} vj\textsuperscript{d}.  It'm I gyve to \\

\textsc{art} & thomeis hynde y\textsuperscript{t} was my p'ntice \emph{\textbf{an apron}} \& a new \\

\textsc{art} & fyshe knyffe. | It'm I gyve to thomas capstone \emph{\textbf{a}}\\
\textsc{art} & \emph{\textbf{napron}}. | It'm I gyve to thomas boswell \emph{\textbf{a napron}}.| \\
\textsc{art} & It'm I gyve to luke hanynge \emph{\textbf{a napron}} \& a fyshe borde. \\
& (I.327, 1570)\\
\textsc{art} & And to alles Barnes a gowne of worsted \& \textbf{\emph{a napron of}}\\
& \emph{\textbf{worsted}} (I.305, 1569)\\
\bottomrule
\end{tabular}
\end{table}

Only two contexts occur in this small corpus.\is{corpus} One context is represented
by two tokens, in which the indefinite article and noun are separated by
a modifier (\emph{a linn Apron}, \emph{a blewe apron}). The modifier
makes these constructions novel. This pair of examples suggests two
thoughts: that the innovative\is{innovation} form \emph{apron} is favored to the extent
the context in which the noun is used is non-idiomatic, novel; and the
concept of idiomaticity is a gradated (not discretely binary) parameter.
To continue down this path, both in the example from 1562 and the 1570
example (from volume III of these documents), \emph{an apron} occurs in
the middle of a miniature list of three bequeathed items. Lists by their
nature hint that a set of entities could be extended, so they promise a
modal, possible, open-endedness. Thus it appears that open-endedness
favors the innovative\is{innovation} form \emph{an apron.} In contrast, in the will of
cook William Hawkesley (presented in Table 4 as a block of four tokens),
the second through fourth tokens have a fixed phrase \emph{a napron},
and the whole construction, `I give to \emph{x a napron}', is a standard
\isi{idiom} of bequests. Thus fixed idioms use the inherited older form
\emph{a napron.} In 1569 Alice Barnes receives two worsted items;
possibly the parallel in material is the critical information, and the
fact that one is a \emph{(n)apron} is incidental.

\newpage 
The most interesting example is the first example of the 1570 set.
Throughout this will of William Hawkesley, the recipients are identified
in an unambiguous but not expansive fashion; the recipients are listed
by name alone (22 xx) or with name and geographical location (3 xx) or
name and relationship (9 xx), such as mother-in-law or midwife. That is,
the recipients are presumed to be known by name with the briefest of
descriptions, almost titles. Against that background, the description of
the first recipient of aprons, \emph{thomeis hynde y\textsuperscript{t}
was my p'ntice,} stands out; given its relative clause
\emph{y\textsuperscript{t}} (`that'), the identification of Thomas is
relatively elaborate. Indirectly, this means that the bequest -- an apron
and a fish knife -- is out of the ordinary, atypical. In contrast, in the
three bequests that follow immediately thereafter are idiomatic. It
appears, then, that novel or unexpected bequests of aprons -- the bequest
itself or the recipient -- are expressed by the innovative\is{innovation} \emph{(an)
apron}, while less novel scenarios are expressed by the older form
\emph{a napron}.

This takes us back to the one example in the Liber Niger Domus Regis
Edward IV which had \emph{aprons} (other than \emph{none aprons}):
\emph{{[}he taketh{]} at eueryche of the iiij festes of the yere, of the
clerk of greete spycery, ij elles of lynen clothe \textbf{for aprons}},
\emph{price the elle, xij}\textsuperscript{d}. In and of itself, the
sentence is unremarkable and indistinguishable from the other examples
with prepositions which had \emph{naprons}. What might be atypical is
the office described here, which is that of \emph{sewar}, the highest
\isi{ranking} and first mentioned of the king's servants: \emph{\textsc{a
sewar for the kyng}, wich owith to be full cunyng, diligent, and
attendaunt. He receueth the metes by sayez and} \emph{saufly so
conueyeth hit to the kinges bourde with saucez according therto, and all
that commith to that bourde he settith and dyrectith} (§33, p. 112). In
this instance, although the act of taking aprons is not exceptional, the
recipient -- the \emph{sewar} -- is unique. This is then similar to
\emph{thomeis hynde y\textsuperscript{t} was my p'ntice} from \emph{Durham
wills and inventories}, in the sense that the non-idiomatic character of the example
derives from the recipient, not the \emph{(n)apron} phrase. That should
not be surprising, since the act of bequeathing includes a recipient as
well as the item bequeathed. The innovative\is{innovation} \emph{aprons} here acts
effectively as an honorific to draw attention to the unusual status of
the recipient, as it did with \emph{thomeis hynde y\textsuperscript{t}
was my p'ntice}.

In general, it appears the innovative\is{innovation} form is favored if the transfer of
\emph{apron} is novel, not typical, and this extends to the recipient of
the transfer (relative to other recipients). This principle applies to
both example sets from different stages of the change. This
distribution -- unidiomatic context prefers the novel form -- turns out to
match other instances of the competition between equivalent
morphological forms. Thus in contemporary Czech the locative singular
(used with certain prepositions) can be either the traditional ending
\{-e\} or a new ending \{-u\}. (That ending is original with nouns of
the \ili{Indo-European} \emph{u-}stem declension, but its use with
\emph{o-}stem masculine nouns is new.) The parallel is that the
traditional \mbox{\{-e\}} is used with ``typical'' combinations while
innovative\is{innovation} \{-u\} is used with atypical contexts \citep{BERM}.

There is another \isi{regularity} of some interest that applies to both texts.
We saw above that in Liber Niger there were two instances in which
\emph{(n)apron} followed the numeral \emph{j} (`one'), and in both the
older form \emph{napron} was used. The examples are more or less
equivalent in meaning to a true indefinite article as in \emph{a
napron}; the two examples of \emph{j napron} (with \emph{napron}) with
the numeral invite the suspicion that true indefinite articles at this
time might have \emph{napron}, if they were attested. Conveniently,
there is a contemporaneous will that has two tokens of an indefinite
article one after the other: \emph{Also I gyve to Margarete Holton my
best kyrtill \& \textbf{a napron}. Also I gyve to Elisabeth Wike a smok
\& \textbf{a napron}} (will of Jone Montor, 1489, \emph{Surrey Wills},
p. 95). A slightly later example is consistent: \emph{A jak \& a salet,
a gorget, ij gussettes, \textbf{a napron}, and iij gauntlettes}
(\emph{York wills}, p. 35, 1512). These examples at least suggest that
the context with an indefinite article used the more conservative form.

To return to the other text under discussion here, \textit{Durham wills and
inventories}, there were two recognizable contexts. One involves an
indefinite article split from the noun; it does show that the change had
progressed to novel (unidiomatic) contexts. The other context had seven
tokens with an indefinite article (not separated from the noun); the
older form \emph{napron} was used 4 times, the novel form time 3 times.
That is to say, the novel form \emph{apron} was slow to appear in the
context in which the indefinite article immediately preceded the noun.
For both periods (late 1400s, third quarter of 1500s) it appears that a
construction with an indefinite article uses \emph{apron} less (or at
least not more) than other contexts.

\largerpage
Now the standard analysis is that the ambiguous combination of
indefinite article and \emph{napron} led to a \isi{reanalysis} of \{a+napron\}
to \{an+apron\}. If so, it would be natural to expect that \emph{apron}
would be used first in the context of \isi{reanalysis} and only later in other
contexts -- that is, it should appear earliest with the indefinite
article. But we just saw that in both texts, \emph{apron} was used in
other contexts when \emph{an apron} was not yet used (the first text
plus the auxiliary wills) or not used as frequently (the second text).

This suggests a revision of the account of \isi{reanalysis}. Since the
appearance of \emph{apron} is in fact not tied to the indefinite
article, the unit \emph{apron} appears to have some degree of \isi{autonomy}.
The \isi{reanalysis} consists not of replacing the underlying shape of the
noun, but it consists of imagining the possibility of an alternate word
form \{apron\}, which co-exists, for a time, with an alternate sublexeme
\{napron\}. Imagined \{apron\} becomes real only when it is actually
used. Following the general principle that innovative\is{innovation} forms appear first
in novel contexts, \{napron\} was maintained with the indefinite
article -- in fact, the most conventional and idiomatized
construction -- while the sublexeme \{apron\} was used in novel contexts
cited above, such as \emph{{[}takith{]}\ldots{} ij elles of lynen
clothe} \emph{\textbf{for aprons}} and \ldots{} \emph{to thomeis hynde
y\textsuperscript{t} was my p'ntice \textbf{an apron}}. Over time,
\{apron\} and \{napron\} compete; \{apron\} keeps on increasing, in a
fashion that could be understood as the other half of Newton's first
law: once in motion, a body, or linguistic subsystem, will remain in
motion.

Semantically the new demilexeme \{apron\} must be basically similar to
traditional \{napron\}. For example, both demilexemes refer to
protective coverings, usually of cloth, though in artisanry, aprons
could be sheepskins. In the York fabric rolls -- the record of expenses
of the York expenses, including irregular expenses of masons and their
equipment -- we observe naprons used at the beginning of the fifteenth
century -- \emph{In ij pellibus emptis et datis eisdem pro naporons} `two
hides were bought and given to them to serve as aprons' (1423) -- and
then the form is \emph{aprons} at the end of the fifteenth century{]}:
\emph{\textbf{pro ij le aprons} de correo pro les setters per spacium ij
mensium, 12}d\emph{.} `for two aprons of hide for the setters for the
period of two months, 12 shillings' (1504). That is only to say that
\emph{napron} and \emph{apron} seem to have the same extension.

Still, despite the overlap in extension, there are indications that the
two sublexemes began to develop slightly different
connotations.\footnote{In a fashion consistent with Bréal's \citeyearpar[ch. 2]{breal1900} ``law of differentiation'' of synonyms.} Two facts argue for this.

The first, perhaps unexpectedly, has to do with translations of the
Bible. As is well-known, John Wyfcliffe translated much of the Bible
from the Vulgate, around 1382. (His translation was finished by his
followers after his death.) A passage of interest is Genesis 3:7 -- the
famous story of the nakedness of Adam and Eve -- for which Wycliffe (or
his followers) translated Vulgate \ldots{}\emph{cognovissent esse se
nudos consuerunt folia ficus et fecerunt sibi perizomata} as
\emph{\ldots{}and when they knew that they were naked, they sewed the
leaves of a fig tree, and made breeches to themselves}.

\largerpage
Wycliffe's Bible became the model for an extended tradition of \ili{English}
translations thereafter, but with a difference in this passage. Starting
with Tynsdale (1534), the subsequent translations have a different noun
in Genesis 3:7: \ldots{}\emph{vnderstode how that they were naked. Than
they sowed fygge leves togedder and made them apurns}. The translation
with \emph{aprons} continues through Cloverdale (1535), the Great Bible
(1540), Matthew's Bible (1549), the Catholic Bishops' Bible (1568), the
Geneva Bible (1587), and finally the King James (1604--1611). All have
\emph{aprons} (variant spellings) except for the Geneva Bible, a
retrograde Protestant Bible which returned to Wycliffe's
\emph{breeches}.


The improvised fig-leaf garment of Genesis 3:7 wasn't exactly an apron
in the sense of linen or hide aprons, but it was somewhat similar. Why
was \emph{apron} used instead of \emph{napron}? One reason might be that
\emph{apron} was the innovative form, and innovative\is{innovation} forms are more
appropriate than conventional forms for encoding semantic extensions.
There is another possibility. As we saw in the Liber Niger,
\emph{naprons} were something that would result from linen, and their
value was defined by the price of the linen used to make them. In
earlier wills \emph{naprons} were classified with other items of cloth
with different functions; \emph{naprons} belonged to the \emph{naperie}
(the collection of similar cloths) along with \emph{napkins} (same sense
as modern) and \emph{borde clothes} or \emph{table cloths}. So the
sublexeme \{napron\} emphasized the origin of the entity in cloth or
hide; secondarily, such a flat piece of material could be donned for
protection. With the sublexeme \{apron\} the dominant feature is not
that it was made from material (or hide); the dominant feature is that it is
a garment worn to provide protective covering. This difference in the
\isi{ranking} of \isi{features} -- \textsc{material} as opposed
\textsc{function} -- might be why the corrections to Wycliff's Bible used
\emph{apron.}

\largerpage
A second indication is the way items are grouped in \emph{Chesterfield wills and inventories of
household goods}. For example, from
Derbyshire \#177 (Margaret Capper, 1588) here is a partial list of items
(omitting tools and animals). Items are listed in the inventory in
natural classes. Categories are added here:

\begin{quote}
\emph{\textless{}furniture\textgreater{} /} 4 bedstids in the Chamber /
2 bedstids in the parlar with pented Cloates about them / 1 bed teaster
of Cloathe / 3 bed stids in the nether Chamber /
\emph{\textless{}bedding\textgreater{} /} 8 pillobears / 8 hand towels /
4 shietes / \emph{\textless{}garment\textgreater{} /} 1 smock / \emph{2
aperns} / 1 bruse and a grater / 1 Coat and a pear of house / 1 Gone and
a for kertle {[}?{]} / 1 buckrame savgard /
\emph{\textless{}utensils\textgreater{} /} 2 Chamber pots / 1 morter and
a Cresset / 3 Chaffindishes / 1 Skomar and a ladle of brase / 9 bear
potes and 2 black potes / 1 falling bord in the house / 3 pans 2 ketles
/ 1 basson brase / 4 brase potes
\end{quote}

Note that \emph{aprons} is listed next to smock and other garments. (The
listing of a ``bruse and grater'' below \emph{aperns} seems out of
place.) By this time, in the late sixteenth century, an \{apron\} was
classified as a garment.

Does this explain why \{apron\} continued to displace \{napron\}?
Possibly. The demilexeme \{apron\} removes aprons from the domain of the
\emph{naperie} (the collection of pieces of fabric) to the domain of
garments. The extension may be the same, but the intension changes, by
the re-\isi{ranking} of the semantic \isi{features} of the two demilexemes:
\{napron\} ranks the material over the function, whereas \{apron\},
while it does not that fabric may be involved, ranks garment and its
function of covering as more important.

There are several conclusions here. The two demilexemes have a certain
\isi{autonomy}: they have overlapping but not identical \isi{semantics}; they have
different preferred contexts in which they appear. From this it follows
that the change of \emph{napron} to \emph{apron} is not a simple
substitution of one form for the other. Next, the newer form
\emph{apron} seems not to appear in the context of an indefinite article
ahead of other contexts, as might be expected if \emph{apron} merely
replaced \emph{napron}. This again implies that \emph{apron} and
\emph{napron} are somewhat separate entities. Third, the ambiguity of
{[}anapron{]} made the change possible, but the change was the creation
of two demilexemes here, \{napron\} and \{apron\}, not a reparsing.

\section{Conclusion}\label{conclusionT}

The examples above suggest that \emph{parole} exists, that it has a role
in language. Saussure, as we saw, did his best to hide \emph{parole}
from view, but it ended up that \emph{parole} has a life of its own: it
required its own set of rules and it was maintained (the accent stayed
on the same \isi{syllable} as in \ili{Latin}) without justification in the
system.\footnote{Boris Gasparov \citeyearpar{gasparov2012} has argued that Saussure's
  thinking was more complex (and less rigidly categorial and
  structuralist) than the subsequent reception would have it (especially
  chapter 4, pp. 111--37).}  All the action of \isi{stress} in Romance was in
\emph{parole}, not \emph{système}. In other examples, we saw that
\emph{parole} is maintained from one generation to the next, not because
it is motivated by higher principles, but because it was the usage and
it was then transmitted as usage. \emph{Parole} can also shape other
changes (such as the apocope of post-tonic vowels in the transition from
\ili{Latin} to \ili{French} and the restriction on \isi{intrusive /r/} to low vowels).
Variants of lexemes, such as \{napron\} and \{apron\}, have partially
separate existences and properties, including \isi{semantics}.

\emph{Parole} is not always static and it is not one-dimensional.
\emph{Parole} is, after all, activity, and human activity implies the
possibility of more activity and other paths of activity, which may
differ from inherited activity. \emph{Parole} is habit infused with
potential.

\nocite{XYORK,XYDURHAMI,XYDURHAMIII,XCHEST,XEDWARD,Xsurrey}
\section*{Abbreviations and texts cited}
\begin{tabularx}{\linewidth}{>{\raggedright}p{.21\linewidth}Q}
\lsptoprule
Abbreviation & Explication of abbreviation\\
\midrule
1790 &   Society of Antiquaries of London, 1790. \emph{V. Liber Niger Domus Regis Edward IV. From a MS. in the Harleian Library, № 642, fol. 1--196}, {A collection of ordinances and regulations for the government of the royal household, made in divers reigns.  From King Edward III. to King William and Queen Mary. Also receipts in ancient cookery}. Printed for the Society of Antiquaries by John Nichols.
\\
\tablevspace
Chesterfield wills and inventories  &Bestall, J. M.; Fowkes, D. V., ed., a glossary by Rosemary Milward with an introduction by David Hey \& an index by Barbara Bestall. 1977. \emph{Chesterfield wills and inventories 1521–1603.} Vol. 1 (Derbyshire Record Society).  Derbyshire:  Derbyshire Record Society.\\
\tablevspace 
Durham wills and inventories &
	\emph{Wills and inventories illustrative of the history, manners, language, statistics, \&c., of the northern counties of England, from the eleventh century downwards.} 1835.  Vol. I (Publications of the Surtees Society, vol. 2).  London: J. B. Nichols \& Son;  1906.  Vol. III  (Publications of the Surtees Society, 112.).  London:  J. B. Nichols \& Son.  \\
\tablevspace 
Fabric rolls &\emph{The fabric rolls of York Minster with an appendix of illustrative documents.} 1859.  Vol. 35.  (Publications of the Surtees Society).  Durham: Published for the Society by G. Andrews.\\
\tablevspace
Surrey wills &\emph{Surrey wills}. (Archdeaconry Court, Spage Register). 1922. Vol. 5 (Surrey Record Society). Surrey:  Roworth \& Co., for the Surrey Record Society.\\
\tablevspace 
Wills and inventories of Bury St. Edmonds&\emph{Wills and inventories from the registers of the commissary of Bury St. Edmunds and the archdeacon of Sudbury.} 1850.   London:  Printed for the Camden Society.\\
\tablevspace 
York wills &\emph{Testamenta eboracensia, or Wills registered at York: illustrative of the history, manners, language, statistics, \&c., of the province of York, from the year MCCC downwards.}  London:  J. B. Nichols \& Son, 1836--1902.\\
\lspbottomrule
\end{tabularx}
\tablevspace

%\section*{Acknowledgements}
\newpage 

\printbibliography[heading=subbibliography,notkeyword=this]


\end{document}
