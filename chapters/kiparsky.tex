\documentclass[output=paper,
modfonts
]{LSP/langsci}


%% add all extra packages you need to load to this file 
% \usepackage{todo} %% removed,cna use todonotes instead. % Jason reactivated
% \usepackage{graphicx} % not needed because forest loads tikz, which loads graphicx
\usepackage{tabularx}
\usepackage{amsmath} 
\usepackage{multicol}
\usepackage{lipsum}
\usepackage{longtable}
\usepackage{booktabs}
\usepackage[normalem]{ulem}
%\usepackage{tikz} % not needed because forest loads tikz
\usepackage{phonrule} % for SPE-style phonological rules
\usepackage{pst-all} % loads the main pstricks tools; for arrow diagrams in Hale.tex
%\usepackage{leipzig} % for gloss abbreviations
\usepackage[% for automatic cross-referencing
compress,%
capitalize,% labels are always capitalized in LSP style
noabbrev]% labels are always spelled out in LSP style
{cleveref}

% based on http://tex.stackexchange.com/a/318983/42880 for using gb4e examples with cleveref
\crefname{xnumi}{}{}
\creflabelformat{xnumi}{(#2#1#3)}
\crefrangeformat{xnumi}{(#3#1#4)--(#5#2#6)}
\crefname{xnumii}{}{}
\creflabelformat{xnumii}{(#2#1#3)}
\crefrangeformat{xnumii}{(#3#1#4)--(#5#2#6)}

%\usepackage[notcite,notref]{showkeys} %%removed, not helping CB.
%\usepackage{showidx} %%remove for final compiling - shows index keys at top of page.
 
\usepackage{langsci/styles/langsci-gb4e}  
 \usepackage{pifont}
% % OT tableaux                                                
% \usepackage{pstricks,colortab}  
\usepackage{multirow} % used in OT tableaux
\usepackage{rotating} %needed for angled text%
\usepackage{colortbl} % for cell shading
 
 \usepackage{avm}  
\usepackage[linguistics]{forest} 
\usetikzlibrary{matrix,fit} % for matrix of nodes in Kaisse and Bat-El


\usepackage{hhline}
\newcommand{\cgr}{\cellcolor[gray]{0.8}}
\newcommand{\cn}{\centering}



\newcommand{\reff}[1]{(\ref{#1})}
%\usepackage{newtxtext,newtxmath}


%\usepackage[normalem] {ulem}
\usepackage{qtree}
%\usepackage{natbib}
%\usepackage{tikz}
%\usepackage{gb4e}
\usepackage{phonrule}  
%\bibliographystyle{humannat}



\usepackage{minibox}

%\include{psheader-metr}

\def\bl#1{$_{\textrm{{\footnotesize #1}}}$}

%%add all your local new commands to this file

\newcommand{\form}[1]{\mbox{\emph{#1}}}
\newcommand{\uf}[1]{\mbox{/#1/}}

% borrowed from expex and converted from plan tex to latex
\newcommand{\judge}[1]{{\upshape #1\hspace{0.1em}}}
\newcommand{\ljudge}[1]{\makebox[0pt][r]{\judge{#1}}}

\newcommand\tikzmark[1]{\tikz[remember picture, baseline=(#1.base)] \node[anchor=base,inner sep=0pt, outer sep=0pt] (#1) {#1};} % for adding decorations, arrows, lines, etc. to text
\newcommand\tikzmarknamed[2]{\tikz[remember picture, baseline=(#1.base)] \node[anchor=base,inner sep=0pt, outer sep=0pt] (#1) {#2};} % for adding decorations, arrows, lines, etc. to text
\newcommand\tikzmarkfullnamed[2]{\tikz[remember picture, baseline=(#1.base)] \node[anchor=base,inner sep=0pt, outer sep=0pt] (#1) {\vphantom{X}#2};} % for adding decorations, arrows, lines, etc. to text; this one works best for decorations above a line of text because it adds in the heigh of a capital letter and takes two arguments - one for the node name and one for the printed text

\newcommand{\sub}[1]{$_{\text{#1}}$} % for non-math subscripts
\newcommand{\subit}[1]{\sub{\textit{#1}}} % for italics non-math subscripts
\newcommand{\1}{\rlap{$'$}\xspace} % for the prime in X' (the \rlap command allows the prime to be ignored for horizontal spacing in trees, and the \xspace command allows you to use this in normal text without adding \ afterwards). This isn't crucial, but it helps the formatting to look a little better.

% Aissen:
\newcommand\tikzmarkfull[1]{\tikz[remember picture, baseline=(#1.base)] \node[anchor=base,inner sep=0pt, outer sep=0pt] (#1) {\vphantom{X}#1};} % for adding decorations, arrows, lines, etc. to text; this one works best for decorations above a line of text because it adds in the heigh of a capital letter and takes one argument that serves as the name and the printed text
\newcommand{\bridgeover}[2]{% for a line that creates a bridge over text, connecting two nodes
	\begin{tikzpicture}[remember picture,overlay]
	\draw[thick,shorten >=3pt,shorten <=3pt] (#1.north) |- +(0ex,2.5ex) -| (#2.north);
	\end{tikzpicture}
}
\newcommand{\bridgeoverht}[3]{% for a line that creates a bridge over text, connecting two nodes and specifing the height of the bridge
	\begin{tikzpicture}[remember picture,overlay]
	\draw[thick,shorten >=3pt,shorten <=3pt] (#2.north) |- +(0ex,#1) -| (#3.north);
	\end{tikzpicture}
}
\newcommand{\bridgeoverex}{\vspace*{3ex}} % place before an example that has a \bridgeover so that there is enough vertical space for it

% Chung:
\newcommand{\lefttabular}[1]{\begin{tabular}{p{0.5in}}#1\end{tabular}}

% Kaisse:
\newcommand{\mgmorph}[1]{|(#1)| {#1}}
\newcommand{\mgone}[2][$\times$]{\node at (#2.base) [above=2ex] (1#2) {\vphantom{X}#1};}
\newcommand{\mgtwo}[2][$\times$]{\mgone{#2} \node at (#2.base) [above=4.5ex] (2#2) {\vphantom{X}#1};}
\newcommand{\mgthree}[2][$\times$]{\mgtwo{#2} \node at (#2.base) [above=7ex] (3#2) {\vphantom{X}#1};}
\newcommand{\mgftl}[1]{\node at (1#1) [left] {(};}
\newcommand{\mgftr}[1]{\node at (1#1) [right] {)};}
\newcommand{\mgfoot}[2]{\mgftl{#1}\mgftr{#2}}
\newcommand{\mgldelim}[2]{\node at (#2.west) [left,inner sep = 0pt, outer sep = 0pt] {#1};}
\newcommand{\mgrdelim}[2]{\node at (#2.east) [right,inner sep = 0pt, outer sep = 0pt] {#1};}

\newcommand{\squish}{\hspace*{-3pt}}

% Kavitskaya:
\newcommand{\assoc}[2]{\draw (#1.south) -- (#2.north);}
\newcolumntype{L}{>{\raggedright\arraybackslash}X}

% Lepic & Padden:
\newcommand{\fitpic}[1]{\resizebox{\hsize}{!}{\includegraphics{#1}}} % from http://tex.stackexchange.com/a/148965/42880
\newcommand{\signpic}[1]{\includegraphics[width=1.36in]{#1}}
\newcolumntype{C}{>{\centering\arraybackslash}X}

% Spencer:

\newcommand{\textex}[1]{\textit{#1}\xspace}
\newcommand{\lxm}[1]{\textsc{#1}\xspace}

% Thrainsson:

\renewcommand{\textasciitilde}{\char`~} % for use with TTF/OTF fonts (see comments on http://tex.stackexchange.com/a/317/42880)
\newcommand{\tikzarrow}[2]{% for an arrow connecting two nodes
\begin{tikzpicture}[remember picture,overlay]
\draw[thick,shorten >=3pt,shorten <=3pt,->,>=stealth] (#1) -- (#2);
\end{tikzpicture}
}

\newlength{\padding}
\setlength{\padding}{0.5em}
\newcommand{\lesspadding}{\hspace*{-\padding}}
\newcommand{\feat}[1]{\lesspadding#1\lesspadding}

% Hammond

\usepackage[]{graphicx}\usepackage[]{xcolor}
%% maxwidth is the original width if it is less than linewidth
%% otherwise use linewidth (to make sure the graphics do not exceed the margin)
\makeatletter
\def\maxwidth{ %
  \ifdim\Gin@nat@width>\linewidth
    \linewidth
  \else
    \Gin@nat@width
  \fi
}
\makeatother

\definecolor{fgcolor}{rgb}{0.345, 0.345, 0.345}
\newcommand{\hlnum}[1]{\textcolor[rgb]{0.686,0.059,0.569}{#1}}%
\newcommand{\hlstr}[1]{\textcolor[rgb]{0.192,0.494,0.8}{#1}}%
\newcommand{\hlcom}[1]{\textcolor[rgb]{0.678,0.584,0.686}{\textit{#1}}}%
\newcommand{\hlopt}[1]{\textcolor[rgb]{0,0,0}{#1}}%
\newcommand{\hlstd}[1]{\textcolor[rgb]{0.345,0.345,0.345}{#1}}%
\newcommand{\hlkwa}[1]{\textcolor[rgb]{0.161,0.373,0.58}{\textbf{#1}}}%
\newcommand{\hlkwb}[1]{\textcolor[rgb]{0.69,0.353,0.396}{#1}}%
\newcommand{\hlkwc}[1]{\textcolor[rgb]{0.333,0.667,0.333}{#1}}%
\newcommand{\hlkwd}[1]{\textcolor[rgb]{0.737,0.353,0.396}{\textbf{#1}}}%
\let\hlipl\hlkwb

\usepackage{framed}
\makeatletter
\newenvironment{kframe}{%
 \def\at@end@of@kframe{}%
 \ifinner\ifhmode%
  \def\at@end@of@kframe{\end{minipage}}%
  \begin{minipage}{\columnwidth}%
 \fi\fi%
 \def\FrameCommand##1{\hskip\@totalleftmargin \hskip-\fboxsep
 \colorbox{shadecolor}{##1}\hskip-\fboxsep
     % There is no \\@totalrightmargin, so:
     \hskip-\linewidth \hskip-\@totalleftmargin \hskip\columnwidth}%
 \MakeFramed {\advance\hsize-\width
   \@totalleftmargin\z@ \linewidth\hsize
   \@setminipage}}%
 {\par\unskip\endMakeFramed%
 \at@end@of@kframe}
\makeatother

\definecolor{shadecolor}{rgb}{.97, .97, .97}
\definecolor{messagecolor}{rgb}{0, 0, 0}
\definecolor{warningcolor}{rgb}{1, 0, 1}
\definecolor{errorcolor}{rgb}{1, 0, 0}
\newenvironment{knitrout}{}{} % an empty environment to be redefined in TeX

\usepackage{alltt}

%revised version started: 12/17/16

%NEEDS: allbib.bib - already added to the master bibliography file.
%cited references only: bibexport -o mhTMP.bib main1-blx.aux
%PLUS sramh-img*, sramh.tex

%added stuff
\newcommand{\add}[1]{\textcolor{blue}{#1}}
%deleted stuff
\newcommand{\del}[1]{\textcolor{red}{(removed: #1)}}
%uncomment these to turn off colors
\renewcommand{\add}[1]{#1}
\renewcommand{\del}[1]{}

%shortcuts
\newcommand{\w}{\ili{Welsh}}
\newcommand{\e}{\ili{English}}
\newcommand{\io}{Input Optimization}




 \newcommand{\hand}{\ding{43}}
% \newcommand{\rot}[1]{\begin{rotate}{90}#1\end{rotate}} %shortcut for angled text%  
% \newcommand{\rotcon}[1]{\raisebox{-5ex}{\hspace*{1ex}\rot{\hspace*{1ex}#1}}}

%% add all extra packages you need to load to this file 
% \usepackage{todo} %% removed,cna use todonotes instead. % Jason reactivated
% \usepackage{graphicx} % not needed because forest loads tikz, which loads graphicx
\usepackage{tabularx}
\usepackage{amsmath} 
\usepackage{multicol}
\usepackage{lipsum}
\usepackage{longtable}
\usepackage{booktabs}
\usepackage[normalem]{ulem}
%\usepackage{tikz} % not needed because forest loads tikz
\usepackage{phonrule} % for SPE-style phonological rules
\usepackage{pst-all} % loads the main pstricks tools; for arrow diagrams in Hale.tex
%\usepackage{leipzig} % for gloss abbreviations
\usepackage[% for automatic cross-referencing
compress,%
capitalize,% labels are always capitalized in LSP style
noabbrev]% labels are always spelled out in LSP style
{cleveref}

% based on http://tex.stackexchange.com/a/318983/42880 for using gb4e examples with cleveref
\crefname{xnumi}{}{}
\creflabelformat{xnumi}{(#2#1#3)}
\crefrangeformat{xnumi}{(#3#1#4)--(#5#2#6)}
\crefname{xnumii}{}{}
\creflabelformat{xnumii}{(#2#1#3)}
\crefrangeformat{xnumii}{(#3#1#4)--(#5#2#6)}

%\usepackage[notcite,notref]{showkeys} %%removed, not helping CB.
%\usepackage{showidx} %%remove for final compiling - shows index keys at top of page.
 
\usepackage{langsci/styles/langsci-gb4e}  
 \usepackage{pifont}
% % OT tableaux                                                
% \usepackage{pstricks,colortab}  
\usepackage{multirow} % used in OT tableaux
\usepackage{rotating} %needed for angled text%
\usepackage{colortbl} % for cell shading
 
 \usepackage{avm}  
\usepackage[linguistics]{forest} 
\usetikzlibrary{matrix,fit} % for matrix of nodes in Kaisse and Bat-El


\usepackage{hhline}
\newcommand{\cgr}{\cellcolor[gray]{0.8}}
\newcommand{\cn}{\centering}



\newcommand{\reff}[1]{(\ref{#1})}
%\usepackage{newtxtext,newtxmath}


%\usepackage[normalem] {ulem}
\usepackage{qtree}
%\usepackage{natbib}
%\usepackage{tikz}
%\usepackage{gb4e}
\usepackage{phonrule}  
%\bibliographystyle{humannat}



\usepackage{minibox}

%\include{psheader-metr}

\def\bl#1{$_{\textrm{{\footnotesize #1}}}$}
\usepackage{arydshln}
\usepackage{rotating}

%%add all your local new commands to this file

\newcommand{\form}[1]{\mbox{\emph{#1}}}
\newcommand{\uf}[1]{\mbox{/#1/}}

% borrowed from expex and converted from plan tex to latex
\newcommand{\judge}[1]{{\upshape #1\hspace{0.1em}}}
\newcommand{\ljudge}[1]{\makebox[0pt][r]{\judge{#1}}}

\newcommand\tikzmark[1]{\tikz[remember picture, baseline=(#1.base)] \node[anchor=base,inner sep=0pt, outer sep=0pt] (#1) {#1};} % for adding decorations, arrows, lines, etc. to text
\newcommand\tikzmarknamed[2]{\tikz[remember picture, baseline=(#1.base)] \node[anchor=base,inner sep=0pt, outer sep=0pt] (#1) {#2};} % for adding decorations, arrows, lines, etc. to text
\newcommand\tikzmarkfullnamed[2]{\tikz[remember picture, baseline=(#1.base)] \node[anchor=base,inner sep=0pt, outer sep=0pt] (#1) {\vphantom{X}#2};} % for adding decorations, arrows, lines, etc. to text; this one works best for decorations above a line of text because it adds in the heigh of a capital letter and takes two arguments - one for the node name and one for the printed text

\newcommand{\sub}[1]{$_{\text{#1}}$} % for non-math subscripts
\newcommand{\subit}[1]{\sub{\textit{#1}}} % for italics non-math subscripts
\newcommand{\1}{\rlap{$'$}\xspace} % for the prime in X' (the \rlap command allows the prime to be ignored for horizontal spacing in trees, and the \xspace command allows you to use this in normal text without adding \ afterwards). This isn't crucial, but it helps the formatting to look a little better.

% Aissen:
\newcommand\tikzmarkfull[1]{\tikz[remember picture, baseline=(#1.base)] \node[anchor=base,inner sep=0pt, outer sep=0pt] (#1) {\vphantom{X}#1};} % for adding decorations, arrows, lines, etc. to text; this one works best for decorations above a line of text because it adds in the heigh of a capital letter and takes one argument that serves as the name and the printed text
\newcommand{\bridgeover}[2]{% for a line that creates a bridge over text, connecting two nodes
	\begin{tikzpicture}[remember picture,overlay]
	\draw[thick,shorten >=3pt,shorten <=3pt] (#1.north) |- +(0ex,2.5ex) -| (#2.north);
	\end{tikzpicture}
}
\newcommand{\bridgeoverht}[3]{% for a line that creates a bridge over text, connecting two nodes and specifing the height of the bridge
	\begin{tikzpicture}[remember picture,overlay]
	\draw[thick,shorten >=3pt,shorten <=3pt] (#2.north) |- +(0ex,#1) -| (#3.north);
	\end{tikzpicture}
}
\newcommand{\bridgeoverex}{\vspace*{3ex}} % place before an example that has a \bridgeover so that there is enough vertical space for it

% Chung:
\newcommand{\lefttabular}[1]{\begin{tabular}{p{0.5in}}#1\end{tabular}}

% Kaisse:
\newcommand{\mgmorph}[1]{|(#1)| {#1}}
\newcommand{\mgone}[2][$\times$]{\node at (#2.base) [above=2ex] (1#2) {\vphantom{X}#1};}
\newcommand{\mgtwo}[2][$\times$]{\mgone{#2} \node at (#2.base) [above=4.5ex] (2#2) {\vphantom{X}#1};}
\newcommand{\mgthree}[2][$\times$]{\mgtwo{#2} \node at (#2.base) [above=7ex] (3#2) {\vphantom{X}#1};}
\newcommand{\mgftl}[1]{\node at (1#1) [left] {(};}
\newcommand{\mgftr}[1]{\node at (1#1) [right] {)};}
\newcommand{\mgfoot}[2]{\mgftl{#1}\mgftr{#2}}
\newcommand{\mgldelim}[2]{\node at (#2.west) [left,inner sep = 0pt, outer sep = 0pt] {#1};}
\newcommand{\mgrdelim}[2]{\node at (#2.east) [right,inner sep = 0pt, outer sep = 0pt] {#1};}

\newcommand{\squish}{\hspace*{-3pt}}

% Kavitskaya:
\newcommand{\assoc}[2]{\draw (#1.south) -- (#2.north);}
\newcolumntype{L}{>{\raggedright\arraybackslash}X}

% Lepic & Padden:
\newcommand{\fitpic}[1]{\resizebox{\hsize}{!}{\includegraphics{#1}}} % from http://tex.stackexchange.com/a/148965/42880
\newcommand{\signpic}[1]{\includegraphics[width=1.36in]{#1}}
\newcolumntype{C}{>{\centering\arraybackslash}X}

% Spencer:

\newcommand{\textex}[1]{\textit{#1}\xspace}
\newcommand{\lxm}[1]{\textsc{#1}\xspace}

% Thrainsson:

\renewcommand{\textasciitilde}{\char`~} % for use with TTF/OTF fonts (see comments on http://tex.stackexchange.com/a/317/42880)
\newcommand{\tikzarrow}[2]{% for an arrow connecting two nodes
\begin{tikzpicture}[remember picture,overlay]
\draw[thick,shorten >=3pt,shorten <=3pt,->,>=stealth] (#1) -- (#2);
\end{tikzpicture}
}

\newlength{\padding}
\setlength{\padding}{0.5em}
\newcommand{\lesspadding}{\hspace*{-\padding}}
\newcommand{\feat}[1]{\lesspadding#1\lesspadding}

% Hammond

\usepackage[]{graphicx}\usepackage[]{xcolor}
%% maxwidth is the original width if it is less than linewidth
%% otherwise use linewidth (to make sure the graphics do not exceed the margin)
\makeatletter
\def\maxwidth{ %
  \ifdim\Gin@nat@width>\linewidth
    \linewidth
  \else
    \Gin@nat@width
  \fi
}
\makeatother

\definecolor{fgcolor}{rgb}{0.345, 0.345, 0.345}
\newcommand{\hlnum}[1]{\textcolor[rgb]{0.686,0.059,0.569}{#1}}%
\newcommand{\hlstr}[1]{\textcolor[rgb]{0.192,0.494,0.8}{#1}}%
\newcommand{\hlcom}[1]{\textcolor[rgb]{0.678,0.584,0.686}{\textit{#1}}}%
\newcommand{\hlopt}[1]{\textcolor[rgb]{0,0,0}{#1}}%
\newcommand{\hlstd}[1]{\textcolor[rgb]{0.345,0.345,0.345}{#1}}%
\newcommand{\hlkwa}[1]{\textcolor[rgb]{0.161,0.373,0.58}{\textbf{#1}}}%
\newcommand{\hlkwb}[1]{\textcolor[rgb]{0.69,0.353,0.396}{#1}}%
\newcommand{\hlkwc}[1]{\textcolor[rgb]{0.333,0.667,0.333}{#1}}%
\newcommand{\hlkwd}[1]{\textcolor[rgb]{0.737,0.353,0.396}{\textbf{#1}}}%
\let\hlipl\hlkwb

\usepackage{framed}
\makeatletter
\newenvironment{kframe}{%
 \def\at@end@of@kframe{}%
 \ifinner\ifhmode%
  \def\at@end@of@kframe{\end{minipage}}%
  \begin{minipage}{\columnwidth}%
 \fi\fi%
 \def\FrameCommand##1{\hskip\@totalleftmargin \hskip-\fboxsep
 \colorbox{shadecolor}{##1}\hskip-\fboxsep
     % There is no \\@totalrightmargin, so:
     \hskip-\linewidth \hskip-\@totalleftmargin \hskip\columnwidth}%
 \MakeFramed {\advance\hsize-\width
   \@totalleftmargin\z@ \linewidth\hsize
   \@setminipage}}%
 {\par\unskip\endMakeFramed%
 \at@end@of@kframe}
\makeatother

\definecolor{shadecolor}{rgb}{.97, .97, .97}
\definecolor{messagecolor}{rgb}{0, 0, 0}
\definecolor{warningcolor}{rgb}{1, 0, 1}
\definecolor{errorcolor}{rgb}{1, 0, 0}
\newenvironment{knitrout}{}{} % an empty environment to be redefined in TeX

\usepackage{alltt}

%revised version started: 12/17/16

%NEEDS: allbib.bib - already added to the master bibliography file.
%cited references only: bibexport -o mhTMP.bib main1-blx.aux
%PLUS sramh-img*, sramh.tex

%added stuff
\newcommand{\add}[1]{\textcolor{blue}{#1}}
%deleted stuff
\newcommand{\del}[1]{\textcolor{red}{(removed: #1)}}
%uncomment these to turn off colors
\renewcommand{\add}[1]{#1}
\renewcommand{\del}[1]{}

%shortcuts
\newcommand{\w}{\ili{Welsh}}
\newcommand{\e}{\ili{English}}
\newcommand{\io}{Input Optimization}




 \newcommand{\hand}{\ding{43}}
% \newcommand{\rot}[1]{\begin{rotate}{90}#1\end{rotate}} %shortcut for angled text%  
% \newcommand{\rotcon}[1]{\raisebox{-5ex}{\hspace*{1ex}\rot{\hspace*{1ex}#1}}}

%% add all extra packages you need to load to this file 
% \usepackage{todo} %% removed,cna use todonotes instead. % Jason reactivated
% \usepackage{graphicx} % not needed because forest loads tikz, which loads graphicx
\usepackage{tabularx}
\usepackage{amsmath} 
\usepackage{multicol}
\usepackage{lipsum}
\usepackage{longtable}
\usepackage{booktabs}
\usepackage[normalem]{ulem}
%\usepackage{tikz} % not needed because forest loads tikz
\usepackage{phonrule} % for SPE-style phonological rules
\usepackage{pst-all} % loads the main pstricks tools; for arrow diagrams in Hale.tex
%\usepackage{leipzig} % for gloss abbreviations
\usepackage[% for automatic cross-referencing
compress,%
capitalize,% labels are always capitalized in LSP style
noabbrev]% labels are always spelled out in LSP style
{cleveref}

% based on http://tex.stackexchange.com/a/318983/42880 for using gb4e examples with cleveref
\crefname{xnumi}{}{}
\creflabelformat{xnumi}{(#2#1#3)}
\crefrangeformat{xnumi}{(#3#1#4)--(#5#2#6)}
\crefname{xnumii}{}{}
\creflabelformat{xnumii}{(#2#1#3)}
\crefrangeformat{xnumii}{(#3#1#4)--(#5#2#6)}

%\usepackage[notcite,notref]{showkeys} %%removed, not helping CB.
%\usepackage{showidx} %%remove for final compiling - shows index keys at top of page.
 
\usepackage{langsci/styles/langsci-gb4e}  
 \usepackage{pifont}
% % OT tableaux                                                
% \usepackage{pstricks,colortab}  
\usepackage{multirow} % used in OT tableaux
\usepackage{rotating} %needed for angled text%
\usepackage{colortbl} % for cell shading
 
 \usepackage{avm}  
\usepackage[linguistics]{forest} 
\usetikzlibrary{matrix,fit} % for matrix of nodes in Kaisse and Bat-El


\usepackage{hhline}
\newcommand{\cgr}{\cellcolor[gray]{0.8}}
\newcommand{\cn}{\centering}



\newcommand{\reff}[1]{(\ref{#1})}
%\usepackage{newtxtext,newtxmath}


%\usepackage[normalem] {ulem}
\usepackage{qtree}
%\usepackage{natbib}
%\usepackage{tikz}
%\usepackage{gb4e}
\usepackage{phonrule}  
%\bibliographystyle{humannat}



\usepackage{minibox}

%\include{psheader-metr}

\def\bl#1{$_{\textrm{{\footnotesize #1}}}$}
\usepackage{arydshln}
\usepackage{rotating}

%%add all your local new commands to this file

\newcommand{\form}[1]{\mbox{\emph{#1}}}
\newcommand{\uf}[1]{\mbox{/#1/}}

% borrowed from expex and converted from plan tex to latex
\newcommand{\judge}[1]{{\upshape #1\hspace{0.1em}}}
\newcommand{\ljudge}[1]{\makebox[0pt][r]{\judge{#1}}}

\newcommand\tikzmark[1]{\tikz[remember picture, baseline=(#1.base)] \node[anchor=base,inner sep=0pt, outer sep=0pt] (#1) {#1};} % for adding decorations, arrows, lines, etc. to text
\newcommand\tikzmarknamed[2]{\tikz[remember picture, baseline=(#1.base)] \node[anchor=base,inner sep=0pt, outer sep=0pt] (#1) {#2};} % for adding decorations, arrows, lines, etc. to text
\newcommand\tikzmarkfullnamed[2]{\tikz[remember picture, baseline=(#1.base)] \node[anchor=base,inner sep=0pt, outer sep=0pt] (#1) {\vphantom{X}#2};} % for adding decorations, arrows, lines, etc. to text; this one works best for decorations above a line of text because it adds in the heigh of a capital letter and takes two arguments - one for the node name and one for the printed text

\newcommand{\sub}[1]{$_{\text{#1}}$} % for non-math subscripts
\newcommand{\subit}[1]{\sub{\textit{#1}}} % for italics non-math subscripts
\newcommand{\1}{\rlap{$'$}\xspace} % for the prime in X' (the \rlap command allows the prime to be ignored for horizontal spacing in trees, and the \xspace command allows you to use this in normal text without adding \ afterwards). This isn't crucial, but it helps the formatting to look a little better.

% Aissen:
\newcommand\tikzmarkfull[1]{\tikz[remember picture, baseline=(#1.base)] \node[anchor=base,inner sep=0pt, outer sep=0pt] (#1) {\vphantom{X}#1};} % for adding decorations, arrows, lines, etc. to text; this one works best for decorations above a line of text because it adds in the heigh of a capital letter and takes one argument that serves as the name and the printed text
\newcommand{\bridgeover}[2]{% for a line that creates a bridge over text, connecting two nodes
	\begin{tikzpicture}[remember picture,overlay]
	\draw[thick,shorten >=3pt,shorten <=3pt] (#1.north) |- +(0ex,2.5ex) -| (#2.north);
	\end{tikzpicture}
}
\newcommand{\bridgeoverht}[3]{% for a line that creates a bridge over text, connecting two nodes and specifing the height of the bridge
	\begin{tikzpicture}[remember picture,overlay]
	\draw[thick,shorten >=3pt,shorten <=3pt] (#2.north) |- +(0ex,#1) -| (#3.north);
	\end{tikzpicture}
}
\newcommand{\bridgeoverex}{\vspace*{3ex}} % place before an example that has a \bridgeover so that there is enough vertical space for it

% Chung:
\newcommand{\lefttabular}[1]{\begin{tabular}{p{0.5in}}#1\end{tabular}}

% Kaisse:
\newcommand{\mgmorph}[1]{|(#1)| {#1}}
\newcommand{\mgone}[2][$\times$]{\node at (#2.base) [above=2ex] (1#2) {\vphantom{X}#1};}
\newcommand{\mgtwo}[2][$\times$]{\mgone{#2} \node at (#2.base) [above=4.5ex] (2#2) {\vphantom{X}#1};}
\newcommand{\mgthree}[2][$\times$]{\mgtwo{#2} \node at (#2.base) [above=7ex] (3#2) {\vphantom{X}#1};}
\newcommand{\mgftl}[1]{\node at (1#1) [left] {(};}
\newcommand{\mgftr}[1]{\node at (1#1) [right] {)};}
\newcommand{\mgfoot}[2]{\mgftl{#1}\mgftr{#2}}
\newcommand{\mgldelim}[2]{\node at (#2.west) [left,inner sep = 0pt, outer sep = 0pt] {#1};}
\newcommand{\mgrdelim}[2]{\node at (#2.east) [right,inner sep = 0pt, outer sep = 0pt] {#1};}

\newcommand{\squish}{\hspace*{-3pt}}

% Kavitskaya:
\newcommand{\assoc}[2]{\draw (#1.south) -- (#2.north);}
\newcolumntype{L}{>{\raggedright\arraybackslash}X}

% Lepic & Padden:
\newcommand{\fitpic}[1]{\resizebox{\hsize}{!}{\includegraphics{#1}}} % from http://tex.stackexchange.com/a/148965/42880
\newcommand{\signpic}[1]{\includegraphics[width=1.36in]{#1}}
\newcolumntype{C}{>{\centering\arraybackslash}X}

% Spencer:

\newcommand{\textex}[1]{\textit{#1}\xspace}
\newcommand{\lxm}[1]{\textsc{#1}\xspace}

% Thrainsson:

\renewcommand{\textasciitilde}{\char`~} % for use with TTF/OTF fonts (see comments on http://tex.stackexchange.com/a/317/42880)
\newcommand{\tikzarrow}[2]{% for an arrow connecting two nodes
\begin{tikzpicture}[remember picture,overlay]
\draw[thick,shorten >=3pt,shorten <=3pt,->,>=stealth] (#1) -- (#2);
\end{tikzpicture}
}

\newlength{\padding}
\setlength{\padding}{0.5em}
\newcommand{\lesspadding}{\hspace*{-\padding}}
\newcommand{\feat}[1]{\lesspadding#1\lesspadding}

% Hammond

\usepackage[]{graphicx}\usepackage[]{xcolor}
%% maxwidth is the original width if it is less than linewidth
%% otherwise use linewidth (to make sure the graphics do not exceed the margin)
\makeatletter
\def\maxwidth{ %
  \ifdim\Gin@nat@width>\linewidth
    \linewidth
  \else
    \Gin@nat@width
  \fi
}
\makeatother

\definecolor{fgcolor}{rgb}{0.345, 0.345, 0.345}
\newcommand{\hlnum}[1]{\textcolor[rgb]{0.686,0.059,0.569}{#1}}%
\newcommand{\hlstr}[1]{\textcolor[rgb]{0.192,0.494,0.8}{#1}}%
\newcommand{\hlcom}[1]{\textcolor[rgb]{0.678,0.584,0.686}{\textit{#1}}}%
\newcommand{\hlopt}[1]{\textcolor[rgb]{0,0,0}{#1}}%
\newcommand{\hlstd}[1]{\textcolor[rgb]{0.345,0.345,0.345}{#1}}%
\newcommand{\hlkwa}[1]{\textcolor[rgb]{0.161,0.373,0.58}{\textbf{#1}}}%
\newcommand{\hlkwb}[1]{\textcolor[rgb]{0.69,0.353,0.396}{#1}}%
\newcommand{\hlkwc}[1]{\textcolor[rgb]{0.333,0.667,0.333}{#1}}%
\newcommand{\hlkwd}[1]{\textcolor[rgb]{0.737,0.353,0.396}{\textbf{#1}}}%
\let\hlipl\hlkwb

\usepackage{framed}
\makeatletter
\newenvironment{kframe}{%
 \def\at@end@of@kframe{}%
 \ifinner\ifhmode%
  \def\at@end@of@kframe{\end{minipage}}%
  \begin{minipage}{\columnwidth}%
 \fi\fi%
 \def\FrameCommand##1{\hskip\@totalleftmargin \hskip-\fboxsep
 \colorbox{shadecolor}{##1}\hskip-\fboxsep
     % There is no \\@totalrightmargin, so:
     \hskip-\linewidth \hskip-\@totalleftmargin \hskip\columnwidth}%
 \MakeFramed {\advance\hsize-\width
   \@totalleftmargin\z@ \linewidth\hsize
   \@setminipage}}%
 {\par\unskip\endMakeFramed%
 \at@end@of@kframe}
\makeatother

\definecolor{shadecolor}{rgb}{.97, .97, .97}
\definecolor{messagecolor}{rgb}{0, 0, 0}
\definecolor{warningcolor}{rgb}{1, 0, 1}
\definecolor{errorcolor}{rgb}{1, 0, 0}
\newenvironment{knitrout}{}{} % an empty environment to be redefined in TeX

\usepackage{alltt}

%revised version started: 12/17/16

%NEEDS: allbib.bib - already added to the master bibliography file.
%cited references only: bibexport -o mhTMP.bib main1-blx.aux
%PLUS sramh-img*, sramh.tex

%added stuff
\newcommand{\add}[1]{\textcolor{blue}{#1}}
%deleted stuff
\newcommand{\del}[1]{\textcolor{red}{(removed: #1)}}
%uncomment these to turn off colors
\renewcommand{\add}[1]{#1}
\renewcommand{\del}[1]{}

%shortcuts
\newcommand{\w}{\ili{Welsh}}
\newcommand{\e}{\ili{English}}
\newcommand{\io}{Input Optimization}




 \newcommand{\hand}{\ding{43}}
% \newcommand{\rot}[1]{\begin{rotate}{90}#1\end{rotate}} %shortcut for angled text%  
% \newcommand{\rotcon}[1]{\raisebox{-5ex}{\hspace*{1ex}\rot{\hspace*{1ex}#1}}}

%\input{localpackages.tex}
\usepackage{arydshln}
\usepackage{rotating}

%\input{localcommands.tex}
\newcommand{\tworow}[1]{\multirow{2}{*}{#1}}


\newcommand{\tworow}[1]{\multirow{2}{*}{#1}}


\newcommand{\tworow}[1]{\multirow{2}{*}{#1}}



\ChapterDOI{10.5281/zenodo.495452}
\author{Paul Kiparsky\affiliation{Stanford University}}

\title{Nominal verbs and transitive nouns: Vindicating lexicalism}

\abstract{Event nominalizations and agent nominalizations provide evidence that all affixation is
morphological, and that phrasal categories are projected from words in the syntax. Departing
from both transformational and earlier lexicalist approaches to nominalizations, I first argue
on the basis of English and Finnish evidence that gerunds are not DPs built on heads that embed
an extended verbal projection \citep{baker2009,kornfilt2011}, but IPs
that need Case.  They are categorially verbal at all levels of the syntax, including having
structural subjects rather than possessor specifiers.  Their nominal behavior is entirely due
to the unvalued Case feature borne by their Infl head, which they share with all participial
verb forms.  I then argue that agent nominalizations are categorially nominal at all levels of
the syntax, and that the verb-like case assignment of transitive\is{transitivity} agent nominalizations is due
to the verbal Aspect feature borne by their nominalizing head.  Vedic Sanskrit, Northern
Paiute, and Sakha evidence is shown to favor this analysis over B\&V's analysis of intransitive
agent nominalizations as nominal equivalents of Voice heads and transitive agent
nominalizations as Aspect heads.  The two ``mixed" categories -- gerunds and transitive
nominalizations -- thus prove to be formally duals:  respectively verbs with Case and nouns with
Aspect.
}

%\usepackage{needspace}
%\usepackage{tikz}       %these added for rounded boxes
%\usetikzlibrary{shapes,decorations}
%\usepackage{pgfplots}
\usepackage{qtree}
\usepackage{tree-dvips}

 \newcommand{\rf}[1]{(\ref{#1})}
 \newcommand{\rfa}[2]{(\ref{#1}{#2})}


%\def\ā{\textacutemacron{a}}
%\def\R{{r̥}}
%\def\I*{\'{ī}}
%\def\U*{\'{ū}}
%\def\tsr#1{\textsubring{#1}}
\def\tam#1{\textacutemacron{#1}}
\def\ts#1{\textsuperscript{#1}}
\def\trf#1{$_{\textrm{\scriptsize{#1}}}$}
\def\urf#1{$^{\textrm{\scriptsize{#1}}}$}

\def\newthreepar#1#2#3{\noindent
\begin{tabular}[t]{@{}p{.25\textwidth}@{\hspace{.02\textwidth}}p{.30\textwidth}@{\hspace{.02\textwidth}}p{.40\textwidth}	@{}}
\textit{#1} & #2 & #3
\end{tabular}}

\begin{document}

\maketitle


\section{Nominalizations}\label{eventnomsection}
\largerpage[-1]
The earliest generative work derived all nominalizations syntactically \citep{chomsky1955,lees1960}.  \citet{chomsky1970} then argued that only \form{-ing} gerunds are derived syntactically,
while all other types of event nominals, such as \textit{refutation, acceptance, refusal}, are
derived morphologically\is{morphology} in the \isi{lexicon} from bases that are unspecified between nouns and verbs.
The suffix\is{suffixation} \form{-ing} was shown to serve both as the \isi{gerund} formative and as one of the
 formatives that derive lexical event nominals.  Chomsky's main argument was based on the fact
that \isi{gerund} phrases have the structure of verb phrases whereas other event nominals have the
structure of noun phrases.\footnote{Chomsky also contrasted the uniformity, \isi{regularity} and full
  \isi{productivity} of gerunds with the morphological and semantic diversity, idiosyncrasies, and
  limited \isi{productivity} of derived event nominals. As \citealt{anderson2016} notes, these points played a
  subsidiary role in Chomsky's argument.  Indeed, they are not compelling criteria by
  themselves, for there is no shortage of \isi{productivity} and \isi{regularity} in the lexicon,\is{lexicon} and
  syntax has its share of idiosyncrasy.} The differences are completely systematic.  Unlike
derived event nominals, gerunds are modifiable by adverbs, assign structural \isi{case} to their
complements\is{complement} (\cref{gernom}{a,b}), disallow articles and other determiners (\cref{gernom}{c}) and
plurals (\cref{gernom}{d}), allow \isi{aspect} (\cref{gernom}{e}) and negation (\cref{gernom}{f}), and they
have a grammatical subject which is assigned a Th-role as in finite clauses (\cref{gernom}{g}),
and which may be an expletive (\cref{gernom}{h}). In (\cref{gernom}{g}), \form{reading} refers to an event and \form{her} to its \isi{agent}, the reader. In the derived nominal, \form{her} could also be a
sponsor, organizer, or some other participant of a reading event not necessarily identical with
the reader \citep{kratzer1996,kratzer2004}, and \form{reading} could also mean `manner of reading',
`interpretation'.  Without an \form{of} complement,\is{complement} derived nominals are also interpretable
in a passive sense: in \textit{Mary's confirmation}, Mary could be the confirmer or the
confirmee.

\begin{table}
\begin{tabularx}{\linewidth}{llLL}
\lsptoprule
& & \multicolumn{1}{c}{Gerunds} & \multicolumn{1}{c}{Nominals} \\
& & \multicolumn{1}{c}{(\form{-ing\urf{V}})}&	 \multicolumn{1}{c}{(\form{-ing\urf{N},} \form{-ion,} \form{-al,} \form{-ance}\ldots{})}	\\
\midrule
a.& Adjectives	&\ljudge{*}her quick signing the document	& her quick signing of the document		\\
b. &Adverbs	& her immediately reciting it	& \ljudge{*}her immediately recital of it		\\	   
c. &Determiners &\ljudge{*}the/a/this/each performing it		& the/a/this/each performance of it	\\
d. &Plural	&\ljudge{*}her readings it / 	& her readings of it		\\
& & \ljudge{*}her reading its & \\
e. &Aspect	& by her having sung it		& \ljudge{*}by her having sung of it		\\
f. &Negation	&by her not approving it		& \ljudge{*}by her not approval of it		\\
g. &Subject	&we remembered her reading it	&we remembered her reading of it   \\
h. &Expletives	&it(s) seeming to me that I exist\footnotemark	& \ljudge{*}it(s) appearance to me that I exist	 \\
\lspbottomrule
\end{tabularx}
\caption{Gerunds vs.\ Nominals.}
\label{gernom}
\end{table}
\footnotetext{\textit{I can't help but feel a little despondant \textbf{due to it seeming to me} that the TIE/fo and the T-70 make the original TIE and T-65 somewhat redundant}  (Internet), \textit{evidence that ``explains away'' \textbf{its seeming to me} that p is the    case} (James Pryor, The Skeptic and the Dogmatist, \textit{Noûs} 34: 534, 2000). The  variation between Poss\form{-ing} and Acc\form{-ing} gerunds seen here is briefly  addressed in \ref{accing} below.}



The lexicalist line of analysis continues to be developed in different ways \citep{malouf2000,blevins2003k,kim2016}.  But many recent treatments have reverted to a uniformly syntactic
\isi{derivation} of nominalizations, in which nominalizing heads project a nominal structure and have
a verbal \isi{complement} whose type determines the nominalization's properties.  The differences
between the two types in \cref{gernom} is captured by introducing them at different levels in the
functional structure.  The \isi{gerund} \form{-ing\urf{V}} is structurally high, and derived nominals in
\form{-ing\urf{N},} \form{-ion,} \form{-al,} \form{-ance} are structurally low.

\citet{kornfilt2011} dub this the \textsc{Functional Nominalization Thesis} (FNT), and
propose a \isi{typology} of four levels of nominalization, CP, TP, vP and VP. In this \isi{typology},
\ili{English} gerunds are TP nominalizations, while derived nominals are VP
nominalizations.\footnote{A syntactic \isi{derivation} of gerunds from TP/IP is also developed by
  \citealt{pires2006}.  The aspectual content of the \isi{gerund} is treated in \citealt{pustejovsky1995,alexiadou2001}, and \citealt{alexiadou2010}.  Alexiadou and her co-workers conclude
  that gerunds are \isi{imperfective} Aspect heads that dominate VoiceP and vP, while nominalizers
  are n heads that also dominate VoiceP and vP, but under NumberP and ClassifierP, housing
  adjective modifiers, determiners, and \isi{plural}.} This paper vindicates a uniform treatment of
nominalizations in a different way:  all true nominalizations are derived lexically; gerunds
are not nominalizations at all – they are neither DPs nor NPs but IPs that need Case.\is{case}

As my point of departure I take Baker \& Vinokurova's \citeyearpar{baker2009} theory, which extends the FNT from
event nominalizations to \isi{agent} nominalizations.  For gerunds, B\&V posit the structure
\rfa{gerund}{a}, based on the version of the DP analysis originated (along with the DP itself) by
\citealt{abney1987}.  The DP's \isi{complement} here is an NP headed by the \isi{gerund} nominalizer
\form{-ing\urf{V}}, below which the structure is entirely verbal:  an AspP which hosts
aspectual material and certain adverbs, and which has a vP \isi{complement} whose v (v=Voice) head
assigns structural \isi{case} and introduces an external \isi{agent} argument.\is{arguments} This external \isi{agent} argument
shows up as a genitive in D head. B\&V do not say exactly how it gets there in \rfa{gerund}{a};
perhaps it is base-generated in D and bears a control relation to the PRO in the Spec-vP
position where the \isi{agent} role is assigned.

For derived nominals B\&V propose the structure \rfa{gerund}{b}, where the head
(\form{-ing\urf{N}}, \form{-ion}, etc.)  takes a bare VP complement.\is{complement} Because it has no Asp
or v projection, it contains neither adverbs, agents, nor structural case.\is{case}\footnote{All
  analyses have to contend with the fact that certain adverbs can occur as postmodifiers with
  derived nominals, and even with some underived ones \citep{payne2010}; they
  cite examples such as \textit{the opinion generally of the doctors}, \textit{a timber
    shortage nationally}, \textit{the people locally}, and \textit{the \isi{intervention} again of
    Moscow}.  We shall see similar Finnish data in \S\ref{agentsection} below.}

\largerpage[-1]    
The structures in \rf{gerund} take care of the contrasting properties \cref{gernom}{b}, {e}, {f}, and {g}, but leave the remaining four properties
(\cref{gernom}{a}, {c}, {d}, and {h}) to fend for
themselves.  On the one hand, the DP in \rfa{gerund}{a} provides too little structure:
expletive \form{it}-subjects are believed to occupy Spec-IP or Spec-TP, but \cref{gerund}{a}
provides no Spec-IP or Spec-TP for them.\footnote{Expletive \form{there}, which likewise
  appears in gerunds, may sit in a lower subject position, since it is sensitive to the
  \isi{argument structure} of the predicate – the absence of Cause according to \citealt{deal2009}, who puts
  it in the \isi{specifier} of v.  Like expletive \form{it}, \form{there} does not appear in
  derived nominals (\textit{*there's appearance to be a problem}).  On Deal's analysis, the
  distribution of expletive \form{there} is consistent with my IP analysis of gerunds, but
  adds no further support to it.} On the other hand, the DP, needed in the analysis as a site
for the \isi{gerund}'s subject, generates unwanted structure.  Since DPs can have \isi{plural} heads,
adjective modifiers, determiners, and quantifiers, the DP analysis wrongly predicts that
\cref{gernom}{a}, {c}, and {d} should be grammatical.  To maintain it
one must somehow prevent functional projections like AP, QP, and NumP from appearing in DPs
that have NP complements\is{complement} that have AspP complements,\is{complement} while allowing them in other kinds of DPs,
and one must prevent the head of a DP whose \isi{complement} is an NP whose \isi{complement} is an AspP
from being an article or a demonstrative pronoun.\footnote{Some of the overgeneration could be
  curbed by by eliminating the DP layer, or by eliminating the NP layer and having D select for
  AspP directly. But these projections cannot be struck from \rf{gerund} because their heads
  are essential to the analysis.  The D head serves as the site of the structural subject, and
  the N head houses the nominalizer \form{-ing}.  Neither of these elements can be
  accommodated in the Asp head, for that is required for the aspectual auxiliary
  \form{have}.}

\begin{exe}
\ex\label{gerund}\label{typ}\label{ex:kip:1}
\minibox[t]{a.\quad his reading the book (high)\\
	\hphantom{a.}\quad\begin{forest}
	[DP
	[DP, fit=band]
	[D\1
	[D
	['s]
	]
	[NP
	[N
	[{-ing}]
	]
	[AspP
	[Asp, fit=band]
	[vP
	[(PRO), fit=band]
	[v\1
	[v, fit=band, name=v]
	[VP
	[V
	[read]
	]
	[DP
	[the book, roof, name=DP]
	]
	]
	]
	]
	]
	]
	]
	]
	\draw[->,thick,>=stealth] (v.south) |- node[near end,below] {case} ([yshift=-3ex]DP.south) -|  (DP.south);
\end{forest}}
\hspace*{-10em}
\minibox[t]{b.\quad the reading of the book (low)\\
	\hphantom{b.}\quad\begin{forest}
	[DP
		[D
			[the]
		]
		[NP
			[N
				[-ing]
			]
			[VP
				[V
					[read]
				]
				[DP
					[of the book, roof]
				]
			]
		]
	]
	\end{forest}}
\end{exe}
%\begin{exe}
%\ex\label{gerund}
%\label{low}\label{typ}
%\setlength\treelinewidth{1pt}
%\setlength\arrowwidth{7pt}
%\setlength\arrowwidth{7pt}
%\setlength{\tabcolsep}{4pt}
%\hspace{-5pt}\begin{tabular}{ccccccccccccccc}
%\multicolumn{8}{l}{\quad (a) his reading the book (high)}\\[2.5ex]
%	&\node{DP}{DP}	&	&	&	&	&			\\[2.5ex]
%\node{DPA}{DP}	&	&\node{DD}{D′}	&	&	&			\\[2.5ex]
%&\node{D}{D}     &       &\node{NP}{NP}	&	&	&			\\[2.5ex]
%&\node{S}{'s}&\node{N}{N}	&	&\node{AspP}{AspP}	&	&	\\[2.5ex]
%&	&\node{er}{-ing}	&\node{Asp}{Asp}	&	&\node{VoiceP}{vP}	\\[2.5ex]
%&	&	&	&\node{PRO}{(PRO)}	&	&\node{Voi}{v′}	&		\\[2.5ex]
%&	&	&	&	&\node{Voice}{v}	&	&\node{VP}{VP}		\\[2.5ex]
%&	&	&	&	&	&\node{V}{V}	&	&\node{DP2}{DP}		\\[2.5ex]
%&	&	&	&	&	&\node{read}{read}	&
%        &\node{book}{\begin{minipage}[h]{8ex}{the book} \end{minipage}}	\\[2ex]
%&        &
%		&&&&\multicolumn{2}{c}{\footnotesize \ \ \ \ \ \ case}	&	&	
%\end{tabular}
%\nodeconnect{DP}{DPA}
%\nodeconnect{DD}{NP}
%\nodeconnect{DD}{D}
%\nodeconnect{D}{S}
%\nodeconnect{DP}{DD}
%\nodeconnect{NP}{N}
%\nodeconnect{N}{er}
%\nodeconnect{NP}{AspP}
%\nodeconnect{AspP}{Asp}
%\nodeconnect{AspP}{VoiceP}
%\nodeconnect{VoiceP}{PRO}
%\nodeconnect{VoiceP}{Voi}
%\nodeconnect{Voi}{Voice}
%\nodeconnect{Voi}{VP}
%\nodeconnect{VP}{V}
%\nodeconnect{VP}{DP2}
%\nodeconnect{V}{read}
%\nodetriangle{DP2}{book}
%\abarnodeconnect[-80pt]{Voice}{book}
%\hspace{-50pt}\begin{tabular}{ccccccccccccccc}
%\multicolumn{7}{l}{(b) the reading of the book (low)}\\[2ex]
%		&\node{DP}{DP}	&  &	      &	      &	      \\[2.5ex]
%
%\node{D}{D}	&	&\node{NP}{NP}	   &	   &	   &	   &	   \\[2.5ex]
%
%\node{the}{the}&		\node{N}{N} &       &\node{VP}{VP}       &       &       \\[2.5ex]
%
%&               \node{er}{-ing}        &\node{V}{V}	      &	        &\node{DP2}{DP}\\[2.5ex]
% &        &\node{read}{read}	      &         &
% \node{book}{\begin{minipage}[h]{6ex}{of the book} \end{minipage}}\\
%\nodeconnect{DP}{D}
%\nodeconnect{D}{the}
%\nodeconnect{DP}{NP}
%\nodeconnect{NP}{N}
%\nodeconnect{NP}{VP}
%\nodeconnect{VP}{V}
%\nodeconnect{VP}{DP2}
%\nodeconnect{N}{er}
%\nodetriangle{DP2}{book}
%\nodeconnect{V}{read}
%\end{tabular}
%\end{exe}


Contrary to what the FNT seems to promise, then, the \isi{morphosyntactic} properties of a
nominalization cannot be fixed just by locating its nominal head in a universal hierarchy of
verbal \isi{functional categories}, or even in a language-specific one.  In that approach to mixed
categories, it seems that the functional content that a given nominalizing head may combine
with must be specified on an item-specific basis.  But not just any arbitrary mixed category is
possible.  Consider the awesome unused power unleashed by the FNT.  If functional N heads can
convert AspPs into NPs in the syntax, as in \rfa{gerund}{a}, why aren't there such things as Q
heads with vP complements\is{complement} (*\textit{[some [he read it]\trf{vP} ]\trf{QP}}), let alone
multiple verbalizing and nominalizing syntactic heads interspersed to generate phrases in which
layers of verbal and nominal structure alternate in various combinations?

The empirical problem of overgeneration is a direct result of the theoretical approach behind
the FNT-style analysis.  The \isi{derivation} of gerunds in \rfa{gerund}{a} involves syntactic
affixation\is{affix} of \form{-ing} to the phrasal projection AspP.\footnote{A similar earlier proposal
  is \citet{yoon1996}.}  A lexicalist perspective rules out affixation to phrases.  It dictates an
entirely different kind of derivation,\is{derivation} in which the \isi{gerund} suffix\is{suffixation} \form{-ing} is added to
verbs in the \isi{morphology} to build words (e.g.\ \form{reading}), which are then inserted in
terminal nodes in the syntax.  On this view, a \isi{gerund} phrase is the syntactic projection of a
\isi{gerund}, not of a determiner as in \rfa{gerund}{a}. On that basis we can build a simple and
restrictive theory of nominalizations that explains \textit{all} the data in \cref{gernom}.

The key idea is that gerunds are \isi{participles}, and that participial suffixes,\is{suffixation} \form{-ing}
included, are Infl heads that differ from finite and infinitive Infl heads in that they bear a
Case\is{case} feature.\is{features}  The extended projection of a \isi{gerund} is then an IP with a Case\is{case} feature,\is{features} which
needs to be checked (or, from a non-lexicalist perpective, valued) in the syntax.  The Case\is{case}
feature restricts participial phrases to two syntactic functions:  \isi{arguments} – gerunds – in
positions where their value for Case\is{case} can be checked by a predicate, and participial modifiers
in positions where their value for Case\is{case} can be checked by head-modifier agreement.

Lexicalism excludes not only FNT-style analyses of gerunds, but every kind of syntactic
affixation\is{affix} to phrasal categories.  This means that no syntactic process can have the effect of
changing the category of a word.  That holds for all types of nominalization:  event
nominalizations, result nominalizations, and \isi{agent} nominalizations.
All ``mixed categories'' must then arise from morphological specifications of lexical heads,
rather than from syntactic embedding as in \rf{typ}.  In \S\ref{agentsection} I support
this more general prediction by showing that transitive\is{transitivity} \isi{agent} nouns do not have an embedded vP
projection and that their verbal properties come from a Tense/Aspect\is{tense} feature\is{features} on the \isi{agent}
suffix.

I assume that a phrasal constituent is a projection of its head, which inherits its category
(Noun, Verb, etc.), its inflectional\is{inflection} \isi{features} (such as Aspect and Case),\is{case} and its thematic roles
(Agent, Patient, Instrument, Event, etc.).\footnote{E.g.\ ⟦-er⟧ =
  λPλxλe[P(e) $\wedge$ Agent(e,x)] (the set of human individuals
  that are the Agent of some event), ⟦-ee⟧ = λPλxλe[P(e) $\wedge$
  human(x) $\wedge$ Undergoer(e,x)] (the set of human individuals that are the Undergoer of
  some event).}  Mixed categories are verbs, nouns, and adjectives that have an extra
phi-feature.\is{features}  Their extended projections behave like extended projections of ordinary verbs,
nouns, and adjectives, modulo the properties enforced by that feature\is{features} content.  The
language-specific syntax of gerunds is determined by their Case\is{case} feature.\is{features} A \isi{gerund} that can bear
any Case\is{case} projects a phrase with the distribution of a DP.  A \isi{gerund} that has a partially
specified Case\is{case} feature\is{features} projects a phrase that is restricted to positions compatible with the
specified values of the feature.  For example, Finnish gerunds are restricted to internal
argument\is{arguments} positions (section \ref{finnsection}).  Similarly, the verbal properties of transitive\is{transitivity}
\isi{agent} nouns are due to a Tense/Aspect\is{tense} feature\is{features} assigned to these nouns by the \isi{agent} \isi{affix} that
forms them.  This feature\is{features} may likewise be lexically unvalued and specified by additional
aspectual \isi{morphology} (as in Northern Paiute), or inherently specified on the \isi{agent} noun \isi{affix}
itself (as in \ili{Sanskrit} and Sakha), see \ref{vedtensesection}.  Since the mixed categories under
lexicalist assumptions are projected from a single head, we correctly predict the absence of
mixed categories in which verbal and nominal structure is alternately layered in weird
combinations, of vPs that function as DPs, and of the other abovementioned monstrosities.

A theoretical gain is that we need not divide nominalizations into a syntactic type and a
lexical type, as in standard lexicalist analyses.  Once gerunds are recognized as IPs, we can
maintain that all nominalizations are derived morphologically in the lexicon.\is{lexicon}  This can be done
either in a realizational \isi{morphology} of the type pioneered by \citet{anderson1992}, or in a
morpheme-based\is{morpheme} one such as the minimalist \isi{morphology} of \citet{wunderlich1996}.  It remains to be seen
whether the analysis can be recast in a DM-friendly syntax-based format.  What is clear is that
it does not \textit{follow} from any theory that countenances structures like \rf{typ}.  To
that extent at least, its empirical success constitutes new empirical support for lexicalism.


I begin in \S\ref{gerundsection} with ``high'' event nominalizations.  I show that
the lexicalist approach correctly predicts the syntax of \ili{English} and Finnish \isi{gerund} phrases,
including aspects that go unexplained in FNT analyses, and that it curbs the \isi{typology} in a good
way.  In \S\ref{agentsection} I apply the same idea to \isi{agent} nominals, and support the
resulting analyses with data from \ili{Vedic} \ili{Sanskrit} and Finnish that is new to the theoretical
literature.

\section{Gerunds}\label{gerundsection}\label{sec:kikarsky:2}
\subsection{\ili{English} gerunds}

Gerunds and \isi{participles} are formally identical in \ili{English} \citep{pullum1991,yoon1996,huddleston2002,blevins2003k}. For example, they are the only verb forms that overtly
distinguish perfect \isi{aspect} but not progressive or \isi{past tense}.  Given the modest morphology of
\ili{English} this identity might be dismissed as an accident, but the testimony of richly inflected
languages, such as Finnish \rfa{part}{b}, Classical Greek \rfa{part}{c}, \ili{Sanskrit}
\rfa{part}{d}, and \ili{Latin} \rfa{part}{e} leaves no doubt that \isi{participles} are systematically used
in two functions: adjectivally as modifiers and nominally as arguments.\is{arguments}
\begin{exe}
\ex\label{part}\label{ex:kip:2}
	\ea \label{ex:kip:2a}\ili{English} \form{-ing} participle
		\ea \textit{Modifier:}      I saw Bill reading the book.  ($\Rightarrow$ I saw Bill.)
		\ex \textit{Argument:}        I hated Bill's reading the book.  ($\not\Rightarrow$ I hated Bill.)
		\z    
	
	\ex \label{ex:kip:2b} Finnish \isi{participles}
		\ea \textit{Modifier:} 

		\gll Muist-i-n hunaja-a syö-vä-n karhu-n.\\
		 remember-\textsc{pst}-\textsc{1sg} honey-\textsc{prtc} eat-\textsc{ptc}-\textsc{gen} bear-\textsc{acc}\\
		\glt `I remembered the/a bear (that was) eating honey.' 

		\ex \textit{Argument:}
		
		\gll Muist-i-n  karhu-n  syö-vä-n hunaja-a.\\
		 remember-\textsc{prtc}-\textsc{1sg} bear-\textsc{gen} eat-\textsc{ptc}-\textsc{gen} honey-\textsc{prtc}\\
		\glt `I remembered that the/a bear ate honey.' 
		\z		    
\newpage		    
	\ex \label{ex:kip:2c} Classical Greek \isi{participles}
		\ea \textit{Modifier:} 
 
 		\gll tòn adikoũ-nt-a Phílippo-n apéktein-a\\
		the-\textsc{acc} act-unjustly-\textsc{prtc}-\textsc{acc} Philip-\textsc{acc} kill-\textsc{aor}-\textsc{1sg}\\
		\glt `I killed the unjustly acting Philip.' 

		\ex \textit{Argument:}
		
		\gll adikoũ-nt-a Phílippo-n eksḗlenk-s-a\\
		act-unjustly-\textsc{prtc}-\textsc{acc} Philip-\textsc{acc} prove-\textsc{aor}-\textsc{1sg}\\
		\glt `I proved that Philip acted unjustly.' 
		\z    

	\ex \label{ex:kip:2d} \ili{Sanskrit} \isi{participles}
		\ea \textit{Modifier:}
 
 		\gll rājān-am ā-ga-ta-ṃ śṛ-ṇo-mi \\
		king-\textsc{acc} to-go-\textsc{prtc}-\textsc{acc} hear-\textsc{pres}-\textsc{1sg}\\
		\glt `I hear the king (who has) arrived.' 

		\ex \textit{Argument:}

		\gll rājān-am  ā-ga-ta-ṃ śṛ-ṇo-mi \\
		king-\textsc{acc} to-go-\textsc{prtc}-\textsc{acc} hear-\textsc{pres}-\textsc{1sg}\\
		\glt `I hear that the king has arrived.'
		\z  

	\ex \label{ex:kip:2e} \ili{Latin} \isi{participles}
		\ea \textit{Modifier:}
 
 		\gll Hannibal vic-tu-s ad Antiochu-m confug-i-t\\
		Hannibal.\textsc{nom} defeat-\textsc{prtc}.\textsc{masc}-\textsc{nom} to Antiochus-\textsc{acc} flee-\textsc{perf}-\textsc{3sg}\\
		\glt `Defeated, Hannibal took refuge with Antiochus.' 

		\ex \textit{Argument:}

		\gll Hannibal vic-tu-s Romano-s metu libera-vi-t\\
		Hannibal.\textsc{nom} defeat-\textsc{prtc}.\textsc{masc}-\textsc{nom} Roman.\textsc{acc}.\textsc{pl}
		fear.\textsc{abl} free-\textsc{perf}-\textsc{3sg}\\
		\glt `Hannibal's being defeated freed the Romans from fear.'
		\z  
	\z    
\end{exe}
Traditional grammars of these languages treat \isi{participles} as verb forms which are inflected for Case,\is{case}
for good reasons.  Participles distinguish the verbal categories of voice and tense/\isi{aspect},\is{tense} and
they are formed off the same tense/\isi{aspect}\is{tense} stems\is{stem} as the finite verbs.  They supply the
periphrastic forms that complete gaps in inflectional\is{inflection} paradigms.\is{paradigm}  They assign the same cases\is{case} to
their objects as the corresponding finite verbs and infinitives do.  They are modified by
adverbs, not by adjectives. They select for the same prefixes as the corresponding finite verbs
and infinitives, with the same (often idiosyncratic) meanings.  Those languages that disallow
noun+verb compounds\is{compound} (such as classical Greek and \ili{Sanskrit}) also disallow noun+participle
compounds.  As I show below, \isi{participles} have structural subjects.

\newpage 
So there must be some property that distinguishes \isi{participles} from finite verbs and
infinitives, and which supports the double function of \isi{participles} as nominal \isi{arguments} and
adjectival modifiers.  The obvious candidate is Case.\is{case}  Suppose then that participial formatives
are Infl heads that need Case.\is{case}  On lexicalist assumptions, they are affixed\is{affix} in the \isi{morphology}
to a verb to make a participle, which is then inflected for \isi{case} if the language has \isi{case}
morphology, and enters the syntax with a specified Case\is{case} feature\is{features} that – like any Case\is{case} feature –
must be checked in the syntax. In a language that lacks \isi{case} morphology,\is{morphology} such as \ili{English}, the
participle remains unvalued for Case,\is{case} and projects an IP with a Case\is{case} feature\is{features} that must be
valued in the syntax.  Both ``checking'' and ``valuing'' can be formalized as identical
operations of feature\is{features} unification, or as optimal matching in OT Correspondence Theory.  

As an illustration consider first the \isi{derivation} of gerunds in \ili{English}.  Prescinding from vP,
AspP, VoiceP, and other possible functional projections, their syntactic structure is as in
\rf{our}.

\begin{exe}
\ex\label{our} \label{ex:kip:3}
\begin{forest}
	[IP\sub{[uCase]}
		[DP\sub{[Gen]}
			[The man's, roof]
		]
		[I\1
			[Infl\sub{[uCase]}, fit=band, name=Infl]
			[VP
				[V, fit=band, name=V]
				[DP
					[it]
				]
			]
		]
	]
	\coordinate (mid) at ($(Infl.south)!0.5!(V.south)$);
	\node (reading) at ($(mid)!1.3cm!90:(Infl.south)$) [anchor=north] {reading};
	\draw (Infl.south) -- (reading.north) -- (V.south);
\end{forest}
\end{exe}
%\begin{exe}
%\ex\label{our}
%\begin{tabular}[t]{cccccccc}
%&\multicolumn{2}{c}{\node{iea}{IP\trf{[uCase]}}}&&&\\[3ex]
%		  	&	&\node{iec}{I′}\\[3ex]
%&&&\multicolumn{1}{l}{\node{ieea}{VP}}\\[6ex]
%\node{ieb1}{DP\trf{[Gen]}}	&\node{ied1}{Infl\trf{[uCase]}}&\node{j}{V}&&
%\node{ieg}{DP}\\[3ex]
%\node{1}{The man's}	&\multicolumn{2}{c}{\node{2}{reading}}&& \node{4}{it}\\
%\end{tabular}	       
%
%\setlength\treelinewidth{1pt}
%\setlength\arrowwidth{7pt}
%\setlength\arrowwidth{7pt}
%\setlength{\tabcolsep}{4pt}
%\nodetriangle{ieb1}{1} 
%\nodeconnect{ied1}{2}
%\nodeconnect{j}{2}
%\nodeconnect{ieg}{4}
%
%\nodeconnect{iea}{ieb1}
%\nodeconnect{iea}{iec}
%\nodeconnect{iec}{ieea}
%\nodeconnect{iec}{ied1}
%\nodeconnect{ieea}{j}
%\nodeconnect{ieea}{ieg}
%\nodeconnect{ieea}{j}
%\end{exe}
Infl\trf{[uCase]} combines with V in the same way as Tensed\is{tense} Infl does. How this happens depends on
the model of grammar. If we assume both lexicalist syntax and lexicalist morphology,\is{morphology} the
case-needing\is{case} Infl \form{-ing} is suffixed\is{suffixation} to V in the \isi{morphology} to form a participle, and
the participle then projects a case-needing\is{case} IP in the syntax, where the Case\is{case} feature\is{features} is valued.
In argumental\is{arguments} \isi{participles} (gerunds), it is valued by the governing Case-assigner,\is{case} and in
participial modifiers it is valued by agreement with the nominal they modify.

If we assume minimalist syntax, we can comply with lexical \isi{morphology} by using spanning
\citep{svenonius2016}, which allows the lexically generated participle to be inserted under the two
corresponding syntactic terminal nodes.  In DM, \form{-ing} would be a syntactic terminal
that is postsyntactically Lowered onto V.  Thus, the idea that gerunds are case-needing\is{case} IPs can
probably be implemented in any grammatical architecture.  However, in non-lexicalist frameworks
this analysis is merely motivated on empirical grounds.  In lexicalist frameworks that prohibit
affixation\is{affix} to phrases it is required on principled grounds as well.

In languages where \isi{participles} are morphologically inflected for Case,\is{case} such as \rfa{part}{b-e},
\isi{participles} project an IP that bears a specified Case\is{case} value that must be checked in the syntax.
The lexicalist approach now makes an interesting prediction: in such languages, gerunds may be
morphologically restricted to particular Case\is{case} features, which restrict their syntactic
distribution to contexts compatible with those features.\is{features}  This prediction is confirmed in
Finnish.  Finnish gerunds bear an oblique Case\is{case} -- glossed as Genitive in \rfa{part}{b.ii} -- which
confines them to internal argument\is{arguments} positions (section \ref{finnsection}).\footnote{There is no
  agreement relation between the genitive subject and the \isi{gerund} in \rf{finnpart}.}
\begin{exe} \label{ex:kip:4}
	\ex\label{finnpart}
	\begin{forest}
		[IP\sub{[Gen]}
		[DP\sub{[Gen]}
		[karhun\sub{[Gen]}, roof]
		]
		[I\1
		[Infl\sub{[Gen]}, fit=band, name=Infl]
		[VP
		[V, fit=band, name=V]
		[DP
		[hunajaa\sub{[Part]}]
		]
		]
		]
		]
		\coordinate (mid) at ($(Infl.south)!0.5!(V.south)$);
		\node (syovan) at ($(mid)!1.4cm!90:(Infl.south)$) [anchor=north] {syövän};
		\draw (Infl.south) -- (syovan.north) -- (V.south);
	\end{forest}
\end{exe}
%\begin{exe}
%\ex\label{finnpart}
%\begin{tabular}[t]{cccccccc}
%&\multicolumn{2}{c}{\node{iea}{IP\trf{[Gen]} }}&&&\\[3ex]
%		  	&	&\node{iec}{I′}\\[3ex]
%&&&\multicolumn{1}{l}{\node{ieea}{VP}}\\[6ex]
%\node{ieb1}{DP\trf{[Gen]}}	&\node{ied1}{Infl\trf{[Gen]}}&\node{j}{V}&&\node{ieg}{DP}\\[3ex]
%\node{1}{karhun\trf{[Gen]}}	&\multicolumn{2}{c}{\node{2}{syövän\trf{[V,Infl,Gen]}}}&& \node{4}{hunajaa\trf{[Part]}}\\
%\end{tabular}	       
%
%\setlength\treelinewidth{1pt}
%\setlength\arrowwidth{7pt}
%\setlength\arrowwidth{7pt}
%\setlength{\tabcolsep}{4pt}
%\nodetriangle{ieb1}{1} 
%\nodeconnect{ied1}{2}
%\nodeconnect{j}{2}
%\nodeconnect{ieg}{4}
%\nodeconnect{iea}{ieb1}
%\nodeconnect{iea}{iec}
%\nodeconnect{iec}{ieea}
%\nodeconnect{iec}{ied1}
%\nodeconnect{ieea}{j}
%\nodeconnect{ieea}{ieg}
%\nodeconnect{ieea}{j}
%\end{exe}

Returning to \ili{English}, the analysis of participial clauses as IPs that have the single special
property of needing structural Case\is{case} explains at a stroke which clausal and nominal properties
they have and which ones they lack.  To start with the latter, it explains why \isi{gerund} phrases,
unlike DPs, have no articles, quantifiers, or numerals, why they cannot be modified by
adjectives and relative clauses, why their head cannot be genitive or
\isi{plural}.\begin{exe}
\ex\label{noo}\label{ex:kip:5}
	\begin{xlist}
	\ex[*]{\label{ex:kip:5a}The/a/each compiling the corpus took over a year.}
	\ex[*]{\label{ex:kip:5b}Both/every compiling corporas took over a year.}
	\ex[*]{\label{ex:kip:5c}His two compilings corpora each took over a year.}
	\ex[*]{\label{ex:kip:5d}His careful compiling the corpus was a turning point.}
	\ex[*]{\label{ex:kip:5e}His editing texts that is funded will take a year.}
	\ex[*]{\label{ex:kip:5f}His compiling corpora's results were dramatic.}
	\end{xlist}
\end{exe}
The missing categories are just the ones that would originate in a DP.

As for the nominal properties that gerunds do have, they are accurately covered by the
generalization that gerunds appear in Case\is{case} positions.  They function as subjects, objects, and
predicates, as objects of prepositions \rfa{pos}{e,f,g}, and as objects of a small set of
transitive\is{transitivity} adjectives \rfa{pos}{f,g}, all diagnosed as Case\is{case} positions by the fact that
full-fledged DPs occur in them.
\begin{exe}
\ex\label{pos}
	\ea {[} Bill's leaving Paris ] was unexpected.
	\ex I regret [ Bill's leaving Paris ].
	\ex The problem is [ Bill's leaving Paris ].
	\ex Because of  Bill's leaving Paris we'll be hiring new personnel.
	\ex We are worried about Bill's leaving Paris.
	\ex This event is worth my visiting Paris.
	\ex It's no good my playing this sort of game.\footnotemark 
	\z  
\end{exe}
\footnotetext{Cf.\ \textit{It's no good this sort of game.} (Dickens, \textit{Our Mutual
    Friend)}}  This does not mean that all transitive\is{transitivity} verbs take \isi{gerund} complements.\is{complement}
Particular verbs can select for whether they take gerunds, \form{that-}clauses, or infinitive
complements,\is{complement} just as they can select for whether they take DPs:
\begin{exe}
\ex
	\begin{xlist}
	\ex[*]{I said Bill's leaving Paris.}
	\ex[]{I said it/something/several things.}
	\z    
\end{exe}
What the analysis correctly predicts is that gerunds, unlike \form{that}-clauses and infinitives,
appear \textit{only} in Case\is{case} positions:
\begin{exe}
\ex\label{nocase}
	\ea
		\begin{xlisti}[iiii.]
		\ex[*]{I hope Bill's leaving Paris.}
		\ex[*]{I hope it.}
		\ex[]{I hope that Bill is leaving Paris.}
		\ex[]{I hope to leave Paris.}
		\end{xlisti}
	\ex 
		\begin{xlisti}[iiii.]
		\ex[*]{It is rumored Bill's leaving Paris.}
		\ex[*]{The proposal is rumored.}
		\ex[]{It is rumored that Bill is leaving Paris.}
		\ex[]{It is rumored to be happening.}
		\end{xlisti}
	\ex 
		\begin{xlisti}[iiii.]
		\ex[*]{It seemed / was expected Bill's leaving Paris.}
		\ex[*]{It seemed / was expected this event.}
		\ex[]{It seemed / was expected that Bill would leave Paris.}
		\ex[]{It seemed / was expected to happen.}
		\end{xlisti}
	\z  
\end{exe}

A further consequence is that the subjects of gerunds are IP specifiers.  If overt, they are
Genitive or Accusative,\footnote{On Acc+\form{ing} gerunds see the brief and inconclusive
  remarks in \S\ref{accing} below.} just as the subject of a finite IP is Nominative, and
the overt subject of an infinitive requires a Case-assigning\is{case} \form{for}.  Crucially, they
are true structural subjects analogous to subjects of finite clauses, not necessarily
``agents'' as in B\&V's \cref{gernom}{a}, nor ``possessors'' with their varied functions as in
derived nominals.  This prediction is confirmed by three generalizations.  Unlike genitive
specifiers of nouns (including derived nominals), but like structural subjects of finite
clauses, the specifiers of gerunds can be expletives:
\begin{exe}
\ex\label{dreamt}
	\begin{xlist}
	\ex[]{It(s) seeming to you that you dreamt is not evidence of it(s) being the case that you dreamt.}
	\ex[*]{It(s) appearance to you that  you dreamt is not evidence of it(s) truth that you dreamt.}
	\z    
\end{exe}
Like structural subjects of finite clauses, they are subject to control \citep[1190]{huddleston2002}:
\begin{exe}
\ex\label{lock}
	\ea Mary remembered locking the door.   [the rememberer is the locker]
	\ex Mary remembered the/a locking of the door.  [the rememberer might not  be the locker]
	\z  
\end{exe}
like structural subjects of finite clauses, and unlike genitive agents of nominals, 
they  cannot be paraphrased with \form{of} or \form{by}:
\begin{exe}
\ex\label{persians}
	\ea the Persians' quick run = the quick run of/by the Persians
	\ex the Persians' quick running = the quick running of/by the Persians
	\ex the Persians' quickly running ≠ *quickly running of/by the Persians
	\ex the Persians' quickly attacking the Greeks ≠ *quickly attacking the Greeks of/by the Persians
	\ex the Persians quickly attacked the Greeks ≠ *of/by the Persians quickly attacked the Greeks 
	\z
\end{exe}

To summarize:  the analysis of gerunds as IPs with Case\is{case} explains the cross-linguisti\-cally
common convergence of nominal and adjectival functions in a single morphological class of
verbal forms.  By not positing any DP or NP structure over the IP it avoids the overgeneration
problem that FNT-type analyses face.  It excludes the possibility of multiple alternating
verbalizing and nominalizing syntactic heads to which the FNT opens the theoretical door, and
gets rid of the \isi{constraint} that heads of DPs whose complements\is{complement} are NPs whose complements\is{complement} are
verbal projections may not be articles or demonstrative pronouns.  It provides the basis for a
uniform structure for all DPs, and for a uniform lexical \isi{derivation} of all nominalizations.  It
correctly predicts that gerunds and \isi{participles} have subjects – specifiers of Infl that are
structural counterparts to the subjects of finite clauses.  What is important is that the
analysis is not motivated merely by these empirical arguments; it is a consequence of
lexicalism, and, if correct, supports the lexicalist organization of grammar.

The question arises whether there might be a CP layer above the IP, headed by a null
complementizer.  This additional structure is not justifiable for \ili{English}, because the
distribution of gerunds differs from that of any type of CP.  First, gerunds need Case,\is{case} whereas
CPs do not \citep{vergnaud1977}.  Secondly, gerunds are permitted in clause-medial position, while
\form{that-}clauses and other CP clauses must extrapose.\footnote{However, the \isi{case} marking
  of \isi{participles} in inflected languages could be considered as a kind of complementizer, as
  conjectured for the inherent \isi{case} \isi{affix} \form{-n} on Finnish gerunds in \S\ref{finnsection}.}

That \isi{gerund} phrases are full IPs with a structural subject, that they bear Case,\is{case} and that,
unlike derived nominals, they have no DP or NP projection, and in particular no possessor-type
Specifier, makes many additional predictions that are testable in morphologically richer
languages.  They turns out to be abundantly supported, as demonstrated for Finnish in the next
section.

\subsection{Finnish gerunds are IPs}
\label{finnsection}
Finnish participial propositional \isi{complement} clauses are the closest functional counterparts of
\ili{English} gerunds, and I will call them gerunds here.  They are not DPs with possessors but IPs
with true structural subjects.  Their Case\is{case} is inherently marked by the oblique suffix \form{-n},
arguably functioning as a complementizer, which restricts them to internal argument\is{arguments}
positions.  This illustrates how the \isi{typology} of gerunds emerges from variation in what cases\is{case}
they can bear.\footnote{The data and analysis of Finnish gerunds presented in this section is
  condensed from my treatment of Finnish nonfinite complementation in \citet{kiparskyInPress}, to
  which I refer the reader for the details.}

Unlike \ili{English} gerunds, Finnish gerunds are
never external arguments.\is{arguments}  Thus they can be objects of transitive verbs such as \textit{say},
\textit{think}, \textit{want}, \textit{prove}, \textit{remember} and \textit{hear}, and subjects of presentational
intransitives like \textit{appear} and \textit{become evident}, but they cannot be subjects of such
predicates as \textit{be obvious}, \textit{prove}, and \textit{mean}.\footnote{In the glosses,
  \textsc{acc\trf{gen}} and \textsc{acc\trf{nom}} both refer to \isi{morphosyntactic} Accusative
  structural case.\is{case} The subscripts show three different morphological \isi{case} realizations of this
  \isi{morphosyntactic} Case.\is{case}  They will become important shortly, but for now the reader may ignore
  them.}
\begin{exe}
\ex\label{unacc}
	\begin{xlist}
	\ex[]{\gll Selvis-i Mati-n ampu-nee-n karhu-n. \\
		become-clear-\textsc{pst}.\textsc{3sg} Matti-\textsc{gen} shoot-\textsc{perf}.\textsc{prtc}-\textsc{gen} bear-\textsc{acc\trf{gen}}\\
		\glt `It became clear that Matti had shot the/a bear.'}

	\ex[*]{\gll Mati-n ampu-nee-n karhu-n suututt-i \\
		Matti-\textsc{gen} shoot-\textsc{perf}.\textsc{prtc}-\textsc{gen} bear-\textsc{acc\trf{gen}} anger-\textsc{pst}.\textsc{3sg}\\}
	\sn[]{\gll Liisa-a.\\
		Liisa-\textsc{part}\\
		\glt `That Matti had shot the/a bear angered Liisa.'}
	\end{xlist}
\end{exe}
This distribution suggests that the ending \form{-n} that \isi{participles} bear in their gerundial
function, glossed ``\textsc{gen}'' in \rf{unacc}, marks an object Case\is{case} that is compatible with
internal \isi{arguments} but not with external arguments.\is{arguments}  Historically, it is probably the old
dative ending, which has fallen together phonologically with the genitive, but persists as a
morphosyntactically distinct type of genitive which (unlike the structural genitive) cannot
function as a subject \citep{kiparskyInPress}.

As shown in \rf{luule}, Finnish gerunds behave more like bare finite CP clauses with \form{että-}
(\form{that-}) than like DPs, whether nominal DPs \rfa{luule}{c} or
pronoun-headed finite clauses with \form{se että-} (\form{it that-}) \rfa{luule}{d}.
\begin{exe}
\ex\label{luule}
	\ea	\gll Huomas-i-n / ymmärrä-n / luule-n / otaksu-n tilante-en ole-va-n hankala-n. \\
		  notice-\textsc{1sg} / understand-\textsc{1sg} / think-\textsc{1sg} / assume-\textsc{1sg}
		  situation-\textsc{gen} be-\textsc{prtc}-\textsc{gen} difficult-\textsc{gen}\\ 
 		\glt `I noticed / understand / think / assume that the situation is difficult.' [lit.\ `the
		  situation's being difficult.'] 

	\ex \gll Se-n huomat-tiin / ymmärre-tään / luul-laan / otaksu-taan ole-va-n hankala-n.\\
		  It-\textsc{gen} notice-\textsc{pass}-\textsc{pst} / understand-\textsc{pass} / think-\textsc{pass} /
		  assume-\textsc{pass} be-\textsc{prtc}-\textsc{gen} difficult-\textsc{gen}\\ 
		\glt  `It is noticed / understood / thought / assumed to be difficult.'

	\ex \gll Huomas-i-n / ymmärrä-n / *luule-n / *otaksu-n häne-t / se-n seika-n. \\
		notice-\textsc{pst}-\textsc{1sg} / understand-\textsc{1sg} / think-\textsc{1sg} / assume-\textsc{1sg}
		him-\textsc{acc\trf{acc}} / that-\textsc{acc\trf{gen}} thing-\textsc{acc\trf{gen}}\\ 
		\glt `I noticed / understand / think / assume him / this point (fact).'

	\ex \gll Huomas-i-n / ymmärrä-n / *luule-n / *otaksu-n se-n, että tilanne on hankala. \\
		notice-\textsc{pst}-\textsc{1sg}  / understand-\textsc{1sg} / think-\textsc{1sg} / assume-\textsc{1sg}
		it-\textsc{acc\trf{gen}} that situation.\textsc{nom} is difficult\\ 
		\glt `I noticed / understand / think / assume that the situation is difficult.'
	\z    
\end{exe}
The distinction between verbs that allow DP objects (\form{huomat-} `notice' and
\form{ymmärtä-} in \rf{luule}) and verbs that do not allow DP objects (\form{luule-}
`think' and \form{otaksu-} `assume' in \rf{luule}) is correlated with factivity, but the
correlation is not exact and my argument does not depend on it.

Since gerunds are not DPs but case-marked\is{case} IPs, their genitive subjects behave like structural
subjects and not like genitive specifiers of DPs.  This is shown by five arguments.

The first argument that the genitive \isi{specifier} of gerunds is a grammatical subject is
that it gets assigned exactly the same Th-roles as the subjects of the corresponding finite
clause, not the diverse range of interpretations that ``possessors'' of derived nominals
receive (see above under \cref{gernom}).  So \form{Matin} in \rfa{unacc}{a} 
picks out the \isi{agent} of the shooting event, whereas the \isi{specifier} \form{Matin} of the derived
nominal \rf{Matin} could be, among other things, the organizer or theme of the rescue.
\begin{exe}
\ex\label{Matin}
  \gll Muista-n Mati-n pelastukse-n. \\
remember.\textsc{pres}-\textsc{1sg} Matti-\textsc{gen} rescue-\textsc{nom}-\textsc{acc}\\
		\glt `I remember Matti's rescue.' 
\end{exe}

The second argument comes from extraction.  The
subjects of gerunds can be extracted as readily as objects:
\begin{exe}
\ex\label{nolbc}
	\ea  \gll  Kene-n väit-i-t ampu-nee-n hän-tä?\\
		 who-\textsc{gen} claim-\textsc{pst}-\textsc{2sg} shoot-\textsc{pfp}-\textsc{gen} he-\textsc{part}\\
		\glt `who did you claim shot at him?' 
	
	\ex  \gll  Ke-tä väit-i-t häne-n       ampu-nee-n?\\
		who-\textsc{part} claim-\textsc{pst}-\textsc{2sg} he-\textsc{gen} shoot-\textsc{pfp}-\textsc{gen} \\
		\glt `who did you claim he shot at?' 
	\z
\end{exe}

But possessors cannot be extracted \rfa{lbc}{a}, and neither can genitive specifiers of
tenseless\is{tense} nonfinite complements\is{complement} such as the third infinitive \rfa{lbc}{b} and the second
infinitive \rfa{lbc}{c} (the Left Branch Condition, \citealt[127]{ross1967}).
\begin{exe}
\ex\label{lbc}
	\begin{xlist}
	\ex[*]{\gll Kene-n\bl{i} väit-i-t ammu-tu-n \textit{t}\bl{i} karhu-n / että ammu-ttin \textit{t}\bl{i} karhu?\\
		who-\textsc{gen} claim-\textsc{pst}-\textsc{2sg} shoot-\textsc{perf}.\textsc{prtc}-\textsc{gen} { }
		bear-\textsc{gen} / that shoot-\textsc{pass}.\textsc{pst} {} bear.\textsc{acc\trf{nom}}\\ 
		\glt `Whose bear did you claim (that) was shot?'}
	
	\ex[*]{\gll Kene-n\bl{i} väit-i-t \textit{e}\bl{i} ampu-ma-n 
		karhu-n paina-nee-n 500 kilo-a?  \\
		who-\textsc{gen} claim-\textsc{pst}-\textsc{2sg} {} shoot-\textsc{3inf}-\textsc{gen} bear-\textsc{gen} weigh-\textsc{perf}.\textsc{prtc}-\textsc{gen} 500 kg-\textsc{part}\\
		\glt `The bear  shot by whom did you claim weighed 500 kg?'}

	\ex[*]{\gll Kene-n\bl{i} itk-i-t \textit{e}\bl{i} ampu-e-ssa karhu-n?  \\
	who-\textsc{gen} claim--\textsc{pst}-\textsc{2sg} {} shoot-\textsc{2inf}-\textsc{iness} bear-\textsc{acc\trf{gen}} \\
	\glt `Who did you weep while he shot the/a bear?'}
	\end{xlist}
\end{exe}
	

A third diagnostic which shows that gerunds have structural subjects and not possessors is that
they do not undergo possessor agreement.
Nouns and infinitives agree with their genitive specifiers, as exemplified for nouns in
\rfa{possinca}{a}, for the second infinitive in \rfa{possinca}{b}, and for the third infinitive
in \rfa{possinca}{c}.
\begin{exe}
\ex\label{possinca}
	\ea \gll (Minu-n\bl{i})  karhu-ni\bl{i} paino-i 500 kilo-a \\
  		(My-\textsc{gen}) bear.\textsc{nom}-\textsc{1sg} weigh-\textsc{pst}-\textsc{3sg} 500.\textsc{acc} kg-\textsc{part} \\
 		\glt  `My bear weighed 500 kilograms.'

	\ex \gll Matti itk-i (minu-n\bl{i}) ampu-e-ssa-ni\bl{i} karhu-n 	\\
  		Matti.\textsc{nom} weep-\textsc{pst}.\textsc{3sg} (my-\textsc{gen}) shoot-\textsc{2inf}-\textsc{1sg} bear-\textsc{acc\trf{gen}}  \\
  		\glt `Matti wept as I shot the/a bear.'

	\ex \gll (minu-n\bl{i}) ampu-ma-ni\bl{i} karhu 	\\
  		(my-\textsc{gen}) shoot-\textsc{3inf}-\textsc{1sg} bear.\textsc{nom}  \\
  		\glt `the/a bear I shot.'
	\z
\end{exe}
But gerunds do not possessor-agree with their subjects, as we can see in \rfa{possinc}{a,b} and
(with a raised subject) in \rfa{possinc}{c}.  
\begin{exe}
\ex\label{possinc}
	\ea  \gll Matti ties-i minu-n\bl{i} ampu-nee-n (*ampu-nee-ni\bl{i}) karhu-n \\ 
		Matti.\textsc{nom} know-\textsc{pst}.\textsc{3sg} me-\textsc{gen} 
		shoot-\textsc{prf}.\textsc{prt}-\textsc{gen} (shoot-\textsc{prf}.\textsc{prt}(-\textsc{gen})-\textsc{1sg}) bear-\textsc{acc\trf{gen}}  \\ 
		\glt `Matti knows that I've shot the/a bear.'  

	\ex \gll Selvis-i häne-n\bl{i} ampu-nee-n (*ampu-nee-nsa\bl{i}) karhu-n\\
		{become clear-\textsc{pst}-\textsc{3sg}} he-\textsc{gen}
 		 (shoot-\textsc{prf}.\textsc{prt}(-\textsc{gen})) (shoot-\textsc{perf}.\textsc{prtc}(-\textsc{gen})-\textsc{3p})
 		 bear-\textsc{acc\trf{gen}}  \\ 
  		\glt `it became clear that he had shot the/a bear.'

	\ex   \gll Näytä-t\bl{i} ampu-nee-n (*ampu-nee-si\bl{i}) karhu-n\\
  		seem-\textsc{2sg} shoot-\textsc{prf}.\textsc{prt}-\textsc{gen}
 		 (shoot-\textsc{perf}.\textsc{prtc}-(\textsc{gen})-\textsc{2sg})  bear-\textsc{acc\trf{gen}}   \\
  		\glt `you seem to have shot the/a bear.'
	\z
\end{exe}
Of course the subjects of gerunds cannot subject-predicate agree with the gerunds like
nominative subjects of finite clauses agree with the finite verb, for genitive subjects never
subject-predicate agree in Finnish.

The fourth argument that gerunds have structural subjects comes from the distribution
of accusative Case\is{case} morphology.\is{morphology|(}  Descriptively, Finnish \isi{morphosyntactic} Case\is{case} is realized as
morphological \isi{case} as follows.\footnote{For details see \citealt{kiparsky2001}; a sophisticated OT
  treatment of the variation is developed by \citet{anttila2016}.}
\begin{exe}
\ex\label{genpart}
	\ea The subject of a participial clause is always Genitive.  
	\ex The object of a participial clause can be \isi{morphosyntactic} Accusative or Partitive.
  Partitive is assigned to objects under the same conditions as in finite clauses:
  		\ea Objects under overt or implicit negation are Partitive.  
		\ex Objects of certain predicates (such as \form{love} and \form{touch}) are Partitive.\footnotemark
		\ex Otherwise objects are Accusative. 
		\z    
	\z  
\end{exe}
\footnotetext{The class of partitive-assigning predicates is often called ``telic'' (e.g.\
  \citealt{kratzer2002}). This is not quite correct; for an attempt at a more accurate formulation see
  \citealt{kiparsky2005b}.} Morphosyntactic Partitive is always realized as morphological partitive.
And now comes the essential and trickiest part.    Morphosyntactic
Accusative is realized by three morphological cases:\is{case}
\begin{exe}
\ex\label{case}
	\ea as morphological accusative on personal pronouns, 
	\ex otherwise as morphological genitive if the object is \isi{plural}, or if the clause has a
  subject with structural \isi{case} (this last condition is called \textsc{Jahnsson's Rule}),
	\ex otherwise as morphological nominative. 
	\z
\end{exe}
Clause types that lack subjects with structural \isi{case} for purposes of Jahnsson's Rule include
imperatives, bare infinitives (``to see Naples and to die''), passives (which in Finnish do not
involve ``promotion'' of the object), and clauses with ``quirky case'' subjects.  

Since the argument to be presented below uses Jahnsson's Rule as a diagnostic for the presence
or absence of a structural subject, I will gloss the examples in such a way that the reader can
see whether Jahnsson's Rule has taken effect in them.  This means glossing not only
\isi{morphosyntactic} Accusative Case,\is{case} but whether \isi{morphosyntactic} Accusative Case\is{case} is realized as
morphological accusative \isi{case} or nominative case.\is{case} So I will mark \isi{morphosyntactic} Case\is{case} by the
main gloss and morphological \isi{case} with a subscript on it.  For example, in \rf{ammu} both
objects bear \isi{morphosyntactic} Accusative Case,\is{case} realized in \rfa{ammu}{a} as morphological
genitive \isi{case} and in \rfa{ammu}{b} as morphological nominative case.\is{case}
\begin{exe}
\ex\label{ammu}
	\ea \gll Matti ampu-i karhu-n \\
		Matti.\textsc{nom} shoot-\textsc{pst}(\textsc{3sg}) bear-\textsc{acc\trf{gen}}\\
		\glt `Matti shot the/a bear.' 
	\ex \gll ammu karhu! \\
		shoot-\textsc{imper} bear-\textsc{acc\trf{nom}}\\
		\glt `shoot the bear!'
	\z
\end{exe}
Through the rest of the text in this section I use capitalization to distinguish
\isi{morphosyntactic} Case\is{case} (such as Accusative) from morphological \isi{case} (nominative, accusative,
etc.).

At last we are ready for the argument.  Nonfinite \isi{complement} clauses are translucent to the
triggering of Accusative and Partitive Case\is{case} and to the realization of Accusative \isi{case} as
genitive or nominative, in the sense that \rf{genpart} and \rf{case}\is{case} can be conditioned
either within the \isi{gerund} clause or in the larger domain of the higher clause with its \isi{gerund}
complement.\is{complement}
So in \rfa{tienn}{a} the object of the lower clause, which contains no negation, can have
either Accusative Case\is{case} (realized as morphological genitive \isi{case} by \rfa{case}{a}), or Partitive
Case\is{case} from the negated main clause by \rfa{genpart}{b.ii}. In \rfa{tienn}{b} the \isi{morphosyntactic}
Accusative Case\is{case} on the object of the \isi{gerund} is realized either as morphological genitive \isi{case}
because the main clause has a subject, or as morphological nominative case,\is{case} because the
participle, being passive, is subjectless (Jahnsson's Rule, \rfa{case}{b}).\footnote{The
  variation between \isi{case} governed locally within the subordinate\is{subordination} clause and in the larger
  domain that includes the main clause is sensitive to as yet poorly understood semantic,
  stylistic and \isi{discourse} factors.  The distribution of the Partitive in particular is affected
  by factivity and the \isi{scope} of negation (\citealt[31]{hakulinen1970}, \citealt[365]{hakulinen1979}).  For example, in
  \rfa{tienn}{a}, the Partitive registers surprise or skepticism, and in \rfa{tienn}{b} the
  Accusative (realized as nominative) is likely to be interpreted factively.} In
\rfa{tienn}{c} the \isi{morphosyntactic} Accusative Case\is{case} on the object can only be realized as
morphological nominative \isi{case} because both the matrix verb and the participle are subjectless.
\begin{exe}
\ex\label{tienn}
	\ea \gll  En tien-nyt heidä-n ampu-nee-n / ampu-va-n karhu-n / karhu-a \hfill \\
		Not-\textsc{1sg} know-\textsc{perf}.\textsc{ptc} they-\textsc{gen} shoot-\textsc{perf}.\textsc{prtc}-\textsc{gen} / 
		shoot-\textsc{prs}.\textsc{prtc}-\textsc{gen} bear-\textsc{acc\trf{gen}} / bear-\textsc{part} \\
		\glt `I didn't know that they had shot / were (would be) shooting the/a bear.' 
	\ex   \gll Ties-i-n metsä-ssä ammu-tu-n  karhu-n / karhu\\
  		know-\textsc{pst}-\textsc{1sg} forest-\textsc{illat}  shoot-\textsc{pass}.\textsc{prtc}-\textsc{acc\trf{gen}}
  		bear-\textsc{acc\trf{gen}}  / bear.\textsc{acc\trf{nom}}\\ 
  		\glt `I knew a bear to have been shot in the forest.'\hfill 
 	\ex \gll Eilen ilmen-i ammu-tu-n *karhu-n / karhu\\
		Yesterday turn.out-\textsc{pst}.\textsc{3sg}  shoot-\textsc{pass}.\textsc{prtc}-\textsc{gen} bear-\textsc{acc\trf{gen}} / bear.\textsc{acc\trf{nom}}\\
		\glt `It turned out yesterday that a bear was shot.'
	\z 
\end{exe}
The crucial case is \rf{casex}, where the \isi{morphosyntactic} Accusative Case\is{case} of the object may be
realized as morphological genitive case.\is{case}  Since the matrix verb is subjectless, the object's
realization as morphological genitive \isi{case} must be licensed by the subject of the \isi{gerund},
\form{Matin}.\is{morphology|)}  Therefore the subject has structural Case.\is{case}
\ea\label{casex}
  \gll  Ilmen-i Mati-n ampu-nee-n karhu-n / karhu. \\
  turn.out-\textsc{pst}.\textsc{3sg}  Matti-\textsc{gen}  shoot-\textsc{perf}.\textsc{prtc}-\textsc{gen} bear-\textsc{acc\trf{gen}} / bear.\textsc{acc\trf{nom}} \\ 
  \glt `It turned out that Matti shot the/a bear.'
\z
This completes the fourth argument that the genitive subject of gerunds is a structural subject.

In contrast, the fact that ``quirky'' genitive subjects induce the nominative form of the
object tells us, by Jahnsson's Rule, that they are non-structural: 
\begin{exe}
\ex\label{shortobj}
	\ea   \gll Hänen pitää osta-a auto. \\ 
He-\textsc{gen} must-\textsc{3sg} buy-\textsc{1inf} car.\textsc{acc\trf{nom}}  \\ 
\glt `He has to buy the/a car.'

	\ex   \gll Hänen on helppo nosta-a tämä säkki. \\ 
He-\textsc{gen} be-\textsc{3sg} easy lift-\textsc{1inf} this.\textsc{acc\trf{nom}} sack.\textsc{acc\trf{nom}}   \\ 
\glt `It is easy for him to lift this sack.'
	\z
\end{exe}
This is as expected, since they are not assigned structurally but idiosyncratically by
particular predicates.


A fifth argument that gerunds have structural subjects is that they can have a
generic null subject \textit{pro$_{arb}$}. In Finnish \textit{pro$_{arb}$} can be a subject \citep{hakulinen1973} but it cannot be
a possessor: contrast \rfa{pro}{a} and \rfa{pro}{b}.  So, the fact that gerunds can have a generic
\textit{pro$_{arb}$} subject, as seen in \rfa{pro}{c}, is another datum in support of the claim
that gerunds have structural subjects and not possessors.  Moreover, gerunds can be subjectless
under the same conditions as subjects of finite clauses.  For example, gerunds can have the
impersonal passive form, see \rfa{pro}{d}.
\begin{exe}
\ex\label{pro}
	\begin{xlist}
	\ex[]{\gll Siellä ∅ voi tanssi-a.\\
	there \textit{pro} can-\textsc{3sg} dance-\textsc{1inf}\\
	\glt `One can dance there.'}

	\ex[*]{\gll On mukava katsel-la ∅ valokuv-i-a.\\
	be-\textsc{3sg} nice look.at-\textsc{1inf} \textit{pro}  photo-\textsc{pl}.\textsc{part}\\
	\glt \ljudge{*}`It's nice to to look at one's photos.' (OK without ∅: `It's nice to look at photos.'}

	\ex[]{\gll Siellä väite-t-ään ∅ voi-va-n tanssi-a.\\
	there claim-\textsc{pst}.\textsc{pass} \textit{pro} can-\textsc{pres}.\textsc{prtc}-\textsc{gen} dance-\textsc{1inf}\\
	\glt `It is claimed that one can dance there.'}

	\ex[]{\gll Siellä väite-tt-iin voi-ta-va-n tanssi-a.\\
	there claim-\textsc{pst}.\textsc{pass} can-\textsc{pass}-\textsc{pres}.\textsc{prtc}-\textsc{gen} dance-\textsc{1inf}\\
	\glt `It was claimed that there is dancing there.'}
	\end{xlist}
\end{exe}

I conclude that Finnish gerunds are IPs like \ili{English} gerunds, albeit with a different syntactic
distribution due to their oblique Case\is{case} specification.

\subsection{Desultory remarks on Acc-\form{ing}}
\label{accing}
The \ili{English} ``Acc-\form{ing}'' construction differs in many ways from the
``Poss-\form{ing}'' \isi{gerund} considered here so far.  I have no serious analysis of it to
offer.  Its behavior resembles Acc-Inf (``ECM'') constructions in some ways.
First, unlike gerunds with genitive subjects, it is degraded by intervening adverbs,
extraposition, and \isi{fronting}, under roughly the same conditions as nominal objects \citep{Portner1995,pires2006}:%\todo{Portner1995 missing from bib}
\begin{exe}
\ex\label{separation}
	\ea We anticipated (*?eagerly) him leaving Paris.
	\ex (We anticipated his resignation, but) *?him/his leaving Paris we did not anticipate.
	\z  
\end{exe}
This is the same pattern as:
\begin{exe}
\ex\label{separationa}
	\ea We believe (*?strongly) him to have told the truth. 
	\ex (We believed him to have been mistreated, but) *?him to be telling the truth we did not believe.
	\z    
\end{exe}


Acc-Inf gerunds allow extraction,  like Acc-Inf complements\is{complement} and  unlike Poss-\form{ing} gerunds:
\begin{exe}
\ex
	\ea Which city do you remember him/*his describing? (Portner 1995: 637, citing L. Horn)
	\ex Who do you resent Bill/*Bill's hitting?  \citep[263]{williams1975}
	\ex Who/*whose do you resent hitting Bill?  (cf.\ *Who do you resent (it) that hit Bill?)
	\ex Who do you believe to be telling the truth?
	\ex What do you believe him to be saying?
	\z    
\end{exe}
Another frequently noted difference between the constructions is that the genitive subject of
gerunds is preferentially human, and cannot be expletive \form{there} at all, whereas the
accusative is unrestricted in this respect, again like Acc-Inf subjects.
\begin{exe}
\ex
	\begin{xlist}
	\ex[]{There (*there's) being no objection, the proposal is approved.}
	\ex[?]{I imagined the water's being 30 feet deep.}
	\z
\end{exe}
Accusative subjects of gerunds do not seem to be getting their \isi{case} from the main verb, since
they can appear in gerunds that function as subjects.  Possibly the accusative \isi{case} assigner is
a null \isi{preposition} or complementizer, an analog of the overt \form{for} of \form{for-to}
infinitives. \largerpage

\section{Agent nominals, transitive and intransitive}
\label{agentsection}

Like the FNT, my alternative theory of nominalizations is in principle applicable to every type
of nominalization, including \isi{agent} nominalizations and result
nominalizations.  The mixed category that most gravely challenges analyses of \isi{agent} nominals is transitive\is{transitivity} \isi{agent}
nouns, which function as nominals except for assigning structural \isi{case} to their objects and
allowing some adverbial modifiers.  I will make a case that, just as gerunds are categorially
verbal at all levels of the syntax and their noun-like behavior is entirely due to a nominal
Case\is{case} feature\is{features} borne by their Infl head, such transitive\is{transitivity} \isi{agent} nominalizations are categorially
nominal at all levels of the syntax and their verb-like behavior is entirely due to a verbal
feature\is{features} borne by their nominalizing head, namely Aspect.  Gerunds and transitive\is{transitivity}
nominalizations thus prove to be duals in a sense – respectively verbs with Case\is{case} and nouns with
Aspect.

I show that this idea predicts the distinction between transitive\is{transitivity} and intransitive \isi{agent} nouns,
whereas the functional properties of nominalizations neither correlate with
each other as the FNT predicts, nor match the height of their nominalizing heads in syntax or
word structure.  In \ili{Vedic} \ili{Sanskrit} (sections \ref{vedagentsection}-\ref{vedtensesection}) and
in Finnish (\ref{finnsection}), high \isi{agent} nominalizations \textit{do not} assign structural
\isi{case} if they lack Tense/Aspect\is{tense} features,\is{features} and even low \isi{agent} nominalizations \textit{do} assign
structural \isi{case} if they have Tense/Aspect\is{tense} features.\is{features}  

\subsection{Agent nominals and subject nominals}
\label{vedagentsection}

In their illuminating study based on the FNT approach, \citet{baker2009} propose an analysis
and \isi{typology} of \isi{agent} nominalizations similar to the one I have called into question for event
nominalizations.  They begin by noting an asymmetry between \isi{agent} and event nominals.  ``High''
event nominals like \rfa{typ}{a} have no \isi{agent} noun counterpart such as \rf{agnom}.\largerpage

B\&V claim that this is a systematic gap, and propose to explain it on the basis of two key
assumptions.  First, agentive nominalizing \isi{morphology} is added by a nominal head immediately
above VP.\footnote{It is fair to ask why it is added there and not in a higher position.  B\&V
  hint that this is ``a position apparently forced on it by the natural (iconic) semantic
  composition of the clause'' \citep[521]{baker2009}, but this remains to be justified.} Secondly, in some
languages, such as \ili{English}, structural \isi{case} is assigned to objects by an active Voice/v head,
whereas in other languages, structural \isi{case} is assigned configurationally (dependent
case).\footnote{B\&V equate Voice with v, following \citet{kratzer2004}, but contra \citet{alexiadou2001,alexiadou2010,harley2012}, among others.} Together, these assumptions rule out transitive\is{transitivity}
agent-denoting nominalizations, such as \rf{agnom} \textit{*the reader the book}. Instead, they
require the structure \rf{agnoma}.  Here the \isi{agent} nominalizer pre-empts the case-assigning\is{case}
active Voice \isi{morpheme} in v that assigns structural \isi{case} to objects in \ili{English}, but (by
hypothesis) has no case-assigning\is{case} force itself.

\newpage 
\begin{exe}
\ex[*]{the reader of the book\\
\begin{forest}
	[DP
		[D
			[the]
		]
		[NP
			[N
				[-er]
			]
			[AspP
				[Asp, fit=band]
				[vP
					[(PRO), fit=band]
					[v\1
						[v, fit=band, name=v]
						[VP
							[V
								[read]
							]
							[DP
								[the book, roof, name=DP]
							]
						]
					]
				]
			]
		]
	]
	\draw[->,thick,>=stealth] (v.south) |- node[near end,below] {case} ([yshift=-3ex]DP.south) -|  (DP.south);
	\end{forest}
}\label{agnom}
\end{exe}
%\begin{exe}
%\ex\label{agnom}
%\setlength\treelinewidth{1pt}
%\setlength\arrowwidth{7pt}
%\setlength\arrowwidth{7pt}
%\setlength{\tabcolsep}{4pt}
%\label{low}
%\begin{tabular}{ccccccccccccccc}
%\multicolumn{8}{l}{*the reader the book}\\[2.5ex]
%	&\node{DP}{DP}	&	&	&	&	&	&	&		\\[2.5ex]
%\node{D}{D}	&	&\node{NP}{NP}	&	&	&	&	&		\\[2.5ex]
%\node{the}{the}	&\node{N}{N}	&	&\node{AspP}{AspP}	&	&		\\[2.5ex]
%	&\node{er}{-er}	&\node{Asp}{Asp}	&	&\node{VoiceP}{vP}		\\[2.5ex]
%	&	&	&\node{PRO}{(PRO)}	&	&\node{Voi}{v′}	&		\\[2.5ex]
%	&	&	&	&\node{Voice}{v}	&	&\node{VP}{VP}		\\[2.5ex]
%	&	&	&	&	&\node{V}{V}	&	&\node{DP2}{DP}		\\[2.5ex]
%	&	&	&	&	&\node{read}{read}	&
%        &\node{book}{\begin{minipage}[h]{9ex}{the book} \end{minipage}}\\[2ex]
%		&&&&&&\multicolumn{2}{c}{\footnotesize case}	
%\nodeconnect{DP}{D}
%\nodeconnect{D}{the}
%\nodeconnect{DP}{NP}
%\nodeconnect{NP}{N}
%\nodeconnect{N}{er}
%\nodeconnect{NP}{AspP}
%\nodeconnect{AspP}{Asp}
%\nodeconnect{AspP}{VoiceP}
%\nodeconnect{VoiceP}{PRO}
%\nodeconnect{VoiceP}{Voi}
%\nodeconnect{Voi}{Voice}
%\nodeconnect{Voi}{VP}
%\nodeconnect{VP}{V}
%\nodeconnect{VP}{DP2}
%\nodeconnect{V}{read}
%\nodetriangle{DP2}{book}
%\abarnodeconnect[-80pt]{Voice}{book}
%\end{tabular}
%
%\end{exe}

\begin{exe}
	\ex\label{agnoma}\begin{forest}
			[DP
				[D
					[the]
				]
				[NP
					[N, fit=band]
					[AspP
						[Asp, fit=band]
						[vP
							[(PRO), fit=band]
							[v\1
								[-er, fit=band, name=v]
								[VP
									[V
										[read]
									]
									[DP
										[of the book, roof, name=DP]
									]
			]
			]
			]
			]
			]
			]
		\end{forest}
\end{exe}
%\begin{exe}
%\ex\label{agnoma}
%\setlength\treelinewidth{1pt}
%\setlength\arrowwidth{7pt}
%\setlength\arrowwidth{7pt}
%\setlength{\tabcolsep}{4pt}
%\begin{tabular}{ccccccccccccccc}
%	&\node{DP}{DP}	&	&	&	&	&	&	&	&	&\\[2.5ex]
%\node{D}{D}	&	&\node{NP}{NP}	&	&	&	&	&	&	&	&\\[2.5ex]
%\node{the}{the}	&\node{N}{N}	&	&\node{AspP}{AspP}	&	&	&	&	&	&	&\\[2.5ex]
%	&\node{er}{}	&\node{Asp}{Asp}	&	&\node{VoiceP}{vP}	&	&	&	&	&	&\\[2.5ex]
%	&	&	&\node{PRO}{(PRO)}	&	&\node{Voi}{v′}	&	&	&	&	&\\[2.5ex]
%	&	&	&	&\node{Voice}{-er}	&	&\node{VP}{VP}	&	&	&	&\\[2.5ex]
%	&	&	&	&	&\node{V}{V}	&	&\node{DP2}{DP}	&	&	&       &\\[2.5ex]
%	&	&	&	&	&\node{read}{read}	&
%        &\node{book}{\begin{minipage}[h]{10ex}{of the book} \end{minipage}}\\
%\nodeconnect{DP}{D}
%\nodeconnect{D}{the}
%\nodeconnect{DP}{NP}
%\nodeconnect{NP}{N}
%\nodeconnect{NP}{AspP}
%\nodeconnect{AspP}{Asp}
%\nodeconnect{AspP}{VoiceP}
%\nodeconnect{VoiceP}{PRO}
%\nodeconnect{VoiceP}{Voi}
%\nodeconnect{Voi}{Voice}
%\nodeconnect{Voi}{VP}
%\nodeconnect{VP}{V}
%\nodeconnect{VP}{DP2}
%\nodeconnect{V}{read}
%\nodetriangle{DP2}{book}
%\end{tabular}
%\end{exe}
The analysis further predicts that, since voice markers cannot attach to
unaccusatives,\footnote{In fact an incorrect premise:  unaccusative verbs passivize in numerous
  languages, including Finnish and \ili{Sanskrit} \citep{kiparsky2013}.} such \isi{agent} nouns cannot attach to
unaccusative verbs.

B\&V then draw a distinction between agentive and non-agentive ``\isi{agent}'' nominalizers –
let's call the latter \textsc{subject nominalizers}. Subject nominalizers do assign structural
case,\is{case} and can be attached to unaccusative verbs.  B\&V (p.\ 547) analyze them as ``nominal
equivalents of an \textsc{aspect} head'', in the sense in which agentive nominalizers like
\form{-er} are nominal equivalents of a \textsc{voice} head.  Their example is Gikũyũ
\form{-i}, another example is Northern Paiute \form{-dɨ} \citep{toosarvandani2014}.  B\&V propose
the structure \rf{gik}:

%\newpage 

\begin{exe}
\ex\label{gik}Subject nominalizers (high)\\
\begin{forest}
for tree={calign=fixed edge angles, calign angle=60}
	[NP\bl{i}
		[N\bl{i}(/Asp)
			[-i]
		]
		[vP
			[NP\bl{i}
				[PRO]
			]
			[v\1
				[v\1
					[v
						[∅, name=v, fit=band]
					]
					[VP
						[V
							[slaughter]
						]
						[NP
							[goats, name=goats]
						]
					]
				]
			[(Adv)]
			]
		]
	]
	\draw[->,thick,>=stealth] (v.south) |- node[near end,below] {\textsc{acc} case} ([yshift=-3ex]goats.south) -|  (goats.south);
\end{forest}
\end{exe}
%\begin{exe}
%\ex\label{gik}
%\setlength\treelinewidth{1pt}
%\setlength\arrowwidth{7pt}
%\setlength\arrowwidth{7pt}
%\setlength{\tabcolsep}{4pt}
%\begin{tabular}{ccccccccccccccc}
%
%
% \multicolumn{7}{l}{Subject nominalizers (high) }\\[2.5ex]
%	\node{N}{}	&	&\node{AspP}{NP\bl{i}}	&	&	&	&	&	&	&\\[2.5ex]
%	&\node{Asp}{N\bl{i}(/Asp)}	&	&\node{VoiceP}{vP}	&	&	&	&	&	&\\[2.5ex]
%		&\node{er}{-i}		&\node{PRO}{NP\bl{i}}	&	&\node{Voia}{v′}&	&	&	&\\[2.5ex]
%                                &       &\node{PROO}{PRO}       	&\node{Voi}{v′}	&	&\node{Adv}{(Adv)}\\[2.5ex]
%		&	&\node{Voice}{v}	&	&\node{VP}{VP}	&	&	&	&\\[2.5ex]
%		&	&\node{null}{∅}	&\node{V}{V}	&	&\node{DP2}{NP}	&	&	&\\[2.5ex]
%		&	&	&\node{slaughter}{slaughter}	&                &\node{goats}{goats}\\[2ex]
%		&	&	&\multicolumn{2}{c}{\footnotesize \textsc{acc} case}	&	&	\\
%\end{tabular}
%\nodeconnect{Asp}{er}
%\nodeconnect{AspP}{Asp}
%\nodeconnect{AspP}{VoiceP}
%\nodeconnect{VoiceP}{PRO}
%\nodeconnect{PRO}{PROO}
%\nodeconnect{VoiceP}{Voia}
%\nodeconnect{Voia}{Voi}
%\nodeconnect{Voia}{Adv}
%\nodeconnect{Voi}{Voice}
%\nodeconnect{Voi}{VP}
%\nodeconnect{VP}{V}
%\nodeconnect{VP}{DP2}
%\nodeconnect{V}{slaughter}
%\nodeconnect{DP2}{goats}
%\nodeconnect{Voice}{null}
%\abarnodeconnect[-50pt]{null}{goats}
%\end{exe}

As an immediate challenge to the FNT in the domain of \isi{agent} nominalizations, B\&V note that
otherwise low \isi{agent} nominalizations unexpectedly assign structural \isi{case} in some languages.
For B\&V, these languages must be special in that they assign structural \isi{case} by a
dependent \isi{case} mechanism, whereas languages in which low \isi{agent} nouns have oblique complements\is{complement}
assign structural \isi{case} by little v.  The need to maintain two entirely distinct mechanisms of
structural \isi{case} assignment on the basis of evidence that cannot loom large in the learner's experience would
be another disappointing consequence of the FNT.\footnote{\citet{levin2015} propose that
  \textit{all} structural \isi{case} can be assigned by dependent case,\is{case} provided  that the
  algorithm is parametrized in certain ways.  However, they do not touch on the \isi{case} variation
  in objects of \isi{agent} nominalizations, and the parametrization of structural \isi{case} assignment
  that they propose does not account for it, as far as I can tell.} We'll also see that B\&V's
analysis of \isi{agent} nominals imposes a functional overload on little v that makes the FNT's
various criteria for syntactic height mutually irreconcilable.

In summary, B\&V's proposal generates the \isi{typology} of \isi{agent} nouns shown in \cref{tab:agnouns}. In the remainder of this section I show that the predicted correlations do not hold for \isi{agent}
nouns of \ili{Vedic} \ili{Sanskrit} and Finnish, and propose a much simpler alternative that does justice
to the data.

\begin{table}
  \begin{tabular}[t]{ll}
  \lsptoprule
  \multicolumn{1}{c}{Agent Nominalizers (low, v)}	  & \multicolumn{1}{c}{Subject Nominalizers (high, Asp)}\\
  \midrule
  always agentive			  &non-agentives OK	     \\
  no unaccusatives			  &unaccusatives OK	    \\
  structural \isi{case} only if dependent case  &structural case	     \\
  no adverbs				  &adverbs OK			       \\
  no   Aspect				  &compatible with Aspect		       \\
  no Voice & compatible with Voice\\
  \lspbottomrule
\end{tabular}
\caption{Typology of \isi{agent} nouns predicted by \citet{baker2009}.}
\label{tab:agnouns}
\end{table}

\subsection{\ili{Vedic} \isi{agent} nouns}

\ili{Vedic} and Pāṇinian \ili{Sanskrit} has a large number of \isi{agent} noun suffixes,\is{suffixation} which fall into two
clearly demarcated types.  A minimal pair that highlights the contrast are the two \isi{agent} noun
types in accented \form{-tár-\urf{N}} and preaccenting \form{′-tar-\urf{V}}.\footnote{Their
  \ili{Indo-European} provenance is guaranteed by Greek and \ili{Avestan} cognates \citep{lowe2014}.  The
  following exposition of their contrasting \isi{semantics}, morphophology, and syntax draws on the
  generalizations and evidence in \citealt{kiparsky2016}, to which the reader is referred for details.}
Agent nouns in accented \form{-tár-\urf{N}} have genitive objects and get only adjective
modifiers, never adverbs, e.g.\ \rfa{agex}{a}.  Agent nouns in preaccenting
\form{′-tar-\urf{V}} (boldfaced in \rf{agex}) regularly assign structural \isi{case} to their
objects and, can get certain aspectual adverb  modifiers, such as \textit{punaḥ} `again' in
\rfa{agex}{b}.\footnote{Other \isi{agent} nouns with verbal properties are attested in early \ili{Vedic}
  include \form{-i-\urf{V}} RV 9.61.20 \textit{jághnir vr̥trám} `killer of V̥tra',
  \form{-(i)ṣṇu-\urf{V}} RV 1.63.3 \textit{dhr̥ṣṇúr etā́n} `bold against them',
  \form{-u-\urf{V}} AV 12.1.48 \textit{nidhanám titikṣuḥ} `enduring poverty',
  \form{-∅-\urf{V}} RV 1.1.4 \textit{yáṃ yajñám … paribhū́r ási} `the sacrifice that you
  embrace'.}
\begin{exe}
\ex\label{agex}
	\ea \gll tvā́-ṃ  hí  {satyá-m} … vid-má  \textbf{dātā́r-am}  {iṣ-ā́m} \\
	you-\textsc{acc} \textsc{prt} {true-\textsc{acc}} … know-\textsc{1pl} \textbf{giver-\textsc{acc}} {good.thing-\textsc{pl}.\textsc{gen}}\\
	\glt  `we know you as the true giver of good things.' \hfill (\textit{RV.} 8.46.2)

	\ex \gll \textbf{íṣ-kar-tā} {víhruta-m} {púnaḥ}\\
	fixer-\textsc{nom} {wrong-\textsc{acc}}  {again}\\
	 \glt `the maker right again (of) what has gone wrong.' \hfill (\textit{RV} 8.1.12)
	\z 
\end{exe}

Both suffixes\is{suffixation} are true nominalizers: they form nouns, not verbs.  They have a complete nominal
\isi{case} and number \isi{inflection} paradigm,\is{paradigm} take denominal derivational suffixes, such as derived
feminines, and can be compounded.\is{compound}\footnote{E.g.\ \form{kṣirá-hotar-} `milk-offerer' (ŚBr.),
  and  \form{neṣṭā-potā́rau} `leader and purifier' (TS.), co-compounds
  \citep{kiparsky2010b} denoting  pairs of priests.}  They allow adjectival modification (in
addition to adverbial modification, in the case of \form{′-tar-\urf{V}}).  These nominal
properties are unsurprising for the noun-like \form{-tár-\urf{N}} formations; that they hold
also for the more verb-like \form{′-tar-\urf{V}} is documented in \rf{jetar}.
\begin{exe}
\ex\label{jetar}
 	\ea \glll āśúṃ  jétāram\\
	āśú-m  jé-tār-am\\
	quick-\textsc{acc} win-er-\textsc{acc}\\
	\glt `the quick (Acc.) winner (Acc.)' \hfill (\textit{RV} 8.99.7) 

	\ex \glll táṣṭeva {} pṣṭyāmayī́\\
   	tákṣ-tar iva  {pr̥ṣṭya-āmay-ín}\\
	 carve-er.\textsc{nom} like back-ache-ed.\textsc{nom}\\
	\glt `like a notalgic (Nom.) carpenter (Nom.)' \hfill (\textit{RV} 1.105.18)
	\z
\end{exe}

Semantically both \form{-tár-\urf{N}} and \form{′-tar-\urf{V}} are \textit{agent}
nominalizers, not \textit{subject} nominalizers: they are never added to non-agentive verbs or
unaccusatives of any kind, and the meaning of the nominalization is canonically
agentive.\footnote{Thus, the following roots\is{root} do \textit{not} take either \form{-tár-\urf{N}}
  and \form{′-tar-\urf{V}} or any other \isi{agent} suffixes for that matter: \form{as} `be',
  \form{ā́s} `sit', \form{śī} `lie', \form{sru} `flow', \form{plu} `float',
  \form{tras} `tremble', \form{vyath} `sway', \form{bhraṃś} `fall', \form{svap}
  `sleep', \form{kṣudh} `be hungry', \form{tr̥ṣ} `be thirsty', \form{svid} `sweat',
  \form{r̥dh} `flourish', \form{ru(d)h} `grow', \form{pyā} `swell', \form{riṣ}
  `sustain damage', \form{mr̥} `die', \form{śam} `become calm', \form{mad} `get drunk',
  \form{mud} `rejoice', \form{hr̥ṣ} `get excited', \form{dhr̥ṣ} `dare', \form{bhī}
  `fear', \form{hīḍ} `be angry', \form{krudh} `become angry', \form{gr̥dh} `be greedy',
  \form{ruc} `shine', \form{śubh} `shine, be beautiful', \form{bhā} `shine' \form{bhās}
  `gleam', \form{dyut} `to strike' (of lightning), \form{pat} `to fall'.} So by these
criteria both nominalizations are ``low'' in the sense of B\&V, not Gikũyũ-type ``high''
nominalizations.

The \isi{agent} nominalizers \form{′-tar-\urf{V}} and \form{-tár-\urf{N}} form a
\textit{privative} semantic opposition, missed in the modern philological literature but
correctly delineated already by Pāṇini, whose description turns out to tally perfectly with the
\ili{Vedic} data.  The unmarked member of the opposition is \mbox{\form{-tár-\urf{N}}}, which
simply denotes agency (like \ili{English} \form{-er}).  The marked member \form{′-tar-\urf{V}}
has {two} additional meaning components:
\begin{exe}
\ex
  \ea \form{′-tar-\urf{V}} denotes agency in \textsc{ongoing time}.  
  \ex \form{′-tar-\urf{V}} denotes \textsc{habitual, professional}, or \textsc{expert}
  agency. 
  \z
\end{exe}

The criteria of the FNT make contradictory predictions.  Since both nominalizations are
agentive, both should be structurally low ``little v'' heads.  On the other hand,
\form{′-tar-\urf{V}} nominalizations, which have the verbal properties of assigning
structural \isi{case} and allowing adverbial modification, should be structurally high, while
\form{-tár-\urf{N}} nominalizations, which have strictly nominal properties, should be
structurally low.  Neither of these is the case.  In fact, as far as the \isi{case} and adverb
properties are concerned, the structure is just the opposite of what is predicted:  verbal
\form{′-tar-\urf{V}} is low and nominal \form{-tár-\urf{N}} is high.  This is shown by four
arguments (details in \citealt{kiparsky2016}).

The first argument that verbal \form{′-tar-\urf{V}} is low and nominal \form{-tár-\urf{N}}
is high is their morphological position in the word. \form{′-tar-\urf{V}} always follows the
bare verbal \isi{root} directly, without any other intervening suffix;\is{suffixation} it cannot be added to \isi{compound}
or prefixed bases.  \form{-tár-\urf{N}}, on the contrary, can be separated from the \isi{root} by
verb-to-verb suffixes commonly analyzed as Voice/v heads (\isi{causative}, denominative, intensive,
desiderative).  It is affixed\is{affix} to the whole verb base, including the extended \isi{root} plus any
prefix that combines with it:
\begin{exe}
\ex
	\ea RV \form{cod-ay-i-tr-ī́-} `impeller (fem.)' (caus.\ \form{cod-áy-a-ti} `impels')
	\ex  TS \form{pra-dāp-ay-i-tár-} `bestower' (caus.\ \form{prá-dāp-ay-a-ti} `bestows'),
	\ex   \form{ni-dhā-tár-} `one who sets down' (\form{ní-dadhā-ti} `sets down')
 	\z
\end{exe}


%\newpage 
The morphological data point to the respective constituent structures in \rf{cs}:
\begin{exe}
\ex\label{cs}
	\ea \textit{Low:}  [~Prefix [~Root \form{′-tar-\urf{V}}~]~]  
	\ex \textit{High:}  [~[~Prefix [~Root (Caus)…~]~] \form{-tár-\urf{N}}~]   
	\z    
\end{exe}

The second argument that verbal \form{′-tar-\urf{V}} is low and nominal \form{-tár-\urf{N}}
is high comes from word accentuation.  The morphological conditioning of accent placement
provides a convenient probe into the constituent structure of words.  In \ili{Vedic} and Pāṇinian
\ili{Sanskrit}, the accentuation of words is computed cyclically from the accentual properties of the
morphemes\is{morpheme} from which they are composed.  Morphemes\is{morpheme} may be accented or unaccented, and at the
word level, all accents but the first in a word are erased \citep{kiparsky2010}.  Both of our \isi{agent}
suffixes\is{suffixation} (like the majority of derivational suffixes) belong to the accentually
\textsc{dominant} type:  they erase the accent off the bases to which they are added.  The
crucial fact for present purposes is that dominant affixes\is{affix} exercise this erasing effect exactly
on the stems\is{stem} to which they are added, no more and no less.  Thanks to this property we can use
accentuation to diagnose constituent structure in morphologically complex\is{complexity} words.

The empirical generalization is that prefixes always prevail over low (bare-root) suffixes,\is{suffixation}
including \form{′-tar-\urf{V}}, whereas high suffixes always prevail over prefixes, dictating
the place of the word accent.  The reason is that prefixes are added after the low suffix
\form{′-tar-\urf{V}}:
\begin{exe}
\ex
Prefixation to nouns with the the low suffix \form{′-tar-\urf{V}}:\\
\begin{tabular}{@{}ll}
 bhar-              &Root\\ 
bhár-tar\urf{V}     &add dominant preaccenting \form{′-tar-\urf{V}}\\
prá-{[}bhár-tar]   &add accented prefix\\
\form{prábhartar-}   & erase all accents but the first
\end{tabular}

\end{exe}
On the other hand, \mbox{\form{-tár-\urf{N}}} is accentually dominant, causing all accents on
its base to be deleted, and attracting accent to itself. This shows that it is added to the
entire stem including the prefix, causing the resulting word to be accented on the suffix.

\begin{exe}
\ex
Suffixation\is{suffixation} of high \mbox{\form{-tár-\urf{N}}} to prefixed verbs:\\
\begin{tabular}{@{}ll}
bhar-            &Root\\ 
ápa-bhar            &add accented prefix\\
{[}apa-bhar]-tár\urf{N}      &add dominant accented \form{-tár-\urf{N}}\\
\form{apabhartár-}  
\end{tabular}

\end{exe}

The third argument that verbal \form{′-tar-\urf{V}} is low and nominal \form{-tár-\urf{N}}
is high comes from tmesis, the splitting of prefixes from stems.\is{stem}  Prefixes can be separated
from verbs and from nominals formed with low suffixes like verbal \form{′-tar-\urf{V}}.
\begin{exe}
\ex
    	\ea  \textit{sáttā ní yónā (= nísattā yónā)} `a sitter down in the womb' (\textit{RV} 9.86.6) 

	\ex  \textit{úpa sū́re ná dhā́tā} (= \textit{sū́re nópadhātā}) `like the Placer of the
  Sun' (\textit{RV} 9.97.38).  
	\z
\end{exe}
Prefixes are never separated from nominals formed with high suffixes such as nominal
\form{-tár-\urf{N}}.

The explanation comes from  the same constituent structure that accounts for the accentual
difference: low suffixes\is{suffixation} such as the \isi{agent} suffix \form{′-tar-\urf{V}} are added directly to
the \isi{root} to form a noun, which can then be composed with a prefix (see \rfa{baree}{a}), while high
suffixes such as the \isi{agent} suffix \form{-tár-\urf{N}} are added to the entire verb, which may
already bear a prefix and/or another suffix \rfa{baree}{b,c}.
\begin{exe}
\ex\label{baree}\begin{tabularx}{\linewidth}[t]{@{}lLlL}
	a. & \begin{forest}
		where n children=0{tier=word}{}
			[N
				[Prefix]
				[N
					[V]
					[\form{′-{tar-}\urf{V}}]
				]
			]
		\end{forest}
	& b. & \begin{forest}
			where n children=0{tier=word}{}
				[N
					[V
						[Prefix]
						[V]
					]
					[\form{-tár-\urf{N}}]
				]
			\end{forest}\\
	\end{tabularx}\\
	\begin{tabularx}{\linewidth}[t]{@{}lLlL}
	c. & \begin{forest}
			where n children=0{tier=word}{}
				[N
					[V
						[Prefix]
						[V
							[V]
							[Caus\dots{}]
						]
					]
					[\form{-tár-\urf{N}}]
				]
			\end{forest} & & \\
	\end{tabularx}
\end{exe}
%\begin{exe}
%\ex\label{baree}
%\setlength\treelinewidth{1pt}
%\setlength\arrowwidth{7pt}
%\setlength\arrowwidth{7pt}
%\begin{tabular}{cccccccccc}
%(a) 
%&                       \node{NStema}{N}       &       &       \\[2.5ex]
%&                       &\node{NStem}{N}       &       &       \\[2.5ex]
%
%\node{Prefix}{Prefix}&\node{VRoot}{V}  &             &\node{NAgent}{\form{′-{tar-}\urf{V}}}
%
%\nodeconnect{NStema}{NStem}
%\nodeconnect{NStema}{Prefix}
%\nodeconnect{NStem}{VRoot}
%\nodeconnect{NStem}{NAgent}
%\end{tabular}
%\begin{tabular}{cccccccccc}
%(b) 
%&&                       \node{NStema}{N}            &       \\[2.5ex]
%&\node{PrefixV}{V}                       &       &       &       \\[2.5ex]
%\node{Prefix}{Prefix}&&\node{VRoot}{V}              &\node{NAgent}{\form{-{tár-}\urf{N}}}
%\nodeconnect{NStema}{NAgent}
%\nodeconnect{NStema}{PrefixV}
%\nodeconnect{PrefixV}{VRoot}
%\nodeconnect{PrefixV}{Prefix}
%\end{tabular}
%\begin{tabular}{ccccccccccccccc}
%(c)        &               &       \node{NStem}{N}  &       &       &       \\[2.5ex]
%
%        &\node{VStem}{V}           &       &       &       &       &       \\[2.5ex]
%
%        &               &\node{VRoott}{V} &       &       &       &       \\[2.5ex]
%
%\node{Prefix}{Prefix} &\node{VRoot}{V}  &        &\node{Caus}{Caus…}         &\node{NAgent}{\form{-tár-\urf{N}}}
%
%\nodeconnect{NStem}{VStem}
%\nodeconnect{NStem}{NAgent}
%\nodeconnect{VStem}{Prefix}
%\nodeconnect{VStem}{VRoott}
%\nodeconnect{VRoott}{VRoot}
%\nodeconnect{VRoott}{Caus}
%\end{tabular}
%\end{exe}
It will be seen the prefix is an immediate constituent of the word in \rfa{baree}{a}, but not
in \rfa{baree}{b} or in \rfa{baree}{c}.  The natural generalization is that a prefix can only
be split if it is an immediate constituent of the word.

The fourth argument that verbal \form{′-tar-\urf{V}} is low and nominal \form{-tár-\urf{N}}
is high comes from selectional properties of prefixes.  Prefixes that only combine with verb
roots\is{root} require high \form{-tár-\urf{N}}, because the right-branching constituent structure
\rfa{baree}{a} would require them to combine with nouns.\footnote{Many examples are given in
  \citealt{kiparsky2016}. One will have to suffice here. The interjection \form{hiṁ} `the sound
  \form{hmm}' cannot be compounded\is{compound} with nouns.  It can only combine with the \isi{root}
  \form{kṛ} `do', `make'.  The \isi{agent} noun from \form{hiṁ-kṛ-} `to make the
  sound \form{hmm}' must therefore have the high suffix \form{-tár-\urf{N}}, viz.\
  \form{hiṁkartár-}.} Conversely, prefixes and other elements that cannot be combined
with roots,\is{root} only with nouns, require the right-branching constituent structure \rfa{baree}{a},
which is available either with \form{-tár-\urf{N}} or with
\form{′-tar-\urf{V}}.\footnote{Again we must make do with a couple of examples.  There is no
  \isi{compound} verb such as \form{*para-apara-i-} `to go far and near' from which
  \form{párāpara-etar-} `one who goes far and near' might be derived.  In fact
  \form{párāpara-} `far and near' is never compounded\is{compound} with verbs.  Instead, the \isi{agent} noun is
  a nominal \isi{compound} formed from \form{para-apara-} `far and near' plus \form{e-tár-} `goer' (←
  \form{i-tár-\urf{N}}).  Another illustration of this generalization is that the negation
  \form{a-} combines only with nouns.  From \form{hótar} `priest' (←
  \form{hu-′tar-\urf{V}}) we get \form{á-hotar-} `a non-priest'.}


The above arguments establish the morphological constituency displayed in \rf{cs} and
\rf{baree}.  But \isi{Distributed Morphology} is a resourceful theory that makes available various
\isi{movement} operations that cause mismatches between \isi{morphology} and syntax. So could the
morphologically low nominalizing morphemes\is{morpheme} be spelled out high where B\&V predict they should
be, and then undergo Lowering to their actual position?  And conversely, could the high
nominalizing morphemes\is{morpheme} be spelled out low as predicted, and then undergo Raising to their
actual position?  The answer is negative on both counts.

   The way morphologically low suffixes\is{suffixation} such as the \isi{agent} suffix \form{′-tar-\urf{V}} could
   be syntactically high for purposes of the FNT is by DM's \textsc{Lowering} operation, which
   applies before Vocabulary insertion to adjoin a head to the head of its \isi{complement} \citep{embick2001}: \begin{exe}
\ex

X\trf{0} lowers  to Y\trf{0}:

[\trf{XP} X\trf{0} …[\trf{YP} …Y\trf{0} …]] → [\trf{XP} …[\trf{YP} …[Y\trf{0} Y\trf{0}+X\trf{0}]…]]
\end{exe}
In \ili{English}, Lowering of T is assumed to adjoin T to v \citep{embick2008}.
But if \form{′-{tar-}\urf{V}} is generated in the N head of \rf{gik} and is lowered from
there into the v, it would, on B\&V's assumptions, have the properties of a subject
nominalizer, forming nouns from unaccusative and non-agentive verbs.  But it does \textit{not}
have any properties of a subject nominalizer – it is a true \isi{agent} nominalizer, as we showed
above.

The other thing required to maintain the FNT is to generate \form{-{tár-}\urf{N}} low in v
(as in \rf{agnoma}) and then raise it to its actual high position.  This is not going to work
in the DM model either, for morphological head-raising can only adjoin a head to the next head
above it \citep{harizanov2016}, and in this case it would have to skip intervening heads,
including the v head that may be occupied by \isi{causative} and other V→V suffixes,\is{suffixation} which must not
raise.  Moreover, in more recent DM \citep{embick2010}, \isi{phonology} is cyclically interleaved with
morphology,\is{morphology} and this would cause problems with the abovementioned accent erasure and tmesis
phenomena.

It should also be noted that \form{′-{tar-}\urf{V}} is overwhelmingly preferred when its
special meaning and morphological restrictions allow its use, and \form{-{tár-}\urf{N}} is
used elsewhere. Moreover, other \isi{agent} suffixes\is{suffixation} supersede each of them with particular roots\is{root}
and/or in particula special meanings. Since competition in DM obtains only between morphemes\is{morpheme}
that have the same meaning and are realized in the same slot (such as \ili{English} \isi{plural}
\form{-s} and \form{-en}), all these competing suffixes would have to be generated in the
same syntactic position.

The conclusion from the \ili{Vedic} data is that the nominalizations' verbal vs.\ nominal properties
do not correlate with structural height of their heads in the word or in the syntax.  In fact,
the majority of morphologically high nominalizers in \ili{Vedic} have nominal properties, and the
majority of morphologically low nominalizers have verbal properties – the opposite of what the FNT
predicts.


\subsection{Aspect in \ili{Vedic} \isi{agent} nouns}
\label{vedtensesection}
A preliminary survey of nominalizations suggests that the Aspect feature\is{features} of a
nominalizer is the best predictor of verbal properties.  Consider the following alternative to
the FNT.
\begin{exe}
\ex\label{TAH}
The Aspect hypothesis

Nominalizations assign structural \isi{case} if and only if they have Aspect, either as an
inherent feature of the nominalization, or in virtue of combining with overt Aspect morphology.\is{morphology}
\end{exe}
By \isi{aspect} \isi{features} I mean outer \isi{aspect} \isi{features} such as \isi{imperfective} and habitual, not inner
\isi{aspect} (Aktionsart), such as \isi{telicity}.  \ili{Sanskrit} \isi{agent} nouns in \form{′-tar-\urf{V}} are
inherently present/\isi{imperfective} and habitual.  Those in \form{-{tár-}\urf{N}}, \ili{English}
\isi{agent} nouns made with  \form{-er}, and Finnish \isi{agent} nouns made with
\form{-ja} have no inherent \isi{aspect}: a \textit{driver} can be a habitual or professional
driver, or just someone who happens to be at the wheel on a particular
occasion.\footnote{Gerunds are arguably inherently \isi{imperfective} \citep{alexiadou2001,alexiadou2010}.}

The Aspect hypothesis is not implausible a priori because Aspect \isi{features} are
cross-linguistically known to affect \isi{case} assignment  – think of \isi{split ergativity}
based on imperfect vs.\ perfect \isi{aspect}, and accusative vs.\ partitive objects in Finnish
depending on gradability \citep{kiparsky2005b}.  It looks promising
for \ili{Vedic} \ili{Sanskrit} in particular because nominalizing endings with verbal properties, such as
\form{′-tar-\urf{V}}, are added to the bare root,\is{root} in the same morphological position as the
Aorist and Perfect Tense/Aspect\is{tense} suffixes.  It is also consistent with the fact that Northern
Paiute deverbal nominalizations, which assign structural accusative case,\is{case} can have overt
aspectual \isi{morphology} below them \citep[793, fn.\ 6]{toosarvandani2014}.

The Aspect hypothesis is compatible with a lexicalist treatment of morphology.\is{morphology}  A Distributed
Morphology\is{morphology} analysis of \ili{Vedic} \isi{agent} nouns is problematic because of the conflicting criteria for
structural height.  In addition, they show a type of competition between morphemes\is{morpheme} that DM
rejects.  The semantically nondescript \form{-tár-\urf{N}}, structurally low by B\&V's
criteria but high in word structure, is the default (elsewhere) case.  The semantically
restricted \form{′-tar-\urf{V}} suffix, structurally high by B\&V's syntactic criteria but
low in the word morphology,\is{morphology} is strongly preferred whenever it is applicable, namely to denote
habitual agency in ongoing time with morphologically simple verbs.  Elsewhere the default is
\form{-tár-\urf{N}}, structurally low by B\&V's syntactic criteria but high in the word
\isi{morphology} – for past or future agency, or occasional agency, \textit{or} when the
verb is morphologically complex\is{complexity} (\isi{causative}, intensive, desiderative, denominative, or prefixed.
Suppose then that a structurally low \isi{agent} is added in a syntactic derivation.\is{derivation}  The \isi{derivation}
must crash if and only if a structurally high \isi{agent} can be successfully added in a competing
derivation.  But DM does not allow rules that spell out syntactico/semantic \isi{features} in
different positions to compete with each other.  Moreover, if we assume bottom-up morphological
spellout of the syntax (by cycles or phases), the syntactically low \isi{agent} would have to
``know'' about the upstairs high \isi{agent} in order to be blocked by it.  On the other hand, in a
\textit{morphological}\is{morphology} theory of word-formation, morphologically low items naturally block
morphologically high items.  Besides, blocking of affixes\is{affix} with general meanings by affixes with
special meanings regardless of the locus of affixation is straighforward in lexicalist
approaches such as those of \citet{wunderlich1996k,wunderlich2001k} and \citet{kiparsky2005}.

\subsection{The Finnish subject nominalizer \form{-ja}}
\label{finnsectionja}
B\&V's formulation of the FNT entails that \isi{agent} nominalizations don't assign \isi{case} and subject
nominalizations do (\sectref{sec:kikarsky:2}).  We have seen that \ili{Vedic} falsifies the first of these claims.
Finnish (among many other languages) falsifies the second.  The fully productive Finnish suffix\is{suffixation}
\form{-ja} is not an \isi{agent} nominalizer, but a subject nominalizer, which is to say a high
nominalizer in B\&V's \isi{typology}.  It can go on non-agentive/unaccusative verbs, and freely
attaches to causatives, often assumed to be under v, as well as denominatives, reflexives,
inchoatives, and inner \isi{aspect} morphemes\is{morpheme} such as frequentatives and semelfactives, thus testing
positively for high position by several diagnostics.
\begin{exe}
\ex
  \ea
  \form{kuolija} `one who dies', `dier', \form{eläjä} `one who lives', `liver',
  \form{toipuja} `one who gets well', `convalescent', \form{olija} `one who is',
  \form{osaaja} `one who is able to', \form{syntyjä} `one who is born', \form{hikoilija}
  `one who sweats', \form{putoaja} `one who falls', \form{turpoaja} `one who swells',
  \form{pelkääjä} `one who fears', \form{luulija} `one who supposes', \form{tuntija} `one
  who knows, expert', \form{muistaja} `one who remembers', \form{jääjä} `one who remains',
  \form{palelija} `one who feels cold', \form{tarvitsija} `one who needs', \form{hukkuja}
  `one who drowns'
\ex
	\ea Frequentative \form{-ele-}: \form{kys-eli-jä} `inquirer', from \form{kys-ele-} `to
  make inquiries' (cf.\ \form{kysy-jä} `asker', from \form{kysy-} `ask'). Similarly
  \form{ryypiskelijä} `tippler', \form{lähentelijä} `harasser', \form{myyskentelijä}
  `peddler', \form{rehentelijä} `bragger', \form{riitelijä} `quarreler'.

	\ex Causative: \form{laula-tta-ja} `one who makes sing', from \form{laula-tta-} `to make
  sing', cf. \form{lauja-ja} `singer', from \form{laula-} `to sing'.

	\ex Inchoative + \isi{causative}: \form{selv-en-tä-jä} `clarifier', ←  \form{selv-en-tä-}
  `make clear, clarify' (← \form{selv-en-} `become clear' ← \form{selvä} `clear'). 


	\ex Causative + frequentative: \form{sopi-ja} `agreer', \form{sovi-tta-ja} `fitter, arranger',
  \form{sovi-tt-el-ija} `reconciler, negotiator'


	\ex Reflexive: \form{puolusta-ja} `defender'  \form{puolusta-utu-ja} `(self-)defender'

	\ex Denominative:  \form{testamentt-aa-ja} `bequeather' (← \form{testamentt-at-} `to
  bequeath' ←  \form{testamentti} `testament')
	\z
\z
 \end{exe} 
 However, nominals in the suffix \form{-ja} do not assign structural
case,\is{case}  whether they have any of these suffixes below them or not.  Their object
\isi{complement} (unlike that of passive verbs) can only receive genitive case.\is{case}
\begin{exe}
\ex
\ea \gll palkinto-j-en (*palkinno-t) saa-ja \\
prize-\textsc{pl}-\textsc{gen} (prize-\textsc{pl}.\textsc{acc}) get-er.(\textsc{nom})\\
\glt `the/a winner of the prizes'

\ex \gll minu-n (*minu-t) käv-el-ytt-eli-jä-ni \\
me-\textsc{gen} (*me-\textsc{acc}) walk-\textsc{freq}-\textsc{caus}-\textsc{freq}-er.(\textsc{nom})-\textsc{1sg}.\textsc{poss}\\
\glt `(the) one who frequently takes me around for walks'  
\z
\end{exe}
Nominals in \form{-ja} do take oblique nominal modifiers, as do all nominalizations, including
action nominalizations, and even to some extent ordinary basic nouns.
\begin{exe}
\ex\label{cann}
	\ea \gll Saksa-sta voitta-ja-na palaa-ja / pal-uu \\
    Germany-\textsc{elat} win-er-\textsc{essive} return-er.(\textsc{nom}) / return-ing.(\textsc{nom})\\
    \glt `(the) one who returns / a return from Germany as a winner' 
 
	\ex \gll hallitse-va-ssa asema-ssa oli-ja / ol-o \\
govern-in-\textsc{iness} position-\textsc{iness} be-er.(\textsc{nom}) / be-ing.(\textsc{nom}) \\
\glt `(the) one who is in a governing position'

  	\ex \gll palatsi Cannesi-ssa\\
    palace     Cannes-\textsc{iness}\\
    \glt `the/a palace in Cannes'
    \z 
\end{exe}
They generally do not take adverbs, except for certain perfectivizing adverbs \rfa{particle}{d,e}:
\begin{exe}
\ex\label{particle}
\begin{xlist}
\ex[*]{\gll nopea-sti juoksi-ja\\
quick-ly run-er.(\textsc{nom})\\
\glt `(the) one who runs/ran/will run quickly'}

\ex[*]{\gll kilpailu-n taas voitta-ja\\
competition-\textsc{gen} again win-er.(\textsc{nom})\\
\glt `(the) one who wins/won/will win the/a competition again'}

\ex[*]{\gll aina matkusta-ja\\
always travel-er.(\textsc{nom})\\
\glt `(the) one who always travels'}

\ex[]{\gll kilpailu-sta pois jää-jä\\
competition-\textsc{elat} away remain-er.(\textsc{nom})\\
\glt `one who does not join the competition', `eliminee'}
\ex[]{\gll viime-ksi tuli-ja \\
lat-\textsc{transl} come-er.(\textsc{nom})\\
\glt `the last to arrive'}
\end{xlist}
 \end{exe} 
 In \rf{cann2} (an example adapted from the internet) the adverb
\form{jälleen} `again' appears with an \isi{agent} noun.
\begin{exe}
\ex\label{cann2} 
 \gll Cannesi-ssa jälleen palkinno-tta jää-jä \\
    Cannes-\textsc{iness}  again prize-\textsc{abess} remain-er.(\textsc{nom})\\
    \glt `one who ended up prizeless again in Cannes' (lit. `a remainer prizeless again…')
\end{exe}
Possibly it modifies not the nominalization but the abessive modifier \form{palkinno-tta}
`prizeless'.

Since Finnish \form{-ja} must be high in order to get a non-agentive interpretation and to
\isi{scope} over every kind of verb-to-verb suffix, it should assign structural case,\is{case} which it
doesn't.  So it does not fit into B\&V's syntactic \isi{typology}, and constitutes a problem for the
FNT generalization.  In this case, morphological raising or lowering, even if they were
available and motivated, would not help to resolve the contradiction.

The lexicalist alternative, however, holds up.  Like \ili{English} \form{-er}, nouns in
\form{-ja} are morphologically incompatible with overt Aspect or Voice morphology,\is{morphology} and
 they refer indifferently to prospective, present, or past events, hence
have no inherent Aspect features.\is{features}
\begin{exe}
\ex
  \form{maksaja} `payer' (one who has paid, is paying, or will pay), similarly
  \form{ostaja} `buyer', \form{vuokraaja} `renter', \form{maahanmuuttaja} `immigrant',
 \form{lähtijä} `goer', \form{siittäjä} `inseminator'
\end{exe} 
Since \form{-ja} has no Aspect \isi{features} and no verbal properties, the fact that it
doesn't assign structural \isi{case} is consistent with the Aspect hypothesis but
inconsistent with the FNT.  I conclude that Finnish \form{-ja} supports the lexicalist
analysis of nominalizations.

\subsection{The Sakha \isi{agent} nominalizer \form{-AAccY}}

\citet[536]{baker2009} note that the correlation predicted by the FNT breaks down for Sakha \isi{agent}
nominalizations, which have structural accusative objects but otherwise conform to the low
type, in that they have no Aspect morphology or adverbs.  Their solution is that Sakha
accusative \isi{case} is not assigned by Voice/v but by a different mechanism, \textsc{dependent
  case}\is{case} assigment.
\begin{exe}
\ex\label{depcase}
{Dependent case assignment}  \citep{marantz1991,baker2015}

\ea If there are two distinct NPs in the same spellout domain such that NP\textsubscript{1} c-commands NP\textsubscript{2},
  then value the \isi{case} feature\is{features} of NP\textsubscript{2} as accusative unless NP\textsubscript{1} has already been marked for case.\is{case}
\ex If there are two distinct NPs in the same spellout domain such that NP\textsubscript{1} c-commands NP\textsubscript{2},
  then value the \isi{case} feature\is{features} of NP\textsubscript{1} as \isi{ergative} unless NP\textsubscript{2} has already been marked for case.\is{case}
 \z 
\end{exe}
Their main argument that Sakha accusative is dependent \isi{case} is that objects of passives receive
accusative case.\is{case}  This argument depends on the fragile assumption that the implicit \isi{agent} of
Sakha passives is a syntactically visible but phonologically null NP, which receives nominative
\isi{case} and serves as the NP\textsubscript{1} that triggers the assignment of accusative \isi{case} to the object of the
passive by \rfa{depcase}{a}.

I am skeptical of this solution for both theoretical and empirical reasons.  It would be
strange for UG to offer two entirely different methods of structural \isi{case} assignment, since
their empirical differences are rather obscure, and offer learners of most languages little
core data to choose between them.  Secondly, the analysis of impersonal passives as having an
invisible nominative \isi{agent} subject is excluded on general grounds by any kind of demotion
analysis of passive, including the typologically grounded theory proposed in \citealt{kiparsky2013}.
Finnish provides empirical evidence against the idea that objects of passives receive dependent
structural \isi{case} because of a syntactically visible but phonologically null nominative implicit
\isi{agent}.  The object of passives in Finnish is assigned structural case as in Sakha, but the \isi{case}
cannot possibly be assigned by the dependent \isi{case} algorithm \rf{depcase}, for the implicit
\isi{agent} of passives in Finnish is \textit{invisible} to \isi{case} assignment, as clearly demonstrated
by Jahnsson's Rule \rfa{case}{b}, see e.g.\ \rfa{tienn}{c,d}.

Our approach predicts the \isi{transitivity} of Sakha \isi{agent} nouns in \form{-aaccy} out of the box.
The reason is that they have an \isi{aspect} feature. \is{features} Agent nouns in Sakha \form{-aaccy} denote
specifically habitual or generic agents.  B\&V (p.\ 531) illustrate this generalization with
the following examples:\footnote{This component is foregrounded in the related habitual
  participle function of the same suffix:  e.g.\ \form{salaj-aaccy} means both `manager'
  (\isi{agent} noun) and `habitually managing' (participle), see \citet[123]{vinokurova2005}.}
\begin{exe}
\ex
\begin{xlist}
\ex[]{\gll ynaq-y ölör-ööccü\\
cow-\textsc{acc} kill-\textsc{ag}.\textsc{nom}\\
\glt `a killer of cows, a butcher'}
\ex[*]{\gll Misha-ny ölör-ööccü\\
Misha-\textsc{acc} kill-\textsc{ag}.\textsc{nom}\\
\glt `the killer of Misha'}
\end{xlist}
\end{exe}
The habitual \isi{aspect} feature\is{features} of the Sakha \isi{agent} nouns licenses its structural \isi{case} assignment
just as the habitual \isi{aspect} feature\is{features} of \isi{agent} nouns formed by the \ili{Sanskrit} \isi{agent} suffix
\form{′-tar-\urf{V}} does, as opposed to aspectually void \isi{agent} nouns in
\form{-tár-\urf{N}}, Finnish \form{-ja}, and \ili{English} \form{-er}.  This accounts fully for
the \isi{case} data without resorting to the unsupported syntactic height distinctions demanded by
the FNT.

Summing up our conclusions about \isi{agent} nominalizations so far:  the syntactic FNT is falsified
by \ili{Vedic} \isi{agent} nominalizations in one direction, and by Finnish subject nominalizations in the
other, and it requires an otherwise unsupported parametric choice between two heterogeneous
structural \isi{case} assignment algorithms.  The analysis reveals that little v can't do
\textit{all} of the following things: (1) introduce Agents, (2) host voice heads or \isi{agent}
nominalizer heads, (3) host \isi{causative} V affixes,\is{affix} (4) host aspectual material, and (5) assign
structural case.\is{case}  In \isi{agent} nominals it is not possible to place nominalizing heads above or
below little v in a consistent way that satisfies all of (1)-(5).  (1) and (5) cannot be
reconciled with an \isi{agent} nominalizer that assigns structural \isi{case} such as \ili{Sanskrit}
\form{′-tar-\urf{V}}, or with a subject nominalizer that does not assign structural \isi{case} such
as Finnish \form{-ja}.  The Sakha nominalizer \form{-AAccY} can dominate causatives (high)
but not aspectual adverbs – a conflict between (3) and (4) – and introduces agents (low) but
assigns structural \isi{case} (high) – a conflict between (2) and (5).

These difficulties fall away if we assume that that \isi{agent} nominals are nouns, and that nouns assign
structural \isi{case} if and only if they have Aspect features. \is{features}


\section{Conclusion}

The Functional Nominalization Thesis claims that so-called ``mixed categories'' arise when a nominal head is affixed to an extended verbal projection that is its syntactic complement.\is{complement}
My findings instead support a lexicalist approach, in which mixed categories are projections of
a nominal or verbal heads with an extra phi-feature.\is{features}  Their extended projections behave like
normal extended projections modulo the properties enforced by that feature.

In \S\ref{gerundsection} I argued that \isi{gerund} phrases are not DPs/NPs with AspP complements.\is{complement}
They are not even nominalizations.  They are participial phrases – IPs with a Case\is{case} feature\is{features} that
is checked or valued in an argument\is{arguments} position.  In all other respects their syntax is entirely clausal:
they lack DP material such as articles, demonstratives, quantifiers, and adjectives, and they
are formally built like IPs, complete with structural subjects.  The lexicalist analysis explains 
these properties.


In \S\ref{agentsection} I argued that \isi{agent} nominalizations that assign structural
\isi{case} to their objects are not nouns with vP complements\is{complement} (or with any other phrasal
complements), but deverbal nouns derived by \isi{agent} suffixes\is{suffixation} that have an Aspect feature.\is{features}  The
Aspect feature\is{features} makes the nouns transitive, \is{transitivity}and modifiable by aspectual adverbs.  Otherwise
their syntax is entirely nominal.  The merit of this analysis is that it tightly correlates the
\isi{transitivity} of \isi{agent} nouns with their aspectual meaning.  Also, by relieving the burden on
little v, it eliminates the mismatches between word structure and syntax that we found in \ili{Vedic}
and Finnish \isi{agent} nouns under the FNT analysis.

The lexicalist approach retains the key idea of the FNT without the typologically unwarranted
overgeneration caused by allowing syntactic affixation.\is{affix} It preserves a uniform mechanism of
structural \isi{case} assigment, a unified analysis of true nominalizations, and the insights that
originally led Chomsky to lexicalism.


%\section*{Abbreviations}
\section*{Acknowledgements}
I am grateful to Vera Gribanova and to an
    anonymous reviewer for their very helpful comments and queries.

{\sloppy
\printbibliography[heading=subbibliography,notkeyword=this]
}
% \todos
\end{document}
