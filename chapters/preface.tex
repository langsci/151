\addchap{Preface}
\begin{refsection}

%content goes here
On Looking into Words is a wide-ranging volume spanning current research into word structure and morphology, with a focus on historical linguistics and linguistic theory. The papers are offered as a tribute to Stephen R. Anderson, the Dorothy R. Diebold Professor of Linguistics at Yale, who retired at the end of the 2016--2017 academic year. The contributors are friends, colleagues, and former students of Professor Anderson, all important contributors to linguistics in their own right. As is typical for a Festschrift, the contributions span a variety of topics relating to the interests of the honorand. In this case, the central contributions that Anderson has made to so many areas of linguistics and cognitive science, drawing on synchronic and diachronic phenomena in diverse linguistic systems, are represented through the papers in the volume. 

The 23 papers that constitute this volume are unified by their discussion of the interplay between synchrony and diachrony, theory and empirical results, and the role of diachronic evidence in understanding the nature of language. Central concerns of the volume include morphological gaps, learnability, increases and declines in productivity, and the interaction of different components of the grammar. Although the volume is not divided into labeled parts, the papers deal with a range of linked synchronic and diachronic topics in phonology, morphology, and syntax (in particular, cliticization), and their implications for linguistic theory. 

Several of the papers take as their general topic synchrony and diachrony, tackling the relationship between synchronic alternations and diachronic change. Thráinsson, for example, uses evidence from Modern Icelandic and Modern Faroese to argue for different trajectories in historical change in \emph{u}-umlaut, and differing degrees of synchronic productivity of the umlaut rules. Maiden argues for the use of comparative and historical evidence in deciding between possible synchronic interpretations of phenomena, a point also made by de Chene for Japanese. %While some papers use diachronic evidence to decide between synchronic interpretations, Kiparsky uses synchronic analysis (of additive and associative plurals across a variety of languages) to argue for a diachronic origin of the dual in Indo-European.

Some  papers focus specifically on morphology. Maiden revisits paradigmatic phenomena in Romansch and investigates whether allomorphy in verb stems is phonologically conditioned or is evidence for paradigm autonomy. In this, he continues a long-standing debate with Anderson about the status of paradigms and their role in shaping change. The theme of residue and productivity is also taken up in Lepic \& Padden’s contribution. Yang’s paper also uses data from Romansch (as well as English) as a test case for a proposal for identifying where rules are likely to be productive and how morphological gaps appear. 

Hyman \& Inkelas present a different type of morphological problem. Lusoga, a Bantu language spoken in Uganda, has multiple exponence of causative, applicative, subjunctive and perfective suffixes, which the authors argue to pose challenges for theories of morphology that minimize redundancy and that treat derivation as distinct (and ordered before) inflection. Lusoga thus provides evidence for a previously unidentified pathway to multiple exponence. Stump also considers multiple exponence in Limbu, a Tibeto-Burman language of Nepal. 

Several papers concentrate on phonology at the level of the word (and below). Kavitskaya provides a generalization about a class of apparent exceptions to rules of compensatory lengthening. Blevins looks at the typology of TR cluster resolution strategies and the role of language contact. Kaisse draws on evidence from Macedonian to investigate why stress assignment is word-bounded, despite the existence of higher-level prosodic interactions –- that is, she discusses the reasons why lexical stress is specifically lexical. Round’s contribution takes a different approach to learnability, discussing Yidiny (Pama-Nyungan) word-final deletion as a case of exceptional phonological patterns.

Steve Anderson is perhaps best known for his work in phonology and morphology, but his research also profoundly informs our understanding of syntax, an interaction reflected in several of the contributions to the volume. Hale investigates the syntax that underlies the distribution of clitics that fall under “Wackernagel’s Law”. Hale looks at the areas where there are multiple demands on a Wackernagel position, and how those conflicts are repaired. Aissen similarly considers exceptional clitic placement, but in Tsotsil (Mayan). While Hale argues for syntactic constraints on exceptionality based on Greek, Aissen focuses on the role of phonology in clitic placement. Chung’s paper examines the phonology, morphology and syntax of Chamorro causative marking, historically analyzed as a derivational prefix that appears outside inflection; she argues instead that the causative is a prosodically deficient verb. This has implications for morphological structure and dissolves an apparent counterexample to the morphological generalization that inflection is outside derivation. 

The syntactic contributions also include Hendrick’s argument for English of as a phrasal affix rather than a syntactic head in certain contexts and Horvath \& Siloni’s use of evidence from idioms to argue against theories such as Construction Grammar that treat knowledge of language as a network of stored constructions. Both Hendrick’s and de Chene’s papers look at the relationship of morphology to syntax, though in different languages and domains. Finally, Deo’s article traces the morphosyntax of ergative alignment loss in Indo-Aryan.

Some papers investigate properties of languages and language change in the aggregate, rather than being tied to individual languages or language families. That is, they directly concern the nature of linguistic theory. Newmeyer, for example, examines the theoretical status of parameters, while Timberlake tackles the place of performance and usage in theories of transmission. Spencer’s paper concerns the place of morphology within the general theory of grammar. Klein’s paper, though couched in a different framework, also engages with the question of how we know what we know, and how our theories provide insight to the phenomenon of language. Newmeyer, Timberlake, and Klein focus on syntax, Spencer and Stump (discussed above) on morphology, while Goldstein’s domain is articulatory phonology and its application to birdsong. Finally, both Spencer and Aronoff discuss Steve Anderson’s and Peter Matthews’s contributions to morphological theory.

The volume offers excellent cross-linguistic depth. Languages discussed in the contributions include Ancient and Modern Greek (Hale), Macedonian (Kaisse), Sanskrit (Kiparsky, Deo, Hale), Indo-Aryan (Deo), Romansch (Maiden, Yang), Icelandic and Faroese (Thráinsson), Welsh (Hammond), American Sign Language (Lepic \& Padden), Limbu (Stump), Lusoga (Hyman \& Inkelas), Yidiny (Round), Dyirbal (Kiparsky), and Japanese (de Chene), and Mayan (Aissen) as well as English.


%\printbibliography[heading=subbibliography]
\end{refsection}

