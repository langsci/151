\documentclass[output=paper,
modfonts
]{LSP/langsci}
% \bibliography{localbibliography}

% % add all extra packages you need to load to this file 
% \usepackage{todo} %% removed,cna use todonotes instead. % Jason reactivated
% \usepackage{graphicx} % not needed because forest loads tikz, which loads graphicx
\usepackage{tabularx}
\usepackage{amsmath} 
\usepackage{multicol}
\usepackage{lipsum}
\usepackage{longtable}
\usepackage{booktabs}
\usepackage[normalem]{ulem}
%\usepackage{tikz} % not needed because forest loads tikz
\usepackage{phonrule} % for SPE-style phonological rules
\usepackage{pst-all} % loads the main pstricks tools; for arrow diagrams in Hale.tex
%\usepackage{leipzig} % for gloss abbreviations
\usepackage[% for automatic cross-referencing
compress,%
capitalize,% labels are always capitalized in LSP style
noabbrev]% labels are always spelled out in LSP style
{cleveref}

% based on http://tex.stackexchange.com/a/318983/42880 for using gb4e examples with cleveref
\crefname{xnumi}{}{}
\creflabelformat{xnumi}{(#2#1#3)}
\crefrangeformat{xnumi}{(#3#1#4)--(#5#2#6)}
\crefname{xnumii}{}{}
\creflabelformat{xnumii}{(#2#1#3)}
\crefrangeformat{xnumii}{(#3#1#4)--(#5#2#6)}

%\usepackage[notcite,notref]{showkeys} %%removed, not helping CB.
%\usepackage{showidx} %%remove for final compiling - shows index keys at top of page.
 
\usepackage{langsci/styles/langsci-gb4e}  
 \usepackage{pifont}
% % OT tableaux                                                
% \usepackage{pstricks,colortab}  
\usepackage{multirow} % used in OT tableaux
\usepackage{rotating} %needed for angled text%
\usepackage{colortbl} % for cell shading
 
 \usepackage{avm}  
\usepackage[linguistics]{forest} 
\usetikzlibrary{matrix,fit} % for matrix of nodes in Kaisse and Bat-El


\usepackage{hhline}
\newcommand{\cgr}{\cellcolor[gray]{0.8}}
\newcommand{\cn}{\centering}



\newcommand{\reff}[1]{(\ref{#1})}
%\usepackage{newtxtext,newtxmath}


%\usepackage[normalem] {ulem}
\usepackage{qtree}
%\usepackage{natbib}
%\usepackage{tikz}
%\usepackage{gb4e}
\usepackage{phonrule}  
%\bibliographystyle{humannat}



\usepackage{minibox}

%\include{psheader-metr}

\def\bl#1{$_{\textrm{{\footnotesize #1}}}$}

% \interfootnotelinepenalty=10000

\title{`Constructions' and grammar: Evidence from idioms}

\author{%
Julia Horvath\affiliation{Tel Aviv University}\and 
Tal Siloni\affiliation{Tel Aviv University}
}

\abstract{The paper presents results of our investigation of the distribution of idioms across diatheses (voice alternations) in English and Hebrew. We propose an account and discuss its consequences for idiom storage and its implications for alternative architectures of grammar. We provide evidence that idioms split into two distinct subtypes, which we label “phrasal” versus “clausal” idioms. Based on idioms surveys, we observe that phrasal idioms can be specific to the transitive, the unaccusative or the adjectival passive diathesis, but cannot be specific to the verbal passive. Clausal idioms, in contrast, do not discriminate between diatheses: they tend to be specific to a single diathesis. These findings, we argue, cannot be accommodated by a Construction Grammar approach, such as \citet{goldberg2006}, which assumes knowledge of language consists merely of an inventory of stored ‘constructions’, and does not distinguish between a storage module versus a computational system, attributing all significant grammatical generalizations to inheritance networks relating stored entities, and general cognitive and functional constraints. An adequate theory of idioms must have recourse to a distinction between stored items and unstored derivational outputs, and to grammatical distinctions. We outline an account of the findings, distinguishing between diatheses according to where they are formed, and assigning different storage to idioms according to whether their head is lexical or functional. }

\begin{document}
\maketitle

\section{Introduction}
Theories of linguistic knowledge all assume a storage component, where
the associations of form and meaning are stored. There is a controversy as to the nature of this component, call it the lexicon: How much does it list? What does it allow? What else is there beyond the lexicon? In
contrast with Generative Grammar, which assumes a modular,
multi-component model (\citealt{chomsky1965} and subsequent work), Usage-based
Construction Grammar (CxG) (e.g. \citealt{goldberg2006}) and similar work assume
that human knowledge of language is nothing more than a network of
stored constructions.\footnote{This approach is also referred to as Cognitive Construction Grammar
(CCxG); see \citet{boas2013} for an overview of this versus other varieties
of construction grammar models.} There is no faculty of language and no language specific mechanisms, no derivations, just a lexicon of constructions, labelled `Construct-i-con', 
which includes morphemes, words, idioms, partially lexically filled as well 
as fully abstract phrasal patterns. Generalizations across languages are explained 
by general cognitive constraints together with the functions of the
particular constructions. Language-specific generalizations across
constructions arise via inheritance networks.

The rationale behind the assumption of a construct-i-con is as follows:
(i) Idioms, for the most part, involve an internal makeup consisting of
phrasal units. Since their meaning is unpredictable and associated with
the whole construction, they are most plausibly stored as constructions.
(ii) The distinction between idioms and `other constructions' (involving
argument realization) is hard to detect in many instances, because often
the specific meaning of a sentence not involving an idiom (in the
traditional sense) seems better specified as a property of the
construction, not as properties of the verb and of its complements
(e.g., the `transfer of possession' meaning of `He sliced Chris a piece
of cake' vs. the `caused motion' interpretation of `He sliced carrots
into the salad', although both sentences feature \emph{sliced}). Hence,
constructions in general should be stored as such.

Indeed, idioms exhibit an inherent duality. On the one hand, they are
phrasal units with internal syntactic structure, and on the other, they
are associated with an unpredictable, conventionalized meaning.
Therefore, the question as to how they are stored is particularly
intriguing. Given that they are grammatical constructs and interact with
grammar (can be embedded, can allow passivization etc.), they must be
stored intra-grammatically, in the lexicon. This paper investigates the
storage of idioms, aiming to shed light on the nature of the lexicon.
Further, idioms are the archetypal construction to be stored à la CxG;
therefore, they constitute a test case (for alternative conceptions of
grammar and the lexicon) most favorable to CxG. So if our investigation
of idioms finds that the storage they require is inconsistent with CxG's
central tenet that grammar is comprised of nothing but networks of
stored `constructions', this must be all the more so for more
productive, prima facie compositional kinds of `constructions'.

Investigating the distribution of idioms across diatheses (transitive,
unaccusative adjectival passive and verbal passive), we observe
contrasts between the cross-diatheses distribution of distinct types of
idioms. One type of idiom (which we will label `phrasal') distributes
differently in the verbal passive diathesis versus the transitive,
unaccusative and adjectival passive diatheses: it cannot be specific to
the verbal passive, but can be specific to the latter diatheses. Another
type of idiom (`clausal'), in contrast, does not discriminate between
diatheses in this way: Idioms of this type tend to be specific to a
single diathesis. We then show that a construct-i-con type of theory
cannot account for these findings. To account for these systematic
distinctions, which idioms (the archetypal `construction' à la CxG)
exhibit, the theory requires more than cognitive principles, functional
needs, and inheritance of properties between stored entities
(`constructions').

The structure of the paper is as follows. Sections 2 and 3 draw a
distinction between lexically headed idioms, which we label `phrasal'
idioms, and idioms headed by a sentential functional head, which we
label `clausal' idioms, and discuss each type (respectively), paying
particular attention to their distinct distribution across diatheses.
Section 4 offers additional evidence for the partition into phrasal and
clausal idioms, and lays out the implications regarding a CxG-type
model. Section 5 sketches an account for the findings in the framework
of a derivational and modular architecture of grammar.

\section{Phrasal Idioms }

It has sporadically been observed in the literature that the verbal
(eventive) passive (e.g., \emph{sold} in `The first costumer was sold
the car') and the adjectival (stative) passive (e.g., \emph{shaven})
differ regarding the distribution of idioms. While there do not seem to
be idioms specific to the verbal (eventive) passive (i.e., idioms in the
verbal passive that have no transitive (active) alternant), there are
idioms specific to the adjectival (stative) passive (see \citealt{ruwet1991} for
English and French, and  \citealt{dubinsky1996} for Chichewa). A first
quantitative survey of idiom dictionaries examining these observations
is reported in Horvath and Siloni's \citeyear{horvath2009} study of Hebrew idioms: Out
of 60 predicates sampled for 4 diatheses -- verbal passive, adjectival
passive, transitive, and unaccusative-- only the verbal passive
exhibited no unique idioms. An idiom is considered `unique' to a given
diathesis α, if α does not share the idiom with its (existing)
root-counterpart β, which α would most directly be related to by
derivation. Specifically, verbal passives, adjectival passives and
unaccusatives are unique if there is no corresponding transitive idiom.
Transitives are unique if there is no corresponding unaccusative idiom.
Except for the verbal passive, all other three diatheses can head unique
idioms.\footnote{The survey proceeded as follows. 60 predicates of each
  diathesis were sampled from a verb dictionary. The number of
  predicates out of the sample of 60 giving rise to unique phrasal
  idioms were counted. This was done by searches of idiom dictionaries,
  followed by Google searches to check occurrences of relevant root-mate
  idioms, and consultation of native speakers regarding the results. The
  number of unique idioms found: 0 verbal passive ones; 21
  unaccusatives, 23 transitives; and 13 adjectival passives.} This will
be illustrated shortly with English examples in (4)-(6).

Two observations are in order. First, the idioms mentioned in the above
studies are all phrasal idioms (VP and AP) involving no sentential
functional categories such as auxiliaries, negation, etc. Second, verbal
passives in Hebrew are known to be rarer in spoken language in
comparison to say English (\citealt{berman2008}), which may affect the inventory
of verbal passive idioms in the language.

In light of the above, we ran a parallel survey of English idiom
dictionaries. We believe such surveys are necessary for the study of
idiom distribution, as speakers may sometimes have a hard time
distinguishing whether a certain idiom variant exists and is commonly
used or only could exist, i.e., is a priori possible, but is not
documented. This is so because the spontaneous formation and learning of
novel idiomatic expressions is part of speakers' linguistic competence.
Also, knowledge of idioms varies considerably among speakers (similar to
vocabulary knowledge).

We have systematically distinguished between phrasal and clausal idioms,
as defined in (1) and illustrated in (2).

\ea Phrasal vs. clausal idioms
	\ea Phrasal Idioms are headed by a lexical head (e.g., (2a)).
	\ex Clausal Idioms are headed by a sentential functional head (a fixed
	tense or mood, a modal, obligatory (or impossible) sentential negation
	or CP-material); they are not necessarily full clauses (e.g., (2b))
	\z
\z
Fixed sentential material is specified in parentheses. Non-idiomatic material within idioms is marked by italics.
\ea
	\ea land on \emph{one}'s feet\\
	`make a quick recovery'
	\ex butter wouldn't melt in \emph{someone}'s mouth\hfill (modal, negation)\\
	 `someone is acting innocent'
	\z
\z

Given that `idiom' is a pre-theoretic term referring to various types of
fixed expressions, we defined a core set. The set consisted of
conventionalized multilexemic expressions whose meaning is figurative
(metaphoric) and unpredictable by semantic composition. A property often
mistakenly conflated with the unpredictability of idioms' meaning is the
level of opacity or transparency of their meaning. Idioms indeed differ
from one another in the level of their transparency (opacity). For
example, the phrasal and clausal idioms in (3a) and (3b) respectively
may be felt more opaque than those in (2a-b). However, the degree of
opacity can be determined only once we know the meaning of the idioms;
neither the former nor the latter meanings can be predicted based on the
meaning of their building blocks. Hence, the meanings of the idioms in
(2) just like those of the idioms in (3) are unpredictable (even if a
posteriori, more transparent). Such idioms are therefore part of the
core set we have defined and included in our study.

\ea 
	\ea cool \emph{one}'s heels\\
	`wait'
	\ex The squeaky wheel gets the grease.\hfill(tense)\\
	`The most noticeable (loudest) ones are the most likely to get attention.'
	\z
\z
We first concentrated only on phrasal idioms. This enabled us to examine
a coherent set of idiomatic expressions.

The English survey we ran produced similar results to those of the
Hebrew one. The transitive, unaccusative, and adjectival passive
exhibited unique idioms, just like their Hebrew counterparts.\footnote{The
English survey was conducted following the guidelines in \citet{horvath2009} (see note 2). The number of predicates out of the sample
of 60 giving rise to unique phrasal idioms in English: 15
unaccusatives; 18 transitives; 10 adjectival passives.} Examples of
unique unaccusative (4), adjectival passive (5), and transitive (6)
idioms are given below. Notice that the nonexistent idiomatic version is
no less plausible than the existing idiom. (\# means the relevant
sequence of words has no idiomatic meaning.)

\ea
	\ea burst at the seams\hfill (unaccusative)\\
	`filled (almost) beyond capacity'
	\ex \#burst \emph{something} at the seams\hfill (transitive)
	\z
\ex 
	\ea caught in the middle\hfill (adjectival passive)\\
	`trapped between two opposing sides'
	\ex \#catch \emph{someone} in the middle\hfill (transitive)
	\z
\ex
	\ea turn \emph{something} on its ear\hfill (transitive)\\
	`change something in a surprising and exciting way'
	\ex \#turn on its ear\hfill (unaccusative)
	\z
\z

However, unlike in Hebrew, the verbal passive in English turned out,
prima facie, to present unique verbal passive idioms for 2 out of the 60
predicates, namely for \emph{caught} and \emph{bitten}. These idioms are
given in (7).

\ea 
	\ea caught in the crossfire\\
	`hurt by opposing groups in a disagreement'
	\ex bitten by the \emph{x} bug (where \emph{x} forms a compound with
\emph{bug})\\
	`having the need/desire/obsession for \emph{x}'
	\z
\z

These phrasal idioms can be suspected at first to constitute unique
verbal passive idioms, due to their listing in idiom dictionaries in the
passive form, and not in the active, in contrast to the norm of listing
verb phrase idioms in dictionaries in the active form.

However, on closer examination, both of these turned out not to
constitute true counterexamples to the generalization that there are no
unique idioms to the verbal passive. Starting with (7a), the idiom
\emph{caught in the crossfire}, which indeed appears in the verbal
passive, in fact is attested -- based on Google searches accompanied by
native speakers' judgments -- also in the transitive (active) form, as
in (8), for instance; hence it is not a unique verbal passive idiom.

\ea
	\ea This \textbf{caught him in the crossfire} between radical
	proponents of independence and French opponents of anti-colonialism.\\
	\hfill (Scheck, 2014:282)\footnote{Scheck, Raffael. 2014. \emph{French Colonial
  	Soldiers in German Captivity during World War II}. Cambridge:
  	Cambridge University Press. Available at \url{https://goo.gl/QAGf9E}. All online examples 			accessed 9 December 2016.}
	\ex \ldots{}the Israeli-Palestinian conflict, which has often \textbf{caught them in the }
	\textbf{crossfire}.\hfill \url{https://goo.gl/f2FbbG}
	\z
\z

The idiom in (7b) is instantiated by versions such as \emph{bitten by
the travel bug}, \emph{bitten by the acting bug}, etc. These, just like
(7a), can be true verbal passive forms; however, again, Google searches
turn up a significant number of active transitive examples of the same
idiom, e.g., ((9)--(10)).

\ea 	Before the \textbf{acting bug bit me} I had dreamed of being another
	Glenn Cunningham.\hfill \href{https://books.google.co.il/books?isbn=1429969016}{(Halbrook} 			2001, 66)\footnote{Halbrook, Hal. 2011. Harold -- The Boy Who Became Mark Twain. New
 	York: Farrar, Straus and Giroux. Available at \url{https://goo.gl/ivWkAQ}.}
	
\ex It was during my time in the Army in the 1960s and 1970s that \textbf{the travel bug bit me}.\hfill (\href{https://books.google.co.il/books?isbn=1412078369}{MacKrell} 2006,
Introduction)\footnote{MacKrell, Thomas. 2006. One Orbit -- Around the
  World in 63 Days. Victoria, Oxford: Trafford. Available at
  \url{https://goo.gl/bRlHKA}.}
\z

The listing of (7a--b) in the passive participial form may well be due to
the fact that in addition to occurring as a verbal (eventive) passive,
they are also attested in the adjectival (stative) passive; the latter
point is demonstrated by the idioms' occurrence as complements of verbs
selecting APs but not VPs, such as \emph{seem} and \emph{remain} (Wasow
1977), as illustrated by ((11)--(12)).

\ea
	\ea Everyone else seems \textbf{caught in the crossfire} between these two, 
	I honestly feel bad about everyone involved.\hfill \url{https://goo.gl/trJp5o}
	\ex The Starbucks coffee chain remains \textbf{caught in the crossfire}
	of a dispute over "open carry" laws\ldots{}\hfill \url{https://goo.gl/PMCiMF}
	\z
\ex
	\ea \ldots{}and Kevin remains \textbf{bitten by the travel} -- \textbf{and}
	\textbf{mapping} -- \textbf{bug}.\\
	\hfill \url{https://goo.gl/xC8kWp}
	\ex It made an impression on Bowley, and he too seems \textbf{bitten by
	the renovation bug}\ldots{}\hfill \url{https://goo.gl/H04LWn}
	\z
\z

More generally, in the case of English in particular, it is important to
keep in mind that there is the interfering factor of the common identity
of form between verbal passives and adjectival passives, and only
diagnostics can establish whether or not the particular idiom is indeed
a verbal passive, and not (only) an adjectival passive one (see \citealt{wasow1977} for diagnostics).

We can thus conclude that the idioms in (7) are not exceptions to the
generalization that there is no unique idiom in the verbal
passive.\footnote{Additional idioms (headed by predicates not included
  in our sample) that may be suspected to be unique verbal passive
  idioms are discussed by \citet{horvath2016}, and are shown to
  also conform to the generalization.} The next question is what can
explain this.

A priori, two alternative types of explanations for the above
generalization come to mind: a derivation-based account in the spirit of
derivational approaches of Generative Grammar or alternatively, an
inheritance-based account, along the guidelines proposed by CxG.
Abstracting away from details, a derivation-based account would have the
verbal passive formed beyond the domain of special meanings, which would
prevent verbal passives from having their own special/idiomatic meaning.
An inheritance-based account would have the verbal passive inherit the
inability to give rise to idioms that it does not share with its
transitive alternant from the inability of verbal passives to lack a
transitive alternant.

A first indication that an inheritance-based account is not on the right
track comes from inspection of the transitive-unaccusative alternation.
This alternation manifests regularity at the verb level, but pervasive
uniqueness at the idiom level. Intransitive unaccusative verbs have a
transitive alternant (with a Cause external role) and vice versa (13),
except for isolated instances (\citealt{haertl2003}, \citealt{reinhart2002}, among
others).\footnote{For example, the transitive alternant may be missing
idiosyncratically and sporadically in a given language for a few
instances, but these instances have a transitive alternant with a
Cause role at some other stage in the evolution of the same language
(e.g., the recently developing transitive \emph{faint} in Hebrew (i))
or in other languages at present (e.g., existence of the transitive
\emph{fall} in Hebrew (ii), but not in English).
\ea
\gll Barur še-hu xavat bo dey xazak im hu ilef oto.\\
	evident that-he hit in.him rather strong if he fainted.\textsc{transitive} him\\
	\glt `It is evident that he hit him rather strongly if he made him faint.'\\
	\hfill \url{https://goo.gl/GK7MWR}
\ex \gll Dan hipil šney sfarim.\\
	Dan fell.\textsc{transitive} two books\\
\z
}

\ea
	\ea Dan / The storm / The stone broke the window.
	\ex The window broke.
	\z
\z

In other words, there are sporadic, isolated gaps in the
transitive-unaccusative verbal alternation but the paradigm is rather
regular. Nonetheless, there is pervasive uniqueness, namely,
unpredictable gaps are common, at the idiom level. If so, then, the
distribution of phrasal idioms across diatheses is not determined by or
inherited from the degree of productivity of their respective
predicates.

In sum, an inheritance-based account does not seem to be able to account
for the observation that the verbal passive, unlike the transitive,
unaccusative and adjectival passive cannot head unique phrasal idioms.
Additional evidence against an inheritance-based account comes from
clausal idioms, which are discussed in the next section.

\section{Clausal Idioms }

As defined in (1b), clausal idioms are not necessarily full clauses;
they are headed by a sentential functional element: a fixed tense or
mood, a modal, obligatory (or impossible) sentential negation or
CP-material. Examples of clausal idioms are given below: (2b) and (3b)
repeated as (14a--b) and additional examples in (14c--e).

\ea
	\ea butter wouldn't melt in \emph{someone}'s mouth\hfill (modal, negation)\\
	`someone is acting innocent'
	\ex The squeaky wheel gets the grease.\hfill (tense)\\
	 `The most noticeable (loudest) ones are the most likely to get attention.'
	\ex \ldots{}can't see the forest for the trees\hfill (modal, negation)\\
	`doesn't perceive the whole situation clearly due to focusing on the
	details'
	\ex not have a leg to stand on\hfill (negation)\\
	`have no support (for your position)'
	\ex Where's the beef?\hfill (wh-phrase, (contracted) copula)\\
	`Where is the important content?'
	\z
\z

One may wonder at this point whether some of what we consider clausal
idioms here would not be classified more appropriately as proverbs
rather than (clausal) idioms. Indeed, the common though informal
distinction between proverbs vs. idioms is worth some clarification.

Proverbs have no precise linguistic definition. Just like our clausal
idioms, they too are headed by some functional, rather than lexical,
head. The definition we have given to delineate the core set of idioms,
given the goals of our study, is aimed at obtaining evidence about
lexical storage; therefore, our idioms all have properties that force
them to be stored, and specifically stored in the grammar (not in
extralinguistic storage in general memory). Consequently, the questions
we need to ask regarding any clausal idiom suspected to be a proverb
are: (a) Is the meaning of the expression unpredictable based on
composition of its parts and does it involve figuration? If so it must
be stored; (b) Is there evidence that it is stored in the storage
component of the grammar, and not extragrammatically? The clausal idioms
used in our study satisfy both of these criteria, thus they are properly
falling within the set of relevant idiom data to be considered. As for
whether some of them may be felt to be proverb-like (due to some
additional, stylistic, aspectual or other properties) this is not a
factor that effects the validity of the conclusions drawn based on them,
as long as they meet the criteria for intra-grammatical (lexical)
storage, as explained above.\footnote{Observe that there is a difference
  between the various (fixed) clausal expressions in terms of the
  presence/absence of figuration they manifest. Expressions such as (i)
  are fixed in form and are felt to be proverbs (as pointed out by an
  anonymous referee), but involve no figuration and hence are not
  classified as idioms according to our criteria; in contrast the
  expressions in (ii) do manifest figuration and constitute idioms under
  our definition. At the same time, both (i) and (ii) may be felt to be
  proverbs. This intuitive notion does not seem to be associated with
  figuration. A property that does appear to play a role in the
  perception of a fixed clausal expression as a proverb is that it
  applies to a generic, rather than episodic, situation. (This property
  is orthogonal to qualifying as an idiom.)
  \ea
  	\ea Two wrongs don't make a right.
	\ex When the going gets tough, the tough get going.
	\z
\ex
	\ea A stitch in time saves nine.
	\ex A chain is only as strong as its weakest link.
	\z
\z
  }

Unlike phrasal idioms, clausal idioms do occur as unique to the verbal
passive. Examples are given in (15--16) for English and (17--18) for
Hebrew. As mentioned in section 1, it is often difficult to decide
whether a certain idiom variant exists or only could exist, but not
documented, and constitutes an ad hoc `playful' intended distortion,
alluding to an existing idiom. Our data therefore are based on idiom
dictionaries and the diathesis/es that they list the idioms in. In
addition, however, we have googled idioms to check their existence in
root-mate variants. We did not consider isolated occurrences, including
playful distortions, which mostly appear in specific styles, such as
media language, as evidence of existence.

\ea
	\ea might/may as well be hung/hanged for a sheep as (for) a lamb\hfill (modal)\\
	`may as well commit a larger transgression, as the same punishment will
	result'
	\ex \#(They) might/may as well hang \emph{someone} for a sheep as (for) a
	lamb.\footnote{A reviewer called our attention to the existence of
  	occurrences of the idiom in the unaccusative. One online dictionary
  	(out of eight) listed the clausal idiom in the unaccusative form, not
  	in the verbal passive: \emph{One may/might as well hang for a sheep as
  	a lamb}.}
	\z
\ex
	\ea Gardens are not made by sitting in the shade.\hfill (negation, tense)\\
	Nothing is achieved without effort.
	\ex \#One doesn't make gardens by sitting in the shade.
	\z
\ex
	\ea \gll Nigzezu maxlafot-av.\hfill (tense)\\
	sheared.V\textsc{passive} hair-his\\
	\glt `lost one's power/influence.'
	\ex \gll \#gazezu et maxlafot-av.\\
	sheared.\textsc{transitive.impersonal} \textsc{accusative} hair-his\\
	\z
\ex
	\ea \gll Hutla ha-kubiya\hfill (tense).\\
	cast.\textsc{Vpassive} the-die\\
	\glt `The process is past the point of return.'
	\ex \gll \#Hetilu et ha-kubiya.\\
	cast.\textsc{transitive.impersonal} \textsc{accusative} the-die\\
	\z
\z

Thus, while there are no unique phrasal idioms in the verbal passive,
there appear to exist clausal idioms unique to the verbal passive. This
provides additional evidence against an inheritance-based account of the
lack of phrasal idioms unique to the verbal passive, that is, against
the proposal that the verbal passive inherits the inability to give rise
to phrasal idioms that it does not share with its transitive alternant
from its lack of a transitive alternant. Initial evidence that such an
inheritance-type account is not on the right track was presented in
section 2 based on inspection of the transitive-unaccusative
alternation. This alternation, as we noted, manifests regularity at the
verb level, but pervasive uniqueness at the idiom level. Its behavior
thus is incompatible with the idea that there is inheritance of
properties from the verb level to the idiom level. Our initial findings
regarding the existence of clausal idioms unique to the verbal passive
are also incompatible with such an inheritance-based account. If it was
indeed merely inheritance by the verbal passive idiom of the
non-uniqueness property of the verbal passive diathesis (i.e., necessary
existence of a transitive alternant), then there does not seem to be any
reason why phrasal idioms would~inherit `non-uniqueness', while clausal
idioms in the verbal passive would not do so. So not only is there no
inheritance of distribution from the verb level to the idiom level, as
shown by the transitive-unaccusative alternation, but in addition, an
inheritance-based account could not explain the distributional
distinction between phrasal versus clausal idioms regarding the verbal
passive. Note also that the discrepancy between phrasal and clausal
idioms with regard to uniqueness in the verbal passive seems to hold
across languages, yet it certainly cannot be attributed to general
cognitive constraints or functional needs of the constructions. If all
the theory has at its disposal is inheritance networks, cognitive
constraints, and functional needs to explain generalizations exhibited
by members in the construct-i-con, the above findings cannot be
accounted for.

One could perhaps suggest that unlike phrasal idioms, clausal idioms are
stored extra-grammatically, outside the construct-i-con (similar to
memorized language material such as lines of poems etc.), and therefore
they do not inherit the non-uniqueness property from the verbal passive.
Such a line of explanation however does not seem to be tenable. Unlike
memorized language material, clausal idioms interact with the grammar
and it is thus hardly plausible that their storage is extra-grammatical.
First, they can appear as embedded clauses within various matrix
contexts (19a--b). Further, they need not be full clauses and can include
a non-idiomatic argument (20a--b). Moreover, the non-idiomatic element
can occur within a sub-constituent (21a--b). Finally, they can include
variable pronouns obligatorily bound by a non-idiomatic noun phrase
(22a--b), (the variable pronoun indicated by \emph{one} has to be bound
by the subject in (22a) and (22b)).

\ea 
	\ea One should take into account the fact that {[}the squeaky wheel
	gets the grease{]}.\hfill (tense)\\
	`One should take into account the fact that the most noticeable
	(loudest) ones are the most likely to get attention{]}.'
	\ex They had to realize that {[}the leopard does not change his spots{]}. (negation)\\
	`They had to realize that {[}one remains as one is even if one pretends
	otherwise/tries hard{]}.
	\z
\ex 
	\ea can't see the forest for the trees\hfill (modal, negation)\\
	 `doesn't perceive the whole situation clearly due to focusing on the details'
	\ex wouldn't touch \emph{someone/something} with a ten-foot pole\hfill (modal,
	negation)\\
	`wouldn't have anything to do with someone/something'
	\z
\ex
	\ea wouldn't put it {[}past \emph{someone}{]}\hfill (modal, negation)\\
	`consider it possible that someone might do something wrong or
	unpleasant'
	\ex butter wouldn't melt in {[}\emph{someone}'s mouth{]}\hfill (modal,negation)\\
	`someone is acting innocent'
	\z
\ex
	\ea can't fight \emph{one's} way out of a paper bag\hfill (modal, negation)\\
	`be an extremely inept'
	\ex would give \emph{one}'s right arm (for\ldots{})\hfill (modal)\\
	`would like something very much'
	\z
\z

Below we turn to an additional distinction between phrasal and clausal
idioms in order to reinforce our conclusion thus far.

\section{Diathesis Sharing vs. Rigidity}

In both English and Hebrew, phrasal idioms can be common to, i.e.,
shared between, root-alternants. The verbal passive always shares its
idiomatic meaning with the corresponding transitive (e.g., (23)), as
discussed in section 2. Moreover, the other diatheses (the transitive,
unaccusative, and adjectival passive), which appear in unique idioms,
can also share their idiomatic meaning with their root-alternants
(24--25).\footnote{We have conducted two surveys of shared idioms. The
  results are as follows. The number of English transitive predicates
  (out of the sample of 60) sharing phrasal idioms with the verbal
  passive: 35, with unaccusative: 17, and with adjectival passive: 21.
  The number of Hebrew transitive predicates (out of the sample of 60)
  sharing phrasal idioms with the verbal passive: 10, with unaccusative
  16, and with adjectival passive: 5. Note that while idioms in the
  verbal passive always have a transitive alternant; it is not the case
  that any transitive idioms has a corresponding verbal passive idiom,
  as discussed in section 5.}

\ea
	\ea spill the beans\hfill (transitive)\\
	`divulge the secret'
	\ex The beans were spilled.\hfill (verbal passive)
	\z
\ex
	\ea burst \emph{someone}'s bubble\hfill (transitive)\\
	 `destroy someone's illusion'
	 \ex someone's bubble burst\hfill (unaccusative)
	 \z
\ex
	\ea carve s\emph{omething} in stone\hfill (transitive)\\
	 `fix some idea/agreement permanently'
	\ex carved in stone\hfill  (adjectival passive\emph{)}
	\z
\z

In contrast, the clausal idioms in our preliminary investigation, unlike
the phrasal ones, fail to exhibit sharing across diatheses. Clausal
idioms seem to be unique, as illustrated by examples (26)-(29) below.
\\

\noindent Transitive vs. verbal passive
\ea
	\ea can't see the forest for the trees\hfill (modal, negation)\\
	`doesn't perceive the whole situation clearly due to focusing on the
	details'
	\ex \#The forest can't be seen for the trees.\footnote{This idiom does
  	have occurrences in the verbal passive (found by Google searches).
  	However, the idiom shows signs of being in the process of developing a
  	phrasal version. This process is indicated by the existence of a large
  	number of occurrences of this idiom in a phrasal version headed by a
  	variety of lexical verbs, each yielding the same meaning as the
  	original clausal idiom: \emph{ignore the forest for the trees},
  	\emph{miss the forest for the trees}, \emph{neglect the forest for the
  	trees}. The evolving use of this idiom in a phrasal form may be the
  	reason for the occurrences of a verbal passive version. See also fn.
  	17.}
	\z
\z

\noindent Transitive vs. unaccusative (in the adjunct clause)
\ea
	\ea You can't make an omelet without breaking a few eggs.\hfill (modal, negation)\\
	`It is difficult to achieve something important without causing any unpleasant effects.'
	\ex \#You can't make an omelet without a few eggs breaking.
	\z
\z

\noindent Adjectival passive vs. transitive
\ea
	\ea The road to hell is paved with good intentions.\hfill  (tense)\\
	`People often mean well but do bad things.'
	\ex \#Good intentions pave the road to hell.
	\z
\z

\noindent Unaccusative vs. transitive
\ea
	\ea do(es) not grow on trees\hfill (auxiliary, negation)\\
	`is not abundant, not to be wasted'
	\ex \#do(es) not grow \emph{something} on trees
	\z
\z

One might think at this point that the emerging lack of cross-diatheses
flexibility of clausal idioms could be due to the fact that in English
the relevant diathesis alternations involve syntactic movements
reordering subparts of the idiom. These movements might be suspected to
be incompatible with the idiomatic reading for reasons of information
structure, independent of the diathesis change itself. However,
examining clausal idioms with regard to parallel diathesis alternations
in Hebrew, a language in which diathesis alternations do not have to
involve such potentially interfering factors, seems to point in the same
direction. For instance, the Hebrew clausal idiom in (30a) does not
require reordering (nor addition of words), when undergoing the
diathesis alternation in (30b); still the latter is impossible.
\ea
	\ea \gll kše-xotvim ecim, nitazim švavim.\hfill (tense)\\
	when-chop.\textsc{transitive.impersonal} trees, sprinkle.\textsc{unaccusative} chips\\
	\glt `When you act, there are risks.' ``Where trees are felled chips will fly.''
	\ex \gll \#kše-xotvim ecim, metizim švavim.\\
	when-chop.\textsc{transitive.impersonal} trees, sprinkle.\textsc{transitive.impersonal} chips\\
	\z
\z

If knowledge of language were nothing more than an inventory of
constructions whose properties derive from cognitive constraints,
functional needs and inheritance hierarchies, there would be no way to
explain why the clausal idioms we have examined (full and partial
sentential structures) are unique to their diathesis, while phrasal
idioms are commonly shared across diatheses.

In sum, an inheritance-based account cannot explain why idioms headed by
members of the unaccusative alternation show pervasive uniqueness at the
idiom level, although the verbal alternation is rather systematic.
Moreover, under such an account, it is completely unclear why clausal
idioms can be unique to the verbal passive as well as to other
diatheses, and even seem to be unique generally, while phrasal idioms
cannot be unique to the verbal passive, but can be unique to other
diatheses.

Below we consider what an alternative approach, one that can provide a
principled account for the above generalizations, should look like, and
sketch our proposal of idiom storage, which derives these findings.

\section{Alternative, derivational accounts}

CxG imposes no principled limitation on lexically stored syntactic
objects and assumes no syntactic (online) derivation, only stored
objects (`constructions'), whose interrelations are expressed via
inheritance networks. The inability of CxG to capture the distributional
asymmetries of diatheses in idioms established in the preceding sections
is a direct consequence of these fundamental characteristics of the
model. We believe that in contrast to the CxG model, modular
derivation-based theories, namely theories incorporating a fundamental
distinction between lexically stored entities versus syntactic objects
derived by the computational system of grammar have the potential to
provide an adequate account for the above findings. Before sketching the
particular account that we propose, observe what assumptions are
available in derivation-based modular architectures -- and absent in
non-derivational, construction-based models -- that seem prerequisites
for accounts aiming to capture the diathesis asymmetries discussed
above.

What seems crucial for conceiving a syntactic account is the incremental
building of structure in the syntactic derivation, yielding units in the
course of the derivation (`phases') that impose locality limitations on
the accessibility of special/idiomatic meanings. As for lexical accounts
(involving the storage component of grammar) what is crucial would be
principled constraints on what can be stored in the lexicon and in what
manner, as will be explained in what follows.

In the remainder of this section, we sketch an account along the latter
lines within our model of Type Sensitive Storage (TSS) \citep{horvath2016}. The model derives the diathesis asymmetries discussed in
the previous sections from a different storage technique motivated for
phrasal versus clausal idioms. Under this proposal, the distinct storage
technique of phrasal versus clausal idioms is a direct consequence of
their having a lexical versus a functional head, respectively. Each
storage strategy, in turn, results in a different pattern of
distribution across diatheses. As summarized in (31), the Type-Sensitive
Storage model suggests that phrasal idioms are stored as subentries of
existing lexical entries, whereas clausal idioms constitute independent
lexical entries on their own, that is, are not stored as subentries.

\ea The Type-Sensitive Storage (TSS) Model
	\ea Idioms are stored as part of our linguistic knowledge (not as
	general, non-linguistic information).
	\ex Phrasal idioms -- Subentry Storage: Phrasal idioms are stored as	
	subentries of the lexical entry/ies representing their subconstituent(s)
	in the lexicon.
	\ex Clausal Idioms -- Independent Storage: Clausal idioms are stored as
	independent entries on their own.
	\z
\z

Let us see how this would account for the findings. If phrasal idioms
are stored as subentries of their subconstituents, it means they must be
stored as subentries of the lexical entry of their head.\footnote{The
  question as to whether they are also stored as subentries of the
  lexical entries of their other constituents is important but
  irrelevant for our purposes here. We therefore abstract away from it
  here.} Subentry storage is contingent upon the listing, i.e., the
existence, of the (mother) entry in the lexicon. The verbal passive is
formed beyond the storage component, the lexicon (\citealt{baker1989}, \citealt{collins2005}, \citealt{horvath2008}, \citealt{meltzer2012}, among
others). It follows that the verbal passive is not stored; it is not a
lexical entry. Hence, the verbal passive cannot have subentries. Thus,
under (31b), phrasal idioms cannot be unique to the verbal passive
because such idioms cannot be stored. Phrasal idioms in the verbal
passive can only be formed by passivization of their transitive
counterparts. Hence, they always share their idiomatic meanings with the
corresponding transitive. The transitive, unaccusative and adjectival
passive, in contrast, are formed in the lexicon (\citealt{horvath2008,horvath2011}, \citealt{reinhart2002}), and stored there; therefore, they can have
subentries.

It should be observed that unlike the existence of a transitive (active)
version for every verbal passive phrasal idiom, we, correctly, do not
predict the automatic existence of a verbal passive version for every
transitive idiom. Since verbal passives are derived in the syntax, the
question determining whether or not a transitive idiom will exist in the
verbal passive depends on whether the idiom is able to undergo the
syntactic operation of passivization resulting in a well-formed output.
This in turn involves interpretive factors, such as whether the idiom
chunk to become the derived subject of the passivized idiom has the
appropriate semantic properties, e.g., referentiality, to be compatible
with the information structure consequences of being in subject
position. Hence, the contrast between \emph{The beans were spilled}
vs. \emph{*The bucket was kicked} (see for instance \citealt{nunberg1994}, \citealt{ruwet1991}, \citealt{punske2014} on what factors may
determine whether or not a verbal passive version of a transitive idiom
is possible).

Clausal idioms in contrast are stored as independent entries. Let us
first motivate this claim. The head of clausal idioms is a functional,
not a lexical, element. Functional elements unlike lexical ones are
closed class items, have no descriptive content (\citealt{abney1987}), and bear
no thematic relation to their complement. Functional elements have often
been argued to be stored in a separate lexicon, e.g., Emonds' \citeyear{emonds2000}
`Syntacticon', or as `f-morphemes' (Distributed Morphology). One storage
option for clausal idioms would be storage as subentries of their
functional head. This would, for instance, mean storage of \emph{Where's
the beef?} as a subentry of its functional head, the interrogative
complementizer. This would be storage of entities that have descriptive
content in the `functional lexicon', where entries do not have
descriptive content. This seems to us incoherent. We therefore do not
pursue this option.

Two additional options come to mind: (i) Clausal idioms are stored as
subentries of the lexical head of the `extended projection' (in
Grimshaw's \citeyear{grimshaw1991} terms) constituting the clausal idiom, namely,
storage under the verb on a par with VP idioms. (ii) Clausal idioms are
independent entries on their own; they are not stored as subentries of
another lexical entry (31c). Subentry storage under the lexical head
(option (i)) predicts the absence of unique clausal idioms in the verbal
passive (just like in the case of phrasal idioms); this is contradicted
by our findings (e.g. (15--18)).

Independent storage (31c) predicts occurrence of unique clausal idioms
in the verbal passive, in concert with our findings. Under independent
storage, clausal idioms get lexicalized in one piece (following
consistent use of the expression in the relevant contexts). Clausal
idioms thus do not require that their subconstituents be represented as
entries in the lexicon. They get stored as a whole and can therefore
include any diathesis (or any other syntactic output). Hence, there
should be clausal idioms unique to the verbal passive. There are thus
reasons to adopt the independent storage strategy for clausal
idioms.\footnote{Interesting questions arise with regard to the storage
  of idioms with no recognizable internal structure (e.g., \emph{trip
  the light fantastic}), as well as idioms (arguably) headed by a
  non-sentential functional element (a light verb, a functional
  preposition, or a conjunction, as in \emph{take a shower}, \emph{in a
  rut,} and \emph{cut and dried}, respectively). These important
  questions are beyond the scope of this paper and are not directly
  relevant for the issue of cross-diatheses distribution.}

If phrasal idioms were stored as independent constructions, on a par
with clausal idioms, there would be no reason why they could not be
unique to the verbal passive. Precisely because phrasal idioms are not
stored constructions (contra CxG's assumptions), they cannot be unique
to the verbal passive.

The difference in storage that the two types of idioms employ can also
explain the second distinction revealed between them. While phrasal
idioms commonly share idiomatic meanings across root-counterparts,
clausal idioms tend not to be shared across diatheses (section 4). As
already discussed above, a verbal passive phrasal idiom must share its
idiomatic meaning with the corresponding transitive because it is formed
by syntactic passivization of the latter (it is not stored). Further,
under the TSS model, sharing of phrasal idioms between the transitive
and its (lexically derived) unaccusative or adjectival passive
alternants is the result of the links between root-related entries in
the lexicon, which can induce spread of special meanings and idiomatic
expressions between the entries. Sharing is not automatic though, as it
requires additional listing under the entry of the relevant alternant;
hence there are also unique phrasal idioms in these diatheses, as
discussed in section 2.

In contrast, under (31c), clausal idioms are stored as independent
entries, not as subentries of other entries that may be linked to
root-mates. The model therefore predicts that nothing would induce
sharing of idiomatic meaning between the transitive and its unaccusative
or adjectival passive alternants; such sharing thus should be unattested
or rare, as our preliminary results show.\footnote{A priori, nothing
  rules out the independent development and storage of a clausal idiom
  in a root-related diathesis. However, we predict this to be very rare
  (if at all attested) as nothing induces this.}

Verbal passives, unlike the transitive, unaccusative, and adjectival
passive, are derived in the syntax. So there is no a priori reason not
to expect the application of this operation to (some) transitive clausal
idioms. If that occurred, at least some clausal idioms would be
available in the verbal passive.\footnote{Whether or not syntactic
  passivization would apply to a particular clausal idiom would depend
  on whether the idiom has the semantic properties compatible with the
  changes in information structure induced by passivization (as
  mentioned above concerning phrasal idioms).} If no sharing occurs in
the case of clausal idioms, it could be due to the inaccessibility of
clausal idioms to internal syntactic operations, resulting from their
being lexical entries inserted into the syntax as single one-piece
units.\footnote{A Google search reveals that the verbal passive version
  of the idiom in (i) does have some occurrences, though substantially
  fewer than the transitive form.
\ea could've knocked me over with a feather\\
  `I was extremely surprised, astonished'
\ex(\#)I could've been knocked over with a feather.
\z

  The question is whether or not these occurrences indeed are clausal
  idioms at all. This cannot be unequivocally determined because along
  with the clausal idiom (i), this idiom turns out to have also a
  phrasal transitive version: `knock (someone) over with a feather'
  (listed in this form, with no fixed tense, no modal (and no fixed
  subject or object) (see the online Free Dictionary
  \url{https://goo.gl/cv7RlT})\emph{.} See also fn. 12.}

\section{Conclusion}

We have distinguished between two different types of idioms -- phrasal
idioms vs. clausal idioms -- and investigated their cross-diathesis
distribution. Phrasal idioms distribute differently in the verbal
passive vs. other diatheses: they cannot be specific (unique) to the
former but can be specific to the later. Clausal idioms do not seem to
discriminate between diatheses in this way: They seem specific to a
single diathesis. These systematic distinctions show that even the
properties of idioms, the archetypal `construction' à la CxG, require
more than cognitive principles, functional needs, and inheritance
networks of stored entities ('constructions') to be accounted for. An
adequate theory of idioms must have recourse to a distinction between
stored items and unstored derivational outputs, and to grammatical
distinctions such as those between diatheses, and those between
functional versus lexical elements. We sketched an account of the above
findings, distinguishing between diatheses according to where they are
formed, and storing idioms according to the type of element heading them
(lexical or functional). Thus, the domain of idioms (surprisingly, from
the CxG point of view) turns out to reinforce the conclusion that there
must be more to knowledge of language than a hierarchical inventory of
items and extra-grammatical constraints.

%\section*{Abbreviations}
\section*{Acknowledgements}
We are grateful to the editors for the
  opportunity to contribute to this Festschrift in honor of Steve
  Anderson, whose work in the field is incredibly rich, multifaceted and
  profound. Our paper addresses the choice between alternative
  architectures of grammar and examines an issue central to both
  syntactic and phonological/morphological research: the nature of
  lexical representations and the division of labor between the lexicon
  and post-lexical derivation, a theme recurrent in Steve's work. More
  specifically, the paper has consequences with regard to the status of
  lexical vs. syntactic operations for capturing relations between
  diathesis alternants. Its empirical basis involves particular
  differences uncovered between diathesis alternations; our findings
  provide novel reinforcement for the distinction between lexically
  versus syntactically derived diatheses. This is a topic that Steve has
  directly contributed to in his paper \emph{Comments on the paper by
  Wasow}, delivered in 1976 at the ground-breaking UC, Irvine Conference
  on the Formal Syntax of Natural Language, and published as \citet{anderson1977}. We wish to thank two anonymous referees for their helpful
  comments.
  
  
\printbibliography[heading=subbibliography,notkeyword=this]


\end{document}

