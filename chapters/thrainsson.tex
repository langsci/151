\documentclass[output=paper,
modfonts
]{LSP/langsci}


%% add all extra packages you need to load to this file 
% \usepackage{todo} %% removed,cna use todonotes instead. % Jason reactivated
% \usepackage{graphicx} % not needed because forest loads tikz, which loads graphicx
\usepackage{tabularx}
\usepackage{amsmath} 
\usepackage{multicol}
\usepackage{lipsum}
\usepackage{longtable}
\usepackage{booktabs}
\usepackage[normalem]{ulem}
%\usepackage{tikz} % not needed because forest loads tikz
\usepackage{phonrule} % for SPE-style phonological rules
\usepackage{pst-all} % loads the main pstricks tools; for arrow diagrams in Hale.tex
%\usepackage{leipzig} % for gloss abbreviations
\usepackage[% for automatic cross-referencing
compress,%
capitalize,% labels are always capitalized in LSP style
noabbrev]% labels are always spelled out in LSP style
{cleveref}

% based on http://tex.stackexchange.com/a/318983/42880 for using gb4e examples with cleveref
\crefname{xnumi}{}{}
\creflabelformat{xnumi}{(#2#1#3)}
\crefrangeformat{xnumi}{(#3#1#4)--(#5#2#6)}
\crefname{xnumii}{}{}
\creflabelformat{xnumii}{(#2#1#3)}
\crefrangeformat{xnumii}{(#3#1#4)--(#5#2#6)}

%\usepackage[notcite,notref]{showkeys} %%removed, not helping CB.
%\usepackage{showidx} %%remove for final compiling - shows index keys at top of page.
 
\usepackage{langsci/styles/langsci-gb4e}  
 \usepackage{pifont}
% % OT tableaux                                                
% \usepackage{pstricks,colortab}  
\usepackage{multirow} % used in OT tableaux
\usepackage{rotating} %needed for angled text%
\usepackage{colortbl} % for cell shading
 
 \usepackage{avm}  
\usepackage[linguistics]{forest} 
\usetikzlibrary{matrix,fit} % for matrix of nodes in Kaisse and Bat-El


\usepackage{hhline}
\newcommand{\cgr}{\cellcolor[gray]{0.8}}
\newcommand{\cn}{\centering}



\newcommand{\reff}[1]{(\ref{#1})}
%\usepackage{newtxtext,newtxmath}


%\usepackage[normalem] {ulem}
\usepackage{qtree}
%\usepackage{natbib}
%\usepackage{tikz}
%\usepackage{gb4e}
\usepackage{phonrule}  
%\bibliographystyle{humannat}



\usepackage{minibox}

%\include{psheader-metr}

\def\bl#1{$_{\textrm{{\footnotesize #1}}}$}
%%add all your local new commands to this file

\newcommand{\form}[1]{\mbox{\emph{#1}}}
\newcommand{\uf}[1]{\mbox{/#1/}}

% borrowed from expex and converted from plan tex to latex
\newcommand{\judge}[1]{{\upshape #1\hspace{0.1em}}}
\newcommand{\ljudge}[1]{\makebox[0pt][r]{\judge{#1}}}

\newcommand\tikzmark[1]{\tikz[remember picture, baseline=(#1.base)] \node[anchor=base,inner sep=0pt, outer sep=0pt] (#1) {#1};} % for adding decorations, arrows, lines, etc. to text
\newcommand\tikzmarknamed[2]{\tikz[remember picture, baseline=(#1.base)] \node[anchor=base,inner sep=0pt, outer sep=0pt] (#1) {#2};} % for adding decorations, arrows, lines, etc. to text
\newcommand\tikzmarkfullnamed[2]{\tikz[remember picture, baseline=(#1.base)] \node[anchor=base,inner sep=0pt, outer sep=0pt] (#1) {\vphantom{X}#2};} % for adding decorations, arrows, lines, etc. to text; this one works best for decorations above a line of text because it adds in the heigh of a capital letter and takes two arguments - one for the node name and one for the printed text

\newcommand{\sub}[1]{$_{\text{#1}}$} % for non-math subscripts
\newcommand{\subit}[1]{\sub{\textit{#1}}} % for italics non-math subscripts
\newcommand{\1}{\rlap{$'$}\xspace} % for the prime in X' (the \rlap command allows the prime to be ignored for horizontal spacing in trees, and the \xspace command allows you to use this in normal text without adding \ afterwards). This isn't crucial, but it helps the formatting to look a little better.

% Aissen:
\newcommand\tikzmarkfull[1]{\tikz[remember picture, baseline=(#1.base)] \node[anchor=base,inner sep=0pt, outer sep=0pt] (#1) {\vphantom{X}#1};} % for adding decorations, arrows, lines, etc. to text; this one works best for decorations above a line of text because it adds in the heigh of a capital letter and takes one argument that serves as the name and the printed text
\newcommand{\bridgeover}[2]{% for a line that creates a bridge over text, connecting two nodes
	\begin{tikzpicture}[remember picture,overlay]
	\draw[thick,shorten >=3pt,shorten <=3pt] (#1.north) |- +(0ex,2.5ex) -| (#2.north);
	\end{tikzpicture}
}
\newcommand{\bridgeoverht}[3]{% for a line that creates a bridge over text, connecting two nodes and specifing the height of the bridge
	\begin{tikzpicture}[remember picture,overlay]
	\draw[thick,shorten >=3pt,shorten <=3pt] (#2.north) |- +(0ex,#1) -| (#3.north);
	\end{tikzpicture}
}
\newcommand{\bridgeoverex}{\vspace*{3ex}} % place before an example that has a \bridgeover so that there is enough vertical space for it

% Chung:
\newcommand{\lefttabular}[1]{\begin{tabular}{p{0.5in}}#1\end{tabular}}

% Kaisse:
\newcommand{\mgmorph}[1]{|(#1)| {#1}}
\newcommand{\mgone}[2][$\times$]{\node at (#2.base) [above=2ex] (1#2) {\vphantom{X}#1};}
\newcommand{\mgtwo}[2][$\times$]{\mgone{#2} \node at (#2.base) [above=4.5ex] (2#2) {\vphantom{X}#1};}
\newcommand{\mgthree}[2][$\times$]{\mgtwo{#2} \node at (#2.base) [above=7ex] (3#2) {\vphantom{X}#1};}
\newcommand{\mgftl}[1]{\node at (1#1) [left] {(};}
\newcommand{\mgftr}[1]{\node at (1#1) [right] {)};}
\newcommand{\mgfoot}[2]{\mgftl{#1}\mgftr{#2}}
\newcommand{\mgldelim}[2]{\node at (#2.west) [left,inner sep = 0pt, outer sep = 0pt] {#1};}
\newcommand{\mgrdelim}[2]{\node at (#2.east) [right,inner sep = 0pt, outer sep = 0pt] {#1};}

\newcommand{\squish}{\hspace*{-3pt}}

% Kavitskaya:
\newcommand{\assoc}[2]{\draw (#1.south) -- (#2.north);}
\newcolumntype{L}{>{\raggedright\arraybackslash}X}

% Lepic & Padden:
\newcommand{\fitpic}[1]{\resizebox{\hsize}{!}{\includegraphics{#1}}} % from http://tex.stackexchange.com/a/148965/42880
\newcommand{\signpic}[1]{\includegraphics[width=1.36in]{#1}}
\newcolumntype{C}{>{\centering\arraybackslash}X}

% Spencer:

\newcommand{\textex}[1]{\textit{#1}\xspace}
\newcommand{\lxm}[1]{\textsc{#1}\xspace}

% Thrainsson:

\renewcommand{\textasciitilde}{\char`~} % for use with TTF/OTF fonts (see comments on http://tex.stackexchange.com/a/317/42880)
\newcommand{\tikzarrow}[2]{% for an arrow connecting two nodes
\begin{tikzpicture}[remember picture,overlay]
\draw[thick,shorten >=3pt,shorten <=3pt,->,>=stealth] (#1) -- (#2);
\end{tikzpicture}
}

\newlength{\padding}
\setlength{\padding}{0.5em}
\newcommand{\lesspadding}{\hspace*{-\padding}}
\newcommand{\feat}[1]{\lesspadding#1\lesspadding}

% Hammond

\usepackage[]{graphicx}\usepackage[]{xcolor}
%% maxwidth is the original width if it is less than linewidth
%% otherwise use linewidth (to make sure the graphics do not exceed the margin)
\makeatletter
\def\maxwidth{ %
  \ifdim\Gin@nat@width>\linewidth
    \linewidth
  \else
    \Gin@nat@width
  \fi
}
\makeatother

\definecolor{fgcolor}{rgb}{0.345, 0.345, 0.345}
\newcommand{\hlnum}[1]{\textcolor[rgb]{0.686,0.059,0.569}{#1}}%
\newcommand{\hlstr}[1]{\textcolor[rgb]{0.192,0.494,0.8}{#1}}%
\newcommand{\hlcom}[1]{\textcolor[rgb]{0.678,0.584,0.686}{\textit{#1}}}%
\newcommand{\hlopt}[1]{\textcolor[rgb]{0,0,0}{#1}}%
\newcommand{\hlstd}[1]{\textcolor[rgb]{0.345,0.345,0.345}{#1}}%
\newcommand{\hlkwa}[1]{\textcolor[rgb]{0.161,0.373,0.58}{\textbf{#1}}}%
\newcommand{\hlkwb}[1]{\textcolor[rgb]{0.69,0.353,0.396}{#1}}%
\newcommand{\hlkwc}[1]{\textcolor[rgb]{0.333,0.667,0.333}{#1}}%
\newcommand{\hlkwd}[1]{\textcolor[rgb]{0.737,0.353,0.396}{\textbf{#1}}}%
\let\hlipl\hlkwb

\usepackage{framed}
\makeatletter
\newenvironment{kframe}{%
 \def\at@end@of@kframe{}%
 \ifinner\ifhmode%
  \def\at@end@of@kframe{\end{minipage}}%
  \begin{minipage}{\columnwidth}%
 \fi\fi%
 \def\FrameCommand##1{\hskip\@totalleftmargin \hskip-\fboxsep
 \colorbox{shadecolor}{##1}\hskip-\fboxsep
     % There is no \\@totalrightmargin, so:
     \hskip-\linewidth \hskip-\@totalleftmargin \hskip\columnwidth}%
 \MakeFramed {\advance\hsize-\width
   \@totalleftmargin\z@ \linewidth\hsize
   \@setminipage}}%
 {\par\unskip\endMakeFramed%
 \at@end@of@kframe}
\makeatother

\definecolor{shadecolor}{rgb}{.97, .97, .97}
\definecolor{messagecolor}{rgb}{0, 0, 0}
\definecolor{warningcolor}{rgb}{1, 0, 1}
\definecolor{errorcolor}{rgb}{1, 0, 0}
\newenvironment{knitrout}{}{} % an empty environment to be redefined in TeX

\usepackage{alltt}

%revised version started: 12/17/16

%NEEDS: allbib.bib - already added to the master bibliography file.
%cited references only: bibexport -o mhTMP.bib main1-blx.aux
%PLUS sramh-img*, sramh.tex

%added stuff
\newcommand{\add}[1]{\textcolor{blue}{#1}}
%deleted stuff
\newcommand{\del}[1]{\textcolor{red}{(removed: #1)}}
%uncomment these to turn off colors
\renewcommand{\add}[1]{#1}
\renewcommand{\del}[1]{}

%shortcuts
\newcommand{\w}{\ili{Welsh}}
\newcommand{\e}{\ili{English}}
\newcommand{\io}{Input Optimization}




 \newcommand{\hand}{\ding{43}}
% \newcommand{\rot}[1]{\begin{rotate}{90}#1\end{rotate}} %shortcut for angled text%  
% \newcommand{\rotcon}[1]{\raisebox{-5ex}{\hspace*{1ex}\rot{\hspace*{1ex}#1}}}

%% add all extra packages you need to load to this file 
% \usepackage{todo} %% removed,cna use todonotes instead. % Jason reactivated
% \usepackage{graphicx} % not needed because forest loads tikz, which loads graphicx
\usepackage{tabularx}
\usepackage{amsmath} 
\usepackage{multicol}
\usepackage{lipsum}
\usepackage{longtable}
\usepackage{booktabs}
\usepackage[normalem]{ulem}
%\usepackage{tikz} % not needed because forest loads tikz
\usepackage{phonrule} % for SPE-style phonological rules
\usepackage{pst-all} % loads the main pstricks tools; for arrow diagrams in Hale.tex
%\usepackage{leipzig} % for gloss abbreviations
\usepackage[% for automatic cross-referencing
compress,%
capitalize,% labels are always capitalized in LSP style
noabbrev]% labels are always spelled out in LSP style
{cleveref}

% based on http://tex.stackexchange.com/a/318983/42880 for using gb4e examples with cleveref
\crefname{xnumi}{}{}
\creflabelformat{xnumi}{(#2#1#3)}
\crefrangeformat{xnumi}{(#3#1#4)--(#5#2#6)}
\crefname{xnumii}{}{}
\creflabelformat{xnumii}{(#2#1#3)}
\crefrangeformat{xnumii}{(#3#1#4)--(#5#2#6)}

%\usepackage[notcite,notref]{showkeys} %%removed, not helping CB.
%\usepackage{showidx} %%remove for final compiling - shows index keys at top of page.
 
\usepackage{langsci/styles/langsci-gb4e}  
 \usepackage{pifont}
% % OT tableaux                                                
% \usepackage{pstricks,colortab}  
\usepackage{multirow} % used in OT tableaux
\usepackage{rotating} %needed for angled text%
\usepackage{colortbl} % for cell shading
 
 \usepackage{avm}  
\usepackage[linguistics]{forest} 
\usetikzlibrary{matrix,fit} % for matrix of nodes in Kaisse and Bat-El


\usepackage{hhline}
\newcommand{\cgr}{\cellcolor[gray]{0.8}}
\newcommand{\cn}{\centering}



\newcommand{\reff}[1]{(\ref{#1})}
%\usepackage{newtxtext,newtxmath}


%\usepackage[normalem] {ulem}
\usepackage{qtree}
%\usepackage{natbib}
%\usepackage{tikz}
%\usepackage{gb4e}
\usepackage{phonrule}  
%\bibliographystyle{humannat}



\usepackage{minibox}

%\include{psheader-metr}

\def\bl#1{$_{\textrm{{\footnotesize #1}}}$}
\usepackage{arydshln}
\usepackage{rotating}

%%add all your local new commands to this file

\newcommand{\form}[1]{\mbox{\emph{#1}}}
\newcommand{\uf}[1]{\mbox{/#1/}}

% borrowed from expex and converted from plan tex to latex
\newcommand{\judge}[1]{{\upshape #1\hspace{0.1em}}}
\newcommand{\ljudge}[1]{\makebox[0pt][r]{\judge{#1}}}

\newcommand\tikzmark[1]{\tikz[remember picture, baseline=(#1.base)] \node[anchor=base,inner sep=0pt, outer sep=0pt] (#1) {#1};} % for adding decorations, arrows, lines, etc. to text
\newcommand\tikzmarknamed[2]{\tikz[remember picture, baseline=(#1.base)] \node[anchor=base,inner sep=0pt, outer sep=0pt] (#1) {#2};} % for adding decorations, arrows, lines, etc. to text
\newcommand\tikzmarkfullnamed[2]{\tikz[remember picture, baseline=(#1.base)] \node[anchor=base,inner sep=0pt, outer sep=0pt] (#1) {\vphantom{X}#2};} % for adding decorations, arrows, lines, etc. to text; this one works best for decorations above a line of text because it adds in the heigh of a capital letter and takes two arguments - one for the node name and one for the printed text

\newcommand{\sub}[1]{$_{\text{#1}}$} % for non-math subscripts
\newcommand{\subit}[1]{\sub{\textit{#1}}} % for italics non-math subscripts
\newcommand{\1}{\rlap{$'$}\xspace} % for the prime in X' (the \rlap command allows the prime to be ignored for horizontal spacing in trees, and the \xspace command allows you to use this in normal text without adding \ afterwards). This isn't crucial, but it helps the formatting to look a little better.

% Aissen:
\newcommand\tikzmarkfull[1]{\tikz[remember picture, baseline=(#1.base)] \node[anchor=base,inner sep=0pt, outer sep=0pt] (#1) {\vphantom{X}#1};} % for adding decorations, arrows, lines, etc. to text; this one works best for decorations above a line of text because it adds in the heigh of a capital letter and takes one argument that serves as the name and the printed text
\newcommand{\bridgeover}[2]{% for a line that creates a bridge over text, connecting two nodes
	\begin{tikzpicture}[remember picture,overlay]
	\draw[thick,shorten >=3pt,shorten <=3pt] (#1.north) |- +(0ex,2.5ex) -| (#2.north);
	\end{tikzpicture}
}
\newcommand{\bridgeoverht}[3]{% for a line that creates a bridge over text, connecting two nodes and specifing the height of the bridge
	\begin{tikzpicture}[remember picture,overlay]
	\draw[thick,shorten >=3pt,shorten <=3pt] (#2.north) |- +(0ex,#1) -| (#3.north);
	\end{tikzpicture}
}
\newcommand{\bridgeoverex}{\vspace*{3ex}} % place before an example that has a \bridgeover so that there is enough vertical space for it

% Chung:
\newcommand{\lefttabular}[1]{\begin{tabular}{p{0.5in}}#1\end{tabular}}

% Kaisse:
\newcommand{\mgmorph}[1]{|(#1)| {#1}}
\newcommand{\mgone}[2][$\times$]{\node at (#2.base) [above=2ex] (1#2) {\vphantom{X}#1};}
\newcommand{\mgtwo}[2][$\times$]{\mgone{#2} \node at (#2.base) [above=4.5ex] (2#2) {\vphantom{X}#1};}
\newcommand{\mgthree}[2][$\times$]{\mgtwo{#2} \node at (#2.base) [above=7ex] (3#2) {\vphantom{X}#1};}
\newcommand{\mgftl}[1]{\node at (1#1) [left] {(};}
\newcommand{\mgftr}[1]{\node at (1#1) [right] {)};}
\newcommand{\mgfoot}[2]{\mgftl{#1}\mgftr{#2}}
\newcommand{\mgldelim}[2]{\node at (#2.west) [left,inner sep = 0pt, outer sep = 0pt] {#1};}
\newcommand{\mgrdelim}[2]{\node at (#2.east) [right,inner sep = 0pt, outer sep = 0pt] {#1};}

\newcommand{\squish}{\hspace*{-3pt}}

% Kavitskaya:
\newcommand{\assoc}[2]{\draw (#1.south) -- (#2.north);}
\newcolumntype{L}{>{\raggedright\arraybackslash}X}

% Lepic & Padden:
\newcommand{\fitpic}[1]{\resizebox{\hsize}{!}{\includegraphics{#1}}} % from http://tex.stackexchange.com/a/148965/42880
\newcommand{\signpic}[1]{\includegraphics[width=1.36in]{#1}}
\newcolumntype{C}{>{\centering\arraybackslash}X}

% Spencer:

\newcommand{\textex}[1]{\textit{#1}\xspace}
\newcommand{\lxm}[1]{\textsc{#1}\xspace}

% Thrainsson:

\renewcommand{\textasciitilde}{\char`~} % for use with TTF/OTF fonts (see comments on http://tex.stackexchange.com/a/317/42880)
\newcommand{\tikzarrow}[2]{% for an arrow connecting two nodes
\begin{tikzpicture}[remember picture,overlay]
\draw[thick,shorten >=3pt,shorten <=3pt,->,>=stealth] (#1) -- (#2);
\end{tikzpicture}
}

\newlength{\padding}
\setlength{\padding}{0.5em}
\newcommand{\lesspadding}{\hspace*{-\padding}}
\newcommand{\feat}[1]{\lesspadding#1\lesspadding}

% Hammond

\usepackage[]{graphicx}\usepackage[]{xcolor}
%% maxwidth is the original width if it is less than linewidth
%% otherwise use linewidth (to make sure the graphics do not exceed the margin)
\makeatletter
\def\maxwidth{ %
  \ifdim\Gin@nat@width>\linewidth
    \linewidth
  \else
    \Gin@nat@width
  \fi
}
\makeatother

\definecolor{fgcolor}{rgb}{0.345, 0.345, 0.345}
\newcommand{\hlnum}[1]{\textcolor[rgb]{0.686,0.059,0.569}{#1}}%
\newcommand{\hlstr}[1]{\textcolor[rgb]{0.192,0.494,0.8}{#1}}%
\newcommand{\hlcom}[1]{\textcolor[rgb]{0.678,0.584,0.686}{\textit{#1}}}%
\newcommand{\hlopt}[1]{\textcolor[rgb]{0,0,0}{#1}}%
\newcommand{\hlstd}[1]{\textcolor[rgb]{0.345,0.345,0.345}{#1}}%
\newcommand{\hlkwa}[1]{\textcolor[rgb]{0.161,0.373,0.58}{\textbf{#1}}}%
\newcommand{\hlkwb}[1]{\textcolor[rgb]{0.69,0.353,0.396}{#1}}%
\newcommand{\hlkwc}[1]{\textcolor[rgb]{0.333,0.667,0.333}{#1}}%
\newcommand{\hlkwd}[1]{\textcolor[rgb]{0.737,0.353,0.396}{\textbf{#1}}}%
\let\hlipl\hlkwb

\usepackage{framed}
\makeatletter
\newenvironment{kframe}{%
 \def\at@end@of@kframe{}%
 \ifinner\ifhmode%
  \def\at@end@of@kframe{\end{minipage}}%
  \begin{minipage}{\columnwidth}%
 \fi\fi%
 \def\FrameCommand##1{\hskip\@totalleftmargin \hskip-\fboxsep
 \colorbox{shadecolor}{##1}\hskip-\fboxsep
     % There is no \\@totalrightmargin, so:
     \hskip-\linewidth \hskip-\@totalleftmargin \hskip\columnwidth}%
 \MakeFramed {\advance\hsize-\width
   \@totalleftmargin\z@ \linewidth\hsize
   \@setminipage}}%
 {\par\unskip\endMakeFramed%
 \at@end@of@kframe}
\makeatother

\definecolor{shadecolor}{rgb}{.97, .97, .97}
\definecolor{messagecolor}{rgb}{0, 0, 0}
\definecolor{warningcolor}{rgb}{1, 0, 1}
\definecolor{errorcolor}{rgb}{1, 0, 0}
\newenvironment{knitrout}{}{} % an empty environment to be redefined in TeX

\usepackage{alltt}

%revised version started: 12/17/16

%NEEDS: allbib.bib - already added to the master bibliography file.
%cited references only: bibexport -o mhTMP.bib main1-blx.aux
%PLUS sramh-img*, sramh.tex

%added stuff
\newcommand{\add}[1]{\textcolor{blue}{#1}}
%deleted stuff
\newcommand{\del}[1]{\textcolor{red}{(removed: #1)}}
%uncomment these to turn off colors
\renewcommand{\add}[1]{#1}
\renewcommand{\del}[1]{}

%shortcuts
\newcommand{\w}{\ili{Welsh}}
\newcommand{\e}{\ili{English}}
\newcommand{\io}{Input Optimization}




 \newcommand{\hand}{\ding{43}}
% \newcommand{\rot}[1]{\begin{rotate}{90}#1\end{rotate}} %shortcut for angled text%  
% \newcommand{\rotcon}[1]{\raisebox{-5ex}{\hspace*{1ex}\rot{\hspace*{1ex}#1}}}

%% add all extra packages you need to load to this file 
% \usepackage{todo} %% removed,cna use todonotes instead. % Jason reactivated
% \usepackage{graphicx} % not needed because forest loads tikz, which loads graphicx
\usepackage{tabularx}
\usepackage{amsmath} 
\usepackage{multicol}
\usepackage{lipsum}
\usepackage{longtable}
\usepackage{booktabs}
\usepackage[normalem]{ulem}
%\usepackage{tikz} % not needed because forest loads tikz
\usepackage{phonrule} % for SPE-style phonological rules
\usepackage{pst-all} % loads the main pstricks tools; for arrow diagrams in Hale.tex
%\usepackage{leipzig} % for gloss abbreviations
\usepackage[% for automatic cross-referencing
compress,%
capitalize,% labels are always capitalized in LSP style
noabbrev]% labels are always spelled out in LSP style
{cleveref}

% based on http://tex.stackexchange.com/a/318983/42880 for using gb4e examples with cleveref
\crefname{xnumi}{}{}
\creflabelformat{xnumi}{(#2#1#3)}
\crefrangeformat{xnumi}{(#3#1#4)--(#5#2#6)}
\crefname{xnumii}{}{}
\creflabelformat{xnumii}{(#2#1#3)}
\crefrangeformat{xnumii}{(#3#1#4)--(#5#2#6)}

%\usepackage[notcite,notref]{showkeys} %%removed, not helping CB.
%\usepackage{showidx} %%remove for final compiling - shows index keys at top of page.
 
\usepackage{langsci/styles/langsci-gb4e}  
 \usepackage{pifont}
% % OT tableaux                                                
% \usepackage{pstricks,colortab}  
\usepackage{multirow} % used in OT tableaux
\usepackage{rotating} %needed for angled text%
\usepackage{colortbl} % for cell shading
 
 \usepackage{avm}  
\usepackage[linguistics]{forest} 
\usetikzlibrary{matrix,fit} % for matrix of nodes in Kaisse and Bat-El


\usepackage{hhline}
\newcommand{\cgr}{\cellcolor[gray]{0.8}}
\newcommand{\cn}{\centering}



\newcommand{\reff}[1]{(\ref{#1})}
%\usepackage{newtxtext,newtxmath}


%\usepackage[normalem] {ulem}
\usepackage{qtree}
%\usepackage{natbib}
%\usepackage{tikz}
%\usepackage{gb4e}
\usepackage{phonrule}  
%\bibliographystyle{humannat}



\usepackage{minibox}

%\include{psheader-metr}

\def\bl#1{$_{\textrm{{\footnotesize #1}}}$}
\usepackage{arydshln}
\usepackage{rotating}

%%add all your local new commands to this file

\newcommand{\form}[1]{\mbox{\emph{#1}}}
\newcommand{\uf}[1]{\mbox{/#1/}}

% borrowed from expex and converted from plan tex to latex
\newcommand{\judge}[1]{{\upshape #1\hspace{0.1em}}}
\newcommand{\ljudge}[1]{\makebox[0pt][r]{\judge{#1}}}

\newcommand\tikzmark[1]{\tikz[remember picture, baseline=(#1.base)] \node[anchor=base,inner sep=0pt, outer sep=0pt] (#1) {#1};} % for adding decorations, arrows, lines, etc. to text
\newcommand\tikzmarknamed[2]{\tikz[remember picture, baseline=(#1.base)] \node[anchor=base,inner sep=0pt, outer sep=0pt] (#1) {#2};} % for adding decorations, arrows, lines, etc. to text
\newcommand\tikzmarkfullnamed[2]{\tikz[remember picture, baseline=(#1.base)] \node[anchor=base,inner sep=0pt, outer sep=0pt] (#1) {\vphantom{X}#2};} % for adding decorations, arrows, lines, etc. to text; this one works best for decorations above a line of text because it adds in the heigh of a capital letter and takes two arguments - one for the node name and one for the printed text

\newcommand{\sub}[1]{$_{\text{#1}}$} % for non-math subscripts
\newcommand{\subit}[1]{\sub{\textit{#1}}} % for italics non-math subscripts
\newcommand{\1}{\rlap{$'$}\xspace} % for the prime in X' (the \rlap command allows the prime to be ignored for horizontal spacing in trees, and the \xspace command allows you to use this in normal text without adding \ afterwards). This isn't crucial, but it helps the formatting to look a little better.

% Aissen:
\newcommand\tikzmarkfull[1]{\tikz[remember picture, baseline=(#1.base)] \node[anchor=base,inner sep=0pt, outer sep=0pt] (#1) {\vphantom{X}#1};} % for adding decorations, arrows, lines, etc. to text; this one works best for decorations above a line of text because it adds in the heigh of a capital letter and takes one argument that serves as the name and the printed text
\newcommand{\bridgeover}[2]{% for a line that creates a bridge over text, connecting two nodes
	\begin{tikzpicture}[remember picture,overlay]
	\draw[thick,shorten >=3pt,shorten <=3pt] (#1.north) |- +(0ex,2.5ex) -| (#2.north);
	\end{tikzpicture}
}
\newcommand{\bridgeoverht}[3]{% for a line that creates a bridge over text, connecting two nodes and specifing the height of the bridge
	\begin{tikzpicture}[remember picture,overlay]
	\draw[thick,shorten >=3pt,shorten <=3pt] (#2.north) |- +(0ex,#1) -| (#3.north);
	\end{tikzpicture}
}
\newcommand{\bridgeoverex}{\vspace*{3ex}} % place before an example that has a \bridgeover so that there is enough vertical space for it

% Chung:
\newcommand{\lefttabular}[1]{\begin{tabular}{p{0.5in}}#1\end{tabular}}

% Kaisse:
\newcommand{\mgmorph}[1]{|(#1)| {#1}}
\newcommand{\mgone}[2][$\times$]{\node at (#2.base) [above=2ex] (1#2) {\vphantom{X}#1};}
\newcommand{\mgtwo}[2][$\times$]{\mgone{#2} \node at (#2.base) [above=4.5ex] (2#2) {\vphantom{X}#1};}
\newcommand{\mgthree}[2][$\times$]{\mgtwo{#2} \node at (#2.base) [above=7ex] (3#2) {\vphantom{X}#1};}
\newcommand{\mgftl}[1]{\node at (1#1) [left] {(};}
\newcommand{\mgftr}[1]{\node at (1#1) [right] {)};}
\newcommand{\mgfoot}[2]{\mgftl{#1}\mgftr{#2}}
\newcommand{\mgldelim}[2]{\node at (#2.west) [left,inner sep = 0pt, outer sep = 0pt] {#1};}
\newcommand{\mgrdelim}[2]{\node at (#2.east) [right,inner sep = 0pt, outer sep = 0pt] {#1};}

\newcommand{\squish}{\hspace*{-3pt}}

% Kavitskaya:
\newcommand{\assoc}[2]{\draw (#1.south) -- (#2.north);}
\newcolumntype{L}{>{\raggedright\arraybackslash}X}

% Lepic & Padden:
\newcommand{\fitpic}[1]{\resizebox{\hsize}{!}{\includegraphics{#1}}} % from http://tex.stackexchange.com/a/148965/42880
\newcommand{\signpic}[1]{\includegraphics[width=1.36in]{#1}}
\newcolumntype{C}{>{\centering\arraybackslash}X}

% Spencer:

\newcommand{\textex}[1]{\textit{#1}\xspace}
\newcommand{\lxm}[1]{\textsc{#1}\xspace}

% Thrainsson:

\renewcommand{\textasciitilde}{\char`~} % for use with TTF/OTF fonts (see comments on http://tex.stackexchange.com/a/317/42880)
\newcommand{\tikzarrow}[2]{% for an arrow connecting two nodes
\begin{tikzpicture}[remember picture,overlay]
\draw[thick,shorten >=3pt,shorten <=3pt,->,>=stealth] (#1) -- (#2);
\end{tikzpicture}
}

\newlength{\padding}
\setlength{\padding}{0.5em}
\newcommand{\lesspadding}{\hspace*{-\padding}}
\newcommand{\feat}[1]{\lesspadding#1\lesspadding}

% Hammond

\usepackage[]{graphicx}\usepackage[]{xcolor}
%% maxwidth is the original width if it is less than linewidth
%% otherwise use linewidth (to make sure the graphics do not exceed the margin)
\makeatletter
\def\maxwidth{ %
  \ifdim\Gin@nat@width>\linewidth
    \linewidth
  \else
    \Gin@nat@width
  \fi
}
\makeatother

\definecolor{fgcolor}{rgb}{0.345, 0.345, 0.345}
\newcommand{\hlnum}[1]{\textcolor[rgb]{0.686,0.059,0.569}{#1}}%
\newcommand{\hlstr}[1]{\textcolor[rgb]{0.192,0.494,0.8}{#1}}%
\newcommand{\hlcom}[1]{\textcolor[rgb]{0.678,0.584,0.686}{\textit{#1}}}%
\newcommand{\hlopt}[1]{\textcolor[rgb]{0,0,0}{#1}}%
\newcommand{\hlstd}[1]{\textcolor[rgb]{0.345,0.345,0.345}{#1}}%
\newcommand{\hlkwa}[1]{\textcolor[rgb]{0.161,0.373,0.58}{\textbf{#1}}}%
\newcommand{\hlkwb}[1]{\textcolor[rgb]{0.69,0.353,0.396}{#1}}%
\newcommand{\hlkwc}[1]{\textcolor[rgb]{0.333,0.667,0.333}{#1}}%
\newcommand{\hlkwd}[1]{\textcolor[rgb]{0.737,0.353,0.396}{\textbf{#1}}}%
\let\hlipl\hlkwb

\usepackage{framed}
\makeatletter
\newenvironment{kframe}{%
 \def\at@end@of@kframe{}%
 \ifinner\ifhmode%
  \def\at@end@of@kframe{\end{minipage}}%
  \begin{minipage}{\columnwidth}%
 \fi\fi%
 \def\FrameCommand##1{\hskip\@totalleftmargin \hskip-\fboxsep
 \colorbox{shadecolor}{##1}\hskip-\fboxsep
     % There is no \\@totalrightmargin, so:
     \hskip-\linewidth \hskip-\@totalleftmargin \hskip\columnwidth}%
 \MakeFramed {\advance\hsize-\width
   \@totalleftmargin\z@ \linewidth\hsize
   \@setminipage}}%
 {\par\unskip\endMakeFramed%
 \at@end@of@kframe}
\makeatother

\definecolor{shadecolor}{rgb}{.97, .97, .97}
\definecolor{messagecolor}{rgb}{0, 0, 0}
\definecolor{warningcolor}{rgb}{1, 0, 1}
\definecolor{errorcolor}{rgb}{1, 0, 0}
\newenvironment{knitrout}{}{} % an empty environment to be redefined in TeX

\usepackage{alltt}

%revised version started: 12/17/16

%NEEDS: allbib.bib - already added to the master bibliography file.
%cited references only: bibexport -o mhTMP.bib main1-blx.aux
%PLUS sramh-img*, sramh.tex

%added stuff
\newcommand{\add}[1]{\textcolor{blue}{#1}}
%deleted stuff
\newcommand{\del}[1]{\textcolor{red}{(removed: #1)}}
%uncomment these to turn off colors
\renewcommand{\add}[1]{#1}
\renewcommand{\del}[1]{}

%shortcuts
\newcommand{\w}{\ili{Welsh}}
\newcommand{\e}{\ili{English}}
\newcommand{\io}{Input Optimization}




 \newcommand{\hand}{\ding{43}}
% \newcommand{\rot}[1]{\begin{rotate}{90}#1\end{rotate}} %shortcut for angled text%  
% \newcommand{\rotcon}[1]{\raisebox{-5ex}{\hspace*{1ex}\rot{\hspace*{1ex}#1}}}

%\input{localpackages.tex}
\usepackage{arydshln}
\usepackage{rotating}

%\input{localcommands.tex}
\newcommand{\tworow}[1]{\multirow{2}{*}{#1}}


\newcommand{\tworow}[1]{\multirow{2}{*}{#1}}


\newcommand{\tworow}[1]{\multirow{2}{*}{#1}}


\ChapterDOI{10.5281/zenodo.495440}
\title{{U}-umlaut in Icelandic and Faroese: Survival and death}

\author{%
Höskuldur Thráinsson\affiliation{University of Iceland}
}
\abstract{Although Icelandic and Faroese are closely related and very similar in
many respects, their vowel systems are quite different (see e.g.
\citealt{anderson1969a,arnason2011}). This paper compares \emph{u}-umlaut alternations
in Icelandic and Faroese and shows that the Faroese umlaut has a number
of properties that are to be expected if the relevant alternations are
morphological (or analogical) rather than being due to a synchronic
phonological process. In Icelandic, on the other hand, \emph{u}-umlaut has none
of these properties and arguably behaves like a living phonological
process. This is theoretically interesting because the quality of the
vowels involved (both the umlaut trigger and the target) has changed
from Old to Modern Icelandic. In addition, \emph{u}-umlaut in Modern Icelandic
is more opaque (in the sense of \citealt{kiparsky1973}) than its Old Icelandic
counterpart, i.e.\ it has more surface exceptions. An epenthesis rule
inserting a (non-umlauting) /u/ into certain inflectional endings is the
cause of many of these surface exceptions. Yet it seems that \emph{u}-umlaut in
Icelandic is still transparent enough to be acquired by children as a
phonological process. In Faroese, on the other hand, \emph{u}-umlaut became too
opaque and died out as a phonological rule. It is argued that this has
partly to do with certain changes in the Faroese vowel system and partly
with the fact that the \textit{u}-epenthesis rule was lost in Faroese.}

%\protect\hypertarget{__RefHeading__452_2075933062}{}{}\emph{\textbf{\\
%}} % commented out because it prevented compilation

\begin{document}
\maketitle


\section{Introduction}\label{introductionTh}

Anderson put the process of \emph{u-}\isi{umlaut} in \ili{Icelandic} on the modern linguistic map with the analysis he proposed in his dissertation \citep{anderson1969a} and several subsequent publications \citep{anderson1969b,anderson1972t,anderson1973,anderson1974,anderson1976t}. Because of changes in the vowel system from Old to Modern
\ili{Icelandic}, the nature of the \isi{umlaut} process changed somewhat through the ages (see e.g. \citealt{benediktsson1959}). The most important part of \emph{u-}\isi{umlaut}, and the only part that is alive in the modern language, involves /a/ \textasciitilde{} /ǫ/ alternations\is{alternation} in the old language (phonetically {[}a{]} \textasciitilde{} {[}ɔ{]}, as shown in \ref{ex:thrainsson:2}), which show up as /a/ \textasciitilde{} /ö/ alternations\is{alternation} in the modern language (phonetically {[}a{]} \textasciitilde{} {[}œ{]}, cf. \ref{ex:thrainsson:2}). This is illustrated in \REF{ex:thrainsson:1} with the relevant vowel symbols highlighted:

\ea \label{ex:thrainsson:1}
\begin{tabular}[t]{ l l }
Old \ili{Icelandic}: & Modern \ili{Icelandic}: \\
\emph{s\textbf{a}ga} ʽsagaʼ, \textsc{obl} \emph{s\textbf{ǫ}gu}, \textsc{pl} \emph{s\textbf{ǫ}gur} & \emph{s\textbf{a}ga}, \textsc{obl} \emph{s\textbf{ö}gu}, \textsc{pl} \emph{s\textbf{ö}gur}\\

\emph{hv\textbf{a}ss} `sharp', \textsc{dat} \emph{hv\textbf{ǫ}ssum} & \emph{hv\textbf{a}ss}, \textsc{dat} \emph{hv\textbf{ö}ssum}\\

\emph{t\textbf{a}la} `speak', \textsc{1.pl} \emph{t\textbf{ǫ}lum} & \emph{t\textbf{a}la}, \textsc{1pl} \emph{t\textbf{ö}lum}\\
\end{tabular}
\z

\noindent As these examples suggest, the quality of the root vowel /a/ changes when a /u/ follows in the next syllable. The relevant proecesses can be illustrated schematically as in \REF{ex:thrainsson:2}. For the sake of simplicity I use conventional \isi{orthographic} symbols to represent the vowels and only give IPA-symbols for the vowels that are important for the understanding of the \isi{umlaut} processes. The umlaut-triggering vowels are encircled:\footnote{Note that in the representation of the Modern \ili{Icelandic} vowel system, the accents over vowel symbols have nothing to do with quantity but simply denote separate vowel
qualities. Thus /í/ is {[}i{]}, /i/ is {[}ɪ{]}, /ú/ is {[}u{]} and /u/ is {[}ʏ{]}, as the schematic representation in \REF{ex:thrainsson:2} suggests.}

\ea \label{ex:thrainsson:2}
	\ea \label{ex:thrainsson:2a} \emph{u-}\isi{umlaut} in Old \ili{Icelandic} and the system of short vowels:\\
	\begin{tabular}{ccccccc}
           		 && \multicolumn{2}{c}{{[}-back{]}} &  & \multicolumn{2}{c}{{[}+back{]}} \\
           		 && {[}-round{]}   	& {[}+round{]}   &  & {[}-round{]}   & {[}+round{]}   \\
	       {[}+high{]}    & &      i         &    y            &  &                &    \textcircled{u} {[}u{]}            \\
           			 & &      e       &    ø          &  &                &    {\hspace*{-18pt} o}           \\
	       {[}+low{]}  &	  &    ę        &                 &  &          a \tikzmarkfullnamed{a}{[a]}   & \tikzmarkfullnamed{o}{ǫ} {[}ɔ{]} \\           
	\end{tabular}\tikzarrow{a}{o}
%\newpage
	\ex \label{ex:thrainsson:2b} \emph{u-}\isi{umlaut} in Modern \ili{Icelandic} and the system of monophthongs:\footnote{I am 			assuming here, like \citet{thrainsson1994} and \citet[34]{gislason2000}, for instance, that Modern \ili{Icelandic} only distinguishes between three three vowel heights and that /e/ 	{[}ɛ{]} and /ö/ {[}œ{]} are both phonologically {[}+low{]}, like /a/ {[}a{]} and /o/ {[}ɔ{]}. For different 	assumptions see e.g. \citet[60]{arnason2011}.}\\
	\begin{tabular}{ccccccc}
           		 && \multicolumn{2}{c}{{[}-back{]}} &  & \multicolumn{2}{c}{{[}+back{]}} \\
           		 && {[}-round{]}   	& {[}+round{]}   &  & {[}-round{]}   & {[}+round{]}   \\
	       {[}+high{]}  &   &  í        &                &  &                &    ú {[}u{]}            \\
           			&  &   i    &   \textcircled{u} {[}ʏ{]} &  &                &                \\
	       {[}+low{]}  	&  &  e    &    ö \tikzmarkfullnamed{oe}{[œ]}           &  & \tikzmarkfull{a} {[}a{]}	    &          o {[}ɔ{]} \\
	\end{tabular}\tikzarrow{a}{oe}
	\z
\z

The gist of Anderson's analysis of \emph{u-}\isi{umlaut} can then be illustrated semi-formally as in the traditional generative phonological notation in \REF{ex:thrainsson:3}, with the assimilating \isi{features} highlighted (see also
\citealt[31]{roegnvaldsson1981}, \citealt[89--90]{thrainsson2011}):\footnote{Here, and elsewhere in this paper, I will use the kinds of formulations of rules and conditions familiar from classical \isi{generative phonology}\is{phonology} since much of the work on \emph{u-}\isi{umlaut} has been done in that kind of framework. For analyses employing more recent frameworks see \citealt{gibson2000}, \citealt{hansson2013} and \citealt{ingason2016}. Most of the argumentation in this paper should be relatively framework-independent, however.}

\ea \label{ex:thrainsson:3}
	\ea \label{ex:thrainsson:3a} \emph{u-}\isi{umlaut} in \ili{Old Icelandic}:\\
	\phonr{/a/}{\phonfeat{\feat{\textbf{+round}}}}{C\sub{0}V\sub{\normalsize\phonfeat{\feat{\textbf{+round}}\\\feat{+high}\\\feat{+back}}}}	
	\ex \label{ex:thrainsson:3b} \emph{u-}\isi{umlaut} in Modern \ili{Icelandic}:\\
	\phonr{/a/}{\phonfeat{\feat{\textbf{+round}}\\\feat{\textbf{-back}}}}{C\sub{0}V\sub{\normalsize\phonfeat{\feat{\textbf{+round}}\\\feat{\textbf{-back}}\\\feat{-low}}}}
	\z
\z

\noindent As the illustration in \REF{ex:thrainsson:3} shows, the modern version of the \isi{umlaut} is
somewhat more complex than the old one, assimilating two \isi{features} rather
than one. Nevertheless, it is still arguably a phonologically (or
phonetically) natural \isi{assimilation} process, assimilating \isi{rounding} and
\isi{backness}.

Although the \emph{u-}\isi{umlaut} discussion was most lively on the
international scene in the 1970s (see e.g. \citealt{iverson1978t,iverson1976,oresnik1975,oresnik1977}, cf. also \citealt{valfells1967}), the topic
keeps popping up to this day, e.g. in journals and conferences dedicated
to Scandinavian \isi{linguistics} (see e.g. \citealt{gibson2000,indridason2010,thrainsson2011,hansson2013}) and even in recent master's theses
and doctoral dissertations (see \citealt{markusson2012,ingason2016}). The main
reason is that while \emph{u-}\isi{umlaut} in Modern \ili{Icelandic} is obviously
very productive, being applied consistently to new words and loanwords,
it shows a number of intriguing surface exceptions.\is{exception} These have been
discussed extensively in the literature cited but here I will
concentrate on the most common and widespread one, namely the lack of
\isi{umlaut} before a /u/ that has been inserted between the inflectional
ending /r/ and a preceding consonant. This \isi{epenthesis} did not exist in
Old \ili{Icelandic} as illustrated in \REF{ex:thrainsson:4}:

\ea \label{ex:thrainsson:4}
\begin{tabular}[t]{ l l }
Old \ili{Icelandic}: & Modern \ili{Icelandic}: \\
\emph{dalr} `valley', \emph{latr} `lazy'  & \emph{dal\textbf{u}r}, \emph{lat\textbf{u}r}
\end{tabular}
\z

\noindent If \emph{u-}\isi{umlaut} is a phonological rule in the modern language, this
\emph{u-}\isi{epenthesis} has to follow it, as it did historically. This is
one of the properties of \emph{u-}\isi{umlaut} that have been used to argue
for the necessity of relatively abstract phonological\is{phonology} representations
and derivations\is{derivation} (e.g. \citealt{anderson1969a,anderson1974,roegnvaldsson1981,thrainsson2011,hansson2013}) while others have maintained that
\emph{u-}\isi{umlaut} is not a phonological\is{phonology} process anymore in Modern
\ili{Icelandic} and the relevant alternations\is{alternation} are morphologized\is{morphology} and purely
\isi{analogical} (see e.g. \citealt{markusson2012}) or at least ``\isi{morpheme-specific}'',\is{morpheme}
i.e. triggered by particular morphemes\is{morpheme} that may or may not contain a /u/
(\citealt{ingason2016}).\footnote{\citet[220]{ingason2016} formulates his \isi{umlaut} rule as follows:

  \begin{quote}
  Realize an underlying /a/ as /ö/ in the syllable which precedes the
  \isi{morpheme} which triggers the \isi{umlaut}.
  \end{quote}

  \noindent As can be seen here, no mention is made of a triggering /u/ in the
  rule. The reason is that Ingason wants to derive all all paradigmatic\is{paradigm}
  /a/ \textasciitilde{} /ö/ alternations\is{alternation} the same way, including the
  ones where /u/ has been syncopated historically. Thus he argues that
  the \textsc{nom.sg.} \isi{morpheme} -\textsc{ø} in \isi{feminine} nouns like
  \emph{sök} `guilt, case' and the \textsc{nom./acc.pl.} \isi{morpheme}
  -\textsc{ø} in neuter nouns like \emph{börn} `children' triggers
  \isi{umlaut} the same way that the \textsc{dat.pl.} \isi{morpheme} \emph{-um} does
  in \emph{sökum} and \emph{börnum}. But many researchers have wanted to
  distinguish between morphologically\is{morphology} conditioned \isi{umlaut}, where there is
  no triggering /u/, and phonologically\is{phonology} conditioned \isi{umlaut} triggered by
  /u/, e.g. \citet{roegnvaldsson1981}. One reason for doing so comes from the
  behavior of loanwords like the adjective \emph{smart} `smart, chic'.
  Here the \textsc{nom.sg.f} and the \textsc{nom/acc.pl.n} can either be
  \emph{smart} or \emph{smört}, i.e. a morphologically conditioned
  \isi{umlaut} may or may not apply. But once an umlauting inflectional\is{inflection} ending
  containing /u/ is added to the \isi{loanword} \emph{smart}, the
  \emph{u-}\isi{umlaut} becomes obligatory. Thus \textsc{dat.pl} can only be
  \emph{sm\textbf{ö}rt-um} and not *\emph{sm\textbf{a}rt-\textbf{u}m}
  and the \textsc{nom.pl.wk} form has to be
  \emph{sm\textbf{ö}rt-\textbf{u}} and not *\emph{smart-\textbf{u}}.
  This suggests that the morphologically conditioned \isi{umlaut} is more
  prone to exceptions\is{exception} than the phonologically conditioned one, which is
  actually to be expected. Thanks to Eiríkur Rögnvaldsson for pointing
  this out to me.}

In this paper I will compare \emph{u-}\isi{umlaut} alternations\is{alternation} in Modern
\ili{Icelandic} and Modern \ili{Faroese}. This comparison will show very clearly
that \emph{u-}\isi{umlaut} in Modern \ili{Faroese} has a number of properties (e.g.
\isi{paradigm} levelling, various kinds of exceptions,\is{exception} total absence from
certain paradigms,\is{paradigm} inapplicability to loanwords ...) that are to be
expected if the relevant alternations\is{alternation} are no longer due to a synchronic
process. In Modern \ili{Icelandic}, on the other hand, \emph{u-}\isi{umlaut} has
none of these properties and behaves more like a phonological rule. This
is of general theoretical interest since it illustrates how phonological
rules can survive (in the case of \ili{Icelandic}) despite reduced
\isi{transparency} (in the sense of \citealt{kiparsky1973}) and how changes in the
phonological\is{phonology} system can cause the death of a phonological rule (in the
case of \ili{Faroese}) and what the consequences can be.

The remainder of the paper is organized as follows: In \S2 I first
illustrate how the \emph{u-}\isi{epenthesis} works in Modern \ili{Icelandic} and
then present a couple of arguments for the phonological\is{phonology} (as opposed to
morphological)\is{morphology} nature of Modern \ili{Icelandic} \emph{u-}\isi{umlaut}. \sectref{sec:thrainsson:3}
first describes some facts about the \ili{Faroese} vowel system that must have
been important for the development of \emph{u-}\isi{umlaut} and then shows
that \emph{u-}\isi{epenthesis} does not exist anymore as a phonological
process in Modern \ili{Faroese}. It is then argued that these developments led
to the death of \emph{u-}\isi{umlaut} as a phonological process in \ili{Faroese}.
\sectref{sec:thrainsson:4} then contains a systematic comparison of \emph{u-}\isi{umlaut}
alternations\is{alternation} in Modern \ili{Icelandic} and \ili{Faroese}, concluding that the
\ili{Faroese} ones must be \isi{analogical} (and morphological)\is{morphology} in nature as they do
not exhibit any of the crucial phonological properties that Modern
\ili{Icelandic} \emph{u-}\isi{umlaut} alternations\is{alternation} show. In \ili{Icelandic}, on the other
hand, \emph{u-}\isi{umlaut} does not show the non-phonological properties
listed for its \ili{Faroese} counterpart. \sectref{sec:thrainsson:5} concludes the paper.

\section{{u}-\isi{epenthesis} and {u-}\isi{umlaut} in Modern Icelandic}

\subsection{The \isi{epenthesis} rule}\label{the-epenthesis-rule}

The \isi{phoneme} /r/ frequently occurs in Old \ili{Icelandic} (\ili{Old Norse}) as a
marker of various morphological categories, including \textsc{nom.sg} of
strong masculine nouns and adjectives as illustrated in \REF{ex:thrainsson:5}. It
sometimes assimilated to a preceding consonant, e.g. /s, l, n/ (cf.
\ref{ex:thrainsson:5c}),\footnote{Assimilation to stem-final /l, n/ only happened in Old
  \ili{Icelandic} if these consonants were preceded by long vowels, i.e. Old
  \ili{Icelandic} diphthongs and vowels that are standardly represented by
  accented vowel symbols in Old \ili{Icelandic} \isi{orthography}, cf.
  \emph{stól-\textbf{l}} `chair' vs. \emph{dal-r} `valley',
  \emph{fín-\textbf{n}} `fine' vs. \emph{lin-r} `soft, limp',
  \emph{heil-\textbf{l}} `whole' vs. \emph{hol-r} `hollow'.} but it was
deleted after certain consonant clusters, such as /gl, gn, ss/ (cf.
\ref{ex:thrainsson:5d}):

\ea \label{ex:thrainsson:5}
	\ea \label{ex:thrainsson:5a} \emph{stór-r} `big', \emph{mó-r} `peat', \emph{há-r} `high'
	\ex \label{ex:thrainsson:5b} \emph{dal-r} `valley', \emph{lat-r} `lazy', \emph{tóm-r} `empty', \emph{harð-r} `hard'
	\ex \label{ex:thrainsson:5c} \emph{ís-s} `ice', \emph{laus-s} `loose', \emph{stól-l} `chair', \emph{fín-n} `fine'
	\ex \label{ex:thrainsson:5d} \emph{fugl} `bird', \emph{vagn} `wagon', \emph{foss} `waterfall' (stem \emph{foss-})
	\z
\z

\noindent It is likely that the /r/ in words of type \REF{ex:thrainsson:5b} was syllabic in Old
\ili{Icelandic}. There are no \isi{syllabic consonants} in Modern \ili{Icelandic}, on the
other hand. Instead a /u/ appears between the /r/ and the preceding
consonant in the modern version of words of type \REF{ex:thrainsson:5b}. There is
historical evidence for this \emph{u-}insertion from the late
thirteenth century and onwards (see e.g. \citealt{kristinsson1992}
and references cited there) and many linguists have argued that
\emph{u-}\isi{epenthesis} is still a productive phonological process in Modern
\ili{Icelandic} (e.g.  \citealt{anderson1969a,anderson1969b,oresnik1972,roegnvaldsson1981,kiparsky1984}).\footnote{Orešnik later  maintained that
  \emph{u-}\isi{epenthesis} could not be a synchronic rule in Modern \ili{Icelandic}
  because of the existence of exceptional word forms like \emph{klifr}
  `climbing' (from the verb \emph{klifra} `climb'), \emph{sötr}
  `slurping' (from the verb \emph{sötra} `slurp'), \emph{pukr}
  `secretiveness' from the verb \emph{pukra(st)} `be secretive about',
  etc. (\citealt{oresnik1978}; see also the discussion in \citealt{kjartansson1984}). In words of this
  kind one would have expected \emph{u-}\isi{epenthesis} to apply. The
  importance of these exceptions is not very clear since this is a very
  special class of words (all derived from verbs ending in -\emph{ra})
  and it is typically possible or even preferred to apply the \isi{epenthesis}
  rule to these forms, giving \emph{klif\textbf{u}r},
  \emph{söt\textbf{u}r}, \emph{puk\textbf{u}r}, etc. For the sake of
  completeness it should be noted that the final \emph{-r} in word forms
  like \emph{sötr}, \emph{pukr} has to be \isi{voiceless} and this may be
  related to the fact that there are no \isi{syllabic consonants} in Modern
  \ili{Icelandic}, as stated above.} This implies that speakers distinguish
between a \emph{-ur}-ending where the underlying morpheme is \#-r\# and
the /u/ is \isi{epenthetic} (and does not trigger \emph{u-}\isi{umlaut}) and a
-\emph{ur-}ending where the /u/ is not \isi{epenthetic} and the underlying
morpheme is \#-ur\# (and the /u/ triggers \emph{u-}\isi{umlaut}). This
contrast is illustrated in \REF{ex:thrainsson:6a} vs.\ \REF{ex:thrainsson:6b} (see also the examples in \ref{ex:thrainsson:1}
and \ref{ex:thrainsson:4} above):

\ea \label{ex:thrainsson:6}\begin{tabular}[t]{ll@{ }l@{ }l@{ \ }l@{ \ }l@{ }l@{ \ }l@{}}
a. & \#dal+r\# & `valley' & \textsc{nom.sg.m} & → & \emph{d\textbf{a}l-\textbf{u}r} & by \isi{epenthesis} & no \isi{umlaut}\\
&\#lat+r\# &`lazy' &\textsc{nom.sg.m} &→& \emph{l\textbf{a}t-\textbf{u}r} &by \isi{epenthesis} &no \isi{umlaut}\\
b. &\#sag+ur\#& `sagas' &\textsc{nom.pl.f}& →& \emph{s\textbf{ö}g-\textbf{u}r}& &\emph{u-}\isi{umlaut}\\
&\#tal+ur\# &`numbers' &\textsc{nom.pl.f}& →& \emph{t\textbf{ö}l-\textbf{u}r}&& \emph{u-}\isi{umlaut}\\
\end{tabular}
%	\ea \label{ex:thrainsson:} \#dal+r\# `valley' \textsc{nom.sg.m} → \emph{d\textbf{a}l-\textbf{u}r} by \isi{epenthesis} no \isi{umlaut}\\
%	 \#lat+r\# `lazy' \textsc{nom.sg.m} → \emph{l\textbf{a}t-\textbf{u}r} by \isi{epenthesis} no \isi{umlaut}
%	 \ex \label{ex:thrainsson:} \#sag+ur\# `sagas' \textsc{nom.pl.f} → \emph{s\textbf{ö}g-\textbf{u}r} \emph{u-}\isi{umlaut}\\
%	\#tal+ur\# `numbers' \textsc{nom.pl.f} → \emph{t\textbf{ö}l-\textbf{u}r} \emph{u-}\isi{umlaut}
%	\z
\z

\noindent Thus the \textsc{nom.sg} ending \#-r\#, which is both found in strong
masculine nouns like \emph{dalur} `valley' and in the strong masculine
form of adjectives like \emph{latur} `lazy', does not have the same
properties as the \textsc{nom.pl} ending \#-ur\# which is found in
\isi{feminine} nouns like \emph{sögur} `sagas' and \emph{tölur} `numbers'.
Despite their surface similarities in certain environments, speakers can
clearly distinguish these endings. A part of the reason must be that the
\textsc{nom.sg.m} ending \#-r\# only shows up as -\emph{ur} in
phonologically definable environments, i.e. the modern version of words
with stems\is{stem} of type \REF{ex:thrainsson:5b}, whereas the \textsc{nom.pl.f} ending \#-ur\# is
not so restricted and always shows up as -\emph{ur}. This is illustrated
in Table~\ref{tab:-r-ur} (compare the examples in \ref{ex:thrainsson:5}).

\begin{table}
\begin{tabularx}{\textwidth}{@{}l>{\raggedright}p{1in}>{\raggedright}X>{\raggedright}X@{}}
\lsptoprule
& type of stem & phonological realization of the \textsc{nom.sg.m} ending \#-r\# & phonological realization of the \textsc{nom.pl.f} ending \#-ur\#\tabularnewline
\midrule
a. & ending in a vowel & -\emph{r} \newline (\emph{mó-r} `peat', \emph{há-r}
`high') & \emph{\textbf{-ur}} \newline (\emph{ló-ur} `golden
plovers')\tabularnewline
b. & ending in a sin- gle consonant (but see c) & \textbf{-\emph{ur}} \newline (\emph{dal-ur} `valley', \emph{lat-ur} ʽlazyʼ) & \textbf{-\emph{ur}}
\newline (\emph{sög-ur} ʽsagasʼ, \emph{töl-ur} ʽnumbersʼ)\tabularnewline
c. & ending in a high vowel + /l,n/ & \isi{assimilation} \newline (\emph{stól-l} `chair',
\emph{fín-n} `fine') & \textbf{-\emph{ur}} \newline (\emph{spús-ur} `wives',
\emph{súl-ur} ʽcolumnsʼ, \emph{dýn-ur} ʽmattresses')\tabularnewline
d. & ending in /s, r/ or consonant clusters ending in /l, n/ such as /gl,
gn/ & \isi{deletion} \newline (\emph{ís} `ice', \emph{laus} `loose', \emph{foss}
`waterfall'\emph{bjór} `beer', \emph{stór} `big', \emph{fugl} `bird',
\emph{vagn} `wagon') & \textbf{-\emph{ur}} \newline (\emph{ýs-ur} ʽhaddocksʼ,
\emph{aus-ur} `scoops', \emph{hór-ur} `whores', \emph{ugl-ur} `owls',
\emph{hrygn-ur} ʽspawning fish', \emph{byss-ur} ʽgunsʼ)\tabularnewline
\lspbottomrule
\end{tabularx}
\caption{Phonological realization of the inflectional endings
	\#-r\# and \#-ur\# in Modern \ili{Icelandic}.}
\label{tab:-r-ur}
\end{table}

Comparison of Table~\ref{tab:-r-ur} and the Old \ili{Icelandic} examples in \REF{ex:thrainsson:5} reveals a
slight extension of \emph{r-}\isi{deletion}: The /r/ of the morphological
ending \#-r\# is now deleted after /r/ (compare line d of the table to
\ref{ex:thrainsson:5a}) and after all instances of /s/, not just /ss/ (compare line d of
the table to (5c,d)). The \emph{u-}\isi{epenthesis} illustrated in line b of
Table~\ref{tab:-r-ur} is an innovation, of course. Otherwise the \textsc{nom.sg.m}
ending behaves in much the same way as in Old \ili{Icelandic}. The different
behavior of the morphemes\is{morpheme} compared in Table~\ref{tab:-r-ur} can be seen as an argument
for distinguishing them in the underlying form, e.g. for not analyzing
the \textsc{n.sg.m} ending as \#-ur\#.

\subsection{Some phonological properties of Modern \ili{Icelandic} {u}-umlaut}

In this section I will mention two sets of facts which show that
\emph{u-}\isi{umlaut} still has certain properties in Modern \ili{Icelandic} that
are to be expected if it is a phonologically\is{phonology} conditioned process.

First, if \emph{u-}\isi{umlaut} was morphologically conditioned and not
phonologically, we would expect it to be restricted to certain
morphological categories or parts of speech. It is not. It applies in
the paradigms\is{paradigm} of nouns, adjectives and verbs when a /u/ follows in the
inflectional\is{inflection} ending (with the exception of the \isi{epenthetic} /u/ already
mentioned). This is illustrated in \REF{ex:thrainsson:7}:

\ea \label{ex:thrainsson:7}
	\ea \label{ex:thrainsson:7a}	\emph{saga} `saga', \textsc{obl.sg}
		\emph{s\textbf{ö}g-\textbf{u}}, \textsc{nom/acc.pl}
		\emph{s\textbf{ö}g-\textbf{u}r}, \textsc{dat.pl}
		\emph{s\textbf{ö}g-\textbf{u}m}
	\ex \label{ex:thrainsson:7b}	\emph{snjall} `smart', \textsc{dat.sg.m}
		\emph{snj\textbf{ö}ll-\textbf{u}m}, \textsc{nom.pl.wk}
		\emph{snj\textbf{ö}ll-\textbf{u}}
	\ex \label{ex:thrainsson:7c}	\emph{kalla} `call', \textsc{1.pl.prs}
		\emph{k\textbf{ö}ll-\textbf{u}m}, \textsc{3.pl.pst}
		\emph{k\textbf{ö}ll\textbf{u}ð-\textbf{u}}
	\z
\z

\noindent The so-called \emph{i-}\isi{umlaut} is very different in this respect. It is
clearly not alive as a phonological rule anymore but its effects can
still be observed in the modern language in certain morphologically
definable environments. As a result we can find near-minimal pairs of
word forms where \emph{i-}\isi{umlaut} has applied in one member but not the
other although the phonological conditions seem identical. Some examples
are given in \REF{ex:thrainsson:8}:

\ea \label{ex:thrainsson:8}
	\ea \label{ex:thrainsson:8a} 	\emph{háttur} `mode', \textsc{dat.sg}
		\emph{hætt-i}/*\emph{hátt-i}, \textsc{nom.pl} \emph{hætt-ir/*hátt-ir}
	\ex \label{ex:thrainsson:8b}	\emph{sáttur} `satisfied', \textsc{nom.sg.m.wk}
		\emph{*sætt-i/sátt-i}, \textsc{nom.pl.m} \emph{*sætt-ir/sátt-ir}
	\z
\z

\noindent In \REF{ex:thrainsson:8a} we see examples of the paradigmatic\is{paradigm} \isi{alternation} /á
\textasciitilde{} æ/ (phonetically {[}au{]} \textasciitilde{} {[}ai{]}
in the modern language, probably {[}aː{]} \textasciitilde{} {[}ɛː{]} in
Old \ili{Icelandic}) originally caused by \emph{i-}\isi{umlaut}. In the
\textsc{nom.sg} we have /á/ in the stem but in the \textsc{dat.sg} the
only acceptable form is \emph{hætti} and the ``non-umlauted'' version
\emph{*hátti} is unacceptable. Similarly, in the \textsc{nom.pl} only
\emph{hættir} is acceptable and *\emph{háttir} is not. At a first glance
we might think that an /i/ in the inflectional ending is still causing
this ``\isi{umlaut}'' but a comparison with the adjectival forms in \REF{ex:thrainsson:8b}
indicates that this cannot be the case. Here the only acceptable weak
\textsc{nom.sg.m} form is \emph{sátti} and not \emph{*sætti} and the
only \textsc{nom.pl.m} form is \emph{sáttir} and not *\emph{sættir}. So
the \emph{i-}\isi{umlaut} alternations\is{alternation} in Modern \ili{Icelandic} are clearly
morphologically\is{morphology} conditioned and not phonological anymore (see also
\citealt[93]{thrainsson2011} for further examples of this kind).

Second, recall that standard \isi{generative phonology}\is{phonology} formulations of
\emph{u-}\isi{umlaut} in \ili{Icelandic} of the kind illustrated in \REF{ex:thrainsson:3b} above state
explicitly that /u/ only triggers \isi{umlaut} of /a/ in the immediately
preceding syllable. This is illustrated by examples like the following:

\ea \label{ex:thrainsson:9}
	\ea \label{ex:thrainsson:9a}	\emph{bakki} `bank' \textsc{dat.pl}
		\emph{b\textbf{ö}kk-\textbf{u}m/*b\textbf{a}kk-\textbf{u}m}
	\ex \label{ex:thrainsson:9b}	\emph{akkeri} `anchor' \textsc{dat.pl}
		*\emph{\textbf{ö}kker-\textbf{u}m/\textbf{a}kker-\textbf{u}m}
	\z
\z

\noindent In \REF{ex:thrainsson:9a} the \emph{u-}\isi{umlaut} obligatorily applies to the root vowel /a/
in the immediately preceding syllable. In \REF{ex:thrainsson:9b}, on the other hand, the
/u/ in the (same) inflectional ending cannot apply to the root vowel /a/
because there is a syllable intervening. An interesting and much
discussed case, e.g. by Anderson in several of the publications cited
above, involves trisyllabic words with two instances of /a/ in the stem.
Consider the examples in \REF{ex:thrainsson:10}:

\ea \label{ex:thrainsson:10}
	\ea \label{ex:thrainsson:10a}	\emph{k\textbf{a}ll\textbf{a}} ʽcallʼ\\
		\textsc{1.sg.pst} \emph{kalla-ð-i}, \textsc{1.pl.pst}
		\emph{*k\textbf{a}ll\textbf{ö}-ð-um/k\textbf{ö}ll\textbf{u}-ð-um/*k\textbf{a}ll\textbf{u}-ð-		um/*k\textbf{ö}ll\textbf{a}-ð-\textbf{u}m}
	\ex \label{ex:thrainsson:10b}	\emph{b\textbf{a}n\textbf{a}n-i} `banana'\\
		\textsc{dat.pl} \emph{b\textbf{a}n\textbf{ö}n-um/b\textbf{ö}n\textbf{u}n-um/*b\textbf{a}n\textbf{u}n-um/*b\textbf{ö}n\textbf{a}n-\textbf{u}m}
	\z
\z

\noindent Consider first the conceivable \textsc{1.pl.pst} forms of the verb
\emph{kalla} `call'. Based on the formulation \REF{ex:thrainsson:3b} of the
\emph{u-}\isi{umlaut} rule, one might have expected the form \emph{*kallöðum},
where the /u/ in the inflectional ending triggers \emph{u}-\isi{umlaut} of the
/a/ in the preceding syllable. This is not an acceptable form, however.
The reason is that in forms of this sort a ``weakening''\is{weakening} of unstressed
/ö/ to /u/ is obligatory. This \isi{weakening} is found in in many words, e.g.
the \isi{plural} of the word \emph{hérað} `district', \isi{plural}
\emph{hér\textbf{ö}ð} or (preferred) \emph{hér\textbf{u}ð, meðal}
`medicine', \isi{plural} \emph{með\textbf{ö}l} or (preferred)
\emph{með\textbf{u}l}. It is not always obligatory but it seems that in
the \isi{past tense} of verbs of this sort it is. But once the (umlauted) /ö/
in \emph{*kall\textbf{ö}ðum} has been weakened\is{weakening} to /u/ it obligatorily
triggers \emph{u-}\isi{umlaut} of the preceding /a/ so
\emph{k\textbf{ö}ll\textbf{u}ðum} is acceptable but
\emph{*k\textbf{a}ll\textbf{u}ðum} is not. Finally, the form
\emph{*k\textbf{ö}ll\textbf{a}ðum} is not acceptable either, since there
\emph{u-}\isi{umlaut} would be applied across an intervening syllable, which
is not possible, as we have seen (cf. \ref{ex:thrainsson:9b}). The \emph{u-}\isi{umlaut} works
in a similar fashion in the word \emph{banani}, except that here the
\isi{weakening} of the second (and unstressed) syllable from /ö/ to /u/ is not
obligatory. Hence \emph{ban\textbf{ö}n\textbf{u}m} is an acceptable
form, with the /u/ in the final syllable triggering \emph{u-}\isi{umlaut} of
the preceding /a/ to /ö/. But if this /ö/ is further weakened\is{weakening} to /u/,
then \emph{u-}\isi{umlaut} of the first /a/ is obligatory and
\emph{b\textbf{ö}n\textbf{u}num} is an acceptable form but
*\emph{b\textbf{a}n\textbf{u}num} is not.\footnote{It is sometimes
  claimed that \emph{b\textbf{ö}n\textbf{ö}n\textbf{u}m} is also an
  acceptable form for some speakers. If this is so, it is possible that
  the /ö/ in the next-to-last syllable triggers \emph{u-}\isi{umlaut} (i.e.
  \emph{ö-}\isi{umlaut}!) of the /a/ in the first syllable. That would simply
  mean that the feature {[}−low{]} in the definition of the environment
  of the \emph{u-}\isi{umlaut} in \REF{ex:thrainsson:3b} would be omitted. But since there are
  no derivational (nor inflectional) morphemes containing an underlying
  /ö/ (i.e. an /ö/ that cannot have been derived by \emph{u-}\isi{umlaut}),
  this proposal cannot be tested independently of the iterative rule
  application, as pointed out by a reviewer.} As predicted by the
formulation of the \emph{u-}\isi{umlaut} rule in \REF{ex:thrainsson:3b} a form like
*\emph{b\textbf{ö}nanum} is unacceptable because there the
\emph{u-}\isi{umlaut} would have applied across an intervening syllable. Facts
of this sort have been interpreted as showing that \emph{u-}\isi{umlaut} in
Modern \ili{Icelandic} is of a phonological\is{phonology} nature since it depends on
syllabic structure (no syllables can intervene between the \isi{umlaut}
trigger and the target) and it can be applied iteratively (a /u/ which
itself is derived by \emph{u-}\isi{umlaut} and subsequent independently needed
\isi{weakening} can trigger \emph{u-}\isi{umlaut}).

\section{The conditions for {u}-\isi{umlaut} in Modern
Faroese}

\subsection{{u-}\isi{umlaut} and the Modern \ili{Faroese} vowel system}

Modern \ili{Faroese} has preserved some \emph{u-}umlaut-like vowel
alternations.\is{alternation} A couple of examples are given in \REF{ex:thrainsson:11} (see also
\citealt[78, 100, passim]{thrainsson2012}):

\begin{exe}
\ex \label{ex:thrainsson:11} 	\emph{dag-ur} `day', \textsc{dat.pl}
	\emph{d\textbf{ø}g-\textbf{u}m}; 
	\emph{spak-ur} ʽcalmʼ, \textsc{nom.pl.wk}
	\emph{sp\textbf{ø}k-\textbf{u}}
\z

\noindent At first glance, these alternations\is{alternation} seem very similar to the \ili{Icelandic}
ones described in the preceding sections. But while the \emph{u-}\isi{umlaut}
alternations\is{alternation} are arguably phonologically\is{phonology} (or phonetically) natural in
Modern \ili{Icelandic} (see the diagram in \ref{ex:thrainsson:2b} and the formulation in \ref{ex:thrainsson:3b}),
it will be claimed below that this is not the case in \ili{Faroese}. To
demonstrate this, it is necessary to look closely at the \ili{Faroese} vowel
system. Consider first the following schematic representation of \ili{Faroese}
\emph{u-}\isi{umlaut} of the type just illustrated, where the alleged \isi{umlaut}
trigger is encircled (cf. \citealt[98]{thrainsson2011}, \citealt[33]{thrainsson2012}, compare \citealt[248--250]{arnason2011}):\footnote{Vowel length
  is predictable in \ili{Faroese}, as it is in \ili{Icelandic}: Vowels are long in
  stressed open syllables, otherwise short. As illustrated in the
  brackets in \REF{ex:thrainsson:12}, there is often a considerable difference in the
  phonetic realization of the long and short variants. This will be
  illustrated below. --- It should be noted that \citet[76]{arnason2011}
  assumes a different analysis of \ili{Faroese} monophthongs.}

\ea \label{ex:thrainsson:12} \emph{u-}\isi{umlaut} in Modern \ili{Faroese} and the system of monophthongs:
		\begin{tabular}[t]{ccccccc}
		&& \multicolumn{2}{c}{{[}-back{]}} &  & \multicolumn{2}{c}{{[}+back{]}} \\
		&& {[}-round{]}   	& {[}+round{]}   &  & {[}-round{]}   & {[}+round{]}   \\
		{[}+high{]}  &   &  i        &      y          &  &                &    \hspace*{22pt}\textcircled{u} [uː/ʊ]            \\
		&  &   e    &   \tikzmarkfullnamed{oe}{ø} {[}øː/œ{]} &  &                &            o    \\
		{[}+low{]}  	&  &   \hspace*{23pt}\tikzmarkfullnamed{ae}{æ} {[}ɛaː/a{]}   &        &  & a	    &          ɔ \\
	\end{tabular}\begin{tikzpicture}[remember picture,overlay]
\draw[thick,shorten >=3pt,->,>=stealth] ([yshift=2pt]ae.north) -- (oe.west);
\end{tikzpicture}
	\z 


Something like \REF{ex:thrainsson:13} would seem to be a possible formulation of a process
of this kind in traditional \isi{generative phonology}\is{phonology} terms (compare \ref{ex:thrainsson:3b}):

\ea \label{ex:thrainsson:13} Possible phonological formulation of \emph{u-}\isi{umlaut} in Modern
\ili{Faroese}:\\
	\phonr{/æ/}{\phonfeat{\feat{\textbf{+round}}\\\feat{\textbf{-low}}}}{C\sub{0}V\sub{\normalsize\phonfeat{\feat{\textbf{+round}}\\\feat{+back}\\\feat{\textbf{-low}}}}}
\z

Presented this way, \emph{u-}\isi{umlaut} in \ili{Faroese} looks like a fairly
natural \isi{assimilation} rule at a first glance.\footnote{A reviewer
  suggests, however, that a process changing \isi{rounding} and height as
  formulated for \ili{Faroese} in \REF{ex:thrainsson:13}, might be less natural from the point
  of view of acoustic phonetics than a process changing \isi{rounding} and
  \isi{backness} the way the \emph{u-}\isi{umlaut} rule in Modern \ili{Icelandic} does
  according to \REF{ex:thrainsson:3}: The former affects both F1 (for the height
  difference) and F2 (for rounding)\is{rounding} whereas the latter affects F2 in
  opposite directions (raising it for \isi{fronting} but lowering it for
  rounding).\is{rounding} Thus the Modern \ili{Icelandic} \emph{u-}\isi{umlaut} rule would
  ``generate more similar input-output mappings'', which may be
  preferred to less similar ones.} But the facts are somewhat more
complicated.

First, the alleged trigger /u/ is not too stable in Modern \ili{Faroese}. The
reason is that unstressed /i,u/ are not distinguished in all \ili{Faroese}
dialects. In some dialects they merge into an {[}ɪ{]}-like sound, in
others into an {[}ʊ{]}-like sound but some dialects distinguish them as
{[}ɪ{]} and {[}ʊ{]} (see \citealt[27]{thrainsson2012}, and references
cited there). This situation has clearly added to the phonological
\isi{opacity} of \emph{u-}\isi{umlaut} alternations\is{alternation} for speakers acquiring \ili{Faroese}.

Second, the target of the \emph{u-}\isi{umlaut} in \ili{Faroese} is arguably a
``moving'' one. As indicated in \REF{ex:thrainsson:12}, the \isi{umlaut} affects the \isi{phoneme}
represented there as /æ/. As the \isi{orthography} suggests, it is a
descendant of \ili{Old Norse} /a/ in words like \emph{d\textbf{a}gur},
\emph{sp\textbf{a}kur} (see \ref{ex:thrainsson:11}). It is realized phonetically as
{[}ɛaː{]} when long and {[}a{]} when short, as shown in \REF{ex:thrainsson:12}, cf.
\emph{spakur} {[}spɛaː\textsuperscript{h}kʊɹ{]} `calm', \textsc{sg.n}
\emph{spakt} {[}spakt{]} (see \citealt[34]{thrainsson2012} passim). But
in the history of \ili{Faroese} \ili{Old Norse} /a/ {[}a{]} and /æ/ {[}ɛː{]} merged
so the \isi{phoneme} represented here as /æ/ can also be a descendant of Old
Norse /æ/ and then it is represented in the spelling as ʽæʼ, cf.
\emph{trælur} {[}t\textsuperscript{h}ɹɛaːlʊɹ{]} `slave', \emph{æða}
{[}ɛaːva{]} `eider duck'. Words written with ʽæʼ show the same
\isi{alternation} between long {[}ɛaː{]} and short {[}a{]} as demonstrated for
\emph{spakur} and \emph{spakt} above (e.g. \emph{vænur} {[}vɛaːnʊɹ{]}
`beautiful' \textsc{sg.m} vs. \emph{vænt} {[}van̥t{]}, cf. Thráinsson et
al. 2012, p. 34). Yet it seems that \emph{u-}\isi{umlaut} is rarely if ever
found in the ʽæʼ-words. Thus the \textsc{dat.pl} of \emph{trælur} is
\emph{tr\textbf{æ}l\textbf{u}m} and not \emph{*tr\textbf{ø}l\textbf{u}m}
(compare \textsc{dat.pl} \emph{d\textbf{ø}l\textbf{u}m} of \emph{dalur}
`valley') and although the words \emph{æða} `eider duck' and \emph{aða}
`(big) mussel' sound the same, i.e. as {[}ɛaːva{]}, the \textsc{dat.pl}
of the former has to be \emph{æðum} {[}ɛaːvʊn{]} and
\emph{\textbf{ø}vum} {[}øːvʊn{]} can only be \textsc{dat.pl} of
\emph{aða}.\footnote{A reviewer points out that the fact that
  \emph{u-}\isi{umlaut} does not apply do `æ'-words in \ili{Faroese} suggests that
  ``\emph{u}-\isi{umlaut} had already taken on a morphological character
  before /a/ and /æ/ merged.'' But since there are no written records of
  \ili{Faroese} from 1400‒1800, the historical development of the language is
  very murky.}

To further complicate matters, the development of \ili{Old Norse} /a/ in
\ili{Faroese} has left ``room'' for a ``regular /a/'' in the \ili{Faroese} vowel
system, as shown in the diagram in \REF{ex:thrainsson:12}. It occurs in loanwords and is
realized as {[}aː{]} when long and {[}a{]} when short, cf. \emph{Japan}
{[}ˈjaː\textsuperscript{h}pan{]}, \emph{japanskur}
{[}jaˈp\textsuperscript{h}anskʊɹ{]} `Japanese'.\footnote{In the noun
  \emph{Japan} the stress falls on the first syllable, in the adjective
  \emph{japanskur} it falls on the second one as indicated. Hence the
  quantity \isi{alternation} in the first vowel.} It does not seem that this
vowel ever undergoes \emph{u-}\isi{umlaut} in \ili{Faroese} (for further discussion
see \S4).

Finally, there is no \emph{u}-\isi{epenthesis} in Modern \ili{Faroese} to ``explain
away'' apparent exceptions to \emph{u-}\isi{umlaut} as will be shown in the
next section.

\subsection{The lack of {u-}\isi{epenthesis} in Modern Faroese}

Now recall that the most obvious \isi{surface exception}\is{exception} to \emph{u-}\isi{umlaut} in
Modern \ili{Icelandic} is due to the \emph{u-}\isi{epenthesis} described above. This
rule creates -\emph{ur-}endings that do not trigger \emph{u-}\isi{umlaut}. It
was argued that this \isi{epenthesis} rule is still productive in \ili{Icelandic},
witness the fact that it only applies in phonologically definable
environments. Hence there is a clear distributional difference between
-\emph{ur}-endings produced by the \isi{epenthesis} rule (and not triggering
\emph{u-}\isi{umlaut}) and -\emph{ur-}endings where the /u/ is a part of the
underlying form (and triggers \isi{umlaut}). This is not the case in \ili{Faroese},
where the ending -\emph{ur} as a marker of the \textsc{nom.sg} of strong
masculine nouns and adjectives, with a /u/ that was historically
inserted by \isi{epenthesis}, has been generalized to all environments. Hence
it has become distributionally indistinguishable from other
-\emph{ur-}endings. Table~\ref{tab:strong} compares the phonological realization of the
\textsc{nom.sg.m} \#-r\#-ending in Modern \ili{Icelandic} to its Modern
\ili{Faroese} counterpart (see also \citealt[100]{thrainsson2011}):

\begin{table}
	\begin{tabularx}{\textwidth}{@{}l>{\raggedright}p{1in}>{\raggedright}X>{\raggedright}X@{}}
		\lsptoprule
& type of stem & phonological realization of a strong \textsc{nom.sg.m} ending in Modern \ili{Icelandic} & phonological
realization of a strong \textsc{nom.sg.m} ending in Modern
\ili{Faroese}\tabularnewline
\midrule
a. & ending in a vowel & -\emph{r} \newline (\emph{mó-r} `peat', \emph{há-r}
`high') & \emph{-\textbf{ur}} \newline (\emph{mó-ur/mógv-ur} `peat', \emph{há-ur}
ʽhighʼ)\tabularnewline
b. & ending in a sin- gle consonant (but see c) & \emph{\textbf{-ur}} \newline
(\emph{dal-ur} `valley', \emph{lat-ur} ʽlazyʼ) & \emph{\textbf{-ur}} \newline
(\emph{dal-ur} `valley', \emph{lat-ur} ʽlazyʼ)\tabularnewline
c. & ending in a high vowel + /l,n/ & \isi{assimilation} \newline (\emph{stól-l} `chair',
\emph{fín-n} `fine') & \textbf{-\emph{ur}} \newline (\emph{stól-ur} `chair',
\emph{fín-ur} `fine')\tabularnewline
d. & ending in /s, r/ or consonant clusters like /gl, gn/ & \isi{deletion} \newline
(\emph{ís} `ice', \emph{laus} `loose', \emph{foss} `waterfall',
\emph{stór} `big', \emph{fugl} `bird', \emph{vagn} `wagon') &
\textbf{-\emph{ur}} \newline (\emph{ís-ur} `ice', \emph{leys-ur} `loose',
\emph{foss-ur} `waterfall', \emph{stór-ur} `big', \emph{fugl-ur} `bird',
\emph{vagn-ur} `wagon')\tabularnewline
	\lspbottomrule
	\end{tabularx}
\caption{Phonological realization of a strong \textsc{nom.sg.m}-ending in Modern \ili{Icelandic} and Modern \ili{Faroese}.}
\label{tab:strong}
\end{table}

This has clearly made the \emph{u-}\isi{umlaut} rule in \ili{Faroese} more \isi{opaque}
since now the non-umlauting and umlauting \emph{ur-}endings occur in the
same phonological environments. It seems very likely that this has
contributed to the death of \emph{u-}\isi{umlaut} as a phonological process in
\ili{Faroese}.

\section{Testing the predictions}\label{testing-the-predictions}

In the preceding discussions I have described /a
\textasciitilde{} ö/ alternations\is{alternation} in Modern \ili{Icelandic} and their Modern
\ili{Faroese} counterparts. I have argued that the \ili{Icelandic} alternations\is{alternation} are
still governed by a synchronic phonological\is{phonology} process. Although these
alternations\is{alternation} are still found in Modern \ili{Faroese}, I have argued that they
cannot be governed by a phonological rule. Instead they must be
morphologically\is{morphology} governed or analogical.\is{analogical} This analysis makes several
testable predictions (see \citealt[100--102]{thrainsson2011}).

First, we do not a priori expect phonologically
conditioned alternations\is{alternation} to be restricted to particular morphological
categories whereas morphologically conditioned alternations \is{alternation}obviously
are, by definition. As we have already seen, the \ili{Icelandic}
\emph{u-}\isi{umlaut} occurs in the inflectional\is{inflection} paradigms\is{paradigm} of nouns,
adjectives and verbs and in various grammatical categories (cases,\is{case}
numbers, tenses,\is{tense} persons \ldots{}). Its \ili{Faroese} counterpart behaves
differently. It is found in the inflectional\is{inflection} paradigms\is{paradigm} of nouns and
adjectives, as we have seen (cf. \ref{ex:thrainsson:11}), but not in the \isi{past tense} forms
of verbs, where it would be expected on phonological grounds. Thus we
have \emph{við kölluðum} in \ili{Icelandic} vs. \emph{vit kallaðu} in \ili{Faroese}
for \textsc{1.pl.pst} `we called', and \emph{við frömdum} vs. \emph{vit
framdu} in \ili{Faroese} for \textsc{1.pl.pst} `we did,
made'.

Second, a phonological rule should not allow \isi{analogical} extensions to
forms that do not fit its structural conditions. Such extensions are not
found for \ili{Icelandic} \emph{u-}\isi{umlaut} but in \ili{Faroese} they are very common.
Thus the /ø/ of the oblique cases\is{case} \emph{s\textbf{ø}g\textbf{u}} `saga'
has been analogically\is{analogical} extended to the \textsc{nom.sg} form
\emph{s\textbf{ø}ga} and many other words of a similar type. The
corresponding form *\emph{s\textbf{ö}ga} is unacceptable in
\ili{Icelandic}.\footnote{As a reviewer reminds me, the \ili{Icelandic} \isi{neologism}
  for \emph{computer} is interesting in this connection. It was supposed
  to be \emph{tölva} (related to the word \emph{tala} `number' --- this
  was when computers were mainly used for computing) in \textsc{nom.sg},
  oblique singular cases\is{case} \emph{tölvu}. In Proto-Nordic time /v/ could
  trigger \isi{umlaut} of /a/ to /ǫ/ so we have \ili{Old Norse} words like
  \emph{vǫlva} `sooth-sayer, witch'. But since /v/ is not a trigger of
  \isi{umlaut} in Modern \ili{Icelandic} (witness loanwords like \emph{salvi}
  `salve, cream'), speakers tend to use the form \emph{talva} for
  \textsc{nom.sg}, thus in a way undoing the underlying /ö/ in the
  nominative as if they are ``assuming'' that the /ö/ in the oblique
  cases\is{case} is derived by a synchronic \emph{u-}\isi{umlaut} from /a/, as in words
  like \emph{saga} `saga', oblique \emph{sögu} (for some discussion see
\citealt{thrainsson1982}).}

Third, a phonologically conditioned rule should apply whenever its
structural conditions are met. Thus we would not expect to find
inflectional forms in \ili{Icelandic} where \emph{u-}\isi{umlaut} fails to apply in
an appropriate environment. Such examples are very common in \ili{Faroese}, on
the other hand. Thus the \textsc{dat.pl} of the noun \emph{rakstur}
`shave' in \ili{Faroese} is \emph{r\textbf{a}kstr\textbf{u}m} and not the
expected \emph{*r\textbf{ø}kstr\textbf{u}m}, the \textsc{dat.pl} of
\emph{spakur} `calm' can either be \emph{sp\textbf{ø}k\textbf{u}m} or
\emph{sp\textbf{a}k\textbf{u}m}, etc. (see \citealt[79, 100, passim]{thrainsson2012}). Corresponding unumlauted forms are unacceptable in
\ili{Icelandic}.

Fourth, there is evidence for ``iterative'' application of
\emph{u-}\isi{umlaut} in \ili{Icelandic}, with one application of the
\emph{u-}\isi{umlaut} rule feeding another. This was discussed above (second
part of \S2.2) in connection with forms like \textsc{1.pl.pst}
\emph{k\textbf{ö}ll\textbf{u}ð\textbf{u}m} `(we) called' and
\textsc{dat.pl} \emph{b\textbf{ö}n\textbf{u}n\textbf{u}m} `bananas'. No
such evidence is found in \ili{Faroese}, where the corresponding forms are
\emph{k\textbf{a}ll\textbf{a}ðum} and
\emph{b\textbf{a}n\textbf{a}num}.\footnote{The latter form may be
  related to the fact that \emph{banan} `banana' is a \isi{loanword} and
  contains the vowel /a/ (long variant {[}aː{]}) and not /æ/, cf. the
  discussion in \S3.1. See also the next paragraph.}

\newpage 
Finally, \ili{Icelandic} \emph{u-}\isi{umlaut} is so productive that it is naturally
applied in loanwords, as we have seen. This is not so in \ili{Faroese}. Thus
the word \emph{app} (for a small program) has been adopted into both
languages. In \ili{Icelandic} the \textsc{dat.pl} has to be
\emph{\textbf{ö}pp\textbf{u}m} whereas the natural form is
\emph{\textbf{a}pp\textbf{u}m} in \ili{Faroese}. This can easily be verified
by searching for the word combinations \emph{með öppum} and \emph{við
appum} `with apps' on Google. For the first variant one finds a number
of \ili{Icelandic} hits, for the second \ili{Faroese} ones.

The general conclusion, then, is that \emph{u-}\isi{umlaut} in Modern
\ili{Icelandic} has a number of properties that are to be expected if it is a
phonological process but none of the properties one might expect of
morphologically conditioned or \isi{analogical} alternations.\is{alternation}


\section{Concluding remarks}\label{concluding-remarksTh}

While it has often been argued that \isi{phonology} need not be ``natural''
(see e.g. \citealt{anderson1981}), there must obviously be limits to the
``unnaturalness'' and \isi{opacity} of phonological processes. Once they
become too unnatural and \isi{opaque}, they can no longer be acquired as such
and the phonological alternations\is{alternation} originally created by them will be
relegated to morphology.\is{morphology} Then their \isi{productivity} will be limited and it
will at best survive to some extent by \isi{analogy}, but \isi{analogical} processes
are known to be irregular and unpredictable. The fate of \emph{i-}\isi{umlaut}
in \ili{Icelandic} is a case in point, as described above (see the discussion
of the examples in \ref{ex:thrainsson:8}). But whereas we do not have detailed information
about how \emph{i-}\isi{umlaut} died as a phonological process, comparison of
the development of \emph{u-}\isi{umlaut} in \ili{Icelandic} and \ili{Faroese} sheds an
interesting light on how a phonological\is{phonology} rule can die and how it can
survive despite changing conditions.



\section*{Acknowledgements}
Many thanks to Steve Anderson for introducing me to the wonders of synchronic \emph{u-}\isi{umlaut} way back when. Thanks are also due to the editors and to two anonymous reviewers for help, useful comments, suggestions and corrections.

\printbibliography[heading=subbibliography,notkeyword=this]


\end{document}
