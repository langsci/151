\documentclass[output=paper,
modfonts
]{LSP/langsci}


%\input{localpackages.tex}
%\input{localcommands.tex}

% \newcolumntype{L}{>{\raggedright\arraybackslash}X}

\title{Word-based Items-and-processes (WoBIP): Evidence from Hebrew morphology}
\author{Outi Bat-El
	\affiliation{Tel-Aviv University}}

\abstract{In his seminal book \emph{A-Morphous Morphology,} Anderson provides ample evidence supporting the item-and-process approach to morphology, whereby relations between words, and thus the derivation of one word from another is expressed in terms of processes. Although Anderson excluded Semitic languages from the paradigm, I argue in this paper for the advantage of item-and-process in the analysis of Modern Hebrew word relations. Under this approach, the word/stem is the base, and the putative consonant root is just a residue of phonological elements, which are lexically prominent as are consonants in non-Semitic languages. The empirical basis of the arguments is drawn from natural and experimental data of adult Hebrew as well as child Hebrew.}
\begin{document}
	\maketitle
	
	
	\section{Introduction}\label{introduction}
	
	``Items vs.\ processes in morphology'' is the title of section 3.4 in
\citeauthor{anderson1992}'s (\citeyear{anderson1992}) seminal book \emph{A-Morphous Morphology}. In this
section, Anderson compares two models of morphology --
item-and-arrangement and item-and-process (attributed to \citealt{Hockett1954})
-- and argues in favor of the latter. Taking apophony (or ablaut; e.g.
\emph{sing} -- \emph{sang}) as one of the many problems encountered with
the item-and-arrangement model, Anderson claims that ``what presents
\{\textsc{past}\} in \emph{sang} is \ldots{} the \emph{relation} between
\emph{sang} and \emph{sing}, expressed as the \textbf{process} by which
one is formed from the other'' (\citealt[62]{anderson1992}; emphasis original).
The process in this case is replacement (or stem modification); ``the
\textsc{past} form of \emph{sing} is formed by replacing /ɪ/ with /æ/.'' Crucially, /æ/ is not the morpheme designating \textsc{past},
and \emph{sang} is not derived by combining bound morphemes, i.e.
\emph{s-ŋ} and \emph{-æ-}.

The immediately following section 3.5 in Anderson's book is titled
``Word-based vs.\ morpheme-based morphology''. The issues addressed in
these two sections are always considered together, since one is
contingent upon the other. A root-based morphology is usually analyzed
within the item-and-arrangement model. However, if morphology is
word-based, the debate between item-and-arrangement and item-and-process
still holds (see §2). This debate is particularly heated in the study of
Semitic morphology, where a consonantal root has been claimed to be the
core morphological unit in the word.

Paradigms like \emph{sing} -- \emph{sang} are relatively rare in
English, but abundant in Semitic languages, such as Hebrew, where the
relation between words is often expressed with apophony; e.g.\ \emph{χam}
-- \emph{χom} `hot -- heat', \emph{limed} -- \emph{limud} `to teach --
learning', \emph{∫uman} -- \emph{∫émen} `fat -- oil', \emph{gadol}
--\emph{gódel} `big -- size' (stress is final unless otherwise
specified). Since item-and-arrangement has been the traditional approach
to Semitic morphology, and has been supported by traditional Semiticists
(see, however, §6) and generative linguists, Anderson contemplates
whether \emph{sing} -- \emph{sang} can be analyzed as a root \emph{s-ŋ}
plus the markers /ɪ/ `\textsc{present}' and /æ/ `\textsc{past}'. He,
however, rejects this analysis due to the absence of ``substantive
evidence in its favor'' \citep[62]{anderson1992}, and adds in parentheses ``as
there clearly is \ldots{} for something like McCarthy's analysis of
Arabic and other Semitic languages'' (ibid). That is, Anderson accepts
the common view that item-and-arrangement is the appropriate model for
Semitic morphology.

While I support Anderson's approach to morphology, I do not agree with
the exclusion of Semitic languages from the paradigm. On the basis of
data from Modern Hebrew, I provide in this paper evidence supporting the
word-based item-and-process (WoBIP) model for Semitic morphology. That
is, English is not like Hebrew, but rather Hebrew is like English.

In the context of Semitic morphology, I outline in the following §2 the
possible morphological models that can be derived from the four
different approaches: word-based, morpheme-based, item-and-process, and
item-and-arrangement. Then, in §3-§5 I provide supporting evidence for
the word-based item-and-process model, but due to space limitation, I do
not dwell on arguments against competing models. Each piece of evidence
supports only part of the model, but together we get a well-motivated
model of morphology. Given Anderson's commitment to the history of
linguistics (see, in particular, \citealt{anderson1985a}), I devote §6 to two
principal Semiticists from the 19\textsuperscript{th} century, whose
grammar books support the word-based item-and-process model. Concluding
remarks are given in §7.

\section{Models of morphology}\label{models-of-morphology}

Research in morphology often concentrates on two questions: What is
listed in the lexicon and how are words derived? Each of these questions
is associated with competing approaches. The \emph{what}-question is
related to the root-based vs.\ word-based debate, which is of particular
interest in the study of Semitic morphology, where the root is always
bound. The \emph{how}-question is related to the item-and-process vs.\ item-and-arrangement debate. Together, they give rise to three models of
morphology, shown in \figref{fig:models}: root-based item-and-arrangement, word-based
item-and-arrangement, and word-based item-and-process.

\begin{figure}
\begin{tikzpicture}[%
box/.style={text width=3.1cm,minimum height=2cm},%
question/.style={text width=1.3cm,minimum height=2cm},%
framed/.style={draw,thick,rounded corners},%
shaded/.style={fill=gray, fill opacity = 0.3, text opacity = 1},%
]
		\matrix [matrix of nodes,ampersand replacement=\&,every node/.style={anchor=center},column sep=5pt, row sep=5pt]
		{
			|[question]| \emph{What is listed in the lexicon?} \& |[box,framed] (roots)| Roots and configurations \& |[box] (a)|  \& |[box] (b)| \\
			|[question]| \emph{How are words derived?} \& |[box,framed] (c)| Association of roots and configurations \& |[box,framed] (d)| \begin{tabularx}{3.1cm}[t]{l@{\hspace{3pt}}L@{}} i. & Extraction \\ ii. & Association of roots and con- figurations \\ \end{tabularx} \& |[box,framed,shaded] (imposing)| \textbf{Imposing configurations on words} \\
		};
		\draw[shaded,framed] (a.north west) rectangle (b.south east) node[pos=.5] {\textbf{Words and configurations}};
		\node[fit={(a) (b)}, inner sep=0pt] (words) {};
		\node[above] at (roots.north) {\textbf{\emph{Root-based}}};
		\node[above] at (words.north) {\textbf{\emph{Word-based}}};
		\node[below] at (imposing.south) {\textbf{\emph{Item \& Process}}};
		\node[fit={(c) (d)}, inner sep=0pt] (arrangement) {};
		\node[below] at (arrangement.south) {\textbf{\emph{Item \& Arrangement}}};
		\end{tikzpicture}
		\caption{Models of morphology.}
		\label{fig:models}
	\end{figure}
	
	%\ea Models of morphology \\
	%
	%\scalebox{0.9}{
	%\begin{tikzpicture}
	%
	%\node[text width=2cm] at (-1, 2.8) {a.};
	%\node[text width=3cm] at (0.0, 2.6) {\textit{What is listed in the lexicon?}};
	%\node[text width=2cm] at (-1,1.5) {b.};
	%\node[text width=3cm] at (0.0, 1.3) {\textit{How are words derived?}};
	%\node[text width=3cm] at (3.3, 3.3) {\textit{\textbf{Root-based}}};
	%\node[text width=3cm] at (7.5, 3.3) {\textit{\textbf{Word-based}}};
	%\draw (1.3,3) rectangle (4.3, 2) node[pos=.5, align=left] {Roots and \\ configurations};
	%\draw (1.3,1.8) rectangle (4.3, -0.2) node[pos=.5, align=left] {Association of \\ roots and \\ configurations};
	%\draw [fill=gray, fill opacity = 0.3, text opacity = 1] (4.5,3) rectangle (10.0, 2) node[pos=.5, align=left] {\textbf{Words and configurations}};
	%\draw (4.5,1.8) rectangle (7.3, -0.2) node[pos=.5, align=left] {\begin{tabularx}{3cm}[t]{l@{\hspace{2pt}}X}i. & Extraction\\ii. & Association of roots and configurations\\\end{tabularx}};
	%\draw [fill=gray, fill opacity = 0.3, text opacity = 1] (7.5,1.8) rectangle (10.0, -0.2) node[pos=.5, align=left] {\textbf{Imposing} \\ \textbf{configurations} \\ \textbf{on words}};
	%\node at (4.2, -0.5) {\textit{\textbf{Item \& Arrangement}}};
	%\node at (8.8, -0.5) {\textit{\textbf{Item \& Process}}};
	%\end{tikzpicture}
	%}
	%\z 
	
	In this paper I support the word-based item-and-process (WoBIP) model.
	Before displaying the supporting arguments, a short review of the three
	models is given in the three ensuing subsections.\footnote{I do not consider here the pluralistic approaches, whereby some words
  are derived from roots and others from words \citep{McCarthy1979a,arad2005a,berman2012a}, because all phenomena can be accounted for within
  the WoBIP model reviewed in §2.2.}

\subsection{Root-based item-and-arrangement}\label{root-based-item-and-arrangement}

In the context of Semitic morphology, the root-based morphology teams up
with item-and-arrangement. According to the traditional approach, the
root in Hebrew and other Semitic languages consists of 2--4 consonants (3
in most cases) and is combined with a configuration \citep{batel2011a}, where
the latter, traditionally termed \emph{mishkal} for nouns and
\emph{binyan} for verbs, is a shorthand for the grouping of prosodic
structure, vocalic pattern, and affixes (if any).\footnote{Each of these
  elements (i.e.\ the prosodic structure, the vocalic pattern, and the
  affix) is independent \citep{McCarthy1979a,McCarthy1981b}, but here reference to the
  configuration suffices. In this context, we should note that the term
  `Semitic morphology' refers to morphology that employs configurations
  consisting of at least a vocalic pattern and prosodic structure. Of
  course, Hebrew and other Semitic languages employ the more
  conventional affixal morphology, but this type of morphology does not
  concern us here.} In a configuration like \emph{miCCéCet}, for
example, the C-slots host the root consonants, the specified consonants
(\emph{m} and \emph{t}) are affixes, and the vowels are part of the
vocalic pattern (e.g.\ \emph{mivʁé∫et} `brush', \emph{mizχélet} `sleigh',
\emph{mi∫méʁet} `guard'). \cref{tab:rootsconfig} shows examples of words sharing a root and examples of words sharing a configuration.
	
	\begin{table}
		\begin{tabular}[t]{llll}
			\lsptoprule
			\multicolumn{4}{c}{Words sharing the root \form{√sdʁ}} \\
			\midrule
			CiCeC & √sdʁ & sideʁ & `to arrange' \\
			CaCiC & & sadiʁ & `regular'\\
			CiCuC & & siduʁ & `arrangement'\footnotemark{}\\ 
			CeCeC & & sédeʁ & `order' \\ 
			meCuCaC & & mesudaʁ & `arranged' \\ 
			miCCaC & & misdaʁ & `military parade' \\ 
			miCCeCon & & misdeʁon & `corridor' \\
			CaCCan & & sadʁan & `usher' \\ 
			CiCCa & & sidʁa & `series' \\
			\midrule
			\multicolumn{4}{c}{Words sharing a configuration\footnotemark{}}\\
			\midrule
			CaCCan & √sdʁ & sadʁan & `usher'  \\
			& √ʁkd & ʁakdan & `dancer' \\
			& √btl   & batlan   & `lazy' \\
			CeCeC  	& √bgd &  béged  & `garment' \\
			& √jld   & jéled	 & `boy'  \\
			& √dgl &  dégel & `flag'  \\
			CiCeC & √χps & χipes & `to search'  \\
			& √btl & bitel & `to cancel'  \\
			& √χbʁ & χibeʁ & `to connect' \\	
			\lspbottomrule
		\end{tabular}
		\caption{Roots and configurations.}
		\label{tab:rootsconfig}
	\end{table}
	\addtocounter{footnote}{-1}
	\footnotetext{Some words get additional,
		idiomatic meaning. For example, \emph{siduʁ} carries the general
		meaning `arrangement' and the more specific one referring to `a prayer
		book'. Similarly, \emph{sédeʁ} carries the general meaning `order' and
		the more specific one referring to `Passover ceremony' (\emph{sédeʁ pésaχ}).}
\stepcounter{footnote}
\footnotetext{As in other studies, the 
	exponent of the 3\textsuperscript{rd} person masculine past serves as the citation 
	form because it is structurally neutral, i.e.\ it has no affixes. The gloss is still in the
	infinitive, implying reference to the lexeme.}


	%\noindent
	The classical studies seem to suggest a lexical representation
consisting of morphemes, as can be inferred from \citeauthor{moscati1980a}'s (\citeyear[71]{moscati1980a}) account
of the Semitic morphological system: ``The Semitic languages present a
system of consonantal roots (mostly triconsonantal), each of which is
associated with a basic meaning range common to all members of that
root: e.g.\ \emph{ktb} `to write', \emph{qbr} `to bury', \emph{qrb} `to
approach', etc. These roots (root morphemes) constitute a fundamental
category of lexical morphemes.'' If roots are listed, so are the
	configurations, and word formation thus consists of associating roots
	and configurations, i.e.\ item-and-arrangement.
	
As \citet[139]{hoberman2006a} notes, ``students of Semitic languages find the
concept of the root so convenient and useful that one finds it hard to
think about Semitic morphology without it.'' However,
	researchers vary with respect to the definition of the term `root'.
\citet[202]{lipinski1997a}, for example, assumes that ``Semitic roots are
\emph{continuous} morphemes which are instrumental in derivation but
subject to vocalic and consonantal change \ldots{} based on continuous
or discontinuous `pattern morphemes''' (emphasis original). The
`continuous morphemes', which Lipiński calls roots, are not the
traditional consonantal roots, but rather stems consisting of vowels and
consonants; the `pattern morphemes' are what I call configurations.
\citet{aronoff2007a} drains its original morphological
(structural and semantic) properties from the root, claiming that it does not have to
be linked to meaning and its phonology can be vague. Yet another use of
the term `root' is found in \citet{Frost1997} with reference to an
orthographic root, which as the results of their experiments suggest,
has no semantic properties.
	
	\subsection{Word-based item-and-process (WoBIP)}\label{word-based-item-and-process-wobip}
	
	Within this approach, the word or the stem is the core element to which
	all the required processes apply \citep{aronoff1976}. As a core element, it
	does not have an internal morphological structure. The processes are
	\emph{operations} \citep[72]{anderson1992} that \emph{modify} the basic form
\citep[97]{Matthews1974}. Indeed, the most common process in languages is the
one deriving \emph{bats} from \emph{bat}, i.e.\ affixation, but there are
other processes, such as apophony, which derives \emph{teeth} from
\emph{tooth}.
	
Also in the context of Semitic morphology, the input is a word/stem to
which several processes apply (see §3.1.2 for word vs.\ stem as the
base). The processes vary according to the goal, and the goal is that
the output fits into a configuration. Such a goal- or output-oriented
phenomenon, called \emph{stem modification} \citep{steriade1988,mccarthy1990a}, is best analyzed within the framework of Optimality Theory
\citep{prince1993}, as shown in analyses of Semitic
morphology, such as \citet{McCarthy1993a,Ussishkin1999,Ussishkin2000,Gafos2003,batel2003a}.

The details of the required modification depend on the structural
similarity between the base and the output; the more similar they are,
	the fewer the required adjustments. Any element in the configuration can
	be modified -- the vocalic pattern, the prosodic structure, and/or the
	affix. The modification, however, is contingent upon the configuration
	of the output.
	
	%\ea Stem Modification – modifying elements in the configuration\\
	%\begin{tabular}{lllll}
	%\multicolumn{2}{l}{\emph{\textbf{Base form}}} & \multicolumn{2}{c}{\emph{\textbf{Derived form}}}
	% & \emph{\textbf{Modified elements}}\\ \hline
	%sabon & `soap' &  → siben & `to soap' & vocalic pattern \\
	%tipel & `to take care of' & → me-tapel & `caretaker' & vocalic pattern, affix\\
	%matok & `sweet' & →  ma-mtak & `candy' & vocalic pattern, affix, \\
	% & & & & prosodic structure \\
	%\end{tabular}
	%\z
	
	\begin{table}
		\begin{tabularx}{\linewidth}{llllL}
			\lsptoprule
			\multicolumn{2}{l}{Base form} & \multicolumn{2}{l}{Derived form}
			& Modified elements\\
			\midrule
			sabon & `soap' &  siben & `to soap' & vocalic pattern \\
			tipel & `to take care of' & me-tapel & `caretaker' & vocalic pattern, affix\\
			matok & `sweet' & ma-mtak & `candy' & vocalic pattern, affix, prosodic structure \\
			\lspbottomrule
		\end{tabularx}
		\caption{Stem modification – modifying elements in the configuration.}
		\label{tab:stemmod}
	\end{table}
	
	Within this approach, there is no morphological element consisting
	solely of three consonants, and the emphasis here is on a `morphological
	element'. Of course, related words share consonants, but these are
	\emph{stem consonants}, where the stem is a morphological unit (e.g.
	\emph{tapél} in \emph{me-tapel} `caretaker'), but the consonants are
	phonological elements.
	
	\subsection{Word-based item-and-arrangement}\label{word-based-item-and-arrangement}
	Item-and-arrangement can also be applied within the word-based approach,
but only if a root is extracted from the base word \citep{ornan1983a, bolozky1978a}. That is, the base is the word but the root is an intermediate
morphological element in the derivation. The derivation proceeds in two
stages -- extraction and association \citep{batel1986a, batel1989a}. For example,
the word \emph{sabón} `soap' serves as the base for the verb
\emph{sibén} `to soap', which is derived in two stages: (i) extraction
of the consonants \emph{s,b,n}, which automatically become the root
√\emph{sbn} (traditionally called a secondary root), and (ii)
association of this newly formed root with the verb configuration
\emph{CiCeC}. The assumption is that the extracted consonants carry the
semantic properties of their base, which are, in turn, carried over to
the derived form.

However, root extraction is necessary only when one is limited to the
root-based approach, and thus to item-and-arrangement. In this model,
all words are derived via association of a root with a configuration,
regardless of whether the base is a word or a root. Not only is there no
independent reason to prefer root extraction to stem modification
(§2.2), but also there is empirical evidence refuting root extraction. These
are cases of phonological transfer (§3.1), whereby properties that
cannot be carried over by the consonants are transferred from the base
to the derived form.

\section{Phonological and morphological
relations}\label{phonological-and-morphological-relations}

\subsection{Transfer of phonological structure}\label{transfer-of-phonological-structure}

The most striking evidence for a direct relation between words, without
an intermediate stage that derives a root, is provided by cases
exhibiting phonological transfer \citep{clements1985a,hammond1988a,mccarthy1990a}. As shown below, there are cases where structural
information, which cannot be encoded in the consonantal root, is
transferred from the base to the derived form. In the case of Hebrew,
the structural information is both prosodic and segmental \citep{Batel1994}.

\subsubsection{Prosodic transfer}
\emph{Prosodic transfer} includes transfer of the entire configuration
or of a consonant cluster.

\emph{Configurations} are often assigned a grammatical function \citep{doron2003a}, but the question is whether this grammatical function is a
property of the configuration or just a property shared by many (but not
all) words within a morphological class. In general, words that share
meaning are often structurally similar, but it does not necessarily mean
that this shared meaning is a property of a morphological unit. One
striking example is displayed by the nouns in \cref{tab:nconfig} below, most of which
are creative innovations (drawn from
\textless{}http://www.dorbanot.com\textgreater{}). These nouns share the
configuration \emph{CoCCa} and the meaning `related to a computer
program'.

%\ea\label{ex:nconfig}Nouns sharing a configuration\\
%\scalebox{0.9}{
%\begin{tabular}[t]{llll}
%	\multicolumn{2}{l}{\textbf{a.} \textit{\textbf{CoCCa noun}}} &
%	\multicolumn{2}{l}{\textbf{b.} \textit{\textbf{Related word}}} \\ \hline
%	toχna & `computer program' & toχnit & `program' \\
%	gonva & `stolen computer program' &  ganav & `to steal' \\
%	poʁna & `computer program with porno pop-ups'  & poʁna & `pornography' \\
%	t͡soʁva & `illegally burned computer program' &  t͡saʁav & `to burn' \\
%	gomla & `old computer program' & gimlaot & `pension' \\
%\end{tabular} 
%}
%\z

\begin{table}
	\begin{tabular}[t]{llll}
		\lsptoprule
		\multicolumn{2}{l}{CoCCa noun} &
		\multicolumn{2}{l}{Related word} \\
		\midrule
		toχna & `computer program' & toχnit & `program' \\
		gonva & `stolen computer program' &  ganav & `to steal' \\
		poʁna & `computer program with porno pop-ups'  & poʁna & `pornography' \\
		t͡soʁva & `illegally burned computer program' &  t͡saʁav & `to burn' \\
		gomla & `old computer program' & gimlaot & `pension' \\
		\lspbottomrule
	\end{tabular}
	\caption{Nouns sharing a configuration.}
	\label{tab:nconfig}
\end{table}

Since these nouns share a configuration and meaning, the traditional
Semitic morphology would assign the meaning to the configuration. This
is, of course, erroneous because there are other nouns with the
configuration \emph{CoCCa} that do not carry this meaning; e.g.
\emph{jo∫ʁa} `dignity' (cf.\ \emph{ja∫aʁ} `honest'), \emph{χoχma}
`wisdom' (cf.\ \emph{χaχam} `smart'), \emph{ot͡sma} `strength' (cf.
\emph{at͡sum} `huge'), \emph{jozma} `enterprise' (cf.\ \emph{jazam} `to
initiate'). In addition, this meaning is too specific to function as a
morpho-semantic feature.

What we actually have here is a Semitic-type blending. The last four
words in the first column of \cref{tab:nconfig} use the first word \emph{toχna} as a base form, from which
the configuration is drawn, along with the basic meaning. That is,
\emph{toχna} provides the configuration \emph{CoCCa} and the meaning
`relating to a computer program'. The stem consonants are drawn from the
related words in the third column of \cref{tab:nconfig}, along with some specific meaning denoted by this
word. Crucially, such a derivation must be word-based, and the fact that
these words are creative innovations suggests that this model is active
in the Hebrew speakers' grammar.

Other creative examples are found in a children's story written by Meir
Shalev (\emph{ʁoni venomi vehadov jaakov} `Roni and Nomi and the bear
Jacob'). Each invented word in the first column of \cref{tab:shalev} has two bases, one
providing the configuration and another the consonants.

%\ea\label{ex:shalev}Meir Shalev's invented words\\
%\scalebox{0.8}{
%\begin{tabular}{llllll}
%\multicolumn{2}{l}{\textit{\textbf{Invented word}}} & 
%\multicolumn{2}{l}{\textit{\textbf{Source of configuration}}} & 
%\multicolumn{2}{l}{\textit{\textbf{Source of consonants}}} \\ \hline
%
%koféfet & `she wears gloves' &  lové∫et & `she wears' & kfafot & `gloves' \\
%mogéfet & `she puts on boots' &  noélet & `she puts on shoes' & magaf & `boot' \\
%lehitmaheʁ & `to hurry/rush' & lehizdaʁez & `to hurry' & lemaheʁ & `to rush' \\
%laχut͡s & `to run out' & laʁut͡s & `to run' & haχut͡sa & `outside' \\ 
%\end{tabular}}
%\z

\begin{table}
	\begin{tabularx}{\linewidth}{lLlLll}
		\lsptoprule
		\multicolumn{2}{l}{Invented word} & 
		\multicolumn{2}{l}{Source of configuration} & 
		\multicolumn{2}{l}{Source of consonants} \\
		\midrule	
		koféfet & `she wears gloves' &  lové∫et & `she wears' & kfafot & `gloves' \\
		mogéfet & `she puts on boots' &  noélet & `she puts on shoes' & magaf & `boot' \\
		lehitmaheʁ & `to hurry/ rush' & lehizdaʁez & `to hurry' & lemaheʁ & `to rush' \\
		laχut͡s & `to run out' & laʁut͡s & `to run' & haχut͡sa & `outside' \\ 
		\lspbottomrule
	\end{tabularx}
	\caption{Meir Shalev's invented words.}
	\label{tab:shalev}
\end{table}

Given that the invented words draw semantic properties from the two base
words, as is usually the case with blends, direct access to the base
must be assumed. That is, the configuration of one of the base words is
imposed on the other.

\emph{Cluster} transfer is often found in denominative verbs like
\emph{tʁansfeʁ} → \emph{tʁinsfeʁ} `to transfer' and \emph{faks} →
\emph{fikses} `to fax' \citep{bolozky1978a, mccarthy1984a, Batel1994}. In
such cases, the distribution of the sequential order of vowels and
consonants, thus including the clusters, is preserved in the derived
form. For example, \emph{fílteʁ} `filter' is the base of the verb
\emph{filteʁ} (preserved cluster -- \emph{lt}), while \emph{fliʁt}
`flirt' is the base of \emph{fliʁtet} (preserved clusters -- \emph{fl},
\emph{ʁt}), and not *\emph{filʁet}. Note that the higher the structural
similarity between the base and the derived form, the closer the semantic
relation \citep{raffelsiefen1993a}, and thus, the fewer the structural
amendments required in the course of stem modification (§2.2), the
greater the semantic similarity.

\subsubsection{Segmental transfer}
\emph{Segmental transfer} includes vowel transfer as well as the
transfer of an affix consonant to the stem \citep{Batel1994}.

In \emph{vowel transfer}, an exceptional configuration is selected
because its vowel is identical to that of the base (e.g.\ \emph{kod}
`code' → \emph{koded} `to codify', \emph{ot} `sign' → \emph{otet} `to
signal'). It should be noted that in most cases, the regular
configuration is also possible (e.g.\ \emph{kided} `to codify'). However,
the exceptional configuration is used only when the base has an
\emph{o}. That is, there is output-output correspondence between the
base noun \emph{kod} and the derived form, and \emph{koded} is
segmentally more faithful to \emph{kod} then \emph{kided} \citep{batel2003a}.

In \emph{affix transfer}, the consonant that serves as an affix in the
base becomes a stem consonant in the derived form. This is common with
the suffix -\emph{n}, as in \emph{paʁ∫an} `commentator' → \emph{piʁ∫en}
`to commentate' (cf.\ \emph{peʁe∫} `to interpret') and the prefix
\emph{m}-, as in \emph{maχzoʁ} `cycle' → \emph{miχzeʁ} `to recycle' (cf.
\emph{χazaʁ} `to return'). Note that speakers' morphological knowledge
allows them to strip the word of its affixes (more so in regular forms),
and therefore the inclusion of an affix consonant in the derived words
has its purpose, mostly to preserve a semantic contrast, as in
\emph{χizeʁ} `to court' vs.\ \emph{miχzeʁ} `to recycle' (from
\emph{maχzoʁ} `cycle'). But in the paradigm of \emph{∫amaʁ} `to guard'
-- \emph{mi∫maʁ} `guard' there is no *\emph{mi∫meʁ} (though it is a
potential verb).

\subsection{Semantic distance} 

One crucial property distinguishing among the three approaches reviewed in §3 is the
semantic `distance' between related words; among these, only the WoBIP model \cref{ex:worditemarr} allows a 
direct relation between a base and its derived form.

\ea\label{ex:distance}The distance factor\\
\ea \emph{Root-based item-and-process}\\
\begin{forest}
	[√sdʁ
	[sidéʁ\\`to arrange']
	[sédeʁ\\`order']
	]
\end{forest}

\ex \emph{Word-based item-and-process}\\
\begin{tikzpicture}
\matrix [matrix of nodes,anchor=center,row sep=2em,column sep=2em]
{|(seder)| sédeʁ & |(sdr)| √sdʁ \\
	& |(sider)| sidéʁ \\
};
\draw (seder) -- (sdr) -- (sider);
\node at (seder) [below=1ex] {`order'};
\node at (sider) [below=1ex] {`to arrange'};
\end{tikzpicture}

\ex\label{ex:worditemarr}\emph{Word-based item-and-arrangement}\\
\begin{tikzpicture}
\matrix [matrix of nodes,row sep=2em,column sep=2em]
{ & |(sdr)| \phantom{√sdʁ} \\
	|(seder)| sédeʁ	& |(sider)| sidéʁ \\
};
\draw (seder) -- (sider);
\node at (seder) [below=1ex] {`order'};
\node at (sider) [below=1ex] {`to arrange'};
\end{tikzpicture}
\z
%\scalebox{0.9}{
%\begin{tikzpicture}
%\node at (-1, 2.7) {a.};
%\node[text width=3cm] at (1, 2.5) {\small{\textit{\textbf{Root-based \\[-0.1cm] Item-and-process}}}};
%\node at (3, 2.7) {b.};
%\node[text width=3cm] at (5,2.5) {\small{\textit{\textbf{Word-based \\[-0.1cm] Item-and-process}}}};
%\node at (6.8, 2.7) {c.};
%\node[text width=5cm] at (9.8,2.5) {\small{\textit{\textbf{Word-based \\[-0.1cm] Item-and-arrangement}}}};
%\draw (-0.5,0) rectangle (2.7,2) node[pos=.5, align=center] {√sdʁ \\ \\ sidéʁ \hspace{.5cm} sédeʁ \\ \footnotesize{`to arrange' \ `order'}};
%\draw (0.3,1) -- (1.1,1.5);
%\draw (1.1,1.5) -- (1.8,1);
%\draw (3.5,0) rectangle (6.5,2) node[pos=.5, align=left] {sédeʁ --- √sdʁ \\ \footnotesize{`order'} \\ \hfill sidéʁ \\ \ \ \ \ \ \ \ \footnotesize{`to arrange'}} ;
%\draw (5.7, 1.5) -- (5.7,1);
%\draw (7.3,0) rectangle (11,2) node[pos=.5, align=left]{ \\  \\[.3cm] sédeʁ --- sidéʁ \\ \footnotesize{`order'} \ \ \ \footnotesize{`to arrange'}};
%\end{tikzpicture}
%}
\z

The advantage of the direct relation \cref{ex:worditemarr} is that information can be
carried over from input to output, be it structural (§3.1) or semantic.
It is often the case that within a group of words sharing stem
consonants, there is 1st, 2nd or
higher degree of separation between words, as illustrated in \cref{fig:degrees}.

\begin{figure}
	\caption{Degrees of separation.}
	\label{fig:degrees}
	\begin{tikzpicture}
	\matrix [matrix of nodes,every node/.style={draw,thick,rounded corners,align=center,anchor=center,text width=2.2cm},column sep=0.5cm, row sep=0.5cm]
	{
		|(time)| {kédem\\`ancient time'} &       &       &  \\
		|(ancient)| {kadúm\\`ancient'} &       &       &  \\
		|(former)| {kodém\\`former'} & |(precedent)| {takdím\\`precedent'} &       &  \\
		|(before)| {kódem\\`before'} & |(prefix)| {kidómet\\`prefix'} &       &  \\
		|(early)| {mukdám\\`early'} & |(payment)| {mikdamá\\`advanced payment'} &       &  \\
		|(beearly)| {hikdím\\`to be early'} &       & |(promotion)| {kidúm\\`promotion'} &  \\
		|(infront)| {kidmí\\`in front'} & |(progress)| {kidmá\\`progress'} &       & |(promote)| {kidém\\`to promote'} \\
		|(ahead)| {kadíma\\`ahead'} &       & |(toprogress)| {hitkadém\\`to progress'} &  \\
	};
	
	\draw[thick] (time) -- (ancient) -- (former) -- (before) -- (early) -- (beearly) -- (infront) -- (ahead);
	
	\draw[thick] (former) -- (precedent);
	
	\draw[thick] (before) -- (prefix);
	
	\draw[thick] (early) -- (payment);
	
	\draw[thick] (infront) -- (progress) -- (promotion) -- (promote) -- (toprogress) -- (progress);
	\end{tikzpicture}
\end{figure}

Such a network can express different degrees of semantic relations,
depending on how far one word is from another. Needless to say, such a
network cannot be expressed if all words are derived from a single root.
Of course, one can claim that the three words at the middle of the network (\form{takdím,} \form{kidómet,} and \form{mikdamá}),
which are not directly related to one another, are derived from a root,
while all other words are derived from words \citep{McCarthy1979a, arad2005a}.
However, this is an unsupported and unnecessary burden on the system.
All words in the network are connected to one another, directly or
indirectly, where some words are basic and others are derived. The fact
that all the words in \cref{fig:degrees} share the stem consonants is due to the
important role of consonants in conveying lexical information and
lexical relations (see §5.2).

\subsection{Derivation without a configuration} 

A fundamental element of the traditional root-based item-and-arrangement
model is that every word consists of a root and a configuration, where
every configuration is a function. This is particularly essential in the
verbal paradigms, where the configurations are claimed to carry
grammatical categories, such as transitivity \citep{doron2003a, arad2005a}.
Such a theory predicts that the transitivity relation must involve a change
in the configuration. This is true for most cases (e.g.\ \emph{katáv} `to
write' -- \emph{hitkatév} `to correspond', \emph{∫alaχ} `to send' --
\emph{ni∫laχ} `to be sent', \emph{laχat͡s} `to press' -- \emph{hilχit͡s}
`to cause to feel pressured'), but not all.

In an extensive study of labile alternations in Hebrew, \citet[114--115]{lev2016} lists
91 verbs where transitivity does not involve a change of the
configuration; three of his examples are provided in \cref{tab:labileverbs}.
As Lev argues, a root-based morphology cannot accommodate labile verbs
because under this approach the root has to associate with two different
configurations in order to derive verbs contrasting in transitivity. The
word-based approach, on the other hand, can incorporate labile verbs,
assuming that transitivity in such verbs is not lexically specified but
rather derived from the syntactic context. That is, some verbs are
specified for {[}± transitive{]} and others, i.e.\ the labile verbs, are
{[}ø transitive{]}. Many of the examples in Lev's list are recent
innovations, i.e.\ where verbs with transitivity specification become
labile. For example, the verb \emph{tijel} used to have one meaning
only, `to walk', but today it also means (at least for some speakers)
`to walk someone (usually a dog)'. This change can be viewed as a loss
of transitivity specification, i.e.\ {[}-- transitive{]} \textgreater
{[}ø transitive{]}. Crucially, it is the verb that loses its
specification for transitivity, not the configuration. That is, in
historical change too, as shown in the ensuing §4, it is the word that
changes, and not some putative consonantal root.

\begin{table}
	\begin{tabular}{lll}
		\lsptoprule
		Verb & Transitive & Intransitive\\
		\midrule
		hi∫χiʁ  & `to make black' & `to become black'\\
		hivʁik & `to polish' & `to shine'\\
		hivʁi & `to cure' & `to recuperate'\\
		\lspbottomrule
	\end{tabular}
	\caption{Labile verbs \citep[114--115]{lev2016}.}
	\label{tab:labileverbs}
\end{table}

\section{Historical change}\label{historical-change}
\subsection{Configuration change}\label{configuration-change}

Over the course of time, words may change their meaning or their
structure. In his study of instrumental nouns in Hebrew, \citet{laks2015a}
shows that quite a few instrumental nouns undergo change in their
configuration, in particular within a compound, as illustrated in \cref{tab:configchange}.
As Laks shows, the change always goes towards the participial
configuration, and it never occurs when the instrumental noun does not
have a verbal counterpart. That is, while both \emph{maχded} `pencil
sharpener' and \emph{mazleg} `fork' have the same configuration, only
the former adopts the participial configuration \emph{meχaded}, given
its verbal counterpart \emph{χided} `to sharpen'; the latter cannot
adopt a participial configuration because it does not have a verbal
counterpart.

\begin{table}
	\begin{tabularx}{\linewidth}{llllL}
		\lsptoprule
		\multicolumn{2}{l}{Old configuration} &
		\multicolumn{3}{l}{New configuration -- participle}  \\
		\midrule
		maχded & maCCeC & meχaded & meCaCeC & `pencil sharpener' \\
		nakdan & CaCCan & menaked (tekstim) & meCaCeC & `text vocalizer' \\
		masχeta & maCCeCa & soχet (mit͡sim) & CoCeC & `juicer (juice squeezer)' \\
		\lspbottomrule
	\end{tabularx}
	\caption{Change of configuration in instrumental nouns \citep{laks2015a}.}
	\label{tab:configchange}
\end{table}

In order for this restriction to hold, an instrumental noun must be
linked to its verb, from which it can draw its participial
configuration. Otherwise, as Laks argues, the instrumental noun could
adopt any of the five participle configurations, not necessarily the one
associated with its verb. That is, we get the instrumental noun
\emph{meχaded} `pencil sharpener' because \emph{meCaCeC} is the
participial configuration of \emph{χided} `to sharpen'. Similarly, we
get the instrumental nouns \emph{soχet} (\emph{mit͡sim}) `juicer' because
\emph{CoCeC} is the participial configuration of \emph{saχat} `to
squeeze'. Such a change is possible only in a word-based lexicon; a
root-based lexicon does not account for the restrictive generalization
as it allows options that are not attested.

\subsection{Semantic change}\label{semantic-change}

Over the course of time, the meaning of words also changes; crucially,
the semantic change affects words and not putative roots. For example,
the verbs \emph{nimlat} and \emph{himlit} used to differ in transitivity
only, with the former meaning `to escape' and the latter `to help
someone to escape'. Nowadays, these verbs are not related, since
\emph{himlit} means `to give birth (for non-humans)'. Similarly,
\emph{kalat} and \emph{hiklit} used to be related, with the former
meaning `to absorb' and the latter `to cause to absorb'. However, the
meaning of \emph{hiklit} is now `to record', and the two verbs are
vaguely related, if at all. For the traditional root-based approach
(§2.1), it would be rather strange that the change in meaning does not
affect the element that carries it, i.e.\ the root. This inconsistency
does not arise within the word-based approach.

It is quite feasible that the root does not undergo semantic change
because its meaning is just ``a basic meaning range'', according to
\citet{moscati1980a} and other Semiticists, or underspecified, according to
\citeauthor{arad2005a}'s (\citeyear{arad2005a}) analysis within the theory of Distributed Morphology
(\citealt{halle1993} and subsequent studies). That is, semantic
specification of roots may have at least three degrees of specification:
fully specified (e.g.\ \emph{boy}), underspecified (e.g.\ Hebrew roots),
and unspecified (e.g.\ the roots in \emph{refer}, \emph{remit}, and
\emph{resume}; \citealt{aronoff1976}).

The major problem is that the specific meaning of words is derived,
according to \citet{arad2005a}, from the morpho-syntactic context, i.e. the
configurations. This works nicely for some words but not for others.
Consider, for example, the pairs \emph{zaʁak} `to throw' --
\emph{hizʁik} `to inject' and \emph{ma∫aχ} `to pull' -- \emph{him∫iχ}
'to continue'. It is not clear which semantic property can be assigned
to the configurations \emph{CaCaC} and \emph{hiCCiC} such that the
relation within these pairs would be consistent.

\subsection{Segmental change}\label{segmental-change}

Like semantic change, segmental change also affects words and not
consonantal roots, even when the change is in the stem consonants. This
is seen in the case of stop-fricative alternation, which due to its
opacity, suffers from a great degree of change-oriented variation \citep{adam2002a}.


As shown in \cref{tab:stopfric}, normative verb inflectional paradigms are changing under
the force of paradigm uniformity. Although the change affects consonants, it certainly does not affect a
consonantal root because derivationally related words are hardly ever
affected; nonetheless they change, and sometimes they even undergo independent
change. For example, while χ can change to \emph{k} in \emph{katav} --
\emph{jiktov} (normative \emph{jiχtov}) `to write \textsc{past --
	future',} it never changes to \emph{k} in \emph{miχtav} (*\emph{miktav})
`letter'. Note also that while the direction of change in this paradigm
is from a fricative to a stop (\emph{jiχtov → jiktov}), the change in a
related pair is towards a fricative, as in \emph{ktav → χtav}
`handwriting'.

\begin{table}
	
	\begin{tabular}{lllllll}
		\lsptoprule
		\multicolumn{3}{l}{Old paradigm} & 
		\multicolumn{3}{l}{New paradigm} & \\
		\midrule
		k--χ &kisa & jeχase & χ--χ &χisa & jeχase & `to cover \textsc{past -- future}'\\
		k--χ &katav & jiχtov & k--k &katav & jiktov & `to write \textsc{past -- future}'\\
		b--v &bitel & jevatel & v--v &vitel & jevatel & `to cancel \textsc{past -- future}' \\
		\lspbottomrule
	\end{tabular}
	\caption{Change of configuration in instrumental nouns \citep{adam2002a}.}
	\label{tab:stopfric}
\end{table}

\section{Other supporting sources}\label{other-supporting-sources}
\subsection{Children's words}
During the early
stages of acquisition, verbs in the production lexicon of
Hebrew-acquiring children are not derivationally related, i.e.\ they do
not share stem consonants. Derivationally related verbs start appearing
later on, where the new verbs ``are learnt as versions of, and based
upon, the verbs known from before'' \citep[62]{berman1988a}.

This direct derivation in children's speech is not surprising given
\citeauthor{ravid2016a}'s (\citeyear{ravid2016a}) study on the distribution of verbs in spoken and
written Hebrew corpora: child-directed speech (to toddlers age 1;8--2;2)
and storybooks (for preschoolers and
1st--2nd grade). In both corpora,
the average number of verbs per root was below two: 684 verbs for 521
root types in the spoken corpus (1.3) and 1,048 verbs for 744 roots in
the written corpus (1.4). Only around 30\% of the verb types in each
corpus share a root with another verb, and most such verbs share a root
with only one other verb.

These results mean, as the authors admit, that at least until the age of
7, the children have very little input supporting a root-based
morphology. Nevertheless, the authors insist that the children must
``eventually construe the root as a structural and semantic
morphological core'' \citep[126]{ravid2016a}. As argued in the current
paper and elsewhere, starting with \citet{Batel1994}, Hebrew speakers are
free from this burden since Hebrew morphology (and Semitic morphology in
general) is not root-based, but rather word/stem-based.

Previous studies that attribute children and adults' innovations to root
extraction (§2.3 -- word-based item-and-arrangement) must now reconsider
their conclusion at least for children below the age of 7. In an
experimental study reported in \citet{berman2003a}, children at the age of
4--6 years old had a rather high success rate (84--88\%) of morphological
innovation (forming novel verbs from nouns or adjectives) with a very
high percentage of well-formed innovations (91--99\%). If children can
form verbs from nouns/adjectives at the stage where they still do not
have sufficient input that allows them to form a root-based morphology
\citep{ravid2016a}, they probably use another strategy -- the
modification strategy employed within the WoBIP model (§2.3). And if
they can use this model successfully until the age of 7, they have no
reason whatsoever to shift to a root-based model later on. Of course, as
I have argued here and elsewhere, they do not -- Hebrew speakers employ
WoBIP at all ages.

\subsection{Experimental studies}
There are quite a few experimental studies supporting the consonantal root in Hebrew. Most
notable are Berent's studies with the acceptability rating paradigm
(\citealt{berent1997a, berent2001a}, inter alia) and Frost's
studies with the priming paradigm (\citealt{Frost1997, frost2000a}, inter
alia).\footnote{In an additional study, which was design to ask ``is it
  a root or a stem?'' (rather than ``is it a root?''), \citet{berent2007aB} note that although their results do not falsify the root-based
  account, they strongly suggest that the stem can account for the
  restrictions on identical consonants.} However, most experimental
studies supporting the consonantal root in Hebrew morphology adopted a
visual modality. As such, they cannot tease apart the effect of
orthography, which is primarily consonantal (\citealt{batel2002a} and \citealt{berrebi2016a} for a critical view).

A fresh look on the matter is provided in \citeauthor{berrebi2016a}'s (\citeyear{berrebi2016a}) auditory
priming study, which controlled semantic relatedness and orthographic
identity. Word pairs sharing phonological stem consonants were either
semantically related (e.g.\ \emph{kibel} `to receive' -- \emph{hitkabel}
`to be accepted') or semantically unrelated (e.g.\ \emph{ʁigel} `to spy'
-- \emph{hitʁagel} `to get used to'); and when semantically unrelated,
either orthographically identical with respect to the consonants (e.g.
\emph{ʁigel} `to spy' -- \emph{hitʁagel} `to get used to') or
orthographically different (e.g.\ \emph{∫ikeʁ} `to lie' --
\emph{hi∫takeʁ} `to get drunk', where \emph{k} is spelled differently).
The results showed that all conditions had a priming effect, i.e.
whether or not the prime and the target were orthographically identical
or semantically related. As the property shared by the prime and the
target in all conditions was phonological, i.e.\ stem consonants, the
results suggest that there is a phonological priming effect among words
sharing stem consonants. Crucially, the stem consonants are not a
morphological unit since there was also a priming effect when the words
were semantically unrelated and orthographically different (e.g.
\emph{∫ikeʁ} -- \emph{hi∫takeʁ}).

If we assume that priming effects reflect the organization of the
lexicon, then we can conclude that words are also phonologically
organized according to the stem consonant. As emphasized in §2.2, the
stem consonants are phonological elements (consonants) within a
morphological unit (stem); they do not carry meaning and they do not
constitute a morphological unit.

Stem consonants, and not vowels, serve to identify relations between
words because consonants are lexically prominent, while vowels have
syntactic functions \citep{nespor2003a, berent2013a}; this is true not
only for Hebrew but also for non-Semitic languages. In their
experimental study, \citet{cutler2000} asked the participants: ``Is a
\emph{kebra} more like \emph{cobra} or \emph{zebra}?''. They found that
speakers identify similarity between a nonce word (\emph{kebra}) and an
existing word on the basis of shared consonants (\emph{kebra} --
\emph{cobra}) rather than shared vowels (\emph{kebra} -- \emph{zebra}).
That is, the consonants serve as the core of similarity between words in
English, French, Swedish, and Dutch as much as they do in Hebrew and
other Semitic languages (see also \citealt{vanooijen1996a, new2008a, carreiras2009a, winskel2013a}).

Consonants are lexically prominent from the very early stages of
language development. This is reported in \citeauthor{nazzi2007a}'s (\citeauthor{nazzi2007a}) study,
where French 16--20 month old infants could learn in a single trial two
new nonce words if they differed by one consonant (\emph{pize} --
\emph{tize}), but not if they differed by one vowel (\emph{pize} --
\emph{paze}). That is, although vowels are acoustically more prominent
than consonants, when it comes to lexical contrast, consonants are
employed. This is true for children and adults, regardless of the
ambient language, whether it is Semitic or non-Semitic.

Consonants are prominent not only in speech perception and lexical
relations but also in the association between sound and shape revealed
by the \emph{bouba-kiki} effect \citep{kohler1929a}, whereby people pair
labial consonants with round shapes and dorsal consonants with spiky
shapes. One of the many subsequent studies of the \emph{bouba-kiki}
effect is \citet{fort2015a}, which found that the sound--shape
association remains constant regardless of the vowels. That is,
\emph{lomo} was associated with a round shape as much as \emph{limi},
and \emph{toko} with a spiky shape as much as \emph{tiki}. \citet{fort2015a} conclude that consonants have a greater effect than vowels in
sound -- shape association.

\section{19th century Semitic grammarians}\label{th-century-semitic-grammarians}

The root-based item-and-arrangement model of Semitic morphology has been
deeply entrenched for generations, thus presenting the advocates of the
word-based item-and-process approach as revolutionary (see \citealt{horvath1981a, lederman1982a, heath1987a, hammond1988a, mccarthy1990a, Batel1994, batel2002a, batel2003a, ratcliffe1997a, Ussishkin1999, Ussishkin2000, Ussishkin2005, laks2011a, laks2015a, lev2016}.

However, WoBIP has its seeds in the studies of the orientalists Wilhelm
Gesenius (1786--1842) and William Wright (1830--1889), who wrote the
seminal grammar books of Hebrew and Arabic respectively. It is important
to note that both Gesenius and Wright were not native speakers of a
Semitic language (Gesenius was German and Wright British), and thus not
biased like the other Semitic grammarians by the consonantal script of
Hebrew and Arabic.

\citet{gesenius1813a} distinguishes between `primitive' verbs, which
consist of a stem only and are not derived from any other form, and
derived verbs, among which there are verbal derivatives and denominative
verbs. Gesenius used the term `internal modification' when addressing
the processes involved in the derivation. He indicates two types of
``changes in the primitive form'' \citep[115]{gesenius1813a}: internal modification (cf.
stem modification; §2.2) and repetition (i.e.\ reduplication) of one or
two of the stem consonants. Within the internal modification, he
includes vowel change like \emph{gadal} `to grow' -- \emph{gidel} `to
raise', and gemination as in Biblical Hebrew \emph{ga:dal} `to grow' --
\emph{giddel} `to raise' (there is no gemination in Modern Hebrew).
Crucially, Gesenius compares vowel modification in Hebrew to that in
English \emph{lie} -- \emph{lay} and \emph{fall} -- \emph{fell}, and
does not find them different. That is, Gesenius finds stem modification
to be identical in both Hebrew and English, but unlike \citet{anderson1992}
who contemplates whether English is like Hebrew, Gesenius actually
claims that Hebrew is like English.

A similar approach is found in \citeauthor{wright1859a}'s (\citeyear{wright1859a}) grammar book of
Arabic, where he describes the relation between verbs within the WoBIP
model. For example, ``the third form \ldots{} is formed from the first
by lengthening the vowel sound after the first radical'' (p.\ 32) or
``{[}T{]}he second form \ldots{} is formed from the first \ldots{} by
doubling the second radical'' (p.\ 31). This is the format of Wright's
description of each and every binyan in Arabic, and it specifically says
that (i) one form is derived from another, i.e.\ word-based derivation,
and (ii) the derivation involves some process, doubling, lengthening,
etc., i.e.\ item-and-process. Note that Wright uses the term `radical' to
refer to a consonant in the stem, without reference to the stem
consonants being an independent morphological unit.

That is, although it has always been said that the root-based approach
is the one assumed by traditional Semiticists, it is important to
emphasize that the two great 19th century Semiticists,
Gesenius and Wright, were proponents of the WoBIP model of Semitic
morphology.\footnote{A reviewer suggested that Gesenius and Wright
  adopted the word-based approach, which was used for Latin grammar,
  because they worked prior to the introduction of the term morpheme.
  \citet{kilbury1976a} and \citet{anderson1985a} attribute the term morpheme to
  Baudouin de Courtenay's student H. Ułaszyn, in his articles from 1927
  and 1931. However, it is possible to have a notion without a specific
  term. Sibawayhi (760--796), who wrote the first known Arabic grammar
  \emph{Al-kitab}, used the term \emph{kalima} `word' in the sense of a
  morpheme (e.g. the suffix \emph{-ta}) and referred to the radicals
  that make up the words \citep{levin1986}. Gesenius notes that the Jewish
  grammarians call the stem root and the stem consonants radical
  letters. That is, there was a reference to morphological units (stem,
  affixes), but the stem consonants did not constitute a morphological
  unit.}

\section{Concluding remarks}\label{concluding-remarks}

In section 3.6, \citet[71]{anderson1992} concludes: ``\ldots{} the morphology of
a language consists of a set of Word Formation Rules which operates on
lexical stems to produce other lexical stems \ldots{}'' In this
paper I extended the scope of this model to Semitic morphology. That is,
in Semitic languages too, words are derived from words/stems via
modification of the base.

The modification in Semitic morphology is output oriented, as the output
has to fit into a configuration. The constraint-based framework of
Optimality Theory \citep{prince1993a} allows for
output-oriented grammar, where the constraints impose certain
configurations (structural constraints) as well as output-output
identity of consonants (faithfulness constraints). A configuration is
imposed by several constraints, referring to syllabic structure (usually
a foot), syllable structure, and vocalic patterns (where the latter ones
are language specific). Identity among the stem consonants is imposed by
the faithfulness constraints, where preservation of segmental identity
ensures preservation of morphological relation among words.

That is, the stability of the stem consonants is due to phonological
faithfulness constraints that require identity among stem consonants.
Phonological faithfulness enhances morphological relations. ``Any given
\emph{focal} word (that is, a specific word in which we are interested)
is thus surrounded by a vaguely defined family of words which are more
or less acoustically similar to it. The members of the family will in
general have the widest variety of meaning, and yet it may often happen
that some members of the family will resemble the focal word not only in
acoustic shape, but also in meaning'' (\citealt[297]{Hockett1958}, \citeyear[86]{hockett1987a}). That
is, within an acoustic family of words there is a morphological family,
where the words are not only acoustically similar but also semantically
related. For the purpose of membership in a morphological family, the
consonants are more important than the vowels. This status does not
grant the consonants morphological status, neither in English nor in
Hebrew.

\section*{Acknowledgements}
As my advisor and mentor, Steve Anderson set the foundations for my approach to Semitic morphology presented in this paper.

\printbibliography[heading=subbibliography,notkeyword=this]

% \todos

\end{document}

