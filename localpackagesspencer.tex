% add all extra packages you need to load to this file 
\usepackage{todo}
\usepackage{graphicx}
\usepackage{tabularx}
\usepackage{amsmath} 
\usepackage{multicol}
\usepackage{lipsum}
\usepackage{longtable,booktabs}
\usepackage{tikz}


% \makeatletter
% \DeclareOldFontCommand{\rm}{\normalfont\rmfamily}{\mathrm}
% \DeclareOldFontCommand{\sf}{\normalfont\sffamily}{\mathsf}
% \DeclareOldFontCommand{\tt}{\normalfont\ttfamily}{\mathtt}
% \DeclareOldFontCommand{\bf}{\normalfont\bfseries}{\mathbf}
% \DeclareOldFontCommand{\it}{\normalfont\itshape}{\mathit}
% \DeclareOldFontCommand{\sl}{\normalfont\slshape}{\@nomath\sl}
% \DeclareOldFontCommand{\sc}{\normalfont\scshape}{\@nomath\sc}
% \makeatother

%%%%%%%%%%%%%%%%%%%%%%%%%%%%%%%%%%%%%%%%%%%%%%%%%%%%
%%%                                              %%%
%%%           Examples                           %%%
%%%                                              %%%
%%%%%%%%%%%%%%%%%%%%%%%%%%%%%%%%%%%%%%%%%%%%%%%%%%%%
% remove the percentage signs in the following lines
% if your book makes use of linguistic examples
\usepackage{LSP/lsp-styles/lsp-gb4e} 
%% to add additional information to the right of examples, uncomment the following line
% \usepackage{jambox}
%% if you want the source line of examples to be in italics, uncomment the following line
% \def\exfont{\it}

%%%%%%%%%%%%%%%%%%%%%%%%%%%%%%%%%%%%%%%%%%%%%%%%%%%%
%%%                                              %%%
%%%      Optimality Theory                       %%%
%%%                                              %%%
%%%%%%%%%%%%%%%%%%%%%%%%%%%%%%%%%%%%%%%%%%%%%%%%%%%%
% If you are using OT, uncomment the following lines      
% % OT pointing hand
%  \usepackage{pifont}
%  \newcommand{\hand}{\ding{43}}
% % OT tableaux                                                
% \usepackage{pstricks,colortab}  
% \usepackage{multirow} % used in OT tableaux
% \usepackage{rotating} %needed for angled text%
% \newcommand{\rot}[1]{\begin{rotate}{90}#1\end{rotate}} %shortcut for angled text%  

%%%%%%%%%%%%%%%%%%%%%%%%%%%%%%%%%%%%%%%%%%%%%%%%%%%%
%%%                                              %%%
%%%       Attribute Value Matrices               %%%
%%%                                              %%%
%%%%%%%%%%%%%%%%%%%%%%%%%%%%%%%%%%%%%%%%%%%%%%%%%%%%
%If you are using Attribute-Value-Matrices, uncomment the following lines 
% \usepackage{lsp-avm}
% \usepackage{avm}
% \avmfont{\sc} 
% \avmvalfont{\it} 
% % command to fontify the type values of an avm 
% \newcommand{\tpv}[1]{{\avmjvalfont #1}} 
% % command to fontify the type of an avm and avmspan it
% \newcommand{\tp}[1]{\avmspan{\tpv{#1}}}

\usepackage{booktabs}
\usepackage{paralist}
\usepackage{setspace}
\usepackage{xcolor}
\usepackage{amssymb} %for \varnothing
% \usepackage{multicol}
% \usepackage{multirow}

%%%%% AVM style 
\usepackage{avm}
\avmoptions{active}
\avmvalfont{\upshape}
\avmsortfont{\scriptsize\itshape}
%%%%% %%%%%%%%%%%%%%%%%%%

%%%%%%%  Drawing   %%%%%
\usepackage{qtree}

%%%%%%  Author macros %%%%%%%
\newcommand\redmar[1]{\marginpar{\textsf{\color{red}{#1}}}}		

% \newcommand\textex[1]{\textit{#1}}
% \newcommand\lxm[1]{\textsc{#1}}
\newcommand\glossfeat[1]{\textsc{#1}}
\newcommand\featname[1]{#1}

\newcommand\featval[1]{\textit{#1}}
\newcommand{\hpsgtag}[1]{\raisebox{0.2ex}{{\tiny\fbox{#1}}}}

\newcommand\Corrfn{\textbf{\textit{Corr}}}

\newcommand\pmfn{\textbf{\textit{pm}}}

\newcommand\Corr[1]{\textbf{\textit{Corr}}$(\langle${#1}$\rangle)$}

\newcommand\PF[1]{PF$(\langle${#1}$\rangle)$}

\newcommand{\lab}{$\langle$}

\newcommand{\rab}{$\rangle$}

%%%%%%%%%%%%%%%%%%%%%%%

\usepackage[linguistics]{forest}
